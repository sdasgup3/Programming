For $\{\p, \q\} \subseteq  \heap,\ f \in\fields,\ n \in \nat$
and $\mbox{\tt op} \in \{+, -\}$, we have the following eight
basic  statements   that  can  access  or   modify  the  heap
structures.
 \begin{enumerate}
   \item Allocations
   	\begin{enumerate}
     	\item {\tt p = malloc();}
   	\end{enumerate}
   \item Pointer Assignments
   	\begin{enumerate}
     	\item {\tt p = NULL;}
     	\item {\tt p = q;}
     	\item {\tt p = q $\rightarrow$ f;}
     	\item {\tt p = \&(q $\rightarrow$ f);}
     	\item {\tt p = q op n;}
   	\end{enumerate}
   \item Structure Updates
   	\begin{enumerate}
     	\item {\tt p $\rightarrow$ f = q;}
     	\item {\tt p $\rightarrow$ f = NULL;}
   	\end{enumerate}
 \end{enumerate}

Our intend is  to determine, at each program  point, the path
matrix  $P_F$  and  the  boolean  variables  capturing  field
connectivity.   We formulate  the problem  as an  instance of
forward data  flow analysis, where  the data flow  values are
the boolean variables and the path matrix. For simplicity, we
construct   basic  blocks   containing  a   single  statement
each. Before presenting the equations for data flow analysis,
we define  the confluence operator (\merge)  for various data
flow values  as used by our analysis.  Using the superscripts
$x$ and $y$ to denote the values coming along two paths,

\begin{eqnarray*}
  \merge( f_{\p\q}^x, f_{\p\q}^y ) &=& f_{\p\q}^x \vee f_{\p\q}^y , f \in \fields,
  \p,\q \in \heap \\
  \merge(\p_{\subC}^x, \p_{\subC}^y) &=& \p_{\subC}^x \vee
  \p_{\subC}^y, \p \in \heap \\
  \merge(\p_{\subD}^x, \p_{\subD}^y) &=& \p_{\subD}^x \vee
  \p_{\subD}^y, \p \in \heap \\
  %\merge(P_F^{x}, P_F^{y}) &=& P_F^{xy}
  \merge(P_F^{x}, P_F^{y})  &=& P_F^{xy} \mbox{ where }
   P_F^{xy}[{\p}, {\q}]  = \\ 
   && P_F^{x}[{\p}, {\q}] \cup
      P_F^{y}[{\p}, {\q}],   \forall {\p}, {\q} \in \heap\\
\end{eqnarray*}
The transformation of data flow values due to a statement
is captured by the following set of equations:
\begin{eqnarray*}
  P_F^{\dout}[\p,\q] &=& (\remOne{P_F^{\din}[\p,\q]}{P_F^{\dkill}[\p,\q]}) \cup
  P_F^{\dgen}[\p,\q] \\
  I_F^{\dout}[\p,\q] &=& (\remOne{I_F^{\din}[\p,\q]}{I_F^{\dkill}[\p,\q]}) \cup
  I_F^{\dgen}[\p,\q] \\
  \p_{\subC}^{\dout} &=& (\p_{\subC}^{\din} \wedge
  \neg\p_{\subC}^{\dkill} ) \vee \p_{\subC}^{\dgen}  \\
  \p_{\subD}^{\dout} &=& (\p_{\subD}^{\din} \wedge
  \neg\p_{\subD}^{\dkill} ) \vee \p_{\subD}^{\dgen} 
\end{eqnarray*}

Update of field connectivity information, and the computation
of the  $\dgen$ and $\dkill$ components of  various data flow
values depends on the type  of statement. This is described
in details next.

%%%%%%%%%%%%%%%%%%%%%%%%%%%%%%%%%%%%%%%%%%%%%%%%%%%%%
\subsection{Analysis of Basic
  Statements} \label{sec:Analysis_Rules}  
%%%%%%%%%%%%%%%%%%%%%%%%%%%%%%%%%%%%%%%%%%%%%%%%%%%%%
We now present  our analysis for each kind  of statement that
can  modify the heap structures.

%@@@@@@@@@@@@@@@@@@@@@@@@@@@@
\subsubsection{\tt p = NULL}
%@@@@@@@@@@@@@@@@@@@@@@@@@@@@
This statement only kills the existing values of
\p. The heap node pointed-to by \p\ is no longer reachable by
\p. However, it is possible that the node is part of a DAG or
a Cycle reachable from some other node \q. To keep the
analysis safe, we need to first evaluate all terms that involve
\p\ ($f_{pq}^{\anysup}$, $P_F$[\q,\p], etc.) and use
the resulting value in the equations using 
them. Then, we kill the values of \p\ as given by the
equations:
\begin{eqnarray*}
  \KillC{p}  = \InC{p} &\quad& \KillD{p} = \InD{p} \\
  \GenC{p} = \false 	 &\quad& \GenD{p} = \false \\
\end{eqnarray*}
$\forall \s \in \heap,$
\begin{eqnarray*}
  f_{\p\s} =\false  &\quad& f_{\s\p} =\false \\
  P_F^{\dkill}[\p,\s]  =  P_F^{\din}[\p,\s] &\quad&
  P_F^{\dgen}[\p,\s]   =  \emptyset \\ 
  P_F^{\dkill}[\s,\p]  =  P_F^{\din}[\s,\p] &\quad&
  P_F^{\dgen}[\s,\p]   =  \emptyset 
\end{eqnarray*}

\begin{figure}
\begin{center}
\begin{tabular}[t]{@{}c@{}}
  \\
    {\tt
      \renewcommand{\arraystretch}{0.5}
      \begin{tabular}[t]{l|@{}c@{}}
	{\em \scriptsize S1:} p$\rightarrow$f = q;	    & 
	\begin{pspicture}(0,3)(3,4.3)
          %\psframe(0,3)(3,4.4)
	  \pscircle(.7,3.6){.3}
	  \pscircle(2.3,3.6){.3}
	  \psset{linecolor=black}
	  \psline[linewidth=1pt,linearc=.5]{->}(1.0,3.6)(1.5, 3.9)(2.0,3.6)
	  \rput(2.3,3.6){\psframebox*{\scalebox{.8}{$q$}}}
	  \rput(1.5,4.1){\psframebox*{\scalebox{.8}{$f$}}}
	  \rput(.7,3.6){\psframebox*{\scalebox{.8}{$p$}}}
	\end{pspicture}  \\
	& \\ \hline
        {\em \scriptsize S2:} q$\rightarrow$f = p; &
	\begin{pspicture}(0,3)(3,4.4)
	  \pscircle(.7,3.6){.3}
	  \pscircle(2.3,3.6){.3}
	  \psset{linecolor=black}
	  \psline[linewidth=1pt,linearc=.5]{->}(1.0,3.6)(1.5, 3.9)(2.0,3.6)
	  \psline[linewidth=1pt,linearc=.5]{<-}(1.0,3.6)(1.5, 3.3)(2.0,3.6)
	  \rput(2.3,3.6){\psframebox*{\scalebox{.8}{$q$}}}
	  \rput(1.5,4.1){\psframebox*{\scalebox{.8}{$f$}}}
	  \rput(1.5,3.1){\psframebox*{\scalebox{.8}{$f$}}}
	  \rput(.7,3.6){\psframebox*{\scalebox{.8}{$p$}}}
	\end{pspicture} \\ 
	& \\	\hline
        {\em \scriptsize S3:} p = null; &
	\begin{pspicture}(0,3)(3,4.4)
	  \pscircle(.7,3.6){.3}
	  \pscircle(2.3,3.6){.3}
	  \psset{linecolor=black}
	  \psline[linewidth=1pt,linearc=.5]{->}(1.0,3.6)(1.5, 3.9)(2.0,3.6)
	  \psline[linewidth=1pt,linearc=.5]{<-}(1.0,3.6)(1.5, 3.3)(2.0,3.6)
	  \rput(2.3,3.6){\psframebox*{\scalebox{.8}{$q$}}}
	  \rput(1.5,4.1){\psframebox*{\scalebox{.8}{$f$}}}
	  \rput(1.5,3.1){\psframebox*{\scalebox{.8}{$f$}}}
	  \rput(.7,3.6){\psframebox*{\scalebox{.8}{\mbox{}}}}
	\end{pspicture}
      \end{tabular}
    } \\
    \scalebox{0.8}{(a)} \\ \\
    \begin{tabular}[t]{@{}c@{}}
      $P_F$[\p,\q] = \{\fieldD{f}{}\} \quad  $P_F$[\q,\p] = \{\fieldD{f}{}\} \\ 
      $\f_{\p\q}$  = \true           \quad $\f_{\q\p}$  = \true \\
      $q_{\subC}$  = $f_{\q\p} \wedge (\num{P_F[\p,\q] \geq 1})$  \\
    \end{tabular} \\
    \scalebox{0.8}{(b)} \\ \\    
    \begin{tabular}[t]{@{}c@{}}
      $P_F$[\p,\q] = $\emptyset$ \quad  $P_F$[\q,\p] = $\emptyset$ \\ 
      $\f_{\p\q}$  = \false            \quad $\f_{\q\p}$  = \false \\
      $q_{\subC}$  = \true  \\
    \end{tabular} \\
    \scalebox{0.8}{(c)}
\end{tabular}
\end{center}

\caption{The case  for eager evaluation  of boolean functions
  involving \p\ for {\tt p  = NULL}.  (a) shows the updates of
  a heap graph. (b) and  (c) show data-flow values and boolean
  equations  before   and  after  {\em   S3}.  Without  eager
  evaluation,  the  equation  for  $q_{\Cycle}$  will  remain
  $f_{\q\p}  \wedge  (\num{P_F[\p,\q]  \geq  1})$  and  hence
  evaluate         to         false        after         {\em
    S3}.\label{fig:p_null_illustrate}}
\end{figure}

\begin{example}{\rm
In  the  code segment  Figure~\ref{fig:p_null_illustrate}(a),
after statement  {\em S2} there is  a \Cycle\ on  \p\ and \q.
The   cycle   on   \q\    must   remain   after   {\tt   S3}.
Figure~\ref{fig:p_null_illustrate}(b),  presents the relevant
data-flow values after {\em S2}.  Before {\em S3}, we evaluate
terms containing \p\, i.e.\ $f_{\q\p}$ and $\num{P_F[\p,\q] \geq
  1}$  both to  true, and  use  these values  in the  equation
\InC{\q} so as to get $\InC{q}$ as true. Then we kill all the
equations of \p. After {\em  S3}, we still have $\OutC{q}$ as
true, thus  inferring the shape  of \q\ as \Cycle.  This would
not  be the  case if  we do  not eagerly  evaluate  the terms
containing \p\ at {\em S3}.  } \hfill\psframebox{}
\end{example}
%@@@@@@@@@@@@@@@@@@@@@@@@@@@@
\subsubsection {{\tt p = malloc()}}
%@@@@@@@@@@@@@@@@@@@@@@@@@@@@
After this statement all the existing data-flow values of $\p$
get  killed and  \p\  starts pointing  to  a newly  allocated
object. The kill  effect is exactly same as that  of {\tt p =
  NULL}.  After the statement,  $\p$ can  only have  an empty
path to itself.
\begin{eqnarray*}
  \KillC{p}  = \InC{p} &\quad& \KillD{p} = \InD{p} \\
  \GenC{p} = \false 	 &\quad& \GenD{p} = \false 
\end{eqnarray*}
$\forall \s \in \heap,\ \s \not= \p,$
\begin{eqnarray*}
  f_{\p\s} =\false  &\quad& f_{\s\p} =\false \\
  P_F^{\dkill}[\p,\s]  =  P_F^{\din}[\p,\s] &\quad&
  P_F^{\dgen}[\p,\s]    =  \emptyset \\ 
  P_F^{\dkill}[\s,\p]  =  P_F^{\din}[\s,\p] &\quad&
  P_F^{\dgen}[\s,\p]    =  \emptyset \\
  P_F^{\dkill}[\p,\p]  =  P_F^{\din}[\p,\p] &\quad&
  P_F^{\dgen}[\p,\p]    =  \epsilonset
\end{eqnarray*}


%@@@@@@@@@@@@@@@@@@@@@@@@@@@@
\subsubsection{\tt p = q, p = \&(q$\rightarrow$f), p = q op n} 
%@@@@@@@@@@@@@@@@@@@@@@@@@@@@
In our  analysis we  consider these three  pointer assignment
statements  as  equivalent.   After  this statement  all  the
existing values of $\p$ get  killed and it will point to same
heap object  (or NULL)  as pointed to  by $\q$. So  $\p$ will
have the same  paths and field connections as  those of $\q$.
The  kill effect of  this statement  is same  as that  of the
previous  cases.  The  generated boolean  functions  for heap
object $\p$ corresponding to DAG or Cycle attribute will also
be  same as  those  of  $\q$, with  all  occurrences of  $\q$
replaced  by $\p$\footnote{The  notation  $X[\q/\p]$ means  a
  copy  of  boolean  equation  $X$ with  all  occurrences  of
  \q\ replaced by \p.}.

\begin{eqnarray*}
  \KillC{p} = \InC{p} &\quad& \KillD{p} = \InD{p} \\ 
  \GenC{p}   = \InC{q}[q/p]
  &\quad& \GenD{p} = \InD{q}[q/p] 
\end{eqnarray*}


$\forall \s \in \heap, \s \not= \p, \forall f \in \fields$
\begin{eqnarray*}
  f_{\p\s} = f_{\q\s}  &\quad&  f_{\s\p} = f_{\s\q} \\
  P_F^{\dkill}[\p,\s]  =  P_F^{\din}[\p,\s] &\quad&
  P_F^{\dgen}[\p,\s]    =  P_F^{\din}[\q,\s]   \\ 
  P_F^{\dkill}[\s,\p]  =  P_F^{\din}[\s,\p] &\quad&
  P_F^{\dgen}[\s,\p]    =  P_F^{\din}[\s,\q]   \\
  P_F^{\dkill}[\p,\p]  =  P_F^{\din}[\p,\p] &\quad&
  P_F^{\dgen}[\p,\p]    =  P_F^{\din}[\q,\q]  \rule{0mm}{0pt}\\
\end{eqnarray*}


%@@@@@@@@@@@@@@@@@@@@@@@@@@@@
\subsubsection {\tt p$\rightarrow$f = NULL} 
%@@@@@@@@@@@@@@@@@@@@@@@@@@@@
This statement  breaks the  existing link $f$  emanating from
$\p$,  thus killing  equations of  \p\ that  are due  to link
$f$. The statement does not generate new equations.

\begin{eqnarray*}
  \KillC{p}  = \false, &\quad& \KillD{p} = \false \\
  \GenC{p} = \false, &\quad& \GenD{p} = \false  \\
\end{eqnarray*}
$\forall \q,\s \in \heap, \s \not= \p,$
\begin{eqnarray*}
  f_{\p\q} &=& \false \\
  P_F^{\dkill}[\p,\q] &=& \project{P_F^{\din}[\p,\q]}{f}
  \qquad
  P_F^{\dkill}[\s,\q]  = \emptyset
\end{eqnarray*}
%@@@@@@@@@@@@@@@@@@@@@@@@@@@@
\subsubsection {\tt p$\rightarrow$f = q} 
%@@@@@@@@@@@@@@@@@@@@@@@@@@@@
This statement  first breaks the  existing link $f$  and then
re-links the the  heap object pointed to by  $\p$ to the heap
object pointed to by $\q$.  The kill effects are exactly same
as described in the case  of {\tt p$\rightarrow$f = null}. We
only describe the generated values here.
  
The  fact that  the shape  of the  variable $\p$  becomes DAG
after  the statement  is  captured by  the boolean  functions
$\GenD{p}$.   The  function  simply  say that  variable  $\p$
reaches  a  DAG  because   there  are  more  than  one  paths
($\num{I_F[p,q]} > 1$) from $\p$ to $\q$. It also keeps track
of the  path $f_{pq}$ in this case.   The function $\GenC{q}$
(or  $\GenC{p}$) captures  the fact  that cycle  on  $\q$ (or
$\p$) consists  of field $f$  from $\p$ to  $\q$ ($f_{\p\q}$)
and some path from  $\q$ to $\p$ ($\num{\DFM{q}{p}} \geq 1$).
The  function $\GenC{p}$($\GenD{p}$)  also  capture the  fact
that cycle  (DAG) on $\p$ can  be due to  the link $f_{\p\q}$
reaching an already existing  cycle (DAG) on $\q$.  These are
summarized as follows:

\begin{eqnarray*}
  \GenC{p} &=& (f_{\p\q} \wedge \InC{q}) \vee (f_{\p\q} \wedge (\num{\DFM{\q}{\p}} \geq 1)) \\
  \GenD{p} &=& (f_{\p\q} \wedge \InD{q}) \vee (f_{\p\q} \wedge (\num{\IFM{\p}{\q}} > 1) \\
  \GenC{q} &=& f_{\p\q} \wedge (\num{\DFM{\q}{\p}} \geq 1) \\
  \GenD{q} &=& \false\\ 
  f_{\p\q} &=& \true
\end{eqnarray*}

For  nodes  $\s \in  \heap$  other  than  $\p$ or  $\q$,  the
function  $\GenC{\s}$ captures  the fact  that cycle  on $\s$
consists  of   some  path  from   $\s$  to  $\p$   (or  $\q$)
i.e.\  $\num{\DFM{\s}{\p}}  \geq  1$ (or  $\num{\DFM{\s}{\q}}
\geq 1$) and the fact that a Cycle on $\p$ (or $\q$) has just
created due to the  statement. Again the function $\GenD{\s}$
simply say that variable $\s$ reaches a DAG because there are
more  than one  way  of interference  between  $\s$ and  $\q$
i.e.\ $\num{\IFM{\s}{\q}}  > 1$.  It also keeps  track of the
paths $f_{pq}$ and $\DFM{\s}{\p}$ in this case.

\begin{eqnarray*}
  \GenC{\s} &=&  ((\num{\DFM{\s}{\p}} \geq 1) \wedge f_{\p\q} \wedge \InC{q}) \\
  && \vee\ ((\num{\DFM{\s}{\p}} \geq 1) \wedge f_{\p\q} \wedge (\num{\DFM{\q}{\p}} \geq 1)) \\ 
  && \vee\ ((\num{\DFM{\s}{\q}} \geq 1) \wedge f_{\p\q} \wedge (\num{\DFM{\q}{\p}} \geq 1)), \\
  && \qquad \qquad  \forall \s \in \heap, \s \not= \p, \s \not=\q \\
  \GenD{\s}   &=& (\num{\DFM{\s}{\p}} \geq 1) \wedge  f_{\p\q} \wedge (\num{\IFM{\s}{\q}} > 1), \\
  && \qquad \qquad \forall \s \in \heap, \s \not= \p, \s \not=\q 
\end{eqnarray*}

After the  statement, all the  nodes that have  paths towards
$\p$ (including  $\p$) will have  path towards all  the nodes
reachable  from $\q$  (including  $\q$).  Thus,

For  $\myr, \s  \in
\heap$:
\begin{eqnarray*}
  P_F^{\dgen}[\myr, \s] &=& \num{P_F^{\din}[\q, \s]} \star
  P_F^{\din}[\myr, \p],\ \s \not= \p,\ \myr \not\in
  \{\p, \q\}\\ 
  P_F^{\dgen}[\myr, \p] &=& \num{P_F^{\din}[\q,\p]} \star
  P_F^{\din}[\myr, \p], \ \myr \not= \p\\
 P_F^{\dgen}[\p, \myr] &=& \num{P_F^{\din}[\q, \myr]} \star
 (\remOne{P_F^{\din}[\p, \p]}{\epsilonset} \cup \{f^{\indrct
   1}\}),\\ 
   && \qquad \qquad \myr \not= \q \\ 
  P_F^{\dgen}[\p, \q] &=&
  \{\fieldD{f}{}\}\ \cup\ (\num{P_F^{\din}[\q, \q] -
    \epsilonset} \star
  \{\fieldI{f}{}{1}\})\ \cup\ \\
  && (\num{P_F^{\din}[\q, \q]} \star ( \remOne{P_F^{\din}[\p, \p]}{\{\project{P_F^{\din}[\p,\p]}{f} \cup \epsilonset \}})) \\ 
  P_F^{\dgen}[\q, \q] &=& 1 \star P_F^{\din}[\q, \p] \\
  P_F^{\dgen}[\q, \myr] &=& \num{P_F^{\din}[\q, \myr]} \star
  P_F^{\din}[\q, \p],\ \myr \not\in \{\p,
  \q\} \\ 
\end{eqnarray*}

%@@@@@@@@@@@@@@@@@@@@@@@@@@@@
\subsubsection {\tt p = q$\rightarrow$f} 
%@@@@@@@@@@@@@@@@@@@@@@@@@@@@
The values killed by  the statement are  same as  that in
case  of {\tt  p  =  NULL}.  The  values  created by  this
statement  are heavily approximated  by our  analysis.  After
this  statement, $\p$  points  to the  heap  object which  is
accessible  from  pointer $\q$  through  $f$  link. The  only
inference  we can  draw is  that  \p\ is  reachable from  any
pointer  \myr\  such  that  \myr\ reaches  $\q\rightarrow  f$
before  the  assignment.   

As \p\ could potentially point to a cycle (DAG) reachable from
\q, we set:
\[\GenC{p} = \InC{q} \qquad \GenD{p} = \InD{q}\]
Note that shape of no other pointer variable gets affected by
this statement.

We  record the  fact that  \q\ reaches  \p\ through  the path
$f$. Also, any  object reachable from \q\ using  field $f$ is
marked as reachable from \p\ through any possible field.
\begin{eqnarray*}
  f_{\q\p} &=& \true \\ %\qquad
  {h_{\p\myr}} &=&
  \num{\project{P_F^{\din}[\q,\myr]}{f}} \geq 1\quad
  \forall h \in \fields, \forall \myr \in \heap 
\end{eqnarray*}

The equations to compute the generated values for  $P_F$ 
can be divided into three components. We explain each of the
component, and give the equations.

As  a side-effect  of the  statement,  any node  \s\ that  is
reachable from  \q\ through  field $f$ before  the statement,
becomes reachable  from \p. However, this  information is not
sufficient to determine the  path from \p\ to \s.  Therefore,
we conservatively assume that  any path starting from \p\ can
potentially  reach \s. This  is achieved  in the  analysis by
using a universal path  set \upath\ for $P_F[\p,\s]$. The set
\upath\ is defined as:
\[ \upath \quad=\quad \epsilonset \cup \bigcup_{f\in\fields}
\{f^{\drct},  f^{\indrct\infty}\} \] 
%
Because it is also not possible to determine if there exist a
path from $\p$  to itself, we safely conclude  a self loop on
$p$   in  case  a   cycle  is   reachable  from   \q\  (i.e.,
\q.\shape\ evaluates  to \Cycle).  These  observations result
in the following equations:
\[ \forall \s \in \heap, \s \not=\p \wedge
  \project{P_F^{\din}[\q, \s]}{f} \not= \emptyset \]
\begin{eqnarray*}
  P_1[\p, \s] &=&  \left\{\begin{array}{@{}ll}
    \epsilonset &\quad(\project{P_F^{\din}[\q, \s]}{f} =
    \fieldD{f}{}) \\ 
    & \quad \wedge\ \q.\shape \mbox{ evaluates to } \Tree
    \mbox{ or } \Dag \\
    \upath & \quad\mbox{Otherwise}
  \end{array}\right.
  \\
  P_1[\p, \p] &=& \left\{\begin{array}{@{}ll}
    \upath&\quad \q.\shape \mbox{ evaluates to } \Cycle \\
    \epsilonset &\quad \mbox{Otherwise}
  \end{array}\right.
\end{eqnarray*}


Any node \s\ (excluding \p\ and \q), that has paths to \q\ before the
statement,    will   have    paths   to    \p\    after   the
statement. However, we can not know the exact number of paths
\s\ to  \p, and  therefore use upper  limit ($\infty$)  as an
approximation:
\begin{eqnarray*}
  P_2[\s, \p] &=& \infty \star P_F^{\din}[\s, \q] \quad
  \forall \s \in \heap,\ \s \notin \{\p, \q\} 
\end{eqnarray*}
If $\p\not=\q$, then we record the path from \q\ to \p\ as:
\begin{eqnarray*}
  P_2[\q, \p] &=& \left\{\begin{array}{ll}
  \{ f^{\drct} \}  &\quad \q.\shape \mbox{ evaluates to } \Tree \\
  \upath &\quad \mbox{Otherwise}
  \end{array}\right.
\end{eqnarray*}


The  third  category of  nodes  to  consider  are those  that
interfere with  the node corresponding  to $\q\rightarrow f$,
without going through \q.  Any such node \s\ will have paths to
\p\ after the statement. Thus, we have:
\[ \forall \myr,\s \in \heap,\ \myr \not\in \{\p,\q\} \]
\begin{eqnarray*}
  P_3[\s, \p] &=& \bigcup_{\myr}\{ \alpha \ \vert 
  \ f^{\drct}  \in P_F^{\din}[\q, \myr] 
  \wedge \alpha \in  P_F^{\din}[\s, \myr] \ominus P_F^{\din}[\s, \q] \}
\end{eqnarray*}

Finally, we compute the entries generated for $P_F$ as:
\begin{eqnarray*}
P_F^{\dgen}[\myr, \s] &= & P_1[\myr, \s] \cup P_2[\myr, \s]
\cup P_3[\myr, \s] \quad \forall\myr, \s \in \heap  \\ 
\end{eqnarray*}
