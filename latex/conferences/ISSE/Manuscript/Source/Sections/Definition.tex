We view the heap structure at a program point as a directed
graph, the nodes of which represent the allocated objects and
the edges represent the connectivity through pointer fields.
Pictorially, inside a node we show all the relevant pointer
variables that can point to the heap object corresponding to
that node. The edges are labeled by the name of the
corresponding pointer field. In this paper, we only label
nodes and edges that are relevant to the discussion, to avoid
clutter.

Let $\heap$ denotes the set  of all heap directed pointers at
a particular  program point and $\fields$ denotes  the set of
all  pointer  fields  at   that  program  point.   Given  two
heap-directed pointers $\p$, $\q$  $\in$ $\heap$, a path from
$\p$ to $\q$  is the sequence of pointer  fields that need to
be traversed  in the  heap to reach  from $\p$ to  $\q$.  The
length of a  path is defined as the  number of pointer fields
in the path.  As the path length between two heap objects may
be  unbounded, we keep  track of  only the  first field  of a
path\footnote{The decision to use  only first field is guided
  by  the  fact  that  in many  intermediate  languages  (for
  example, GIMPLE for GCC), a  statement is allowed to use at
  most  one field,  i.e.  {\tt  p$\rightarrow$f =  \ldots} or
  {\tt \ldots = p$\rightarrow$f}.   Therefore, a long path is
  broken into  several small paths.  While it  is possible to
  use  prefixes of  any  fixed length  by reconstructing  the
  path,  the  process  is   complex  and  does  not  add  any
  fundamental  value  to   our  analysis.}.   To  distinguish
between a  path of  length one (direct  path) from a  path of
length  greater than one  (indirect path)  that start  at the
same field, we use the  superscript $\drct$ for a direct path
and $\indrct$ for an indirect path. In pictures, we use solid
edges for direct paths, and dotted edges for indirect paths.

It  is  also possible  to  have  multiple  paths between  two
pointers starting at a given  field $f$. There can be at most
one  such  direct  path  $f^\drct$. However,  the  number  of
indirect paths  starting at a field $f$,  $f^\indrct$, may be
unbounded.  We  include  the  count for  the  indirect  paths
between two pointer  variables in the path set.  To bound the
size of the set, we put  a limit $k$ on number of repetitions
of  a particular  field. If  the number  goes beyond  $k$, we
treat the number of paths with that field as $\infty$. The
following example illustrates these concepts.

\begin{example} {
\begin{figure}[t]
\centering
\newcommand{\smf}{$f$}%%\scriptstyle f$}
\newcommand{\smh}{$h$}%%\scriptstyle h$}
\newcommand{\smg}{$g$}%%\scriptstyle g$}
\begin{tabular}{@{}c|c@{}}
  {\tt
    \begin{tabular}[b]{l}
      {\em \scriptsize S1:} q = p; \\
      {\em \scriptsize S2:} {\bf while}(...) \{ \\
      {\em \scriptsize S3:} \ \ \ \ q$\rightarrow$g = s; \\
      {\em \scriptsize S4:} \ \ \ \ q = q$\rightarrow$f; \\
      {\em \scriptsize S5:} \} \\
    \end{tabular}
  } &
  \scalebox{0.8}{ \psset{unit=1mm}
    \begin{pspicture}(0,0)(50,30)
      %%\psframe(0,0)(50,30)
      \putnode{q0}{origin}{3}{3}{}
      \putnode{q1}{q0}{0}{0}{\pscirclebox{\mbox{\p}}}
      \putnode{q2}{q1}{10}{0}{\pscirclebox[framesep=2.5]{\mbox{}}}
      \putnode{q3}{q2}{10}{0}{\mbox{\ldots}}
      \putnode{q4}{q3}{10}{0}{\pscirclebox{\mbox{\q}}}
      \putnode{q5}{q4}{10}{0}{\pscirclebox[framesep=2.5]{\mbox{}}}
      
      \ncline{->}{q1}{q2}
      \Bput[0.2]{\smf}
      \ncline{->}{q2}{q3}
      \Bput[0.2]{\smf}
      \ncline{->}{q3}{q4}
      \Bput[0.2]{\smf}
      \ncline{->}{q4}{q5}
      \Bput[0.2]{\smf}
      
      \putnode{s0}{q4}{15}{15}{\pscirclebox{\mbox{\s}}}
      \nccurve[angleA=90,angleB=145,ncurv=1]{->}{q1}{s0}
      \Aput[.1]{\smg}
      
      \putnode{t0}{q2}{0}{15}{\pscirclebox[framesep=2.5]{\mbox{}}}
      \nccurve[linestyle=dotted, dotsep=.5, angleA=-55,angleB=-165,
        ncurv=.2,nodesepA=-.3]{->}{t0}{s0}
      
      \ncline[linestyle=dotted, dotsep=.5]{->}{t0}{s0}
      \nccurve[linestyle=dotted, dotsep=.5,angleA=55,angleB=165,
        ncurv=.3,nodesepA=-.3]{->}{t0}{s0}
      
      
      \ncline[nodesepA=-.5,nodesepB=-.5]{->}{q1}{t0}
      \Aput[.1]{\smh}
      
      \ncline{->}{q2}{s0}
      \Bput[0.2]{\smg}
      \ncline[nodesepA=-.75,nodesepB=-.75]{->}{q4}{s0}
      \Bput[0.2]{\smg}
  \end{pspicture}} \\
  \scalebox{0.80}{ (a) } &
  \scalebox{0.80}{ (b) }
\end{tabular}
\caption{Paths in a heap graph.  Graph in (b) is a possible
  heap graph for code in (a). Solid edges are the direct
  paths, dotted edges are the indirect paths.
  \label{fig:pathsAndSets}}
\end{figure}

Figure~\ref{fig:pathsAndSets}(a)  shows a  code  fragment and
Fig.~\ref{fig:pathsAndSets}(b) shows a possible heap graph at
a program point after line  {\tt S5}. In any execution, there
is one path  between $\p$ and $\q$, starting  with field $f$,
whose  length  is  statically  unknown. This  information  is
stored      by     our      analysis      as     the      set
$\{\fieldI{f}{}{1}\}$. Further, there are unbounded number of
paths  between  $\p$  and   $\s$,  all  starting  with  field
$f$.  There is also  a direct  path from  $\p$ to  $\s$ using
field $g$, and  3 paths starting with field  $h$ between $\p$
and $\s$. Assuming the limit $k \geq 3$, this information can
be  represented by  the set  $\{  g^\drct, f^{\indrct\infty},
h^{\indrct 3}  \}$. On the other  hand, if $k <  3$, then the
set     would    be    $\{     g^\drct,    f^{\indrct\infty},
h^{\indrct\infty} \}$.  } \hfill\psframebox{}
\end{example}

\newcommand{\anysup}{\ensuremath{\ast}}  For brevity,  we use
$f^{\anysup}$  for   the  cases  when  we  do   not  want  to
distinguish between  direct or indirect path  starting at the
field $f$. We now define the field sensitive path matrix used
by our analysis.

%FIELD SENSITIVE PATH MATRIX 
\begin{definition}
\label{DFM_matrix}
Field  sensitive Path matrix  $P_F$ is  a matrix  that stores
information about paths between  two pointer variables in the
following form: Given $\p, \q \in \heap, f \in \fields$:
\begin{eqnarray*}
  \epsilon & \in \DFM{p}{p}& \mbox{where $\epsilon$
    denotes the empty path.} \\
  f^\drct  &\in  \DFM{p}{q} & \mbox{if there is a direct
    path $f$ from $\p$ to $\q$.}\\
  f^{\indrct m} & \in  \DFM{p}{q} & 
  \mbox{\begin{tabular}[t]{@{}p{55mm}}if there are $m$ indirect
      paths starting with field $f$ from $\p$ to $\q$ and $m
      \leq k.$
    \end{tabular}
  } \\
  f^{\indrct\infty} & \in  \DFM{p}{q} &
  \mbox{\begin{tabular}[t]{@{}p{55mm}}if there are $m$ indirect
      paths starting with field $f$ from $\p$ to $\q$ and $m >
      k.$
  \end{tabular}}  
\end{eqnarray*}
\end{definition}

Let \nat\ denote the set of natural numbers. We define the
following partial order for approximate paths used by our
analysis. For $ f \in \fields,\ m,n \in \nat,\ n \leq m$:
$$
\epsilon \sqsubseteq \epsilon, \quad 
f^\drct \sqsubseteq  f^\drct,  \quad
f^{\indrct\infty}  \sqsubseteq  f^{\indrct\infty}, \quad
f^{\indrct m} \sqsubseteq f^{\indrct\infty}, \quad
f^{\indrct n} \sqsubseteq f^{\indrct m}. 
$$
The partial order is extended to set of paths $S_{P_1},
S_{P_2}$ as\footnote{Note that for our analysis, for a given
  field $f$, these sets contain at most one entry of type
  $f^\drct$ and at most one entry of type $f^\indrct$}:
\begin{eqnarray*}
  S_{P_1} \sqsubseteq S_{P_2} &\Leftrightarrow& \forall \alpha \in
  S_{P_1}, \exists \beta \in S_{P_2}\ s.t. \alpha \sqsubseteq \beta
\end{eqnarray*}

Two pointers  $\p,\q \in  \heap$ are said  to interfere  at a
program point if  there exists $\s \in \heap$  such that both
$\p$ and  $\q$ have paths  reaching $\s$ at that  point. Note
that $\s$ could  be $\p$ (or $\q$) itself,  in which case the
path  from  $\p$  (from   $\q$)  is  $\epsilon$.   Thus,  the
interference  relation between $\p$  and $\q$,  $\IFM{p}{q}$
can be defined in terms of path matrix ($P_F$) as:
\begin{eqnarray}
\IFM{p}{q} &\Rightarrow& \exists s. \DFM{p}{s} \wedge \DFM{q}{s}
\label {eq:rel-df-if}
\end{eqnarray}

As   we   will   see   later  in   the   data-flow   equations
(Sec.~\ref{sec:Analysis}), we are only interested in the {\em
  maximum count}  of pair of  paths that are  interfering for
two pointers at a given heap node.  We use $\num{\IFM{p}{q}}$
to denote this count for interfering paths for nodes $\p$ and
$\q$.   Our  analysis  computes over-approximations  for  the
$P_F$    matrix   at   each    program   point,    and   uses
Equation~(\ref{eq:rel-df-if}) to compute an over-approximation
for  $\num{I_F}$. This  can  result in  a conservative  shape
(Cycle or DAG instead of a  Tree, Cycle instead of a DAG) for
a pointer, which is a safe inference.

\begin{example}{\rm
\begin{figure}[!t]
  \centering
  \begin{tabular}{@{}c|c@{}}
    \psset{unit=1.2mm}
    \begin{pspicture}(0,0)(18,20)
      %  \psframe(0,0)(15,20)
      \putnode{q0}{origin}{3}{9}{\pscirclebox{\q}}
      \putnode{p0}{q0}{5}{7}{\pscirclebox{\p}}
      \putnode{r0}{p0}{0}{-14}{\pscirclebox{\myr}}
      \putnode{s0}{q0}{10}{0}{\pscirclebox{\s}}
      
      \ncline[nodesep=-.3]{->}{p0}{q0}
      \bput[0.2](0.2){$\scriptstyle f$}
      \ncline{->}{q0}{s0}
      \aput[0.2](0.2){$\scriptstyle g$}
      \ncline[nodesep=-.5]{->}{s0}{r0}
      \aput[0.2](0.2){$\scriptstyle h$}
      \nccurve[angleA=45,angleB=0,linestyle=dotted,dotsep=0.2,ncurv=1,nodesepA=-.7]{->}{p0}{s0}
      \aput[0.2](0.2){$\scriptstyle g$}
      \nccurve[angleA=180,angleB=225,linestyle=dotted,dotsep=0.2,ncurv=1,nodesepB=-.5]{->}{q0}{r0}
      \bput[0.2](0.2){$\scriptstyle f$}
    \end{pspicture} &
    \scalebox{0.8}{
      \renewcommand{\arraystretch}{1.2}
      \begin{tabular}[b]{|c|c|c|c|c|}
        \hline
        $P_F$& \p & \q &  \s &  \myr \\ \hline \hline 
        \p   & $\{\epsilon\}$ & $\{\fieldD{f}{}\}$ & $\{\fieldI{f}{}{1}, \fieldI{g}{}{1}\}$ & $\{\fieldI{f}{}{2}, \fieldI{g}{}{1}\}$  \\ \hline 
        \q   & $\emptyset$ & $\{\epsilon\}$ & $\{\fieldD{g}{}\}$ & $\{\fieldI{g}{}{1}, \fieldI{f}{}{1}\}$ \\ \hline
        \s   & $\emptyset$ & $\emptyset$ & $\{\epsilon\}$ & $\{\fieldD{h}{}\}$ \\ \hline
        \myr & $\emptyset$ & $\emptyset$ & $\emptyset$ & $\{\epsilon\}$ \\ \hline
    \end{tabular} } \\
    \scalebox{0.80}{ (a) } &
    \scalebox{0.80}{ (b) }  \\
  \end{tabular}
  \caption{A heap graph (a) and its field sensitive path
    matrix (b).\label{fig:DFM_IFM}}
\end{figure}


Figure~\ref{fig:DFM_IFM} shows a heap graph and the
corresponding path matrix as computed by our
analysis.} \hfill\psframebox{}
\end{example}

As mentioned  earlier, for each variable $\p  \in \heap$, our
analysis  uses attributes  $\p_{\subD}$  and $\p_{\subC}$  to
store boolean functions telling  whether $\p$ can reach a DAG
or  cycle respectively  in the  heap.  The  boolean functions
consist  of the  values from  matrices $P_F$,  and  the field
connectivity  information. For  $f  \in \fields,  \p, \q  \in
\heap$, field  connectivity is captured  by boolean variables
of the  form $f_{pq}$, which is  true when $f$  field of $\p$
points directly to $\q$.  The  shape of \p, \p.\shape, can be
obtained  by  evaluating  the  functions for  the  attributes
$\p_{\subC}$      and       $\p_{\subD}$,      and      using
Table~\ref{tbl:det_shape}.  It  is important to  note that we
are only interested in  those boolean functions that evaluate
to true for attributes $\p_{\subC}$ and $\p_{\subD}$. This is
because  we want  to capture  the transitions  of  shape from
\Cycle\ to  \Dag, from \Cycle\  to \Tree, and from  \Dag\ to
\Tree.   If the function  for $\p_{\subC}$  ($\p_{\subD}$) is
already   false,   then  such   transitions   can  not   take
place. Therefore, we only store boolean functions if they can
potentially  evaluate  to  true  value.  Further,  for  these
functions,  we  store  the  functions  themselves  as  Binary
Decision  Diagrams (BDDs)~\cite{bdd},  and not  the evaluated
values.  The  evaluation of  functions takes place  only when
the shape  of the corresponding  pointer is required  by some
client of  the analysis  or by the  analysis itself.  We call
this {\em lazy} evaluation of boolean functions.
\begin{table}[t]
\caption{Determining shape from boolean
  attributes\label{tbl:det_shape}}
\begin{center}
  \begin{tabular}{|c|c|c|}
    \hline
    $\p_{\subC}$ & $\p_{\subD}$ & $\p.\shape$ \\ 
    \hline
    \true  & Don't Care  & Cycle        \\ 
    \false  & \true          & DAG    \\ 
    \false  & \false          & Tree   \\ 
    \hline
  \end{tabular}
\end{center}
\end{table}

We use the following operations in our analysis. Let $S$
denote the set of approximate paths between two nodes, $P$
denote a set of pair of paths, and $k \in \nat$ denotes the
limit on maximum indirect paths stored for a given
field. Then,
\begin{itemize}\is
\item      Projection:     For      $f      \in     \fields$,
  \project{S}{f}\ extracts the paths starting at field $f$.
  $$\project{S}{f} \equiv S \cap \{f^{\drct}, f^{\indrct 1}, \ldots,
  f^{\indrct k}, f^{\indrct \infty}\}$$

\item Counting: The  count on the number of  paths is defined
  as :
  $$\num{\epsilon} = 1, \qquad   \num{f^{\drct}} = 1,
  \qquad   \num{f^{\indrct\infty}} =     \infty $$
  $$\num{f^{\indrct j}} =  j \mbox{ for } j \in \nat$$
  $$\num{S} =   \sum_{\alpha \in S}\num{\alpha}$$

\item  Path  removal,  intersection  and union  over  set  of
  approximate paths: For singleton sets of paths $\{\alpha\}$
  and           $\{\beta\}$,           path           removal
  (\remOne{\{\alpha\}}{\{\beta\}}), intersection ($\{\alpha\}
  \cap  \{\beta\}$)  and  union($\{\alpha\} \cup  \{\beta\}$)
  operations      are      defined      as      given      in
  Table~\ref{tab:path-ops}. These definitions can be extended
  to set of paths in  a natural way. For example, for general
  sets of  paths, $S_1$ and $S_2$, the  definition of removal
  can be extended as:
  \begin{eqnarray*}
    \remOne{S_1}{S_2} &=&
    \bigcap_{\beta \in {S_2}} \bigcup_{\alpha \in {S_1}}
        (\remOne{\{\alpha\}}{\{\beta\}})
  \end{eqnarray*}

\begin{table}[t]
\caption{Path operations\label{tab:path-ops}}
\begin{center}
\scalebox{1}{
  \renewcommand{\arraystretch}{1.1}
  \begin{tabular}{@{}c@{}}
    $i, j  \in \nat$,$\quad m = {\tt max}(i-j, 0)$,  $\quad n = {\tt
      min}(i, j)$, \\
    $\tau = \left\{\begin{array}{lcl}
    i+j    && \mbox { if } i+j \leq k \\
    \infty && \mbox{ Otherwise} 
    \end{array}\right.$. \\ 
    $\eta = \left\{\begin{array}{lcl}
    i*j    && \mbox { if } i*j \leq k \\
    \infty && \mbox{ Otherwise} 
    \end{array}\right.$. \\  \\
    (a) Path removal \\
    \begin{tabular}{@{}c@{}c@{}|c|c|c|c|c}
      \hline
      \remOne{}{}&$\{\beta\} $& $\{\epsilon\}$ &
      $\{f^{\drct}\}$ & $\{f^{\indrct j}\}$&
      $\{f^{\indrct\infty}\}$ & $g^{\anysup}$ \\
      $\{\alpha\}$ && & & & & \\ \hline %\hline
      $\{\epsilon\}$ && $\emptyset$  & $\{\epsilon\}$ &
      $\{\epsilon\}$ & $\{\epsilon\}$ & $\{\epsilon\}$ \\ %\hline
      $\{f^{\drct}\}$ && $\{f^{\drct}\}$ & $\emptyset$ &
      $\{f^{\drct}\}$ & $\{f^{\drct}\}$ & $\{f^{\drct}\}$ \\ %\hline
      $\{f^{\indrct i}\}$ && $\{f^{\indrct i}\}$ & $\emptyset$
      & $\{f^{\indrct m}\}$& $\emptyset$ & $\{f^{\indrct i}\}$ \\ %\hline
      $\{f^{\indrct\infty}\}$ && $\{f^{\indrct\infty}\}$
      &$\emptyset$ & $\{f^{\indrct \infty}\}$& $\{f^{\indrct
        \infty}\}$ & $\{f^{\indrct\infty}\}$ \\ \hline
    \end{tabular} \\ \\
    (b) Intersection \\ 
    \begin{tabular}{@{}c@{}c@{}|c|c|c|c|c@{}}
      \hline
      $\cap$&$\{\beta\} $& $\{\epsilon\}$ & $\{f^{\drct}\}$ &
      $\{f^{\indrct j}\}$& $\{f^{\indrct\infty}\}$ & $g^{\anysup}$\\ 
      $\{\alpha\}$ && & & & & \\ \hline %\hline
      $\{\epsilon\}$ && $\epsilon$  & $\emptyset$ & $\emptyset$ & $\emptyset$ & $\emptyset$ \\ %\hline
      $\{f^{\drct}\}$ && $\emptyset$ & $\{f^{\drct}\}$ & $\emptyset$ & $\emptyset$ & $\emptyset$ \\ %\hline
      $\{f^{\indrct i}\}$ && $\emptyset$ & $\emptyset$ & $\{f^{\indrct n}\}$& $\{f^{\indrct i}\}$ & $\emptyset$ \\ %\hline
      $\{f^{\indrct\infty}\}$ && $\emptyset$ &$\emptyset$ & $\{f^{\indrct j}\}$& $\{f^{\indrct \infty}\}$ & $\emptyset$ \\ \hline
    \end{tabular} \\ \\
    (c) Union \\
    \begin{tabular}{@{}c@{}c@{}|@{}c@{}|@{}c@{}|@{}c@{}|@{}c@{}|@{}c@{}}
      \hline
      $\cup$&$\{\beta\} $& $\{\epsilon\}$ & $\{f^{\drct}\}$ &
      $\{f^{\indrct j}\}$& $\{f^{\indrct\infty}\}$ & $g^{\anysup}$ \\
      $\{\alpha\}$ && & & & &  
      \\ \hline %\hline
      $\{\epsilon\}$ && $\{\epsilon\}$  & $\{\epsilon, \fieldD{f}{}\}$ & $\{\epsilon, \fieldI{f}{}{j} \}$ & $\{\epsilon, \fieldI{f}{}{\infty}\}$ & $\{\epsilon, g^{\anysup}\}$  \\ %\hline
      $\{f^{\drct}\}$ && $\{\fieldD{f}{}, \epsilon\}$ & $\{\fieldD{f}{}\}$ & $\{\fieldD{f}{}, \fieldI{f}{}{j}\}$ & $\{\fieldD{f}{}, \fieldI{f}{}{\infty}\}$ & $\{\fieldD{f}{}, g^{\anysup}\}$  \\ %\hline
      $\{f^{\indrct i}\}$ && $\{\fieldI{f}{}{i}, \epsilon\}$ & $\{\fieldI{f}{}{i}, \fieldD{f}{}\}$ & $\{\fieldI{f}{}{\tau}\}$& $\{\fieldI{f}{}{\infty}\}$ & $\{\fieldI{f}{}{i}, g^{\anysup}\}$  \\ %\hline
      $\{f^{\indrct\infty}\}$ && $\{\fieldI{f}{}{\infty},
      \epsilon\}$ & $\{\fieldI{f}{}{\infty}, \fieldD{f}{}\}$ &
      $\{\fieldI{f}{}{\infty}\}$& $\{\fieldI{f}{}{\infty}\}$ &
      $\{\fieldI{f}{}{\infty}, g^{\anysup}\}$  \\ \hline
    \end{tabular} \\ \\
    (d) Multiplication by a Scalar \\
    \begin{tabular}{l|c|c|c|c}
      \hline
      $\star\qquad \alpha$ & $\epsilon$ & $f^{\drct}$      &
      ${f^{\indrct j}}$ & ${f^{\indrct \infty}}$ \\ 
      $i$ &&&& \\ \hline 
      $i$ & $\epsilon$ & ${f^{\indrct i}}$ & $f^{\indrct \eta}$
      & ${f^{\indrct \infty}}$ \\ 
      $\infty$ & $\epsilon$ &${f^{\indrct \infty}}$ &
      ${f^{\indrct \infty}}$& ${f^{\indrct \infty}}$\\ \hline 
    \end{tabular}
  \end{tabular}}
\end{center}
\end{table}
	

\item Multiplication by a scalar($\star$): Let $i, j
  \in \nat, i \leq k, j \leq k$. Then, for a path $\alpha$,
  the multiplication by a scalar $i$, $i\star\alpha$ is
  defined in Table~\ref{tab:path-ops}(d). The operation is
  extended to set of paths as:
\begin{eqnarray*}
  i \star {S} &=&   \left\{ \begin{array}{lcl}
    \emptyset &\quad& i = 0 \\
    \{ i\star \alpha \ \vert\ \alpha \in S\} &\quad& i\in
    \nat \cup \{\infty\}, i \not= 0
  \end{array} \right.
\end{eqnarray*}


\end{itemize}
