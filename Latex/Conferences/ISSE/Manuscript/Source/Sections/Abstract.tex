We  present a static  shape analysis  technique to  infer the
shapes of  the heap  structures created by  a program  at run
time. Our technique is field  sensitive in that it uses field
information to  compute the shapes.   The shapes of  the heap
structures  are computed  using two  components:  (a) Boolean
functions  that  capture the  shape  transitions  due to  the
update  of a  field  in  a structure,  and  (b) through  path
matrices that store  approximate path information between two
pointer  variables.  We classify  the shapes  as one  of {\em
  Tree, Directed  Acyclic Graph (DAG)} and  {\em Cycle}.  The
novelty  of  our  approach  lies  in the  way  we  use  field
information  to  remember  the   fields  that  cause  a  heap
structure to  have a particular  shape (Tree, DAG  or Cycle).
This  allows us  to easily  identify the  field  updates that
cause shape transitions from Cycle to DAG, from Cycle to Tree
and from DAG  to Tree.  This makes our  analysis more precise
as compared to earlier  shape analyses that ignore the fields
participating in the formation of a shape.

We implemented our analysis in GCC as a dynamic plug-in as an
interprocedural data-flow  analysis and evaluated  it on some
standard  benchmarks  against   a  field  insensitive,  shape
analysis technique  as a baseline  approach.  We are  able to
achieve  significant precision  as compared  to  the baseline
analysis  (at the  cost of  increase in  analysis  time).  In
particular, we  are able to  infer more precise shapes  for 4
out 7  {\em Olden} benchmarks,  and never detect  more cycles
than the baseline  analysis.  We further suggest enhancements
to  improve   the  precision  of  our   analysis  under  some
constraints and to  improve the analysis time at  the cost of
precision.
