%%%% Plan we agreed on:

%% Lecture 1-3 (prepare 10 questions, so we can select 5 questions)
%% SCM, Jenkins, CodingTracker -- Amarin (MP0, MP2)

%% Lecture 4 (prepare 10 questions, so we can select 5 questions)
%% XP (including User Stories and Planning Game) -- Xiaoyu (MP4)

%% Lecture 5-7 (prepare 10 questions, so we can select 5 questions)
%% Testing (1-3) -- Sandeep (MP1)

%% Lecture 8-10 (prepare 10 questions, so we can select 5 questions)
%% Reverse Enginering, Refactoring, Code Smells -- Owolabi (MP3)

%% Lecture 11-12
%% Metrics, OO Metrics -- Darko

\documentclass[11pt]{exam}
\usepackage{listings}
\lstset{numbers=left, numberstyle=\tiny, tabsize=2, numbersep=5pt, captionpos=b,
basicstyle=\footnotesize, showstringspaces=false,
emphstyle=\color{LightMagenta}\bfseries, language=Java, frame=single}

\usepackage{color} 
\definecolor{LightMagenta}{cmyk}{0.1,0.8,0,0.1}

\usepackage[pdftex]{hyperref}
\hypersetup{
    colorlinks,%
    citecolor=black,%
    filecolor=black,%
    linkcolor=black,%
    urlcolor=blue
}

% This is now the recommended way for checking for PDFLaTeX:
\usepackage{ifpdf}

\ifpdf 
\usepackage{subfigure} 
\usepackage[pdftex]{graphicx} \else 
\usepackage{graphicx} \fi

%  Update these values for running headers
\headrule \lhead[CS 427]{CS427 (Continued)} \chead{\textbf{Midterm Exam}}
\rhead{netid: \hbox to 0.75 in{}}

\footrule \lfoot{} \cfoot{Page \thepage\ of \numpages} \rfoot{\iflastpage{End of
exam.}{Please go on to the next page\ldots}}

% Add the total points in the exam
\addpoints \boxedpoints \pointsinmargin

% Change the layout of the solution
\renewcommand{\solutiontitle}{
\noindent\textbf{Solution:}\par
\noindent}

\newcommand{\Comment}[1]{}
\newcommand{\Ch}[1]{Chapter #1 of Code Complete~2}
\newcommand{\Book}[1]{Chapter #1 of the book \emph{Object 
Oriented Reengineering Patterns (OORP)}}

%\widowpenalties 1 10000
%\raggedbottom

\begin{document}

\lstset{language=Java,numbers=left,tabsize=4} \begin{coverpages}
\coverheader{}{\Large \textbf{University of Illinois at
Urbana-Champaign\\Department of Computer Science}}{} \begin{center}
{\Large\textbf{Midterm Exam}}

 \vspace{0.1 in}

CS 427: Software Engineering I\\ Fall 2013

 \vspace{0.1 in}

October 10, 2013

 \vspace{0.1 in}

TIME LIMIT = 1 hour 15 minutes \\ COVER PAGE + \numpages\ PAGES \end{center}

 \noindent Write your name and netid neatly in the space provided below; \textbf{write
your netid} in the upper right corner of \textbf{every page}.

 \vspace{0.1 in}

 \hbox to 3 in{Name:\enspace\hrulefill} \hbox to 3 in{Netid:\enspace\hrulefill}

 \vspace{0.1 in}

 \noindent \emph{This is a closed book, closed notes examination. You may not
use calculators or any other electronic devices. Any sort of cheating on the
examination will result in a zero grade.}

 \vspace{0.1 in}

 \noindent \textbf{We cannot give any clarifications about the exam questions
during the test}. If you are unsure of the meaning of a specific question, write
down your assumptions and proceed to answer the question on that basis.

 \vspace{0.1 in}

 \noindent Do all the problems in this booklet. Do your work inside this
booklet, using the backs of pages if needed. The problems are of varying degrees
of difficulty so please pace yourself carefully, and answer the questions in the
order which best suits you. \textbf{Answers to essay-type questions should be as brief
as possible.} If the grader cannot understand your handwriting you will get 0
points.

 \vspace{0.1 in}

 \noindent There are \numquestions\ questions on this exam and the maximum grade
on this exam is \numpoints\ points.

 \vspace{0.2 in}

 \gradetable[v][pages] \end{coverpages}

% Questions start here:
\begin{questions}

%\newcommand{\XP}{

%%%%%%%%%%%%%%%%%%%%%%%%%%%%%%%%%%%%%%%%%%%%%%%%%%%%%%%%%%%%%%%%
%%%%%%%%%%%%%%%% START XIAOYU
%% Lecture 4 (prepare 10 questions, so we can select 5 questions)
%% XP (including User Stories and Planning Game) -- Xiaoyu (MP4)

%%%%%%%%%%%%%%%%% BEGIN XP %%%%%%%%%%%%%%%%%%%%%%%%%%%%%%%%%%%%%
\question \textsc{eXtreme Programming (XP)}

\begin{parts}

% ALL
  \part[2] If you could choose \textbf{one} core XP practice for a ``waterfall'' team to
   adopt, what practice would you choose? Why?

  \begin{solution} [2.5 in]
  % Any one advantage of XP over waterfall is fine
       Practices:
       TDD
       Planning Game
       User stories
       Pair programming
       
       Why: 
       All the techniques listed are better than waterfall in responding to changes.
       
       Project  requirements are constantly changing. Altering requirements can mean the
       design and/or functionality must be redone to accommodate client's requests.
       In Waterfall, Such changes require more time, and therefore, project costs increases.

       In Agile, Customer feedback occurs simultaneously with development, unlike the waterfall
       approach that performs all testing after the completion of the project
   \end{solution}

% ALL
\part[2] Describe the interplay between automated unit tests and
refactorings in XP.
\begin{solution}[2.2 in]
You can easily run automated tests to check if your refactoring has 
broken some functionality. This fast feedback – compared to manual 
tests – allows you to make changes more confidently and quickly.
\end{solution}

\Comment{
% MAYBE REMOVE OR REPHRASE ``HOW'' PART
\part[3] According to XP and Planning Game, \textbf{state who (customer or
developers) and briefly describe how} performs the following activities:

\begin{itemize}
\item who and how prioritizes user stories?
\vspace{0.2 in}
\item who and how splits user stories into tasks?
\vspace{0.2 in}
\item who and how implements user stories?
\vspace{0.2 in}
\end{itemize}


% WOULD NEED TO CONFIRM THAT IT'S INDEED IN THE READING!
\part[2] 
According to the readings from MP4 about ``a suggested way to write a user story'', why is it good to write user stories on small plain cards? What features should a good user story have? 
\begin{solution}[1 in]
That's because this serves as a physical constraint which limits the possible length of the story.  
a good user story is:
1. written from a customer's perspective as opposed to a programmer's perspective;
2. specific enough that it is testable so that you show the customer that the feature is working.
3. clear and concise, with a certain goal based on the scenario.
 \end{solution}
}

% potential Q: WE COULD ASK SOMETHING ABOUT ``VELOCITY'' IF MENTIONED ON SLIDES/READING.

\Comment{
\part[2]
The waterfall development model originated in the manufacturing and construction industries. However, it might not be appropriate as a software development process. Why?
\begin{solution}
In the manufacturing and construction industries, requirement changes are rare. In software development, requirement changes are common. The software development process must be more adaptable to changes, and the waterfall model does not allow so.
\end{solution}
}

% 2012
% \part[2] List two differences between acceptance tests and unit tests in XP
%   (e.g., in terms of who writes these tests, in what language they are written, or when they are written)?
%   \begin{solution}[2 in]
%     acceptance testing is a test conducted to determine if the requirements of a 
%     specification or contract are met.  
%     Unit test: Tests the smallest unit of functionality, typically a method/function
%     Acceptance tests tell you whether your code is working and
%     complete; unit tests tell you where it's failing.
%   \end{solution}
   
\Comment{
% IS THIS IN CHAPTER 12??
\part [2] One of the main practices of XP is test-driven development (TDD). In fact, the authors of Extreme Software Engineering claim that TDD helps you write better code. Give two reasons for their claim.

\begin{solution} [1 in]
 Testing forces simplicity
 Testing clarifies the task at hand
 Testing frees you from on-the-fly editing
\end{solution}
}

\end{parts}
%}

%%%%%% Software Configuration Management Starts %%%%%%%%%%%%%%%%%%
\question[3] \textsc{Software Configuration Management (SCM):} \textbf{Name three} artifacts \textbf{besides source code} that are part of a software project that you might want to keep under SCM. For each artifact that you name, \textbf{describe why} it is good to keep it under SCM.
  \begin{solution}[2.4 in]
    %Code(\Fix{didn't we say above no code}), 
    test suites, manuals, requirements, design documentation, change control, build configuration etc.
  \end{solution}

\Comment{
\begin{parts}

% ALL
\part[2] \textbf{Name two goals} of proper Software Configuration Management.
  \begin{solution}[2 in]
    To keep track of how software changes over time and to be able to reproduce
    any version of the software. 
  \end{solution}
\end{parts}
}

%%%%%%%%%%%%%%%% END Software Configuration Ends %%%%%%%%%%%%%%%%%%%%%%

%\Interview

%%%%%%%%%%%%%%%% BEGIN Version Control  %%%%%%%%%%%%%%%%%%%%%%%%%%%%%%%
 % ALL
% Many groups have run into problems with tagging, but we do want you to
% keep tagging your code so you can learn more about it, because tagging 
% is often a good practice.
\question[2] \textsc{Version Control:} MPs ask you to tag your code. In general software development, not just MPs, what do tags help with?
  \begin{solution}[1.5 in]
    to identify important milestones of production;
  \end{solution}

\Comment{
  \begin{parts}
  %2012
  \part[2] In using Version Control systems, it is recommended to 
  commit your changes very often. Describe one reason why this is recommended.
  \begin{solution}[1 in]
    1. The longer you hold on to your changes, the more you interfere with others
    since you might have critical changes that will change the design of the system and
    these changes should be propagated to the mainline quickly so that everyone else will
    use this new design.
    2. The longer you have code checked out the harder it is to merge (for you and
    other developers) since multiple changes might have occurred.
  \end{solution}
  
  %2012 
  \part[2] Mention \textbf{one} benefit and \textbf{one} cost of using branches in version-control systems?
  \begin{solution}
    Benefits:
    1. Branching increases productivity by allowing parallel development.
    2. Branching helps reduce risk by isolating experimental changes from breaking the mainline
    Cost:
    1. Merging can be complicated in some situations
  \end{solution}
}

% in some previous exam, we had a question about ``branches'', so
% consider reusing it.

\Comment{
  % ONLY IF NEEDED  
  \part[2] From the following list, select all the good reasons to use branches in 
  your use of version control systems. 
  [ ] to work on an experimental version
  [ ] for political reasons
  [ ] to support different hardware platforms
  [ ] to support different customers
  \begin{solution}
    to work on an experimental version
    for political reasons
  \end{solution}
\end{parts}
}
%%%%%%%%%%%%%%%% END VERSION CONTROL %%%%%%%%%%%%%%%%%

%%%%%%%%%%%%%%%% BEGIN REVERSE ENGINEERING %%%%%%%%%%%
\question \textsc{Reverse Engineering}
\begin{parts}
% ALL
\part[2] List \textbf{two reasons} for software system reengineering according \Book{1}.

\begin{solution}[2 in]
1. unbundle a monolithic system so as to sell/deploy its parts separately
2. improve performance
3. port to a new platform
4. extract the design
5. exploit new technology
6. reduce human dependencies by documenting knowledge about the system
\end{solution}

\part [2] During MP3, you were provided with some \texttt{HTML} files, which we generated using \texttt{javadoc}, to help you get a quick overview of an unfamiliar codebase. This follows a reverse engineering pattern that we discussed in class. This pattern is part of a group of patterns called \textit{First Contact} in \Book{1}. \textbf{Name this pattern}. (Hint: In the lecture on \emph{Reverse Engineering}, we also had a slide with the title \textit{First Contact})
\begin{solution}[1 in]
Skim the Documentation
\end{solution}

\Comment{
\part[2] What is a legacy software, according to \Book{1}?
\begin{solution}[1 in]
Valuable software that you have inherited
\end{solution}
}


\Comment{
% ALL
\part[3] Describe how one could use debuggers for reverse engineering.
Which specific function provided by a debugger during program execution is particularly useful?
\begin{solution}[1 in]
\end{solution}
}

\Comment{
% not too exciting
\part[2] How is reverse engineering different from refactoring, in terms 
of how much of the codebase you should understand before carrying out 
each technique?
\begin{solution}[1 in]
refactoring requires less understanding, reverse engineering requires 
more understanding
\end{solution}
}
\end{parts}
%%%%%%%%%%%%%% END REVERSE ENGINEERING %%%%%%%%%%%%%%%%%%%%%%%%%%%%%%%%%%%


%%%%%%%%%%%%%%% BEGIN CODE SMELLS %%%%%%%%%%%%
% ALL
\question[2] \textsc{Code Smells:} One of the code smells you learned in this 
class is called ``non-localized plan''.  Describe this code smell in \textbf{one} 
sentence.
\begin{solution}[1 in]
If adding a feature requires changing many parts of a program, it is a 
``non-localized plan''.
\end{solution}

%%%%%%%%%%%%% END CODE SMELLS %%%%%%%%%%%%%%%%


%%%%%%%%%%%%% BEGIN INTERVIEW STYLE XP QUESTIONS %%%%%%%
\newpage
%\newcommand{\Interview}{
\question \textsc{More XP (Job Interview Style Questions)} \\
You are interviewing for a job, and they find out that you took CS 427.

\begin{parts}

% ALL
% this tests the "Why (not)? of Pair Programming". 
% Question below tests How. Should we keep both
\part [3] Describe \textbf{two benefits and one disadvantage} of pair programming (in
a professional software development setting, not in the classroom
setting where a disadvantage can be ``my teammate doesn't want to
meet''). Give specific examples from your MPs (e.g., for MP1 on
Java/JUnit, or for MP2 on CodingTracker/Jenkins). Please also
\textbf{write your group number and/or teammate name/netid}; if your
pair changed, pick one MP where you did the most pairing.

\begin{solution}[3.3 in]
Benefits:
- developers learn from each other, e.g., new command line tricks
- bugs can be spotted earlier
- develops rapport between team members
Disadvantage:
- for simple tasks, pair programming might waste an extra developer's time, since there's nothing new to learn and there's no need for an extra person to debug code
\end{solution}

% ALL
\part[4] Because you followed XP, they ask you to \textbf{describe two
XP practices that you believe can be good} (name the practice,
describe what it is, and state why you find it good) and \textbf{one
XP practice that you believe can be bad} (name the practice, describe
what it is, and state why you find it bad).
       \begin{solution}[3 in]
       \end{solution}

% KEEP FOR NOW
% This is a good question but preparing a rubric for grading may be a bit tricky
\part[3] Because you followed XP and learned about Waterfall, they ask
you to describe some third software development process --- \textbf{name one}
process model different from XP and Waterfall, and briefly \textbf{describe
how two} of its practices differ or match XP and/or Waterfall. (Hint:
\url{http://en.wikipedia.org/wiki/Software_development_process} lists
several types of software development processes.)
       \begin{solution}[3.3 in]

       \end{solution}


\Comment{
% What is this trying to test?
\part[2] They ask you some common coding question about containers
(e.g., how to use a constant amount of space to find whether a list is
cyclic).  You ace that question (because you refreshed cs225 and
practiced common interview coding questions).  But then they ask you
to discuss design options for the return type of the following method
\texttt{add} in the class \texttt{Container}:
\begin{lstlisting}
  class Container ...
      ???? add(Object newElement) ...
\end{lstlisting}
Some libraries use \textbf{void}, some use \textbf{boolean}, and some
use \textbf{Container}.  (Some even use \textbf{Object} when the
container is a map that should return what was the old value for a
given key when an update is made.)  Discuss two advantages and
disadvantages of different return types.
       \begin{solution}[2 in]
       \end{solution}
}

\Comment{
\part [2] Because you did pair programming, they ask you to
\textbf{describe how exactly you did pair programming for MPs} (e.g.,
for MP1 on Java/JUnit, or for MP2 on CodingTracker/Jenkins).  Please
also \textbf{write your group number and/or teammate name/netid}; if
your pair changed, pick one MP where you did the most pairing.
       \begin{solution}[3.6 in]
       look for mention of how they carried out any of the XP practices.
       (cross grade each group? NO. Too tedious)
       \end{solution}
}
\end{parts}
%}

%\XP

%%%%%%%%%%%%%%%%% END INTERVIEW STYLE XP QUESTIONS %%%%%%%%%%%%%%%%%%%

%%%%%%%%%%%%%%% BEGIN METRICS %%%%%%%%%%%%%%%%%%%%%%%%%%%%%%%%%%
\question \textsc{Metrics}

\begin{parts}

  \part[2] 
Many metrics, such as code size or complexity, can be hard to compare
across different software projects, but they can be used to compare 
(1)~different modules within one version of the same project or (2)~different 
versions of the same project.  Describe \textbf{two decisions} that can
be made based on comparisons of metrics on the same project.
\begin{solution}[1.5 in]
1. Decide where to put more resources into testing, 2. Decide where to
refactor, 3. Decide where to do more code reviews, 4. Decide if the
size/complexity increase over time is warranted by the features that
were added
\end{solution}

  \part[2]
The lecture on OO metrics discussed coupling.  Describe \textbf{two
reasons} why high coupling among software components is bad.
\begin{solution}[1.5 in]
1. High coupling makes designs hard to change.
2. High coupling makes classes hard to reuse.
3. High coupling makes classes hard to test.
\end{solution}

\end{parts}

%%%%%%%%%%%%%%% END METRICS %%%%%%%%%%%%%%%%%%%%%%

%%%%%%%%%%%%%%%% END XIAOYU
%%%%%%%%%%%%%%%%%%%%%%%%%%%%%%%%%%%%%%%%%%%%%%%%%%%%%%%%%%%%%%%%

%%%%%%%%%%%%%%%%%%%%%%%%%%%%%%%%%%%%%%%%%%%%%%%%%%%%%%%%%%%%%%%%
%%%%%%%%%%%%%%%% START AMARIN

%% Lecture 1-3 (prepare 10 questions, so we can select 5 questions)
%% SCM, Jenkins, CodingTracker -- Amarin (MP0, MP2)

%%%%%%%%%%%%%%%% BEGIN ECLIPSE, JENKINS AND CODINGTRACKER %%%%%%%
\question \textsc{Eclipse, Jenkins, and CodingTracker}

\begin{parts}

% ALL
  \part[2] A software development company develops an email client.  However,
  when a user tries to send an email with a .PDF attachment, the program
  crashes.  The user complains that the program is ``buggy'', but ``buggy''
  can mean many things.  The words ``fault'' and ``failure'' are much
  more precise.  In the given scenario, what is the fault and what is
  the failure?
  \begin{solution}[2.0 in]
  - The fault is the part of the source code that caused the program to crash.
  - The failure is that the application crashed.
  \end{solution}

% ALL  
\part[3] Some developers prefer to use text editors (e.g., notepad, gedit) to perform basic operations, such as adding, deleting, copying, and pasting code. On the contrary, some developers prefer to use Integrated Development Environments (IDEs) like Eclipse that, in addition to the basic operations mentioned, comes with extra features that can potentially help them become more productive when writing code. Describe \textbf{two benefits} and \textbf{one disadvantage} of using IDEs rather than text editors when developing software.
  \begin{solution} [1.8 in]
    Benefits
    - IDEs can make refactoring less error-prone.
    - IDEs can make it convenient to interact with version control systems.
    - IDEs can make it easier to navigate code.
    - IDEs can make it easier to format code.
    - IDEs can make it easier to lookup APIs (e.g. auto-complete)
    Disadvantage
    - You may not have IDE on all platforms.
    - Sometimes, you have less control of what happens to your code, e.g.,
    when using Eclipse refactorings to move methods, sometimes an unnecessary method argument is added.
    - Sometimes, IDEs perform unnecessary computations, e.g., automatically build large projects when you're still modifying code.
  \end{solution}

% ALL
\part[2] Why is it good to explicitly specify JAVA\_HOME for Jenkins instead of using the system default?
  \begin{solution} [1 in]
    - There may be multiple Java versions on the system.
  \end{solution}

% ALL
  \part[2] How does CodingTracker replay code changes differently when clicking on the \texttt{Step} button vs.\ the \texttt{Fast} button?
  \begin{solution}[1 in]
    The speed and continuity is different. When I click the Step button, CodingTracker replays my code changes step by step. When I click the Fast button, CodingTracker replays through all my code changes as fast as the machine allows.
  \end{solution}
% mightbe consider asking how these buttons related to break points

\end{parts}
%%%%%%%%% END ECLIPSE, JENKINS AND CODINGTRACKER %%%%%%%

\Comment{
\question Continuous Integration (CI)

\begin{parts}

% unclear what is being compared to what: (1) Jenkins auto trigger
% vs. ``build now'', (2) Jenkins vs. scripts, (3) manually triggerring builds??
  \part[3] Describe \textbf{two} benefits and \textbf{one} cost of using continuous integration (CI) systems.
  \begin{solution}[2 in]
    Benefits:
    - identify failures sooner
    - identify culprit changes precisely
    - reduce time to fix breaks
    - faster iteration
    Costs:
    - requires investment in computing resources
    - requires time to set up at the beginning
  \end{solution}
    
\end{parts}
}
%%%%%%%%%%%%%%%% END AMARIN
%%%%%%%%%%%%%%%%%%%%%%%%%%%%%%%%%%%%%%%%%%%%%%%%%%%%%%%%%%%%%%%%

%%%%%%%%%%%%%%%%%%%%%%%%%%%%%%%%%%%%%%%%%%%%%%%%%%%%%%%%%%%%%%%%
%%%%%%%%%%%%%%%% START SANDEEP
%% Lecture 5-7 (prepare 10 questions, so we can select 5 questions)
%% Testing (1-3) -- Sandeep (MP1)

%\newpage


%\question Developer Testing: Answer the questions below according to \Ch{22}.
%\begin{parts}

\Comment{
\part[2] There are three ways to do
'Scaffolding' to Test Individual Classes. Mention any one of the Scaffolding techniques.
	\begin{solution}[1 in]
           - One kind of scaffolding is a fake routine that calls the real routine being tested.
             This is called a “driver” or, sometimes, a “test harness.”
           - Another is a class that’s dummied up so that it can be used by another class that’s being tested.
              Such a class is called a “mock object” or “stub object”
            - The last is The dummy file, a small version of the real thing that has the same types of components
              that a full-size file has. A small dummy file offers a couple of advantages.
              Since it’s small, you can know its exact contents and can be reasonably sure that the file itself is error-free.
              And since you create it specifically for testing, you can design its contents so that any error
              in using it is conspicuous.
        \end{solution}
}

\Comment{
% potentially okay
\part[2] In the context of 'Scaffolding' to test individual classes,
what is a ``mock object'' or a ``stub object''?
    \begin{solution}[0.5 in]
        A class that's dummied up so that it can be used by another class that's being tested.
    \end{solution}
}

\Comment{
\part[2] As per your reading of \Ch{22}, mention 2 limitations
 of developer testing?
\begin{solution}[2 in]
Developer tests tend to be "clean tests" (``happy path'' testing)
Developer testing tends to have an optimistic view of test coverage
Developer testing tends to skip more sophisticated kinds of test coverage
\end{solution}
}

%%%%%%%%%%%%% BEGIN TESTING TECHNIQUES %%%%%%%%%%%%%%%%%%%%%

%\newpage
\question \textsc{Testing Techniques}
\begin{parts}
% ALL
\part[3] In the lecture on Testing and XP, we discussed the benefits of writing the tests before the code. Describe any \textbf{three benefits}.
\begin{solution} [2.5 in]
- Developers know when they are "done" with regards to software requirements, 
  i.e., passing all unit tests.
- In case of libraries, it gives examples how to use the API, or how classes 
  are created, etc.
- It avoids unnecessary complexity, i.e. creating the simplest code that 
  passes the test.
- Writing test cases before writing the code doesn't take any more effort 
  than writing test cases after the code; it simply resequenced the test-case
  -writing activity.
- When you write test cases first, you detect defects earlier and you can 
  correct them more easily.
- Writing test cases first forces you to think at least a little bit about the
  requirements and design before
- writing code, which tends to produce better code.
- Writing test cases first exposes requirements problems sooner, before the 
  code is written, because it's hard to write a test case for a poor 
  requirement.
- If you save your test cases (which you should), you can still test last,    
  in addition to testing first.
\end{solution}

\Comment{
  %GOOD although copied from 2012
  \part[4] Describe \textbf{two} reasons why manual testing is better than
  automated testing. Similarly, describe \textbf{two} other reasons why
  automated testing is better than manual testing.

  \begin{solution}[1.5 in]

    Manual testing is appropriate when:
    \begin{itemize}
      \item tests need to be run once 
      \item tester doesn't know how to program 
      \item the test is expensive to automate
      \item When writing some kinds of GUI tests, for example
    \end{itemize}

    Automated testing is appropriate when:
    \begin{itemize}
      \item the tests need to be run over and over as the software evolves or is changed
      \item when we want to use them as documentation.
    \end{itemize}		
	
  \end{solution}
}

%Lecture slides 427-6 illustrate the ideas related to ``Equivalence Partitioning''
%         and ``Boundary value Analysis'' in the context of testing.

    \part[1] What is Equivalence Partitioning in testing?
        \begin{solution} [2.1 in]
             A good test case covers a large part of the possible input data. 
             If two test cases flush out exactly the same errors, you need only one of them. 
             The concept of "equivalence partitioning" is a formalization of this idea and helps 
             reduce the number of test cases required.
        \end{solution}
   \part[3] \textbf{What is Boundary Value Analysis in software testing?} Suppose a program validates a numeric field as follows: values less than 10 are rejected, values between 10 and 20 are accepted, and values above 20 are rejected. \textbf{Give two examples} of inputs (numeric values) which are good candidates to test boundary values.
   
         \begin{solution} [1.5 in]
              Boundary value analysis is a software testing technique in which tests are designed to include representatives of boundary values. 

             Any 2 numbers in the  set \{9, 10, 11, 19, 20, 21\} 
         \end{solution}

\end{parts}

%% In the following question I am not sure whether to use the word "scenario" or something else. If you
%% find this question useful, please suggest something
\Comment{
\question
\begin{parts}
\part[2] As per your reading on ``test requirement catalog'', mention 2 testing ideas for each of the 
         following scenarios.
    \part[2] Searching operation that apply to many data types.
        \begin{solution} [1 in]
        \end{solution}
    \part[2] Linked structure, for example trees, queues, graphs, etc.
        \begin{solution} [1 in]
        \end{solution}
\end{parts}
}

%%%%%%%%%%%%%%%% END TESTING TECHNIQUES  %%%%%%%%%%%%%%%%%%%%%%

%%%%%%%%%%%%%%%% END SANDEEP
%%%%%%%%%%%%%%%%%%%%%%%%%%%%%%%%%%%%%%%%%%%%%%%%%%%%%%%%%%%%%%%%

%%%%%%%%%%%%%%%%%%%%%%%%%%%%%%%%%%%%%%%%%%%%%%%%%%%%%%%%%%%%%%%%
%%%%%%%%%%%%%%%% START OWOLABI
%% Lecture 8-10 (prepare 10 questions, so we can select 5 questions)
%% Reverse Enginering, Refactoring, Code Smells -- Owolabi (MP3)


%%%%%%%%%%%%%%%% END OWOLABI
%%%%%%%%%%%%%%%%%%%%%%%%%%%%%%%%%%%%%%%%%%%%%%%%%%%%%%%%%%%%%%%%

%%%%%%%%%%%%%%%%%%%%%%%%%%%%%%%%%%%%%%%%%%%%%%%%%%%%%%%%%%%%%%%%
%%%%%%%%%%%%%%%% START DARKO
%% Lecture 11-12
%% Metrics, OO Metrics -- Darko

%%%%%%%%%%%%%%%%% BEGIN REFACTORING SECTION %%%%%%%%%%%%%%%%%%%%%%%%%%
% ALL
\newpage
\question[2] \textsc{Refactoring:} In a lecture, we discussed that refactoring is sometimes done ``secretly''.  What does it mean: \textbf{from whom, (apart from the users)} do the developers hide that they do refactoring and \textbf{why} do they hide it? 
\begin{solution}[2 in]
\end{solution}

\Comment{
\begin{parts}
%THIS IS OKAYISH, BUT LET'S SEE HOW MANY QUESTIONS WE HAVE FROM READING.
\part[2] In \Ch{24}, two kinds of Software Evolution are discussed: 
(1) changes made during software construction 
(2) changes made during software maintenance. 
Mention one difference between these two kinds of Software Evolution. 
\begin{solution}[1.5 in]
 - original developer vs maintainers, 
 - constraints: pressure to finish vs angry users, 
 - penalty for mistakes: low vs high)
\end{solution}
}

\Comment{
%THIS IS OKAYISH, BUT LET'S SEE HOW MANY QUESTIONS WE HAVE FROM READING.
\part[2] According to \Ch{24}, what is the Cardinal Rule of Software Evolution?
\begin{solution}[1 in]
software evolution should improve the internal quality of the program
\end{solution}
}

% From Darko: I don't understand this question.  If you want to
% discuss something abour refactoring vs. not, you can ask why it's
% recommended to separate refactorings from functionality changes
% (there's a slide in the lecture with XOR in the title).
\Comment{
\part[2] Chapter 24 of Code Complete discusses some Reasons Not to Refactor. 
In \emph{one} sentence, describe any one of these reasons.
\begin{solution}[1 in]
fixing bugs, adding functionality, modifying the design
\end{solution}
}

\Comment{
% massage some more
\part[3] If your MP3 solution used any refactoring different from Extract
Method, Move Method, and Rename, (i)~name that fourth refactoring and (ii)~describe where
and how you performed it.  If you didn't perform any refactoring
different from the three listed refactorings, (i)~name any one refactoring
different from these three, (ii)~describe what code smell it
addresses, and (iii)~describe how this refactoring transforms the
code.
\begin{solution}[1 in]

\end{solution}
\end{parts}
}

%%%%%%%%%%%%%%%% END REFACTORING QUESTIONS %%%%%%%%%%%%%%%%%%%%


%% WE CAN ASK SOMETHING ABOUT MANUAL TEST SELECTION, E.G., IF YOU HAVE
%% CLASSES ``A, B, C'' WITH TESTS ``ATest, BTest, CTest'' AND YOU CHANGE
%% CLASS ``C'', WHICH TEST SHOULD YOU RUN?  QINGZHOU GAVE A TALK ABOUT
%% TEST SELECTION AT GOOGLE.

\question \textsc{Test ordering}
\label{Test-Ordering}
% This may be moved into a question on test ordering.
\begin{parts}
\Comment{
\part[1] When would you use \texttt{\@org.junit.Before} when writing JUnit tests?
\begin{solution}[1 in]
You would use \texttt{\@Before} when you have common setup code to be used by multiple tests, e.g. setting up data structures for tests to run on.
\end{solution}
}

% Version 1: easier, only smell is test ordering
% (See below for a harder version.)
\part [5] 
\Comment{
The test code below has a ``test smell'': the result depends on the
order in which the tests are executed (\texttt{saveLibrary} must be
executed before \texttt{loadLibrary}), and JUnit does not necessarily
execute the tests in the order in which they are written.  Rewrite the
code such that you still have two tests, but the result does not
depend on the order in which they are executed.  Your solution should
not duplicate the common code in both tests but should have only one
copy of the common code.

\begin{lstlisting}[caption={Order-Dependent Tests}]
public class LibraryTest {
    Library library;
    Book book;
    @Before
    void createLibrary() {
        book = new Book("author", "title");
        library = new Library(book); // create library with one book
    }

    @Test
    public void saveLibrary() throws Exception {
        library.saveLibraryToFile("library.txt");
        assert... // content of the file library.txt
    }

    @Test
    public void loadLibrary() throws Exception {
        // create a new library from a given file
        Library loadedLibrary = new Library("library.txt");
        assertEquals(library, loadedLibrary);
    }
}
\end{lstlisting}
}

%%%%%%%%%%%%%%%%%%%%%%%%%%%%%%%%%%%%%%%%%%%%%%%%%%%%%%%%%%%%%%%%

% Version 2: harder, one smell is test ordering, another smell is duplicated code
% (See above for an easier version.)
The test code below has two ``smells'': (1) there's duplicated code
between the two methods, and (2) the test result depends on the order
in which the tests are executed (\texttt{saveLibrary} must be executed
before \texttt{loadLibrary}), while JUnit does not necessarily execute
the tests in the order in which they are written.  Rewrite the code
such that you still have two tests with the same logic, but (1)
there's no duplication (hint: consider a helper method or @Before) and
(2) the test result does not depend on the order in which the tests
are executed.  (Your solution should not duplicate the common code in
both tests but should have only one copy of the common code.)\\

\textbf{NOTE: When you write code for your solution, you need not copy comments.}\\

\begin{lstlisting}[caption={Order-Dependent Tests}, label=Order-Dependent-Tests]
public class LibraryTest {
    @Test
    public void saveLibrary() throws Exception {
        Book book = new Book("author", "title");
        // create library with one book
        Library library = new Library(book);
        library.saveLibraryToFile("library.txt");
        assert... // content of the file library.txt
    }

    @Test
    public void loadLibrary() throws Exception {       
        // create a new library from a given file
        Library loadedLibrary = new Library("library.txt");
        Book book = new Book("author", "title");
        // create library with one book
        Library library = new Library(book);
        assertEquals(library, loadedLibrary);
    }
}
\end{lstlisting}
  \begin{solution}[4 in]
  \end{solution}
\end{parts}

\newpage

\question \textsc{Test Selection} \\
The lecture on testing at Google discussed how one can do test 
selection in regression testing based on what parts of the changed code 
the tests executed in the past.  Consider (I) the test code originally 
given in Listing \ref{Order-Dependent-Tests} of Question \ref{Test-Ordering} 
and (II) the test code you wrote in your answer to Question \ref{Test-Ordering}. 
Suppose that you run both tests from each code and find 
that both tests pass.  You then make a change to the code under test 
and want to rerun these tests for regression testing.  Consider these 
scenarios:
\begin{parts}

\part [2] if you change something in the body of the constructor,
 \texttt{Library(String fileName)}:
\begin{subparts}
\subpart What test(s) should be selected to check the changes in the constructor's method
body if the tests are as in the original code from Listing \ref{Order-Dependent-Tests}
in Question \ref{Test-Ordering}? (If you think that no tests should be selected, please write, \textbf{``no tests to be run''})
\begin{solution}[0.5 in]
\end{solution}
\subpart What test(s) should be selected to check the changes in the constructor's method
body if the tests are as in the code that you wrote in your answer to Question \ref{Test-Ordering}?(If you think that no tests should be selected, please write, \textbf{``no tests to be run''})
\begin{solution}[0.5 in]
\end{solution}
\end{subparts}

\part [2] if you change something in body of the constructor, \texttt{Library(Book book)}:
\begin{subparts}
\subpart What test(s) should be selected to check the changes in the constructor's method
body if the tests are as in original code from Listing \ref{Order-Dependent-Tests}
in Question \ref{Test-Ordering}?(If you think that no tests should be selected, please write, \textbf{``no tests to be run''})
\begin{solution}[0.5 in]
\end{solution}
\subpart What test(s) should be selected to check the changes in the constructor's method
body if the tests are as in the code that you wrote in your answer to Question \ref{Test-Ordering}? (If you think that no tests should be selected, please write, \textbf{``no tests to be run''})
\begin{solution}[0.5 in]
\end{solution}

\end{subparts}

\Comment{
Consider again the code in Listing \ref{Order-Dependent-Tests}.
(1) if you change only the constructor, \texttt{Library(String fileName)}, 
what tests should be selected; (2) if you change only the constructor,
\texttt{Library(Book book)}, does any test need to be run?
\begin{solution} [2 in]
(1) only \texttt{loadLibrary} need to be run
(2) run both \texttt{loadLibrary} and \texttt{saveLibrary}
\end{solution}
}

\end{parts}

\Comment{
% Potential design question (MP1 had keyword ``static''): having two
% similar constructors in the Library class can be confusing.  For
% example, if we had Library(``author:aurhor, title:title''), it'd not
% be clear if we want a new library with one book or to read a library
% from the given fileName.  Library(String bookInfo), Library(String fileName)
% 
% How do you solve that?
% class Library {
%     static Library createFromBook(String bookInfo) { ... }
%     static Library createFromFile(String fileName) { ... }
}

%%%%%%%%%%%%%%%% END DARKO
%%%%%%%%%%%%%%%%%%%%%%%%%%%%%%%%%%%%%%%%%%%%%%%%%%%%%%%%%%%%%%%%

\end{questions}
\end{document}

% LocalWords:  tabsize Urbana Champaign netid eXtreme XP MP4 3x5 refactorings
% LocalWords:  refactoring doesn MPs MP1 JUnit MP2 CodingTracker cs225 boolean
% LocalWords:  newElement PDF IDEs gedit lookup APIs IDE SCM dummied API
% LocalWords:  resequenced
