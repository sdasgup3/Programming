
\section{Implementation Details}

\subsection{DFA framework}
      
    This Field Sensitive analysis is actually a data flow analysis with data flow values being Direction Matrix,Interference Matrix
Boolean Equations,Boolean variables.The data flow values changes whenever any statement belonging to
the 6 class of statement's of Fig.~\ref{fig:stmts} is encountered.We have implemented this analysis
as a GCC dynamic plugin.This next section gives us a small overview of GCC plugins.\\
 
\begin{figure}[h] 
\begin{tabular}[b]{l}
        $\bullet$  p = malloc(); \\
        $\bullet$  p=NULL; \\
        $\bullet$  p=q; p=\&(q$\rightarrow$f); p=q op n; \\
        $\bullet$  p$\rightarrow$f=NULL; \\
        $\bullet$  p$\rightarrow$f=q; \\
        $\bullet$  p=q$\rightarrow$f; \\
\end{tabular}
\\
$\forall$ p,q $\in$ H;

\caption{\label{fig:stmts}}
\end{figure}


  \subsubsection[]{gcc plugins}
      GCC architecture consists of many passes each being either a GIMPLE,IPA(Inter Procedural Analysis) or RTL(Register 
Transfer Language) pass.GIMPLE is a three-address intermediate representation.Our analysis works by inserting a
GIMPLE pass in the GCC architecture which is done by plugins. \\	  
      Plugin's make the developer add new features to the compiler without modifying the compiler itself.
It is a way by which we can add,remove and maintain modules independently.
GCC has two types of plugins ,static and dynamic plugin's. 
Static plugins involve minor changes in the source code and static linking,while dynamic plugin requires
no change to the source code and require dynamic linking \cite{plugin} .For our case we have used dynamic plugin 
because we need not build GCC every time we change the plugin code.

   What our plugin does is ,first it finds out what are the heap pointers present in the program and creates a table for
them.That table will contain details like which structure the pointer points to,what are the fields present
in it,which of those fields are also pointers etc.Similarly another table is created for the field pointers
present in the program.In this process Id's are given to each field and heap pointer.
After this we parse along all the gimple statements one by one and check it belongs to any of the statements of Fig.~\ref{fig:stmts}.
 If yes then the GEN and KILL data flow values are generated for that statement,OUT is calculated by below equation.All the
equations for GEN and KILL of each statement are mentioned in \cite{Sandeep}.\\

OUT = GEN $\cup$ (IN - KILL) 

  This analysis is run until the data flow values reach a fixed point.When considering the fixed point we only check for
the Direction Matrices ,Interference Matrices and Boolean variables,the Boolean equations are left aside because as
mentioned in Sandeep's report \cite{Sandeep11thesis} the Boolean equations reach a fixed point at the end of 
third iteration itself.
\subsection{Optimization's suggested}
\subsection{Results for Benchmarks}
