
\documentclass[11pt]{article}

\usepackage{url}
\usepackage{array}
\usepackage{multirow}
\usepackage{verbatim}
\usepackage{graphicx}
\usepackage{multicol}

\usepackage[top=1in, bottom=1in, left=1in, right=1in]{geometry}
\setlength{\textwidth}{6.5in}
\setlength{\parindent}{0.25in}
\setlength{\parskip}{0.15in}
%%\setlength{\topmargin}{0in}

\newcommand{\p}{\ensuremath{p}}
\newcommand{\s}{\ensuremath{s}}
%\newcommand{\t}{\ensuremath{t}}
\newcommand{\f}{\ensuremath{f}}
 \newcommand{\LEFT}{\ensuremath{LEFT}}
  \newcommand{\RIGHT}{\ensuremath{RIGHT}}
 \newcommand{\PARENT}{\ensuremath{PARENT}}  

\newcommand{\upath}{\ensuremath{\mathcal{U}}}
\newcommand{\num}[1]{\ensuremath{|#1|}}
\newcommand{\drct}{\ensuremath{D}}
\newcommand{\indrct}{\ensuremath{I}}
\newcommand{\fieldD}[2]{\ensuremath{{#1}_{#2}^\drct}}
\newcommand{\fieldI}[3]{\ensuremath{{#1}_{#2}^{\indrct#3}}}
%\newcommand{\myT}{\mbox{\tt T}}
\newcommand{\myL}{\mbox{\tt L}}
%\newcommand{\R}{\mbox{\tt R}}
\newcommand{\false}{\textbf{False}}
\newcommand{\true}{\textbf{True}}
\newcommand{\subC}{\mbox{\scalebox{0.6}{\Cycle}}}
\newcommand{\subD}{\mbox{\scalebox{0.6}{\Dag}}}

\begin{document}
 
\section{Introduction}
\subsection{Brief Introduction}

Shape analysis,the aim of this static analysis technique is to infer some useful properties about the
the programs changing heap structure  which can be used in many areas like garbage collection,parallelization,
compiler time optimization,instruction scheduling etc. \\
  
  In this report we discuss the effectiveness of Sandeep's \cite{Sandeep} work which is about field sensitive shape analysis
in comparison to that of Ghiya's \cite{Ghiya96}.Some of the  implementation details about Sandeep's \cite{Sandeep} analysis
 is also provided .When considering the interprocedural anlysis should we go  for context sensitive or a context
insensitive analysis; feasibility of each of it is discussed along with the supporting arguments and data.
In the context insensitive case a new method of merging contexts at function calls is looked at.\\

  Assume a particular function in a program which uses not all field pointers ,and we were asked
to find the shape of a heap pointer in that function.If we go by Sandeep's\cite{Sandeep} ,it doesn't take into consideration
the field pointers absent in that function because of which the shape may have changed giving less
precise shape i.e instead of considering just a subset of all the fields,everything was considered .We create a subset
for each function and  use the field sensitive information obtained by Sandeep's \cite{Sandeep} work,
to find the shape of the heap pointer inside that function.This subset-based field sensitive
analysis can give even more precise results.
  
\subsection{Organization Of Thesis}
  Some of the earlier related work is discussed in Chapter 2. Chapter 3
tells about the details of implementation of field sensitive analysis, optimization's suggested and results
run on benchmarks.Details about the subset-based analysis is discussed in chapter 4 with the help of examples.The advantages and disadvantages of
some interprocedural analysis methods are given in Chapter 5 along with supporting numbers.Finally details about 
future work in chapter 6.

\section{Related Work}

???????????? ---about history of it ????required\\
??????????-------should I write about the definitions and notations of Dir,Int matrices???-------\\

This work actually has its roots from Ghiya's \cite{Ghiya96} analysis.In this data flow analysis the data flow
values used are direction matrices,interference matrices which has entries for every heap pointer
pair.This analysis doesn't take into consideration any field information so its fast but results are
not precise. \\
    In Sandeep's \cite{Sandeep} analysis limited field sensitivity is used to infer the shape of the heap i.e a
TREE,DAG or a CYCLE. For identifying the shape they capture field sensitivity information in
two ways;first by using Boolean variables to remember the direct connection between two pointer
variables ,second by the use of field sensitive matrices that store the approximate path information.
And at every statement Boolean equations are formed which will be able to infer the shape.
The information coming out of this field sensitive analysis is a set of data flow values named Direction Matrix,
Interference Matrix , Boolean Equations and field based Boolean variables for every
statement(those which access the heap pointer variables) in the program across functions.It's pretty clear
that the data flow values of \cite{Ghiya96} and \cite{Sandeep} are somewhat similar. This
information only is refined and used as per requirement in  subset based analysis.Very little details are given
about the interprocedural analysis of the same,but our report discusses it in detail.


\section{Implementation Details}

\subsection{DFA framework}
      
    This Field Sensitive analysis is actually a data flow analysis with data flow values being Direction Matrix,Interference Matrix
Boolean Equations,Boolean variables.The data flow values changes whenever any statement belonging to
the 6 class of statement's of Fig.~\ref{fig:stmts} is encountered.We have implemented this analysis
as a GCC dynamic plugin.This next section gives us a small overview of GCC plugins.\\
 
\begin{figure}[h] 
\begin{tabular}[b]{l}
        $\bullet$  p = malloc(); \\
        $\bullet$  p=NULL; \\
        $\bullet$  p=q; p=\&(q$\rightarrow$f); p=q op n; \\
        $\bullet$  p$\rightarrow$f=NULL; \\
        $\bullet$  p$\rightarrow$f=q; \\
        $\bullet$  p=q$\rightarrow$f; \\
\end{tabular}
\\
$\forall$ p,q $\in$ H;

\caption{\label{fig:stmts}}
\end{figure}


  \subsubsection[]{gcc plugins}
      GCC architecture consists of many passes each being either a GIMPLE,IPA(Interprocedural Analysis) or RTL(Register 
Transfer Language) pass.GIMPLE is a three-address intermediate representation.Our analysis works by inserting a
GIMPLE pass in the GCC architecture which is done by plugins. \\	  
      Plugin's make the developer add new features to the compiler without modifying the compiler itself.
It is a way by which we can add,remove and maintain modules independently.
GCC has two types of plugins ,static and dynamic plugin's. 
Static plugins involve minor changes in the source code and static linking,while dynamic plugin requires
no change to the source code and require dynamic linking \cite{plugin} .For our case we have used dynamic plugin 
because we need not build GCC every time we change the plugin code.

   What our plugin does is ,first it finds out what are the heap pointers present in the program and creates a table for
them.That table will contain details like which structure the pointer points to,what are the fields present
in it,which of those fields are also pointers etc.Similarly another table is created for the field pointers
present in the program.In this process Id's are given to each field and heap pointer.
After this we parse along all the gimple statements one by one and check it belongs to any of the statements of Fig.~\ref{fig:stmts}.
 If yes then the GEN and KILL data flow values are generated for that statement,OUT is calculated by below equation.All the
equations for GEN and KILL of each statement are mentioned in \cite{Sandeep}.\\

OUT = GEN $\cup$ (IN - KILL) 

  This analysis is run until the data flow values reach a fixed point.When considering the fixed point we only check for
the Direction Matrices ,Interference Matrices and Boolean variables,the Boolean equations are left aside because as
mentioned in Sandeep's report \cite{Sandeep11thesis} the Boolean equations reach a fixed point at the end of 
third iteration itself.
\subsection{Optimization's suggested}
\subsection{Results for Benchmarks}


\section{Subset Based Analysis}
  \subsection{Motivating Example}

EXAMPLE 1:
Consider the code segment in the Fig:~\ref{fig:Code}(a) which has the code  for searching data in a binary tree  and  Fig:~\ref{fig:Code}(b) ,a small code snippet of the
insert function.The datstructure of they binary tree node is also shown.The code snippet of insert is just a small set of statements that are of inserting 
a node into a binary tree. 
  In the table the value of boolean varaibles  $\LEFT_{p,t}$ ,$\PARENT_{t,p}$ are shown after each statement.(The statements S0,S6 were not considered at all
as they donot effect the shape of pointer in any way).The value of $\LEFT_{p,t}$ has become TRUE after statement S4 and value of  $\PARENT_{t,p}$ after S5.
The direction and interference matrices after each statement is also shown in Fig:~\ref{fig:Code}(c).

    In that the pointer parent is used mainly for debugging.As you can notice the parent pointer is not at all used in
the function search.During the build of the binary tree each child's parent pointer is made to point to its parent making shape of
every node in the tree a cycle.But when you see the function search, parent pointer is not used at all.Though the search($\s\rightarrow\LEFT$,key)
and search($\s\rightarrow\RIGHT$,key) functions can be executed in parallel ,its inhibited because the shape at pointer s is identified as a cycle.
If we could identify the field pointers not used in the function and try to find the shape excluding that field pointer we can get
precise shapes. Fig:~\ref{fig:Code}(d) gives the values of boolean equations at the end of statement S5.To each entry we find S4 or S5 appended to the curly brackets just
to tell that those values are to be taken from that equations data flow output values.

One way to find the shape without considering PARENT field for the function search is to  do the following, make  $\forall x,y    PARENT_ {x,y}=0 $
and then substitute the datflow values in the boolean equation.


%\begin{comment}

\begin{figure}[h]

\centering
\scalebox{0.80}{
\begin{tabular}{@{}c@{}}
\begin{tabular}{@{}cc@{}}
  {\small \tt
    \begin{tabular}[b]{l}
      Struct Node \{ \\
             Struct Node *LEFT,*RIGHT,*PARENT;\\
             int key;\\  
      \}  \\
      \\
      bool search(node *s,int key) \{  \\
      if(s)  \\
      return (key==s->data)||search(s->LEFT,key)||search(s->RIGHT,key); \\
      return 0; \\
      \}
    \end{tabular}
  } &
   {\small \tt
    \begin{tabular}[b]{l}
      insert(node *s,int key) \{ \\
             .\\
	     .\\
      S0     Node *p;\\       
      S1.    p=malloc(sizeof(struct Node)); \\
      S2.    p->LEFT=NULL; \\
      S3.    p->RIGHT=NULL; \\
      S4.    s->LEFT=p; \\
      S5.    p->PARENT = s; \\
      S6.    p->data=key;  \\  
             .\\
             .\\
      \}  
    \end{tabular}
  }\\
  \footnotesize (a) A code fragment  %& \footnotesize (b)  \\
  \end{tabular} \\  \\

  \scalebox{0.80}{
    \begin{tabular}[b]{|c||c|c|}
          \hline 
          {\bf After} & $\LEFT_{p,t}$ & $\PARENT_{t,p}$ \\ 
           \hline \hline
	  {\tt S1}       & false	  & false     \\ \hline
	  {\tt S2}       & false          & false     \\ \hline
	  {\tt S3}       & false 	  & false     \\ \hline
	  {\tt S4}       & true	          & false     \\ \hline
          {\tt S5}       & true	          & true     \\ \hline
    \end{tabular} 
      
  } 
  \footnotesize (b) Boolean Variables  \\ \\
  

  \scalebox{0.80}{
\begin{tabular}[b]{|l@{}|@{}c@{}|@{}c@{}|} \hline
 {\bf After} & $D_F$ & $I_F$ \\ 
 {\bf Stmt} & & \\ \hline

{\tt S1} & 
\begin{tabular}{|p{3mm}|p{22mm}p{22mm}|} \hline 
            & $\p$  		& $\s$   \\ \hline
  $\p$ 	& $\epsilon$	& $\emptyset$	 \\ \hline
  $\s$ 	& $\emptyset$	& $\emptyset$	\\ \hline
\end{tabular}
 &
\begin{tabular}{|p{3mm}|p{40mm}p{40mm}|} \hline 
            & $\p$  		& $\s$   \\ \hline
  $\p$ 	& $\{(\epsilon,\epsilon)\}$	& $\emptyset$	 \\ \hline
  $\s$ 	& $\emptyset$	& $\emptyset$	\\ \hline
\end{tabular} \\ \hline

{\tt S2} & 
\begin{tabular}{|p{3mm}|p{22mm}p{22mm}|} \hline 
            & $\p$  		& $\s$   \\ \hline
  $\p$ 	& $\epsilon$	& $\emptyset$	 \\ \hline
  $\s$ 	& $\emptyset$	& $\emptyset$	\\ \hline
\end{tabular}
 &
\begin{tabular}{|p{3mm}|p{40mm}p{40mm}|} \hline 
            & $\p$  		& $\s$   \\ \hline
  $\p$ 	& $\{(\epsilon,\epsilon)\}$	& $\emptyset$	 \\ \hline
  $\s$ 	& $\emptyset$	& $\emptyset$	\\ \hline
\end{tabular} \\ \hline

{\tt S3} & 
\begin{tabular}{|p{3mm}|p{22mm}p{22mm}|} \hline 
            & $\p$  		& $\s$   \\ \hline
  $\p$ 	& $\epsilon$	& $\emptyset$	 \\ \hline
  $\s$ 	& $\emptyset$	& $\emptyset$	\\ \hline
\end{tabular}
 &
\begin{tabular}{|p{3mm}|p{40mm}p{40mm}|} \hline 
            & $\p$  		& $\s$   \\ \hline
  $\p$ 	& $\{(\epsilon,\epsilon)\}$	& $\emptyset$	 \\ \hline
  $\s$ 	& $\emptyset$	& $\emptyset$	\\ \hline
\end{tabular} \\ \hline

{\tt S4} & 
\begin{tabular}{|p{3mm}|p{22mm}p{22mm}|} \hline 
            & $\p$  		& $\s$   \\ \hline
  $\p$ 	& $\epsilon$	& $\emptyset$	 \\ \hline
  $\s$ 	& $\{\fieldD{left}{}\}$	& $\emptyset$	\\ \hline
\end{tabular}
 &
\begin{tabular}{|p{3mm}|p{40mm}p{40mm}|} \hline 
            & $\p$  		& $\s$   \\ \hline
  $\p$ 	& $\{(\epsilon,\epsilon)\}$	& $\{( \epsilon,\fieldD{left}{})\}$ 	 \\ \hline
  $\s$ 	& $\{(\fieldD{left}{}, \epsilon)\}$	& $\emptyset$	\\ \hline
\end{tabular} \\ \hline

{\tt S5} & 
\begin{tabular}{|p{3mm}|p{22mm}p{22mm}|} \hline 
            & $\p$  		& $\s$   \\ \hline
  $\p$ 	& $\epsilon$	& $\{\fieldD{parent}{}\}$	 \\ \hline
  $\s$ 	& $\{\fieldD{left}{}\}$	& $\emptyset$	\\ \hline
\end{tabular}
 &
\begin{tabular}{|p{3mm}|p{40mm}p{40mm}|} \hline 
            & $\p$  		& $\s$   \\ \hline
  $\p$ 	& $\{(\epsilon,\epsilon)\}$	&  $\{( \epsilon,\fieldD{left}{}),(\fieldD{parent}{},\epsilon)\}$	 \\ \hline
  $\s$ 	& $\{(\fieldD{left}{}, \epsilon),(\epsilon,\fieldD{left}{})\}$		& $\emptyset$	\\ \hline
\end{tabular} \\ \hline

\end{tabular} 

}  \\
 \footnotesize (c) Direction ($D_F$) and Interference ($I_F$) matrices  \\ \\
  
\scalebox{0.80}{
\begin{tabular}[b]{|c@{}|@{}c@{}|@{}p{180mm}|}\hline
  Heap Pointer & Shape & BooleanEquation \\ \hline
{\tt p} &
\begin{tabular}{c}
  dag \\ \hline
  cycle \\
\end{tabular}
&
\begin{tabular}{c}
 \{  ($\PARENT_{p,s}  \wedge  s_{cycle}^{in}$) $\vee$ ($\PARENT_{p,s} \wedge ( \num{D[s,p]} >= 1)$) \}S5 $\vee$    \{ $\LEFT_{s,p}  \vee ( \num{D[p,s]} >= 1)$ \}S4  \\ 
  \{ ($\PARENT_{p,s} \wedge   ( \num{I[p,s]} > 1)$) \}S5
\end{tabular}  \\ \hline
{\tt s}&
\begin{tabular}{c} \hline
  dag \\ \hline
  cycle \\
\end{tabular}
&
\begin{tabular}{c}
  \{ $\PARENT_{p,s} \wedge ( \num{D[s,p]} >= 1)$ \}S5 $\vee$ \{  ($\LEFT_{s,p}  \wedge  p_{cycle}^{in}$) $\vee$ ($\LEFT_{s,p} \wedge   ( \num{D[p,s]} >= 1)$) \}S4 \\  
  \{ ($\LEFT_{s,p} \wedge   ( \num{I[s,p]} > 1)$) \}S4
\end{tabular}  \\ \hline

\end{tabular}
} \\
\footnotesize (d) Boolean Equations at the end  \\
      
    

\end{tabular}}
%  \caption{ mirror image of a  binary tree. ``$l$'' and ``$r$'' denotes respectively the left and right fields of tree.\label{fig:limitation}}
\label{fig:Code}
\end{figure}

\section{Interprocedural analysis}
  In this Chapter we discuss what interprocedural analysis method should be considered for implementing the field sensitive analysis of \cite{Sandeep}.
We have implemented this analysis in two methods, one is Callstring based and other a context insensitive analysis which intelligently
merges the contexts at function calls.

\subsection{Callstring Based}
    Before we understand what is this Callstring method we need to know the diffeence between context sensitive and context insensitive .

    ------------Draw a diagram expaliing the difference of both--------------

    ------------What is Callstring based ----(very few details abt implementation)-----------

    ------------Problem ,i.e memory fault-------------------

   -------------System Config,(table showing for which benchmarks it was working and for it was not)

                (benchmark name(C++ programs) ,(tree,dag,cycle),memory???,time

  Why is this happening (memory fault)





\bibliographystyle{unsrt}   % this means that the order of references
			    % is dtermined by the order in which the
			    % \cite and \nocite commands appear
\bibliography{references}  % list here all the bibliographies that


\end{document}
