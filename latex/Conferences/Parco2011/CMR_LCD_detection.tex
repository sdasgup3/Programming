% -*- LaTeX -*-
\documentclass{IOS-Book-Article}

\usepackage{times}
\normalfont
\usepackage[T1]{fontenc}
\usepackage{graphicx}
\usepackage{url}
\usepackage{pstricks}
\usepackage{pst-node}
\usepackage{pst-rel-points}
\usepackage{flexiprogram}
\newtheorem{example}{Example}

\newcommand{\ak}[1]{{\blue #1}}
\newcommand{\sd}[1]{{\green #1}}
\newcommand{\bb}[1]{{\red #1}}
\newcommand{\cmt}[1]{}
\newcommand{\p}{\ensuremath{p}}
\newcommand{\q}{\ensuremath{q}}
\newcommand{\s}{\ensuremath{s}}
\newcommand{\myr}{\ensuremath{r}}
\newcommand{\loc}{\ensuremath{l_0}}
\newcommand{\mynext}{\mbox{\tt next}}
\newcommand{\lt}{\mbox{\tt left}}
\newcommand{\rt}{\mbox{\tt right}}
\newcommand{\isInterfering}{\mbox{isInterfering}}
\newcommand{\interf}[2]{\mbox{\sf interfere}(#1,\ #2)}
\newcommand{\drct}{\ensuremath{D}}
\newcommand{\indrct}{\ensuremath{I}}
\newcommand{\fieldD}[2]{\ensuremath{{#1}_{#2}^\drct}}
\newcommand{\fieldI}[3]{\ensuremath{{#1}_{#2}^{\indrct#3}}}
\newcommand{\ttf}[1]{{\tt #1}}
\newcommand{\rtarrow}{$\rightarrow$}
\bibliographystyle{unsrt}
\begin{document}
\begin{frontmatter}                           % The preamble begins here.

%\pretitle{Pretitle}
\title{Heap Dependence Analysis for Sequential Programs}
%
%\subtitle{An extended abstract}

\author{\fnms{Barnali} \snm{Basak}},
\author{\fnms{Sandeep} \snm{Dasgupta}}
and
\author{\fnms{Amey} \snm{Karkare}}

\address{{\tt \{barnali, dsand, karkare\}@cse.iitk.ac.in}\\
  Department of CSE, IIT Kanpur, India}

\begin{abstract}
In this paper we demonstrate a novel intra-procedural
technique for detecting heap dependences in sequential
programs that use recursive data structures.  The novelty of
our technique lies in the way we compute, for each statement,
abstract {\em heap access paths} that approximate the
locations accessed by the statement, and the way we convert
these paths into equations that can be solved using
traditional tests, e.g. GCD test, Banerjee test and Lamport
test.  The dependence test also uses a field sensitive shape
analysis to detect dependences among heap locations arising
out of sharing within the data structure. In presence of
loops, the technique can be used to discover {\em loop
  dependences}.  i.e. the dependence among two different
iterations of the same loop. This information can be used by
a parallelizing compiler to transform sequential input
program for better parallel execution.
\end{abstract}

\begin{keyword}
Dependence Analysis\sep
Shape Analysis \sep {Parallelizing
  Compilers} \sep {Recursive Data Structures}
\end{keyword}
\end{frontmatter}

\thispagestyle{empty}
\pagestyle{empty}

%%%%%%%%%%%%%%%%%%%%%%%%%%%%%%%%%%%%%%%%%%
\section*{Introduction}

In the recent arena of parallel architectures (multi-cores,
GPUs, etc.), software side lags behind hardware in terms of
parallelism. Parallelization of sequential programs, without
violating their correctness, is a key step in increasing
their performance and efficiency. Over the past years, lot of
work has been done on automatically parallelizing sequential
programs. These approaches have mainly been developed for
programs written in languages, such as FORTRAN, having only
static data structures (fixed sized
arrays)~\cite{Allen87automatic,Banerjee93automatic,Wolf91loop,
  Kennedy01Optimizing}.  Almost all programming languages
today use the heap for dynamic memory structures. Therefore,
any parallelization must also take into account the data
dependency due to the access of common heap
locations. Finding parallelism in sequential programs written
in languages with dynamically allocated data structures, such
as C, C++, JAVA, LISP etc., has been less successful. One of
the reason being the presence of pointer-induced aliasing,
which occurs when multiple pointer expressions refer to same
storage location. Compared to the analysis of static and
stack data, analyzing properties of heap data is challenging
because the structure of heap is unknown at compile time, it
is also potentially unbounded and the lifetime of a heap
object is not limited by the scope that creates it. As a
consequence, properties of heap (including dependence) are
approximated very conservatively. The approximation of the
heap dependence information inhibits the parallelization. The
following example motivates the need for a precise dependence
analysis.
  \begin{figure}[t]
    \centering
    \newcommand{\lnode}[2]{
\psset{unit=1mm}
\begin{pspicture}(0,0)(13,14)
%\psframe(0,0)(13,14)
\putnode{z}{origin}{6}{7}{\rnode{#1}{
    \tt\begin{tabular}{|c|}
      \hline
      next \\\hline
      num\\\hline
    \end{tabular}}}
\end{pspicture}%
}
\scalebox{.95}{
\begin{tabular}{@{}c@{}|c@{}}
\psset{unit=1mm}
\begin{pspicture}(0,28)(62,10)
%\psframe(0,28)(62,40)
\putnode{head}{origin}{2}{11}{p\ }
\putnode{fst}{head}{12}{0}{\lnode{a1}{b1}}
\putnode{snd}{fst}{20}{0}{\lnode{a2}{b2}}
\putnode{thr}{snd}{20}{0}{\lnode{a3}{b3}}
\ncline[offset=2]{->}{a1}{a2}
\ncline[offset=2]{->}{a2}{a3}
\ncline{->}{head}{fst}
\putnode{txt1}{origin}{17}{-2}{{\tt num} is the data field.}
\putnode{txt1}{origin}{30}{-6}{{\tt next} is the pointer to
  the next element.}
\end{pspicture}
 & 
{\tt
\begin{program}{0}
  \FL\ \ldots
  \NL{0} p = list;
  \UNL{0} \WHILE (p$\rightarrow$next != NULL) \{
  \NL{1}     q = p$\rightarrow$next;
  \NL{1}     \ldots = q$\rightarrow$num;
  \NL{1}     r = q$\rightarrow$next;
  \NL{1}     r$\rightarrow$num = \ldots;
  \NL{1}     p = r;
  \UNL{0} \}
  \UNL{0} \ldots
\end{program}
}
\\
(a) A singly linked list & (b) Code traversing the list
\end{tabular}
}

    \caption{A motivating example\label{fig:motiv}}
    \hrule
  \end{figure}
\begin{example}{\rm
  Figure~\ref{fig:motiv}
  shows a singly linked list  and a code fragment traversing that
  list. The performance of the code can be improved if the loop can
  be executed in parallel. However, without the knowledge of
  precise heap dependences, we have to assume worst case
  scenario, i.e.,  the  location read by the statement {\tt
    S3} in some iteration could be the same as the location
  written by the statement {\tt S5} in some other iteration.
  In that case,  it is not possible to parallelize the loop.
 
  Our dependence analysis can show that the locations read by
  {\tt S3} and those written by {\tt S5} are mutually
  exclusive. Further, it also shows the absence of any other
  dependences.  This information, along with the information
  from classical control and data dependence analysis, can be
  used by a parallelizing compiler to parallelize the loop.
}
\hfill\psframebox{}  \end{example}

The rest of this paper explains our approach for a practical
intra-procedural heap data dependence analysis. As it is
understood that we are only talking about data dependences,
we drop the term data in the rest of the paper. We first
describe a shape analysis  that is used by our
analysis to detect sharing (also called interference) among
the data structures created on the heap.  The details of 
dependence analysis are explained next.  We then
present our method to handle loops in a more precise way.  We
finish the paper by giving the directions for the future
research.

%%%%%%%%%%%%%%%%%%%%%%%%%%%%%%%%%%%%%%%%%%
\section{Shape Analysis}
The goal of our shape analyzer is to detect the shape of the
data structure pointed to by the heap directed pointers at
each program point. Our approach is similar to the work
proposed by Ghiya et. al.~\cite{Ghiya96Tree} in that it also
uses the {\em Direction Matrix} and the {\em Interference
  Matrix} to keep track of shapes of data structures.
However, our shape analysis is {\em field sensitive}; it
remembers abstracted paths between two heap nodes. The path
abstraction is done by using fixed length prefixes (sequence
of field names) of the paths between two heap nodes. As the
number of paths starting with the same fixed length prefixes
may be unbounded so we use k-limiting on that number
i.e.\ only k such paths will be considered. 

The novelty of our approach lies in the way we
use field information to remember the paths that result in a
particular shape (Tree, DAG, Cycle). We associate the field
information with a shape in two ways: (a) through boolean
functions that capture the shape transition due to change in
a particular field, and (b) through matrices that store the
field sensitive path information among two pointer variables.
This allows us to easily identify transitions from Cycle to
DAG, from Cycle to Tree and from DAG to Tree, thus making the
shape more precise.  This is an improvement over earlier
approaches like Ghiya et. al.~\cite{Ghiya96Tree} where once a
data structure is marked cyclic, it remains so for the rest
of the analysis.

For dependence detection, our shape analysis technique
provides an interface function {\tt \isInterfering(\p,
  $\alpha$, \q, $\beta$)}. For heap pointers \p, \q\ and
field sequences $\alpha, \beta, \alpha', \beta'$, this
function returns true if the paths \p.$\alpha'$ and
\q.$\beta'$ interfere (potentially reach the same heap node
at run-time), and $\alpha'$ and $\beta'$ are prefixes of
$\alpha$ and $\beta$ respectively. The result of the
interface function is based on the following observations:
\noindent\begin{enumerate}
\item If the shape attribute of a pointer
  variable is Tree, then two access paths rooted at that
  variable cannot interfere. Two access paths can only visit
  a common node if the paths are equivalent. Let \ttf{lp} be
  the pointer variable. Hence \ttf{lp\rtarrow{f}} and
  \ttf{lp\rtarrow{f}} are equivalent paths leading to a
  common node, whereas \ttf{lp\rtarrow{f}} and
  \ttf{lp\rtarrow{g}} lead to different nodes.
\item If the shape attribute is DAG and if it
  is traversed using a sequence of fields, then every
  sub-sequence accesses a distinct node. If an access path is a
  proper sub-path of another access path then they surely
  visit distinct nodes. However, if the paths equivalent or
  distinct (i.e.\ having different pointer field references),
  they may access a common node. For example,
  \ttf{lp\rtarrow{f}} is a proper sub-path of
  \ttf{lp\rtarrow{f}\rtarrow{g}} whereas, \ttf{lp\rtarrow{f}}
  and \ttf{lp\rtarrow{g}} are not. Hence in the former case
  they do not share a common node, whereas in later case they
  might result in sharing of node.
\item If shape attribute of a pointer
  variable is Cycle, we make conservative decision such that
  the access paths originating from that pointer interfere.
\end{enumerate}
 The details of the path abstraction and the shape analysis can
 be found elsewhere~\cite{Sandeep11thesis}.
%%%%%%%%%%%%%%%%%%%%%%%%%%%%%%%%%%%%%%%%%%
\section{Dependence Analysis}

Two statements are said to be heap dependent on each other if
both statements access the same heap location and at least
one of the statements writes to that location. We have
developed a novel technique which finds out heap induced
dependencies, between any two statements in the program. The
novelty of our approach lies in the separation of shape
analysis phase from the dependence detection phase For each
statement our analysis computes two sets of heap access paths
(a) {\em Read set}: the set of paths which are accessed to
read a heap location, and (b) {\em Write set}: the set of
paths which are accessed to write a heap location.

Our approach is conservative in the sense that the read set
and write set we compute for a statement are over
approximations of the actual locations that are read or
written by the statement. Therefore it is possible that our
analysis reports two statement to be dependent when they are
not really dependent on each other. However, this can inhibit
some parallelizing optimization but can not result in an
incorrect parallelization. Function calls are handled by
using conservative read/write sets that over approximate the
heap locations that could potentially be read or written
inside the called function.
\begin{figure}
%\hrule
{\tt
  \begin{program}{0}
  \FL analyze(f, k)  \{
  \UNL{0} initialize(); \COMMENT{Initialize all parameters and
  globals}
  \UNL{0} States = computeStates(k)
  \UNL{0} computeReadWrite(States)
  \UNL{-1} \}
  \end{program}
}
%\hrule
  \caption{Algorithm to analyze a function {\tt f} for
    dependence detection. Parameter {\tt k} is used for
    limiting the length of access paths, to keep the analysis
    bounded. \label{fig:algoTopLevel}}
\hrule  
\end{figure}

Figure~\ref{fig:algoTopLevel} gives top level pseudo code for
analyzing a function {\tt f} in the program. The pseudo codes
for the utility functions used by the above code is given
in~\cite{barnali11thesis}. We describe their functionality in
brief.


Function {\tt initialize()} initializes the parameters of the
function and global variables with symbolic values. The
initialization information is also accessible to the shape
analyzer that requires it for making interference decisions.
  
The function {\tt computeStates(k)} computes the bindings of
pointer variables to the access paths using traditional
iterative data flow analysis.  The access paths are either
symbolic locations, e.g. $\loc$, or symbolic location
followed by pointer fields, e.g.
$\loc\rightarrow$\emph\mynext$\rightarrow$\emph{\tt next}. To
guarantee termination, we limit the length of access paths to
{\tt k}, that is a parameter to {\tt computeStates}. A
special summary field `{$*$}' is used to limit the access
paths. which stands for any field dereferenced beyond length
{\tt k}. Hence, for {\tt k = 1}, all the access paths in the
set \{\loc$\rightarrow${\tt next}$\rightarrow$\mynext,
\loc$\rightarrow${\tt
  next}$\rightarrow$\mynext$\rightarrow$\mynext,
\loc$\rightarrow$\mynext$\rightarrow${\tt
  next}$\rightarrow$\mynext$\cdots\rightarrow${\tt next}\}
can be abstracted as a single summarized path
\loc$\rightarrow$\mynext$\rightarrow{*}$. Similarly, assuming
a data structure has two reference fields \lt\ and \rt, the
summarized path
$\loc\rightarrow\lt\rightarrow\rt\rightarrow{*}$ could stand
for any of the access paths
$\loc\rightarrow\lt\rightarrow\rt\rightarrow\lt,\ \loc\rightarrow\lt\rightarrow\rt\rightarrow\rt,\ \loc\rightarrow\lt\rightarrow\rt\rightarrow\lt\rightarrow\lt,\ \loc\rightarrow\lt\rightarrow\rt\rightarrow\lt\rightarrow\rt$
and more such paths.
  
Given a variable to access path mapping {\tt States}, {\tt
  computeReadWrite(States)} computes read and write sets for
each statement in the program in a single pass. This
information can be used for computation of various
dependencies (flow, anti or output). Let $S$ and $S'$ be
statements in the program such that there exists an execution
path from $S$ to $S'$. Then, the dependence of $S'$ on $S$ is
computed as follows: \\
 %in Figure~\ref{fig:depcomp}.
\newcommand{\flowdep}{\mbox{flow-dep}}
\newcommand{\antidep}{\mbox{anti-dep}}
\newcommand{\outputdep}{\mbox{output-dep}}
\newcommand{\rs}[1]{\mbox{read}(#1)}
\newcommand{\ws}[1]{\mbox{write}(#1)}
\newcommand{\rst}[2]{\mbox{read}(#1, #2)}
\newcommand{\wst}[2]{\mbox{write}(#1, #2)}
\newcommand{\set}{\mbox{set}}
\noindent

$\begin{array}{rcl}
\interf{\set_1}{\set_2} &\equiv& \isInterfering(\p, \alpha, \q,
\beta) \; \mbox{where } 
\p.\alpha \in \set_1   \wedge  \q.\beta \in \set_2 \\ 
\flowdep(S, S') &\equiv& \interf{\ws{S}}{\rs{S'}} \\ 
\antidep(S, S') &\equiv& \interf{\rs{S}}{\ws{S'}} \\
\outputdep(S, S') &\equiv& \interf{\ws{S}}{\ws{S'}}
\end{array}$ \\

\noindent where \isInterfering\ is the function provided by
shape analysis.

\newcommand{\mapp}{$\equiv$}
\begin{table}
\centering
\caption{Simple dependence analysis for code in
  Figure~\ref{fig:motiv} \label{fig:table_simple_dep}}
\begin{tabular}{|c|l|c|c|c|}
\hline 
Stmt & Variables and Locations of Interest & Read Set & Write
Set \\ 
\hline
S1 & \p  \mapp\  \{$\loc$\} & $\emptyset$ & $\emptyset$ \\
S2 & \q \mapp\   \{$\loc\rightarrow$\mynext, $\loc\rightarrow$\mynext$\rightarrow *$\}& $\emptyset$ & $\emptyset$ \\
S3 & \q \mapp\  \{$\loc\rightarrow$\mynext, $\loc\rightarrow$\mynext$\rightarrow *$\} & \{$\loc\rightarrow$\mynext, $\loc\rightarrow$\mynext$\rightarrow *$\} & $\emptyset$ \\
S4 & \myr \mapp\   \{$\loc\rightarrow$\mynext$\rightarrow *$\} & $\emptyset$ & $\emptyset$ \\
S5 & \myr \mapp\ \{$\loc\rightarrow$\mynext$\rightarrow *$\} & $\emptyset$ & \{$\loc\rightarrow$\mynext$\rightarrow *$\} \\
S6 & \p \mapp\ \{$\loc\rightarrow$\mynext$\rightarrow *$\} & $\emptyset$ & $\emptyset$ \\
\hline
\end{tabular}

\noindent \loc\ is the symbolic location representing the
value of the variable {\tt list} at the start of the program.

\hrule
\end{table}

\begin{example} {\label{ex:dep}\rm
Table~\ref{fig:table_simple_dep} shows the state (pointer
variables and symbolic memory locations referred by the
variables) and the read and write sets for each statement in
the example code of Figure~\ref{fig:motiv}. From the table,
we can infer the following dependences:
\begin{enumerate}
\item loop independent anti-dependence  from statement {S3} to
statement {S5}
\item loop carried flow-dependence  from statement {S5} to
statement {S3}
\end{enumerate}
Note that the dependences inferred by our analysis are a
super-set of actual dependences that exist in the program.
  }\hfill\psframebox{} \end{example}

Next we explain how we can further refine our dependence
analysis to filter out some spurious dependences.


%%%%%%%%%%%%%%%%%%%%%%%%%%%%%%%%%%%%%%%%%%
\section{Loop Dependence Analysis}
As can be seen from the Example~\ref{ex:dep} above, our
approach, as explained earlier, does not work well for
loops. This is because it combines the paths being accessed
in different iterations of a loop. To get better result in
presence of loops we need to keep the accesses made by
different iterations of a loop separate. To do so, we have
devised another novel approach, which works as follows: Given
a loop, we first identify the navigators\footnote{a navigator
  consists of a pointer variable and a set of field
  references that are used to traverse a data structure
  inside the loop.}~\cite{ghiya98detecting} for the loop,
then by a single symbolic traversal over the loop, we compute
the read and write accesses made by each statement in terms
of the values of the navigators. Using the iteration number
as a parameter, the read and write sets are generalized to
{\em symbolic access paths}~\cite{Deutsch94klimit} to
represent arbitrary iteration of the loop.

Let $S$ and $S'$ be two statements inside a loop. Further,
let $\wst{S}{i}$ denote the set of access paths written by
statement $S$ in the iteration number $i$, and let
$\rst{S'}{j}$ denote the set of access paths read by
statement $S'$ in the iteration number $j$. Then, 
\begin{itemize}
\item $S'$ is loop independent flow dependent on $S$ if there
  is an execution path from $S$ to $S'$ that does not cross
  the loop boundary and there exist $i$ within loop
  bounds\footnote{in case loop bounds can not be computed at
    compile time, we can assume them to be $(-\infty,
    \infty)$} such that $\interf{\wst{S}{i}}{\rst{S'}{i}}$ is
  true.
\item $S'$ is loop dependent flow dependent on $S$ if there
  exist $i$ and $j$ within loop bounds such that $j > i$, and
  $\interf{\wst{S}{i}}{\rst{S'}{j}}$ is true.
\end{itemize}

We can similarly define loop independent and loop carried
anti-dependence and flow-dependence. The following example
explains our approach for the loop dependence detection.

\begin{figure}[t]
  \begin{center}
    \scalebox{.85}{\begin{tabular}{@{}c@{}|@{}c@{}}
      {\tt
\begin{program}{10}
  \FL\ \ldots
  \UNL{0} p = list;
  \UNL{0} \WHILE (p\rtarrow{next} != NULL) \{
  \NL{1}     \ldots = p\rtarrow{num};
  \NL{1}     p\rtarrow{next}\rtarrow{num} = \ldots;
  \NL{1}     p = p\rtarrow{next}\rtarrow{next};
  \UNL{0} \}
  \UNL{0} \ldots
\end{program}
}
      &
      {\tt
\begin{program}{20}
  \FL\ \ldots
  \UNL{0} p = list;
  \UNL{0} \WHILE (p\rtarrow{next} != NULL) \{
  \NL{1}     \ldots = p\rtarrow{num};
  \NL{1}     p\rtarrow{next}\rtarrow{num} = \ldots;
  \NL{1}     p = p\rtarrow{next};
  \UNL{0} \}
  \UNL{0} \ldots
\end{program}
} \\
      {(a) Loop without Dependence.} &
      {(b) Loop with Dependence.}
    \end{tabular}}
  \end{center}
  \caption{\label{fig:loopdep} Identifying Loop Dependences}
\hrule
\end{figure}
\begin{example}\label{ex:loopdeps}{\rm 
    Consider Figure~\ref{fig:loopdep}. Assume in each case
    \loc\ is the symbolic variable pointed to by the variable
    {\tt list}.
             
    For the code in Figure~\ref{fig:loopdep}(a), the
    navigator is $\langle\loc,
    \mynext\rightarrow\mynext\rangle$. Using $i$ to represent
    the iteration number, the generalized access path read by
    S11 is \loc$\rightarrow\mynext^{2i}\rightarrow\mynext$ and
    the generalized access path written by S12 is
    \loc$\rightarrow\mynext^{2i+1}\rightarrow\mynext$ Clearly
    there is no loop independent dependence. To find out loop
    carried dependence, we have to find out whether for
    iterations $i$ and $j$, the two paths point to the same
    heap location. This reduces to finding out if there is a
    possible solution to the following equation:
      \[ \loc\rightarrow\mynext^{2i}\rightarrow\mynext  =
      \loc\rightarrow\mynext^{2j+1}\rightarrow\mynext \] 
      In other words, we have to find out if  the following
      equation has integer solutions:
      \[2*i  = 2*j + 1\]
      GCD~\cite{Kennedy01Optimizing} or
      Lamport~\cite{Lamport1974parallel} test tell us that
      this equation can not have integer solutions. Thus,
      there is no dependence among the statements.

      For the code in Figure~\ref{fig:loopdep}(b), the navigator
      is $\langle\loc,
    \mynext\rangle$. In this case the equation to find 
      out the loop carried dependences among statements S21
      and S22 reduces to: 
      \[ i = j + 1\]
      which has integer solutions. So we have to
      conservatively report dependence between the two statements

In both the cases, we also need the shape analyzer to assert
that there is no sharing within the underlying data
structure. Had there been a sharing, we would have to report
conservatively that there exists a dependence.  }
\end{example}
%%%%%%%%%%%%%%%%%%%%%%%%%%%%%%%%%%%%%%%%%%
\section{Related Work}
Data dependence analysis for sequential programs, working on
only static and stack related data structure, such as array,
is well explored in literature~\cite{Allen87automatic,
  Wolf91loop, Banerjee93automatic, Kennedy01Optimizing} etc.
Our work extends the work to handle heap data
structure. Various approaches have been suggested for data
flow analysis of programs in the presence of dynamic data
structures. We describe briefly some of the earlier work done
in this area.
  
The work by Hendren et al. in ~\cite{Hendren90parallel}
considers shape information and approximates the
relationships between accessible nodes in larger aggregate
data structures. These relationships are represented by path
expressions, a restricted form of regular expressions, and
are encoded in path matrices. Such matrices are used to
deduce the interference information between any two heap
nodes, and to extract parallelism. Their method focuses on
three levels of parallelization; (a) if two statements can be
executed in parallel, (b) identifies procedure-call
parallelism, and (c) whether two sequences of statements can
be parallelized.

Ghiya et. al.~\cite{ghiya98detecting} uses coarse
characterization of the underlying data structure as Tree,
DAG or Cycle. They compute complete access
paths for each statement in terms of \emph{anchor} pointer,
which points to a fixed heap node in the data structure
within the whole body of the program.  The test for aliasing
of the access paths, relies on connection and shape
information that is automatically computed. They have also
extended their work to identify loop carried dependences for
loop level parallelism.

Hwang and Saltz~\cite{Hwang03Identify} present a technique to
identify parallelism in programs with cyclic graphs. The
method identifies the patterns of the traversal of program
code over the underlying data structure.  In the next step
the shape of the traversal pattern is detected. If the
traversal pattern is acyclic, dependence analysis is
performed to extract parallelism from the program.

Navarro et. al. in ~\cite{navarro05dependence, Navarro05loop}
propose a intra-procedural dependence test which intermixes
shape analysis and dependence analysis together. During the
analysis, the abstract structure of the dynamically allocated
data is computed and is also tagged with read/write tags to
find out dependencies. The resulting analysis is very
precise, but it is costly. Further their shape analysis
component is tightly integrated within the dependence
analysis, while in our approach we keep the two separate as
it gives us modularity and the scope to improve the precision
of our dependence analysis by using a more precise shape
analysis, if available. They have extended their dependence
related work in ~\cite{Navarro08irregular}, where
they have implemented a context-sensitive interprocedural
analysis which successfully detects dependences for both
non-recursive and recursive functions.

Work done by Marron et. al. in~\cite{Kapur08} tracks a two
program location, one read and one write location, for each
heap object field.  The technique uses an explicit store heap
model which captures the tag information of objects for each
program statement. The read and write information are used to
detect dependences. This space effective and time efficient
technique analyses bigger benchmarks in shorter time. But the
effectiveness of this approach lies in the use of predefined
semantics for library functions~\cite{Kapur06shape}, which
recognizes a traversal over a generic structure.


Our approach is closest to the technique proposed by Horwitz
et. al.~\cite{horwitz89dependence}. They also associate read
and write sets with each program statement to detect heap
dependencies. They have also proposed technique to compute
dependence distances for loop constructs. However, there
technique requires iterating over a loop till a fixed point
is reached, which is different from our method of computing
loop dependences as a set of equations in a single pass, and
then solving these equations using classical tests.

%%%%%%%%%%%%%%%%%%%%%%%%%%%%%%%%%%%%%%%%%%
\section{Conclusion and Future Work}

In this report we have presented our work on heap dependence
analysis that can be utilized by a parallelizing compiler to
parallelize sequential programs. Our method is divided into
three phases - the shape analysis phase, the state
computation phase, and the loop analysis phase, with
carefully chosen interfaces between phases to combine work
done by individual phases. This gives us the flexibility to
work on testing and improving each phase independently.  Our
loop dependence analysis abstracts the dependence information
in forms of linear equations, that can be solved using
traditional dependence analysis
tests~\cite{Allen87automatic,Wolf91loop,Banerjee93automatic}
that exist for finding array dependences.

Our analysis is intra-procedural, and we use conservative
approximation of function calls assuming worst case scenario.
Our next challenge is to develop an inter procedural analysis
to handle function calls more precisely. We have to further
develop our shape analysis technique and the loop analysis to
handle more of frequently occurring programming patterns to find
precise dependences for these patterns.  We  also want to
improve our summarization technique. Earlier we have used
graph based approximations of access
paths~\cite{khedker07heap} to compute liveness of heap
data. We plan to explore if the same summarization technique
can also be applied here.

Finally, to show that our analysis is practical, we are
developing a prototype model using GCC
compiler framework to show the effectiveness on large
benchmarks. However, this work is still in very early stages.

\bibliography{parrefs}

\end{document}
