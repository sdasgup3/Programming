\vspace{2in}
\begin{abstract}
Identifying dependences present in the body of sequential 
program is used by parallelizing compilers to 
automatically extract parallelism from the program. 
Dependence detection mechanisms for programs with scalar 
and static variables is well explored and have become a 
standard part of parallelizing and vectorizing compilers. 
However, detecting 
dependences in the presence of dynamic (heap) recursive 
data structures is highly complex because of the unbound and dynamic nature of 
the structure. 
The problem becomes more critical due to the presence of pointer-induced aliasing . 

This thesis addresses the aforementioned problem and gives a  
novel approach for dependence analysis of sequential programs 
in presence of heap data structure.  
The novelty of our technique lies in the two-phase mechanism, 
where the first phase identifies dependences for whole procedure. 
It computes abstract 
{\em heap access paths} for each heap accessing statement 
in the procedure. The access paths approximate the locations accessed 
by the statement. For each pair of statements these access paths are 
checked for interference. 
The second phase refines the dependence analysis in the context of 
loops. The main aspect of the second phase is the way we convert 
the precise access paths, for each statement, into equations
that can be solved using traditional tests, e.g. GCD test, and 
Lamport test. The technique discovers {\em loop dependences}, 
i.e. the dependence among two different iterations of the same 
loop. Further, we extend the intra-procedural analysis to inter-procedural one. 
\end{abstract} 
