We view the heap structure at a program point as a directed
graph, the nodes of which represent the allocated objects and
the edges represent the connectivity through pointer fields.
Pictorially, inside a node we show all the relevant pointer
variables that can point to the heap object corresponding to
that node. The edges are labeled by the name of the
corresponding pointer field. In this paper, we only label
nodes and edges that are relevant to the discussion, to avoid
clutter.

Let $\heap$ denotes the set of all heap directed
pointers at a particular program point and $\fields$
denotes the set of all pointer fields at that program point.
Given two heap-directed pointers $\p$, $\q$ $\in$
$\heap$, a path from $\p$ to $\q$ is the sequence
of pointer fields that need to be traversed in the heap to
reach from $\p$ to $\q$.  The length of a path is
defined as the number of pointer fields in the path.  As the
path length between two heap objects may be unbounded, we
keep track of only the first field of a path\footnote{
%%The
%%  decision to use only first field is guided by the fact that
%%  in our language, a statement can use at most one field,
%%  i.e. {\tt p$\rightarrow$f = \ldots} or {\tt \ldots = p$\rightarrow$f}. 
  While it is possible to use prefixes of any fixed length,
  it does not make any fundamental change to our
  analysis.}. To distinguish between a path of length one
(direct path) from a path of length greater than one
(indirect path) that start at the same field, we use the
superscript $\drct$ for a direct path and $\indrct$ for an
indirect path. In pictures, we use solid edges for direct
paths, and dotted edges for indirect paths.

It is also possible to have multiple paths between two
pointers starting at a given field $f$, with at most one
direct path $f^\drct$. However, the number of indirect paths
$f^\indrct$ may be unbounded. As there can only be a finite
number of first fields, we store first fields of paths,
including the count for the indirect paths, between two
pointer variables in a set. To bound the size of the set, we
put a limit $k$ on number of repetitions of a particular
field. If the number goes beyond $k$, we treat the number of
paths with that field as $\infty$.
%% This approach is similar to the approach
%% of$k$-limiting~\cite{Jones79}.  and
%% $sl$-limiting~\cite{Larus88}.

\begin{example} {\rm
\begin{figure}[t]
\centering
\newcommand{\smf}{$f$}%%\scriptstyle f$}
\newcommand{\smh}{$h$}%%\scriptstyle h$}
\newcommand{\smg}{$g$}%%\scriptstyle g$}
\begin{tabular}{@{}c|c@{}}
  {\tt
    \begin{tabular}[b]{l}
      {\em \scriptsize S1:} q = p; \\
      {\em \scriptsize S2:} {\bf while}(...) \{ \\
      {\em \scriptsize S3:} \ \ \ \ q$\rightarrow$g = s; \\
      {\em \scriptsize S4:} \ \ \ \ q = q$\rightarrow$f; \\
      {\em \scriptsize S5:} \} \\
    \end{tabular}
  } &
  \scalebox{0.8}{ \psset{unit=1mm}
    \begin{pspicture}(0,0)(50,30)
      %%\psframe(0,0)(50,30)
      \putnode{q0}{origin}{3}{3}{}
      \putnode{q1}{q0}{0}{0}{\pscirclebox{\mbox{\p}}}
      \putnode{q2}{q1}{10}{0}{\pscirclebox[framesep=2.5]{\mbox{}}}
      \putnode{q3}{q2}{10}{0}{\mbox{\ldots}}
      \putnode{q4}{q3}{10}{0}{\pscirclebox{\mbox{\q}}}
      \putnode{q5}{q4}{10}{0}{\pscirclebox[framesep=2.5]{\mbox{}}}
      
      \ncline{->}{q1}{q2}
      \Bput[0.2]{\smf}
      \ncline{->}{q2}{q3}
      \Bput[0.2]{\smf}
      \ncline{->}{q3}{q4}
      \Bput[0.2]{\smf}
      \ncline{->}{q4}{q5}
      \Bput[0.2]{\smf}
      
      \putnode{s0}{q4}{15}{15}{\pscirclebox{\mbox{\s}}}
      \nccurve[angleA=90,angleB=145,ncurv=1]{->}{q1}{s0}
      \Aput[.1]{\smg}
      
      \putnode{t0}{q2}{0}{15}{\pscirclebox[framesep=2.5]{\mbox{}}}
      \nccurve[linestyle=dotted, dotsep=.5, angleA=-55,angleB=-165,
        ncurv=.2,nodesepA=-.3]{->}{t0}{s0}
      
      \ncline[linestyle=dotted, dotsep=.5]{->}{t0}{s0}
      \nccurve[linestyle=dotted, dotsep=.5,angleA=55,angleB=165,
        ncurv=.3,nodesepA=-.3]{->}{t0}{s0}
      
      
      \ncline[nodesepA=-.5,nodesepB=-.5]{->}{q1}{t0}
      \Aput[.1]{\smh}
      
      \ncline{->}{q2}{s0}
      \Bput[0.2]{\smg}
      \ncline[nodesepA=-.75,nodesepB=-.75]{->}{q4}{s0}
      \Bput[0.2]{\smg}
  \end{pspicture}} \\
  \scalebox{0.80}{ (a) } &
  \scalebox{0.80}{ (b) }
\end{tabular}
\caption{Paths in a heap graph.  Graph in (b) is a possible
  heap graph for code in (a). Solid edges are the direct
  paths, dotted edges are the indirect paths.
  \label{fig:pathsAndSets}}
\end{figure}


Figure~\ref{fig:pathsAndSets}(a) shows a code fragment and
Fig.~\ref{fig:pathsAndSets}(b) shows a possible heap graph
at a program point after line {\tt S5}. In any execution, there is one path
between $\p$ and $\q$, starting with field $f$, whose
length is statically unknown. This information is stored by
our analysis as the set $\{\fieldI{f}{}{1}\}$. Further, there are
unbounded number of paths between $\p$ and $\s$, all
starting with field $f$. There is also a direct path from
$\p$ to $\s$ using field $g$, and 3 paths starting with
field $h$ between $\p$ and $\s$. Assuming the limit $k
\geq 3$, this information can be represented by the set $\{
g^\drct, f^{\indrct\infty}, h^{\indrct 3} \}$. On the other hand, if $k < 3$,
then the set would be $\{ g^\drct, f^{\indrct\infty}, h^{\indrct\infty} \}$.
  } \hfill\psframebox{}
\end{example}

\newcommand{\anysup}{\ensuremath{\ast}} For brevity, we use
$f^{\anysup}$ for the cases when we do not want to
distinguish between direct or indirect path starting at the
first field $f$. We now define the field sensitive matrices
used by our analysis.

%FIELD SENSITIVE PATH MATRIX 
\begin{definition}
\label{DFM_matrix}
Field sensitive Path matrix
$P_F$ is a matrix that stores information 
about paths between two pointer variables.
Given $\p, \q \in
\heap, f \in \fields$:
\begin{eqnarray*}
  \epsilon & \in \DFM{p}{p}& \mbox{ where $\epsilon$
    denotes the empty path.} \\
  f^\drct  &\in  \DFM{p}{q} & \mbox{ if there is a direct
    path $f$ from $\p$ to $\q$.}\\
  f^{\indrct m} & \in  \DFM{p}{q} & 
  \mbox{\begin{tabular}[t]{p{55mm}}if there are $m$ indirect
      paths starting with field $f$ from $\p$ to $\q$ and $m
      \leq k.$
    \end{tabular}
  } \\
  f^{\indrct\infty} & \in  \DFM{p}{q} &
  \mbox{\begin{tabular}[t]{p{55mm}}if there are $m$ indirect
      paths starting with field $f$ from $\p$ to $\q$ and $m >
      k.$
  \end{tabular}}  
\end{eqnarray*}
\end{definition}

Let \nat\ denote the set of natural numbers. We define the
following partial order for approximate paths used by our
analysis. For $ f \in \fields,\ m,n \in \nat,\ n \leq m$:
$$
\epsilon \sqsubseteq \epsilon, \quad 
f^\drct \sqsubseteq  f^\drct,  \quad
f^{\indrct\infty}  \sqsubseteq  f^{\indrct\infty}, \quad
f^{\indrct m} \sqsubseteq f^{\indrct\infty}, \quad
f^{\indrct n} \sqsubseteq f^{\indrct m}. 
$$
The partial order is extended to set of paths $S_{P_1},
S_{P_2}$ as\footnote{Note that for our analysis, for a given
  field $f$, these sets contain at most one entry of type
  $f^\drct$ and at most one entry of type $f^\indrct$}:
\begin{eqnarray*}
  S_{P_1} \sqsubseteq S_{P_2} &\Leftrightarrow& \forall \alpha \in
  S_{P_1}, \exists \beta \in S_{P_2}\ s.t. \alpha \sqsubseteq \beta
\end{eqnarray*}
For pair of paths:
%%\begin{eqnarray*}
$  (\alpha, \beta) \sqsubseteq (\alpha', \beta') 
  \Leftrightarrow 
   (\alpha \sqsubseteq \alpha')  \wedge
  (\beta \sqsubseteq  \beta'), $
%\end{eqnarray*}
and for set of pairs of paths $R_{P_1}, R_{P_2}$:
%%\begin{eqnarray*}
  $R_{P_1} \sqsubseteq R_{P_2} \Leftrightarrow \forall
  (\alpha, \beta) \in
  R_{P_1}, \exists (\alpha', \beta') \in
  R_{P_2}\ s.t. (\alpha, \beta) \sqsubseteq (\alpha', \beta').$
%%\end{eqnarray*}


Two pointers $\p,\q \in \heap$ are said to
interfere if there exists $\s \in \heap$ such that both
$\p$ and $\q$ have paths reaching $\s$. Note that $\s$ could
be $\p$ (or $\q$) itself, in which case the path from $\p$
(from $\q$) is $\epsilon$.

%FIELD SENSITIVE PATH MATRIX 
\begin{definition}\label{IFM_matrix}
Field sensitive Interference matrix $I_F$ between
two pointers captures the ways in which these pointers are
interfering.  For $\p, \q, \s \in \heap, \p \not= \q$,
the following relation holds for $P_F$ and $I_F$: 
\begin{eqnarray*}
  \DFM{p}{s} \times \DFM{q}{s} &\sqsubseteq& \IFM{p}{q} \label {eq:rel-df-if}
\end{eqnarray*}
\end{definition}

Our analysis computes over-approximations for the matrices
$P_F$ and $I_F$ at each program point. While it is possible
to compute only $P_F$ and use above equation to
compute $I_F$, computing both explicitly results in better
approximations for $I_F$. Note that interference relation is
symmetric, i.e.,
\begin{eqnarray*}
  (\alpha, \beta) \in I_F[\p, \q] \Leftrightarrow
   (\beta, \alpha) \in I_F[\q, \p]
\end{eqnarray*}
While describing the analysis, we use the
above relation to show the computation of only one of the two
entries. 
\begin{figure}[!t]
  \centering
  \begin{tabular}{@{}c|c@{}}
    \psset{unit=1.2mm}
    \begin{pspicture}(0,0)(18,20)
      %  \psframe(0,0)(15,20)
      \putnode{q0}{origin}{3}{9}{\pscirclebox{\q}}
      \putnode{p0}{q0}{5}{7}{\pscirclebox{\p}}
      \putnode{r0}{p0}{0}{-14}{\pscirclebox{\myr}}
      \putnode{s0}{q0}{10}{0}{\pscirclebox{\s}}
      
      \ncline[nodesep=-.3]{->}{p0}{q0}
      \bput[0.2](0.2){$\scriptstyle f$}
      \ncline{->}{q0}{s0}
      \aput[0.2](0.2){$\scriptstyle g$}
      \ncline[nodesep=-.5]{->}{s0}{r0}
      \aput[0.2](0.2){$\scriptstyle h$}
      \nccurve[angleA=45,angleB=0,linestyle=dotted,dotsep=0.2,ncurv=1,nodesepA=-.7]{->}{p0}{s0}
      \aput[0.2](0.2){$\scriptstyle g$}
      \nccurve[angleA=180,angleB=225,linestyle=dotted,dotsep=0.2,ncurv=1,nodesepB=-.5]{->}{q0}{r0}
      \bput[0.2](0.2){$\scriptstyle f$}
    \end{pspicture} &
    \scalebox{0.8}{
      \renewcommand{\arraystretch}{1.2}
      \begin{tabular}[b]{|c|c|c|c|c|}
        \hline
        $P_F$& \p & \q &  \s &  \myr \\ \hline \hline 
        \p   & $\{\epsilon\}$ & $\{\fieldD{f}{}\}$ & $\{\fieldI{f}{}{1}, \fieldI{g}{}{1}\}$ & $\{\fieldI{f}{}{2}, \fieldI{g}{}{1}\}$  \\ \hline 
        \q   & $\emptyset$ & $\{\epsilon\}$ & $\{\fieldD{g}{}\}$ & $\{\fieldI{g}{}{1}, \fieldI{f}{}{1}\}$ \\ \hline
        \s   & $\emptyset$ & $\emptyset$ & $\{\epsilon\}$ & $\{\fieldD{h}{}\}$ \\ \hline
        \myr & $\emptyset$ & $\emptyset$ & $\emptyset$ & $\{\epsilon\}$ \\ \hline
    \end{tabular} } \\
    \scalebox{0.80}{ (a) } &
    \scalebox{0.80}{ (b) }  \\
  \end{tabular}
  \caption{A heap graph (a) and its field sensitive path
    matrix (b).\label{fig:DFM_IFM}}
\end{figure}



\begin{example}{\rm
Figure~\ref{fig:DFM_IFM} shows a heap graph and the
corresponding field sensitive matrices as computed by our
analysis.  } \hfill\psframebox{}
\end{example}

As mentioned earlier, for each variable $\p \in \heap$, our
analysis uses attributes $\p_{\subD}$ and $\p_{\subC}$ to
store boolean functions telling whether $\p$ can reach a DAG or
cycle respectively in the heap. The boolean functions
consist of the values from matrices $P_F$, $I_F$, and the field
connectivity information. For $f \in \fields, \p, \q \in
\heap$, field connectivity is captured by boolean variables
of the form $f_{pq}$, which is true when $f$ field of $\p$ points
directly to $\q$. 
The shape of \p, \p.\shape, can be obtained by evaluating
the functions for the attributes $\p_{\subC}$ and
$\p_{\subD}$, and using Table~\ref{tbl:det_shape}.
\begin{table}
\caption{Determining shape from boolean
  attributes\label{tbl:det_shape}}
\begin{center}
%\scalebox{0.8}{
%\begin{tabular*}{0.75\textwidth}{@{\extracolsep{\fill}} |c|c|c| }
\begin{tabular}{|c|c|c|}
\hline
$\p_{\subC}$ & $\p_{\subD}$ & $\p.\shape$ \\ 
\hline
\true  & Don't Care  & Cycle        \\ 
\false  & \true          & DAG    \\ 
\false  & \false          & Tree   \\ 
\hline
\end{tabular}
%}
\end{center}
\end{table}

We use the following operations in our analysis. Let $S$
denote the set of approximate paths between two nodes, $P$
denote a set of pair of paths, and $k \in \nat$ denotes the
limit on maximum indirect paths stored for a given
field. Then,
\begin{itemize}\is
\item Projection: For $f \in \fields$,
  \project{S}{f}\ extracts the paths starting at field $f$.
  $$\project{S}{f} \equiv S \cap \{f^{\drct}, f^{\indrct 1}, \ldots,
  f^{\indrct k}, f^{\indrct \infty}\}$$

\item Counting: The count on the number of paths is defined
  as :
  \begin{eqnarray*}
   \num{\epsilon} &=&     1 \\ %\qquad \qquad 
   \num{f^{\drct}} &=&     1 \\
   \num{f^{\indrct\infty}} &=&     \infty \\
   \num{f^{\indrct j}} &=&  j \mbox{ for } j \in \nat  \\
   \num{S} &=&   \sum_{\alpha \in S}\num{\alpha}
  \end{eqnarray*}

{\blue  
  Also,  
  \begin{eqnarray*}
   \num{(\alpha, f^{\indrct m})} &=&  \left\{ \begin{array}{@{}ll}  
   			m        &  \mbox{if } \alpha \in \{f^{\drct}, \epsilon\} \\
			m*n      &  \mbox{if } \alpha = f^{\indrct n} \\
			\infty   &  \mbox{if } \alpha = f^{\indrct\infty} \\
			\end{array}\right. \\
   \num{(f^{\indrct m}, \beta)} &=&  \left\{ \begin{array}{@{}ll}  
   			m        &  \mbox{if } \beta \in \{f^{\drct}, \epsilon\} \\
			m*n      &  \mbox{if } \beta = f^{\indrct n} \\
			\infty   &  \mbox{if } \beta = f^{\indrct\infty} \\
			\end{array}\right. \\
   \num{(\alpha, \beta)} &=&  \begin{array}{@{}ll}
   			 1       &  \mbox{if } \alpha, \beta \in \{f^{\drct}, \epsilon\} \\
			\end{array} \\
   \num{(f^{\indrct\infty}, \beta)} &=&  \begin{array}{@{}ll}
   			 \infty       &  \mbox{where }  \beta \in \{f^{\drct}, \epsilon, f^{\indrct\infty}\} \\
			\end{array} \\
   \num{(\alpha, f^{\indrct\infty})} &=&  \begin{array}{@{}ll}
   			 \infty       &  \mbox{where } \alpha \in \{f^{\drct}, \epsilon, f^{\indrct\infty}\} \\
			\end{array} \\
	\num{P} &=&   \sum_{(\alpha, \beta) \in P}\num{(\alpha, \beta)}		
  \end{eqnarray*}
  }

\item Path removal, intersection and union over set of
  approximate paths : For singleton sets of paths
    $\{\alpha\}$ and $\{\beta\}$, path removal
    (\remOne{\{\alpha\}}{\{\beta\}}), intersection
    ($\{\alpha\} \cap \{\beta\}$) and union($\{\alpha\} \cup
    \{\beta\}$) operations are defined as given in
    Table~\ref{tab:path-ops}. These definitions can be extended
    to set of paths in a natural way. For example,
	for general sets of paths, $S_1$ and $S_2$, 
	the definition of removal can be extended as:
\begin{eqnarray*}
  \remOne{S_1}{S_2}  = \bigcap_{\beta \in
    {S_2}}(\bigcup_{\alpha \in {S_1}}  \remOne{\{\alpha\}}{\{\beta\}}  )
\end{eqnarray*}

\begin{table}[t]
\caption{Path operations\label{tab:path-ops}}
\begin{center}
\scalebox{1}{
  \renewcommand{\arraystretch}{1.1}
  \begin{tabular}{@{}c@{}}
    $i, j  \in \nat$,$\quad m = {\tt max}(i-j, 0)$,  $\quad n = {\tt
      min}(i, j)$, \\
    $\tau = \left\{\begin{array}{lcl}
    i+j    && \mbox { if } i+j \leq k \\
    \infty && \mbox{ Otherwise} 
    \end{array}\right.$. \\ 
    $\eta = \left\{\begin{array}{lcl}
    i*j    && \mbox { if } i*j \leq k \\
    \infty && \mbox{ Otherwise} 
    \end{array}\right.$. \\  \\
    (a) Path removal \\
    \begin{tabular}{@{}c@{}c@{}|c|c|c|c|c}
      \hline
      \remOne{}{}&$\{\beta\} $& $\{\epsilon\}$ &
      $\{f^{\drct}\}$ & $\{f^{\indrct j}\}$&
      $\{f^{\indrct\infty}\}$ & $g^{\anysup}$ \\
      $\{\alpha\}$ && & & & & \\ \hline %\hline
      $\{\epsilon\}$ && $\emptyset$  & $\{\epsilon\}$ &
      $\{\epsilon\}$ & $\{\epsilon\}$ & $\{\epsilon\}$ \\ %\hline
      $\{f^{\drct}\}$ && $\{f^{\drct}\}$ & $\emptyset$ &
      $\{f^{\drct}\}$ & $\{f^{\drct}\}$ & $\{f^{\drct}\}$ \\ %\hline
      $\{f^{\indrct i}\}$ && $\{f^{\indrct i}\}$ & $\emptyset$
      & $\{f^{\indrct m}\}$& $\emptyset$ & $\{f^{\indrct i}\}$ \\ %\hline
      $\{f^{\indrct\infty}\}$ && $\{f^{\indrct\infty}\}$
      &$\emptyset$ & $\{f^{\indrct \infty}\}$& $\{f^{\indrct
        \infty}\}$ & $\{f^{\indrct\infty}\}$ \\ \hline
    \end{tabular} \\ \\
    (b) Intersection \\ 
    \begin{tabular}{@{}c@{}c@{}|c|c|c|c|c@{}}
      \hline
      $\cap$&$\{\beta\} $& $\{\epsilon\}$ & $\{f^{\drct}\}$ &
      $\{f^{\indrct j}\}$& $\{f^{\indrct\infty}\}$ & $g^{\anysup}$\\ 
      $\{\alpha\}$ && & & & & \\ \hline %\hline
      $\{\epsilon\}$ && $\epsilon$  & $\emptyset$ & $\emptyset$ & $\emptyset$ & $\emptyset$ \\ %\hline
      $\{f^{\drct}\}$ && $\emptyset$ & $\{f^{\drct}\}$ & $\emptyset$ & $\emptyset$ & $\emptyset$ \\ %\hline
      $\{f^{\indrct i}\}$ && $\emptyset$ & $\emptyset$ & $\{f^{\indrct n}\}$& $\{f^{\indrct i}\}$ & $\emptyset$ \\ %\hline
      $\{f^{\indrct\infty}\}$ && $\emptyset$ &$\emptyset$ & $\{f^{\indrct j}\}$& $\{f^{\indrct \infty}\}$ & $\emptyset$ \\ \hline
    \end{tabular} \\ \\
    (c) Union \\
    \begin{tabular}{@{}c@{}c@{}|@{}c@{}|@{}c@{}|@{}c@{}|@{}c@{}|@{}c@{}}
      \hline
      $\cup$&$\{\beta\} $& $\{\epsilon\}$ & $\{f^{\drct}\}$ &
      $\{f^{\indrct j}\}$& $\{f^{\indrct\infty}\}$ & $g^{\anysup}$ \\
      $\{\alpha\}$ && & & & &  
      \\ \hline %\hline
      $\{\epsilon\}$ && $\{\epsilon\}$  & $\{\epsilon, \fieldD{f}{}\}$ & $\{\epsilon, \fieldI{f}{}{j} \}$ & $\{\epsilon, \fieldI{f}{}{\infty}\}$ & $\{\epsilon, g^{\anysup}\}$  \\ %\hline
      $\{f^{\drct}\}$ && $\{\fieldD{f}{}, \epsilon\}$ & $\{\fieldD{f}{}\}$ & $\{\fieldD{f}{}, \fieldI{f}{}{j}\}$ & $\{\fieldD{f}{}, \fieldI{f}{}{\infty}\}$ & $\{\fieldD{f}{}, g^{\anysup}\}$  \\ %\hline
      $\{f^{\indrct i}\}$ && $\{\fieldI{f}{}{i}, \epsilon\}$ & $\{\fieldI{f}{}{i}, \fieldD{f}{}\}$ & $\{\fieldI{f}{}{\tau}\}$& $\{\fieldI{f}{}{\infty}\}$ & $\{\fieldI{f}{}{i}, g^{\anysup}\}$  \\ %\hline
      $\{f^{\indrct\infty}\}$ && $\{\fieldI{f}{}{\infty},
      \epsilon\}$ & $\{\fieldI{f}{}{\infty}, \fieldD{f}{}\}$ &
      $\{\fieldI{f}{}{\infty}\}$& $\{\fieldI{f}{}{\infty}\}$ &
      $\{\fieldI{f}{}{\infty}, g^{\anysup}\}$  \\ \hline
    \end{tabular} \\ \\
    (d) Multiplication by a Scalar \\
    \begin{tabular}{l|c|c|c|c}
      \hline
      $\star\qquad \alpha$ & $\epsilon$ & $f^{\drct}$      &
      ${f^{\indrct j}}$ & ${f^{\indrct \infty}}$ \\ 
      $i$ &&&& \\ \hline 
      $i$ & $\epsilon$ & ${f^{\indrct i}}$ & $f^{\indrct \eta}$
      & ${f^{\indrct \infty}}$ \\ 
      $\infty$ & $\epsilon$ &${f^{\indrct \infty}}$ &
      ${f^{\indrct \infty}}$& ${f^{\indrct \infty}}$\\ \hline 
    \end{tabular}
  \end{tabular}}
\end{center}
\end{table}
	

\item {\blue Path removal, intersection and union over set of pair
  of paths : For singleton sets of paths
    $\{\alpha, \beta\}$ and $\{\gamma, \delta\}$, union($\{\alpha, \beta\} \cup
    \{\gamma, \delta\}$), intersection
    ($\{\alpha, \beta\} \cap  \{\gamma, \delta\}$) and path removal
    (\remOne{\{\alpha, \beta\}}{\{\gamma, \delta\}}) operations are defined as given in
    Figure~\ref{tab:union_path_pair}, ~\ref{tab:intersection_path_pair} and ~\ref{tab:removal_path_pair} respectively. As before these 
	definitions can be extended to set of pair of paths in a natural way. For example,
	for general sets of paths, $P_1$ and $P_2$, 
	the definition of removal can be extended as:
\begin{eqnarray*}
  \remOne{P_1}{P_2}  = \bigcap_{(\gamma, \delta) \in
    {P_2}}(\bigcup_{(\alpha, \beta) \in {P_1}}  \remOne{\{\alpha, \beta\}}{\{\gamma, \delta\}}  )
\end{eqnarray*}

\begin{figure*}
\caption{Union operation between two singleton sets of pair of paths, where $\{\alpha, \beta\}$ denote any other pair of paths and
$x$, $y$, $m$, $n$ can be any positive integer or $\infty$
\label{tab:union_path_pair}}
\begin{tabular}{c}
\\
\scalebox{0.65}{
\begin{tabular}{|c|c|c|c|c|c|}
\hline
$\cup$ & \{(\fieldD{f}{},\fieldD{g}{})\} & \{(\fieldI{f}{}{m},\fieldD{g}{})\} & \{(\fieldD{f}{},\fieldI{g}{}{n})\} & \{(\fieldI{f}{}{m},\fieldI{g}{}{n})\} & \{($\alpha$, $\beta$)\}\\ \hline

\{($\epsilon$,$\epsilon$)\} & \{($\epsilon$,$\epsilon$),(\fieldD{f}{},\fieldD{g}{}),  & \{($\epsilon$,$\epsilon$),(\fieldI{f}{}{m},\fieldD{g}{}),  & \{($\epsilon$,$\epsilon$),(\fieldD{f}{},\fieldI{g}{}{n}),  & \{($\epsilon$,$\epsilon$),(\fieldI{f}{}{m},\fieldI{g}{}{n}),  & \{($\epsilon$,$\epsilon$),\\
& ($\epsilon$,\fieldD{g}{}),(\fieldD{f}{},$\epsilon$ )\} & ($\epsilon$,\fieldD{g}{}),(\fieldI{f}{}{m},$\epsilon$ )\} & ($\epsilon$,\fieldI{g}{}{n}),(\fieldD{f}{},$\epsilon$ )\} & ($\epsilon$,\fieldI{g}{}{n}),(\fieldI{f}{}{m},$\epsilon$ )\} & ($\alpha$, $\beta$) \}\\ \hline

\{($\epsilon$,\fieldD{g}{})\} & \{($\epsilon$,\fieldD{g}{}),(\fieldD{f}{},\fieldD{g}{})\} & \{($\epsilon$,\fieldD{g}{}),(\fieldI{f}{}{m},\fieldD{g}{})\} & \{($\epsilon$,\fieldD{g}{}),(\fieldD{f}{},\fieldI{g}{}{n}),  & \{($\epsilon$,\fieldD{g}{}),(\fieldI{f}{}{m},\fieldI{g}{}{n}),  & \{($\epsilon$,\fieldD{g}{}), \\
& & & ($\epsilon$,\fieldI{g}{}{n}),(\fieldD{f}{},\fieldD{g}{})\} & ($\epsilon$,\fieldI{g}{}{n}),(\fieldI{f}{}{m},\fieldD{g}{})\} & ($\alpha$, $\beta$)\}\\ \hline

\{($\epsilon$,\fieldI{g}{}{y})\} & \{($\epsilon$,\fieldI{g}{}{y}),(\fieldD{f}{},\fieldD{g}{}) & \{($\epsilon$,\fieldI{g}{}{y}),(\fieldI{f}{}{m},\fieldD{g}{}) & \{($\epsilon$,\fieldI{g}{}{n+y}),(\fieldD{f}{},\fieldI{g}{}{n+y})\} & \{($\epsilon$,\fieldI{g}{}{n+y}),(\fieldI{f}{}{m},\fieldI{g}{}{n+y})\} & \{($\epsilon$,\fieldI{g}{}{y}), \\
& ($\epsilon$,\fieldD{g}{}),(\fieldD{f}{},\fieldI{g}{}{y})\} & ($\epsilon$,\fieldD{g}{}),(\fieldI{f}{}{m},\fieldI{g}{}{y})\} & & & ($\alpha$, $\beta$)\}\\ \hline

\{(\fieldD{f}{},$\epsilon$)\} & \{(\fieldD{f}{},$\epsilon$),(\fieldD{f}{},\fieldD{g}{})\} & \{(\fieldD{f}{},$\epsilon$),(\fieldI{f}{}{m},\fieldD{g}{}) & \{(\fieldD{f}{},$\epsilon$),(\fieldD{f}{},\fieldI{g}{}{n})\} & \{(\fieldD{f}{},$\epsilon$),(\fieldI{f}{}{m},\fieldI{g}{}{n}) & \{(\fieldD{f}{},$\epsilon$), \\
& & (\fieldI{f}{}{m},$\epsilon$),(\fieldD{f}{},\fieldD{g}{})\} & & (\fieldI{f}{}{m},$\epsilon$). (\fieldD{f}{},\fieldI{g}{}{n})\} & ($\alpha$, $\beta$)\}\\ \hline

\{(\fieldI{f}{}{x},$\epsilon$)\} & \{(\fieldI{f}{}{x},$\epsilon$),(\fieldD{f}{},\fieldD{g}{}) & \{(\fieldI{f}{}{x+m},$\epsilon$),(\fieldI{f}{}{x+m},\fieldD{g}{})\} & \{(\fieldI{f}{}{x},$\epsilon$),(\fieldD{f}{},\fieldI{g}{}{n}) & \{(\fieldI{f}{}{x+m},$\epsilon$),(\fieldI{f}{}{x+m},\fieldI{g}{}{n})\} & \{(\fieldI{f}{}{x},$\epsilon$), \\
& (\fieldI{f}{}{x},\fieldD{g}{}),(\fieldD{f}{},$\epsilon$)\} & & (\fieldI{f}{}{x},\fieldI{g}{}{n}),(\fieldD{f}{},$\epsilon$)\} & & ($\alpha$, $\beta$)\}\\ \hline

\{(\fieldD{f}{},\fieldD{g}{})\} & \{(\fieldD{f}{},\fieldD{g}{})\} & \{(\fieldD{f}{},\fieldD{g}{}), & \{(\fieldD{f}{},\fieldD{g}{}),  & \{(\fieldD{f}{},\fieldD{g}{}),(\fieldD{f}{},\fieldI{g}{}{n}), & \{(\fieldD{f}{},\fieldD{g}{}),\\
& & (\fieldI{f}{}{m},\fieldD{g}{})\} & (\fieldD{f}{},\fieldI{g}{}{n})\} & (\fieldI{f}{}{m},\fieldD{g}{}) , (\fieldI{f}{}{m},\fieldI{g}{}{n})\} & ($\alpha$, $\beta$)\} \\ \hline

\{(\fieldI{f}{}{x},\fieldD{g}{})\} & \{(\fieldI{f}{}{x},\fieldD{g}{}), & \{(\fieldI{f}{}{x+m},\fieldD{g}{})\} & \{(\fieldI{f}{}{x},\fieldD{g}{}) , (\fieldD{f}{},\fieldI{g}{}{n}), & \{(\fieldI{f}{}{x+m},\fieldD{g}{}), & \{(\fieldI{f}{}{x},\fieldD{g}{}), \\
& (\fieldD{f}{},\fieldD{g}{})\} & & (\fieldI{f}{}{x},\fieldI{g}{}{n}) , (\fieldD{f}{},\fieldD{g}{})\} & (\fieldI{f}{}{x+m},\fieldI{g}{}{n})\} & ($\alpha$, $\beta$)\}\\ \hline

 \{(\fieldD{f}{},\fieldI{g}{}{y})\} & \{(\fieldD{f}{},\fieldD{g}{}), & \{(\fieldD{f}{},\fieldI{g}{}{y}) , (\fieldI{f}{}{m},\fieldD{g}{}), & \{(\fieldD{f}{},\fieldI{g}{}{n+y})\} & \{(\fieldD{f}{},\fieldI{g}{}{n+y}),  (\fieldD{f}{},\fieldI{g}{}{y}), & \{(\fieldD{f}{},\fieldI{g}{}{y}) , \\
& (\fieldD{f}{},\fieldI{g}{}{y})\} & (\fieldI{f}{}{m},\fieldI{g}{}{y}) , (\fieldD{f}{},\fieldD{g}{}), & & (\fieldI{f}{}{m},\fieldI{g}{}{n+y})\} & ($\alpha$, $\beta$)\}\\ \hline

 \{(\fieldI{f}{}{x},\fieldI{g}{}{y})\} & \{(\fieldD{f}{},\fieldD{g}{}),(\fieldI{f}{}{x},\fieldI{g}{}{y}), & \{(\fieldI{f}{}{x+m},\fieldD{g}{}), & \{(\fieldD{f}{},\fieldI{g}{}{n+y}), & \{(\fieldI{f}{}{m+x},\fieldI{g}{}{n+y})\} & \{(\fieldI{f}{}{x},\fieldI{g}{}{y}),  \\
& (\fieldD{f}{},\fieldI{g}{}{x}),(\fieldI{f}{}{x},\fieldD{g}{})\} & (\fieldI{f}{}{x+m},\fieldI{g}{}{y})\} & (\fieldI{f}{}{x},\fieldI{g}{}{n+y})\} & & ($\alpha$, $\beta$)\} \\ \hline

\end{tabular}
}
\\ \\
\scalebox{0.65}{
\begin{tabular}{|c|c|c|c|c|c|c|}
\hline
$\cup$ & \{($\epsilon$,$\epsilon$)\} & \{($\epsilon$,\fieldD{g}{})\} & \{($\epsilon$,\fieldI{g}{}{n})\} & \{(\fieldD{f}{},$\epsilon$)\} & \{(\fieldI{f}{}{m},$\epsilon$)\} & \{($\alpha$, $\beta$)\}\\ \hline

\{($\epsilon$,$\epsilon$)\} & \{($\epsilon$,$\epsilon$)\} & \{($\epsilon$,$\epsilon$),($\epsilon$,\fieldD{g}{})\} & \{($\epsilon$,$\epsilon$),($\epsilon$,\fieldI{g}{}{n})\} & \{($\epsilon$,$\epsilon$),(\fieldD{f}{},$\epsilon$)\} & \{($\epsilon$,$\epsilon$),(\fieldI{f}{}{x},$\epsilon$)\} & \{($\epsilon$,$\epsilon$)\} \\
&&&&&& ($\alpha$, $\beta$)\} \\\hline

\{($\epsilon$,\fieldD{g}{})\} & \{($\epsilon$,\fieldD{g}{}),($\epsilon$,$\epsilon$)\} & \{($\epsilon$,\fieldD{g}{})\} & \{($\epsilon$,\fieldD{g}{}),($\epsilon$,\fieldI{g}{}{n})\} & \{($\epsilon$,\fieldD{g}{}),(\fieldD{f}{},$\epsilon$), & \{($\epsilon$,\fieldD{g}{}),(\fieldI{f}{}{m},$\epsilon$), & \{($\epsilon$,\fieldD{g}{}),  \\
& & & & ($\epsilon$,$\epsilon$),(\fieldD{f}{},\fieldD{g}{})\} & ($\epsilon$,$\epsilon$),(\fieldI{f}{}{m},\fieldD{g}{})\} & ($\alpha$, $\beta$)\} \\ \hline

\{($\epsilon$,\fieldI{g}{}{y})\} & \{($\epsilon$,\fieldI{g}{}{y}),($\epsilon$,$\epsilon$)\} & \{($\epsilon$,\fieldI{g}{}{y}),($\epsilon$,\fieldD{g}{})\} & \{($\epsilon$,\fieldI{g}{}{y+n})\} & \{($\epsilon$,\fieldI{g}{}{y}),(\fieldD{f}{},$\epsilon$), & \{($\epsilon$,\fieldI{g}{}{y}),(\fieldI{f}{}{m},$\epsilon$), & \{($\epsilon$,\fieldI{g}{}{y}),\\
& & & & ($\epsilon$,$\epsilon$),(\fieldD{f}{},\fieldI{g}{}{y})\} & ($\epsilon$,$\epsilon$),(\fieldI{f}{}{m},\fieldI{g}{}{y})\} & ($\alpha$, $\beta$)\}\\ \hline

\{(\fieldD{f}{},$\epsilon$)\} & \{(\fieldD{f}{},$\epsilon$),($\epsilon$,$\epsilon$)\} & \{(\fieldD{f}{},$\epsilon$),($\epsilon$,\fieldD{g}{}), & \{(\fieldD{f}{},$\epsilon$),($\epsilon$,\fieldI{g}{}{n}), & \{(\fieldD{f}{},$\epsilon$)\} & \{(\fieldD{f}{},$\epsilon$),(\fieldI{f}{}{x},$\epsilon$)\} & \{(\fieldD{f}{},$\epsilon$),\\ 
& & (\fieldD{f}{},\fieldD{g}{}),($\epsilon$,$\epsilon$)\} & (\fieldD{f}{},\fieldI{g}{}{n}),($\epsilon$,$\epsilon$)\} & & & ($\alpha$, $\beta$)\}\\ \hline

\{(\fieldI{f}{}{x},$\epsilon$)\} & \{(\fieldI{f}{}{x},$\epsilon$),($\epsilon$,$\epsilon$)\} & \{(\fieldI{f}{}{x},$\epsilon$),($\epsilon$,\fieldD{g}{}), & \{(\fieldI{f}{}{x},$\epsilon$),($\epsilon$,\fieldI{g}{}{n}), & \{(\fieldI{f}{}{x},$\epsilon$),(\fieldD{f}{},$\epsilon$)\} & \{(\fieldI{f}{}{x+m},$\epsilon$)\} & \{(\fieldI{f}{}{x},$\epsilon$), \\ 
& & (\fieldI{f}{}{x},\fieldD{g}{}),($\epsilon$,$\epsilon$)\} & (\fieldI{f}{}{x},\fieldI{g}{}{n}),($\epsilon$,$\epsilon$)\} & & & ($\alpha$, $\beta$)\}\\ \hline 

\{(\fieldD{f}{},\fieldD{g}{})\} & \{(\fieldD{f}{},\fieldD{g}{}),($\epsilon$,$\epsilon$), & \{(\fieldD{f}{},\fieldD{g}{}),($\epsilon$,\fieldD{g}{})\} & \{(\fieldD{f}{},\fieldD{g}{}),($\epsilon$,\fieldI{g}{}{n}), & \{(\fieldD{f}{},\fieldD{g}{}),(\fieldD{f}{},$\epsilon$)\} & \{(\fieldD{f}{},\fieldD{g}{}),(\fieldI{f}{}{x},$\epsilon$), & \{(\fieldD{f}{},\fieldD{g}{}),  \\
& (\fieldD{f}{},$\epsilon$),($\epsilon$,\fieldD{g}{})\} & & (\fieldD{f}{},\fieldI{g}{}{n}),($\epsilon$,\fieldD{g}{})\} & & (\fieldD{f}{},$\epsilon$),(\fieldI{f}{}{m},\fieldD{g}{})\} & ($\alpha$, $\beta$)\}\\ \hline 

\{(\fieldI{f}{}{x},\fieldD{g}{})\} & \{(\fieldI{f}{}{x},\fieldD{g}{}),($\epsilon$,$\epsilon$), & \{(\fieldI{f}{}{x},\fieldD{g}{}),($\epsilon$,\fieldD{g}{})\} & \{(\fieldI{f}{}{x},\fieldD{g}{}),($\epsilon$,\fieldI{g}{}{n}), & \{(\fieldI{f}{}{x},\fieldD{g}{}),(\fieldD{f}{},$\epsilon$), & \{(\fieldI{f}{}{x+m},\fieldD{g}{}),(\fieldI{f}{}{x+m},$\epsilon$)\} & \{(\fieldI{f}{}{x},\fieldD{g}{}), \\
& (\fieldI{f}{}{x},$\epsilon$),($\epsilon$,\fieldD{g}{})\} & & (\fieldI{f}{}{x},\fieldI{g}{}{n}),($\epsilon$,\fieldD{g}{})\} & (\fieldI{f}{}{x},$\epsilon$),(\fieldD{f}{},\fieldD{g}{})\} & & ($\alpha$, $\beta$)\}\\ \hline

\{(\fieldD{f}{},\fieldI{g}{}{y})\} & \{(\fieldD{f}{},\fieldI{g}{}{y}),($\epsilon$,$\epsilon$), & \{(\fieldD{f}{},\fieldI{g}{}{y}),($\epsilon$,\fieldD{g}{}), & \{(\fieldD{f}{},\fieldI{g}{}{y+n}),($\epsilon$,\fieldI{g}{}{y+n})\} & \{(\fieldD{f}{},\fieldI{g}{}{y}),(\fieldD{f}{},$\epsilon$)\} & \{(\fieldD{f}{},\fieldI{g}{}{y}),(\fieldI{f}{}{m},$\epsilon$), & \{(\fieldD{f}{},\fieldI{g}{}{y}), \\ 
& (\fieldD{f}{},$\epsilon$),($\epsilon$,\fieldI{g}{}{y})\} & (\fieldD{f}{},\fieldD{g}{}),($\epsilon$,\fieldI{g}{}{y})\} & & & (\fieldD{f}{},$\epsilon$),(\fieldI{f}{}{m},\fieldI{g}{}{y})\} & ($\alpha$, $\beta$)\}\\ \hline

\{(\fieldI{f}{}{x},\fieldI{g}{}{y})\} & \{(\fieldI{f}{}{x},\fieldI{g}{}{y}),($\epsilon$,$\epsilon$), & \{(\fieldI{f}{}{x},\fieldI{g}{}{y}),($\epsilon$,\fieldD{g}{}), & \{(\fieldI{f}{}{x},\fieldI{g}{}{y+n}),($\epsilon$,\fieldI{g}{}{y+n})\} & \{(\fieldI{f}{}{x},\fieldI{g}{}{y}),(\fieldD{f}{},$\epsilon$), & \{(\fieldI{f}{}{x+m},\fieldI{g}{}{y}),(\fieldI{f}{}{x+m},$\epsilon$)\} & \{(\fieldI{f}{}{x},\fieldI{g}{}{y}), \\ 
& (\fieldI{f}{}{x},$\epsilon$),($\epsilon$,\fieldI{g}{}{y})\} & (\fieldI{f}{}{x},\fieldD{g}{}),($\epsilon$,\fieldI{g}{}{y})\} & & (\fieldI{f}{}{x},$\epsilon$),(\fieldD{f}{},\fieldI{g}{}{y})\} & & ($\alpha$, $\beta$)\}\\ \hline

\end{tabular}
}\\
\end{tabular}
\end{figure*}

\begin{figure*}
\caption{Intersection operation between two singleton sets of pair of paths, where $\{\alpha, \beta\}$ denote any other pair of paths and $a \star b$ = min(a,b).
\label{tab:intersection_path_pair}}
\begin{tabular}{c}
\\
\scalebox{0.65}{ 
\begin{tabular}{|c|c|c|c|c|c|c|c|c|c|}
\hline

$\bigcap$ & \{($\epsilon$,$\epsilon$)\} & \{($\epsilon$,\fieldD{g}{})\} & \{($\epsilon$,\fieldI{g}{}{n})\} & \{($\epsilon$,\fieldI{g}{}{\infty})\} & \{(\fieldD{f}{},$\epsilon$)\} & \{(\fieldD{f}{},\fieldD{g}{})\} & \{(\fieldD{f}{},\fieldI{g}{}{n})\} & \{(\fieldD{f}{},\fieldI{g}{}{\infty})\} & \{($\alpha$, $\beta$)\} \\ \hline


\{($\epsilon$,$\epsilon$)\} & \{($\epsilon$,$\epsilon$)\} & $\phi$ & $\phi$ & $\phi$ & $\phi$ & $\phi$ & $\phi$ & $\phi$ & $\phi$\\ \hline

\{($\epsilon$,\fieldD{g}{})\} & $\phi$ & \{($\epsilon$,\fieldD{g}{})\} & $\phi$ & $\phi$ & $\phi$ & $\phi$ & $\phi$ & $\phi$ & $\phi$\\ \hline

\{($\epsilon$,\fieldI{g}{}{y})\} & $\phi$ & $\phi$ & \{($\epsilon$,\fieldI{g}{}{y \star n})\} & \{($\epsilon$,\fieldI{g}{}{y})\} & $\phi$ & $\phi$ & $\phi$ & $\phi$ & $\phi$\\ \hline

\{($\epsilon$,\fieldI{g}{}{\infty})\} & $\phi$ & $\phi$ & \{($\epsilon$,\fieldI{g}{}{n})\} & \{($\epsilon$,\fieldI{g}{}{\infty})\} & $\phi$ & $\phi$ & $\phi$ & $\phi$ & $\phi$\\ \hline

\{(\fieldD{f}{},$\epsilon$)\} & $\phi$ & $\phi$ & $\phi$ & $\phi$ & \{(\fieldD{f}{},$\epsilon$)\} & $\phi$ & $\phi$ & $\phi$ & $\phi$\\ \hline

\{(\fieldD{f}{},\fieldD{g}{})\} & $\phi$ & $\phi$ & $\phi$ & $\phi$ & $\phi$ & \{(\fieldD{f}{},\fieldD{g}{})\} & $\phi$ & $\phi$ & $\phi$\\ \hline
 
\{(\fieldD{f}{},\fieldI{g}{}{y})\} & $\phi$ & $\phi$ & $\phi$ & $\phi$ & $\phi$ & $\phi$ & \{(\fieldD{f}{},\fieldI{g}{}{y \star n})\} & \{(\fieldD{f}{},\fieldI{g}{}{y})\} & $\phi$\\ \hline

\{(\fieldD{f}{},\fieldI{g}{}{\infty})\} & $\phi$ & $\phi$ & $\phi$ & $\phi$ & $\phi$ & $\phi$ & \{(\fieldD{f}{},\fieldI{g}{}{n})\} & \{(\fieldD{f}{},\fieldI{g}{}{\infty})\} & $\phi$\\ \hline

\{(\fieldI{f}{}{x},$\epsilon$)\} & $\phi$ & $\phi$ & $\phi$ & $\phi$ & $\phi$ & $\phi$ & $\phi$ & $\phi$ & $\phi$\\ \hline

\{(\fieldI{f}{}{x},\fieldD{g}{})\} & $\phi$ & $\phi$ & $\phi$ & $\phi$ & $\phi$ & $\phi$ & $\phi$ & $\phi$ & $\phi$\\ \hline

\{(\fieldI{f}{}{x},\fieldI{g}{}{y})\} & $\phi$ & $\phi$ & $\phi$ & $\phi$ & $\phi$ & $\phi$ & $\phi$ & $\phi$ & $\phi$\\ \hline

\{(\fieldI{f}{}{x},\fieldI{g}{}{\infty})\} & $\phi$ & $\phi$ & $\phi$ & $\phi$ & $\phi$ & $\phi$ & $\phi$ & $\phi$ & $\phi$\\ \hline

\{(\fieldI{f}{}{\infty},$\epsilon$)\} & $\phi$ & $\phi$ & $\phi$ & $\phi$ & $\phi$ & $\phi$ & $\phi$ & $\phi$ & $\phi$\\ \hline

\{(\fieldI{f}{}{\infty},\fieldD{g}{})\} & $\phi$ & $\phi$ & $\phi$ & $\phi$ & $\phi$ & $\phi$ & $\phi$ & $\phi$ & $\phi$\\ \hline

\{(\fieldI{f}{}{\infty},\fieldI{g}{}{y})\} & $\phi$ & $\phi$ & $\phi$ & $\phi$ & $\phi$ & $\phi$ & $\phi$ & $\phi$ & $\phi$\\ \hline

\{(\fieldI{f}{}{\infty},\fieldI{g}{}{\infty})\} & $\phi$ & $\phi$ & $\phi$ & $\phi$ & $\phi$ & $\phi$ & $\phi$ & $\phi$ & $\phi$\\ \hline


\end{tabular}
}

\\ \\ \\

\scalebox{0.65}{
\begin{tabular}{|c|c|c|c|c|c|c|c|c|c|}
\hline

$\bigcap$ & \{(\fieldI{f}{}{m},$\epsilon$)\} & \{(\fieldI{f}{}{m},\fieldD{g}{})\} & \{(\fieldI{f}{}{m},\fieldI{g}{}{n})\} & \{(\fieldI{f}{}{m},\fieldI{g}{}{\infty})\} & \{(\fieldI{f}{}{\infty},$\epsilon$)\} & \{(\fieldI{f}{}{\infty},\fieldD{g}{})\} & \{(\fieldI{f}{}{\infty},\fieldI{g}{}{n})\} & \{(\fieldI{f}{}{\infty},\fieldI{g}{}{\infty})\} & \{($\alpha$, $\beta$)\}\\ \hline


\{($\epsilon$,$\epsilon$)\} & $\phi$ & $\phi$ & $\phi$ & $\phi$ & $\phi$ & $\phi$ & $\phi$ & $\phi$ & $\phi$\\ \hline

\{($\epsilon$,\fieldD{g}{})\} & $\phi$ & $\phi$ & $\phi$ & $\phi$ & $\phi$ & $\phi$ & $\phi$ & $\phi$ & $\phi$\\ \hline

\{($\epsilon$,\fieldI{g}{}{y})\} & $\phi$ & $\phi$ & $\phi$ & $\phi$ & $\phi$ & $\phi$ & $\phi$ & $\phi$ & $\phi$\\ \hline

\{($\epsilon$,\fieldI{g}{}{\infty})\} & $\phi$ & $\phi$ & $\phi$ & $\phi$ & $\phi$ & $\phi$ & $\phi$ & $\phi$ & $\phi$\\ \hline

\{(\fieldD{f}{},$\epsilon$)\} & $\phi$ & $\phi$ & $\phi$ & $\phi$ & $\phi$ & $\phi$ & $\phi$ & $\phi$ & $\phi$\\ \hline

\{(\fieldD{f}{},\fieldD{g}{})\} & $\phi$ & $\phi$ & $\phi$ & $\phi$ & $\phi$ & $\phi$ & $\phi$ & $\phi$ & $\phi$\\ \hline

\{(\fieldD{f}{},\fieldI{g}{}{y})\} & $\phi$ & $\phi$ & $\phi$ & $\phi$ & $\phi$ & $\phi$ & $\phi$ & $\phi$ & $\phi$ \\ \hline

\{(\fieldD{f}{},\fieldI{g}{}{\infty})\} & $\phi$ & $\phi$ & $\phi$ & $\phi$ & $\phi$ & $\phi$ & $\phi$ & $\phi$ & $\phi$\\ \hline

\{(\fieldI{f}{}{x},$\epsilon$)\} & \{(\fieldI{f}{}{x \star m},$\epsilon$)\} & $\phi$ & $\phi$ & $\phi$ & \{(\fieldI{f}{}{x},$\epsilon$)\} & $\phi$ & $\phi$ & $\phi$ & $\phi$\\ \hline

\{(\fieldI{f}{}{x},\fieldD{g}{})\} & $\phi$ & \{(\fieldI{f}{}{x \star m},\fieldD{g}{})\} & $\phi$ & $\phi$ & $\phi$ & \{(\fieldI{f}{}{x},\fieldD{g}{})\} & $\phi$ & $\phi$ & $\phi$\\ \hline

\{(\fieldI{f}{}{x},\fieldI{g}{}{y})\} & $\phi$ & $\phi$ & \{(\fieldI{f}{}{x \star m},\fieldI{g}{}{y \star n})\} & \{(\fieldI{f}{}{x \star m},\fieldI{g}{}{y})\} & $\phi$ & $\phi$ & \{(\fieldI{f}{}{x},\fieldI{g}{}{y \star n})\} & \{(\fieldI{f}{}{x},\fieldI{g}{}{y})\} & $\phi$\\ \hline

\{(\fieldI{f}{}{x},\fieldI{g}{}{\infty})\} & $\phi$ & $\phi$ & \{(\fieldI{f}{}{x \star m},\fieldI{g}{}{n})\} & \{(\fieldI{f}{}{x \star m},\fieldI{g}{}{\infty})\} & $\phi$ & $\phi$ & \{(\fieldI{f}{}{x},\fieldI{g}{}{n})\} & \{(\fieldI{f}{}{x},\fieldI{g}{}{n})\} & $\phi$\\ \hline

\{(\fieldI{f}{}{\infty},$\epsilon$)\} & \{(\fieldI{f}{}{m},$\epsilon$)\} & $\phi$ & $\phi$ & $\phi$ & \{(\fieldI{f}{}{\infty},$\epsilon$)\} & $\phi$ & $\phi$ & $\phi$ & $\phi$\\ \hline

\{(\fieldI{f}{}{\infty},\fieldD{g}{})\} & $\phi$ & \{(\fieldI{f}{}{m},\fieldD{g}{})\} & $\phi$ & $\phi$ & $\phi$ & \{(\fieldI{f}{}{\infty},\fieldD{g}{})\} & $\phi$ & $\phi$  & $\phi$\\ \hline

\{(\fieldI{f}{}{\infty},\fieldI{g}{}{y})\} & $\phi$ & $\phi$ & \{(\fieldI{f}{}{m},\fieldI{g}{}{y \star n})\} & \{(\fieldI{f}{}{m},\fieldI{g}{}{y})\} & $\phi$ & $\phi$ & \{(\fieldI{f}{}{\infty},\fieldI{g}{}{y \star n})\} & \{(\fieldI{f}{}{\infty},\fieldI{g}{}{y})\} & $\phi$\\ \hline

\{(\fieldI{f}{}{\infty},\fieldI{g}{}{\infty})\} & $\phi$ & $\phi$ & \{(\fieldI{f}{}{m},\fieldI{g}{}{n})\} & \{(\fieldI{f}{}{m},\fieldI{g}{}{\infty})\} &  $\phi$ &  $\phi$ & \{(\fieldI{f}{}{\infty},\fieldI{g}{}{n})\} & \{(\fieldI{f}{}{\infty},\fieldI{g}{}{\infty})\} & $\phi$\\ \hline

\end{tabular}
}\\
\end{tabular}
\end{figure*}

\begin{figure*}
\caption{Removal operation between two singleton sets of pair of paths, where $\{\alpha, \beta\}$ denote any other pair of paths and $a\#b$ = max(a-b,0).
\label{tab:removal_path_pair}}
\begin{tabular}{c}
\\
\scalebox{0.65}{
\begin{tabular}{|c|c|c|c|c|c|c|c|c|c|}
\hline

\remOne{}{} & \{($\epsilon$,$\epsilon$)\} & \{($\epsilon$,\fieldD{g}{})\} & \{($\epsilon$,\fieldI{g}{}{n})\} & \{($\epsilon$,\fieldI{g}{}{\infty})\} & \{(\fieldD{f}{},$\epsilon$)\} & \{(\fieldD{f}{},\fieldD{g}{})\} & \{(\fieldD{f}{},\fieldI{g}{}{n})\} & \{(\fieldD{f}{},\fieldI{g}{}{\infty})\} & \{($\alpha$, $\beta$)\}\\ \hline

\{($\epsilon$,$\epsilon$)\} & $\phi$ & \{($\epsilon$,$\epsilon$)\} & \{($\epsilon$,$\epsilon$)\} & \{($\epsilon$,$\epsilon$)\} & \{($\epsilon$,$\epsilon$)\} & \{($\epsilon$,$\epsilon$)\} & \{($\epsilon$,$\epsilon$)\} & \{($\epsilon$,$\epsilon$)\} & \{($\epsilon$,$\epsilon$)\}\\ \hline

\{($\epsilon$,\fieldD{g}{})\} & $\phi$ & $\phi$ & \{($\epsilon$,\fieldD{g}{})\} & \{($\epsilon$,\fieldD{g}{})\} & \{($\epsilon$,\fieldD{g}{})\} & \{($\epsilon$,\fieldD{g}{})\} & \{($\epsilon$,\fieldD{g}{})\} & \{($\epsilon$,\fieldD{g}{})\} & \{($\epsilon$,\fieldD{g}{})\}\\ \hline

\{($\epsilon$,\fieldI{g}{}{y})\} &  $\phi$ & $\phi$ & \{($\epsilon$,\fieldI{g}{}{y})\} & \{($\epsilon$,\fieldI{g}{}{y})\} & \{($\epsilon$,\fieldI{g}{}{y})\} & $\phi$ & \{($\epsilon$,\fieldI{g}{}{y})\} & \{($\epsilon$,\fieldI{g}{}{y\#n})\} & \{($\epsilon$,\fieldI{g}{}{y})\}\\ \hline

\{($\epsilon$,\fieldI{g}{}{\infty})\} & $\phi$ & $\phi$ & \{($\epsilon$,\fieldI{g}{}{\infty})\} & \{($\epsilon$,\fieldI{g}{}{\infty})\} & \{($\epsilon$,\fieldI{g}{}{\infty})\} & $\phi$ & \{($\epsilon$,\fieldI{g}{}{\infty})\} & \{($\epsilon$,\fieldI{g}{}{\infty})\} & \{($\epsilon$,\fieldI{g}{}{\infty})\}\\ \hline

\{(\fieldD{f}{},$\epsilon$)\} & $\phi$ & \{(\fieldD{f}{},$\epsilon$)\} & \{(\fieldD{f}{},$\epsilon$)\} & \{(\fieldD{f}{},$\epsilon$)\} & $\phi$ & $\phi$ & $\phi$ & $\phi$& \{(\fieldD{f}{},$\epsilon$)\} \\ \hline

\{(\fieldD{f}{},\fieldD{g}{})\} & \{(\fieldD{f}{},\fieldD{g}{})\} & $\phi$ & \{(\fieldD{f}{},\fieldD{g}{})\} & \{(\fieldD{f}{},\fieldD{g}{})\} & $\phi$ & $\phi$ & $\phi$ & $\phi$ & \{(\fieldD{f}{},\fieldD{g}{})\} \\ \hline

\{(\fieldD{f}{},\fieldI{g}{}{y})\} & \{(\fieldD{f}{},\fieldI{g}{}{y})\} & $\phi$ & \{(\fieldD{f}{},\fieldI{g}{}{y\#n})\} & $\phi$ & $\phi$ & $\phi$ & $\phi$ & $\phi$ & \{(\fieldD{f}{},\fieldI{g}{}{y})\}  \\ \hline

\{(\fieldD{f}{},\fieldI{g}{}{\infty})\} & \{(\fieldD{f}{},\fieldI{g}{}{\infty})\} & $\phi$ & \{(\fieldD{f}{},\fieldI{g}{}{\infty})\} & \{(\fieldD{f}{},\fieldI{g}{}{\infty})\} & $\phi$ & $\phi$ & $\phi$ & $\phi$ & \{(\fieldD{f}{},\fieldI{g}{}{\infty})\} \\ \hline

\{(\fieldI{f}{}{x},$\epsilon$)\} & $\phi$ & \{(\fieldI{f}{}{x},$\epsilon$)\} & \{(\fieldI{f}{}{x},$\epsilon$)\} & \{(\fieldI{f}{}{x},$\epsilon$)\} & $\phi$ & $\phi$ & $\phi$ & $\phi$ & \{(\fieldI{f}{}{x},$\epsilon$)\} \\ \hline

\{(\fieldI{f}{}{x},\fieldD{g}{})\} & \{(\fieldI{f}{}{x},\fieldD{g}{})\} & $\phi$ & \{(\fieldI{f}{}{x},\fieldD{g}{})\} & \{(\fieldI{f}{}{x},\fieldD{g}{})\} & $\phi$ & $\phi$ & $\phi$ & $\phi$ & \{(\fieldI{f}{}{x},\fieldD{g}{})\} \\ \hline

\{(\fieldI{f}{}{x},\fieldI{g}{}{y})\} & \{(\fieldI{f}{}{x},\fieldI{g}{}{y})\} & $\phi$ & \{(\fieldI{f}{}{x},\fieldI{g}{}{y\#n})\} & $\phi$ & $\phi$ & $\phi$ & $\phi$ & $\phi$ & \{(\fieldI{f}{}{x},\fieldI{g}{}{y}) \} \\ \hline

\{(\fieldI{f}{}{x},\fieldI{g}{}{\infty})\} & \{(\fieldI{f}{}{x},\fieldI{g}{}{\infty})\} & $\phi$ & \{(\fieldI{f}{}{x},\fieldI{g}{}{\infty})\} & \{(\fieldI{f}{}{x},\fieldI{g}{}{\infty})\} & $\phi$ & $\phi$ & $\phi$ & $\phi$ & \{(\fieldI{f}{}{x},\fieldI{g}{}{\infty})  \\ \hline

\{(\fieldI{f}{}{\infty},$\epsilon$)\} & $\phi$ & \{(\fieldI{f}{}{\infty},$\epsilon$)\} & \{(\fieldI{f}{}{\infty},$\epsilon$)\} & \{(\fieldI{f}{}{\infty},$\epsilon$)\} & $\phi$ & $\phi$ & $\phi$ & $\phi$ & \{(\fieldI{f}{}{\infty},$\epsilon$)\}\\ \hline

\{(\fieldI{f}{}{\infty},\fieldD{g}{})\} & \{(\fieldI{f}{}{\infty},\fieldD{g}{})\} & $\phi$ & \{(\fieldI{f}{}{\infty},\fieldD{g}{})\} & \{(\fieldI{f}{}{\infty},\fieldD{g}{})\} & $\phi$ & $\phi$ & $\phi$ & $\phi$ & \{(\fieldI{f}{}{\infty},\fieldD{g}{})\}  \\ \hline

\{(\fieldI{f}{}{\infty},\fieldI{g}{}{y})\} & \{(\fieldI{f}{}{\infty},\fieldI{g}{}{y})\} & $\phi$ & \{(\fieldI{f}{}{\infty},\fieldI{g}{}{y\#n})\} & $\phi$ & $\phi$ & $\phi$ & $\phi$ & $\phi$ & \{(\fieldI{f}{}{\infty},\fieldI{g}{}{y})\}  \\ \hline

\{(\fieldI{f}{}{\infty},\fieldI{g}{}{\infty})\} & \{(\fieldI{f}{}{\infty},\fieldI{g}{}{\infty})\} & $\phi$ & \{(\fieldI{f}{}{\infty},\fieldI{g}{}{\infty})\} & \{(\fieldI{f}{}{\infty},\fieldI{g}{}{\infty})\} & $\phi$ & $\phi$ & $\phi$ & $\phi$ & \{(\fieldI{f}{}{\infty},\fieldI{g}{}{\infty})\}  \\ \hline

\end{tabular}
}\\ \\
\scalebox{0.65}{
\begin{tabular}{|c|c|c|c|c|c|c|c|c|c|}
\hline

\remOne{}{} & \{(\fieldI{f}{}{m},$\epsilon$)\} & \{(\fieldI{f}{}{m},\fieldD{g}{})\} & \{(\fieldI{f}{}{m},\fieldI{g}{}{n})\} & \{(\fieldI{f}{}{m},\fieldI{g}{}{\infty})\} & \{(\fieldI{f}{}{\infty},$\epsilon$)\} & \{(\fieldI{f}{}{\infty},\fieldD{g}{})\} & \{(\fieldI{f}{}{\infty},\fieldI{g}{}{n})\} & \{(\fieldI{f}{}{\infty},\fieldI{g}{}{\infty})\} & \{($\alpha$,$\beta$)\}\\ \hline


\{($\epsilon$,$\epsilon$)\} & $\phi$ & \{($\epsilon$,$\epsilon$)\} & \{($\epsilon$,$\epsilon$)\} & \{($\epsilon$,$\epsilon$)\} & \{($\epsilon$,$\epsilon$)\} & \{($\epsilon$,$\epsilon$)\} & \{($\epsilon$,$\epsilon$)\} & \{($\epsilon$,$\epsilon$)\} & \{($\epsilon$,$\epsilon$)\}\\ \hline

\{($\epsilon$,\fieldD{g}{})\} & \{($\epsilon$,\fieldD{g}{})\} & $\phi$ & \{($\epsilon$,\fieldD{g}{})\} & \{($\epsilon$,\fieldD{g}{})\} & \{($\epsilon$,\fieldD{g}{})\} & $\phi$ & \{($\epsilon$,\fieldD{g}{})\} & \{($\epsilon$,\fieldD{g}{})\} & \{($\epsilon$,\fieldD{g}{})\}\\ \hline
 
\{($\epsilon$,\fieldI{g}{}{y})\} & \{($\epsilon$,\fieldI{g}{}{y})\} & $\phi$ & \{($\epsilon$,\fieldI{g}{}{y\#n})\} & $\phi$ & \{($\epsilon$,\fieldI{g}{}{y})\} & $\phi$ & \{($\epsilon$,\fieldI{g}{}{y})\} & $\phi$  & \{($\epsilon$,\fieldI{g}{}{y})\}\\ \hline
 
\{($\epsilon$,\fieldI{g}{}{\infty})\} & \{($\epsilon$,\fieldI{g}{}{\infty})\} & $\phi$ & \{($\epsilon$,\fieldI{g}{}{\infty})\} & \{($\epsilon$,\fieldI{g}{}{\infty})\} & \{($\epsilon$,\fieldI{g}{}{\infty})\} & $\phi$ & \{($\epsilon$,\fieldI{g}{}{\infty})\} & \{($\epsilon$,\fieldI{g}{}{\infty})\} & \{($\epsilon$,\fieldI{g}{}{\infty})\}\\ \hline
 
\{(\fieldD{f}{},$\epsilon$)\} & $\phi$ & \{(\fieldD{f}{},$\epsilon$)\} & \{(\fieldD{f}{},$\epsilon$)\} & \{(\fieldD{f}{},$\epsilon$)\} & $\phi$ & \{(\fieldD{f}{},$\epsilon$)\} & \{(\fieldD{f}{},$\epsilon$)\} & \{(\fieldD{f}{},$\epsilon$)\} & \{(\fieldD{f}{},$\epsilon$)\}\\ \hline

\{(\fieldD{f}{},\fieldD{g}{})\} & \{(\fieldD{f}{},\fieldD{g}{})\} & $\phi$ & \{(\fieldD{f}{},\fieldD{g}{})\} & \{(\fieldD{f}{},\fieldD{g}{})\} & \{(\fieldD{f}{},\fieldD{g}{})\} & $\phi$ & \{(\fieldD{f}{},\fieldD{g}{})\} & \{(\fieldD{f}{},\fieldD{g}{})\} & \{(\fieldD{f}{},\fieldD{g}{})\}\\ \hline

\{(\fieldD{f}{},\fieldI{g}{}{y})\} & \{(\fieldD{f}{},\fieldI{g}{}{y})\} & $\phi$ & \{(\fieldD{f}{},\fieldI{g}{}{y\#n})\} &  $\phi$ & \{(\fieldD{f}{},\fieldI{g}{}{y})\} & $\phi$ & \{(\fieldD{f}{},\fieldI{g}{}{y\#n})\} & $\phi$ & \{(\fieldD{f}{},\fieldI{g}{}{y})\}\\ \hline

\{(\fieldD{f}{},\fieldI{g}{}{\infty})\} & \{(\fieldD{f}{},\fieldI{g}{}{\infty})\} & $\phi$ & \{(\fieldD{f}{},\fieldI{g}{}{\infty})\} & \{(\fieldD{f}{},\fieldI{g}{}{\infty})\} & \{(\fieldD{f}{},\fieldI{g}{}{\infty})\} & $\phi$ & \{(\fieldD{f}{},\fieldI{g}{}{\infty})\} & \{(\fieldD{f}{},\fieldI{g}{}{\infty})\} & \{(\fieldD{f}{},\fieldI{g}{}{\infty})\}\\ \hline

\{(\fieldI{f}{}{x},$\epsilon$)\} & \{(\fieldI{f}{}{x},$\epsilon$)\} & \{(\fieldI{f}{}{x\#m},$\epsilon$)\} & \{(\fieldI{f}{}{x\#m},$\epsilon$)\} & \{(\fieldI{f}{}{x\#m},$\epsilon$)\} & \{(\fieldI{f}{}{x},$\epsilon$)\} & $\phi$ & $\phi$ & $\phi$ & \{(\fieldI{f}{}{x},$\epsilon$)\}\\ \hline

\{(\fieldI{f}{}{x},\fieldD{g}{})\} & \{(\fieldI{f}{}{x\#m},\fieldD{g}{})\} & $\phi$ & \{(\fieldI{f}{}{x\#m},\fieldD{g}{})\} & \{(\fieldI{f}{}{x\#m},\fieldD{g}{})\} & $\phi$ & $\phi$ & $\phi$ & $\phi$ & \{(\fieldI{f}{}{x},\fieldD{g}{})\}\\ \hline
 
\{(\fieldI{f}{}{x},\fieldI{g}{}{y})\} & \{(\fieldI{f}{}{x\#m},\fieldI{g}{}{y})\} & $\phi$ & \{(\fieldI{f}{}{x\#m},\fieldI{g}{}{y\#n})\} & $\phi$ & $\phi$ & $\phi$ & $\phi$ & $\phi$ & \{(\fieldI{f}{}{x},\fieldI{g}{}{y})\}\\ \hline

\{(\fieldI{f}{}{x},\fieldI{g}{}{\infty})\} & \{(\fieldI{f}{}{x\#m},\fieldI{g}{}{\infty})\} & $\phi$ & \{(\fieldI{f}{}{x\#m},\fieldI{g}{}{\infty})\} & \{(\fieldI{f}{}{x\#m},\fieldI{g}{}{\infty})\} & $\phi$ & $\phi$ & $\phi$ & $\phi$ & \{(\fieldI{f}{}{x},\fieldI{g}{}{\infty})\}\\ \hline

\{(\fieldI{f}{}{\infty},$\epsilon$)\} & $\phi$ & \{(\fieldI{f}{}{\infty},$\epsilon$)\} & \{(\fieldI{f}{}{\infty},$\epsilon$)\} & \{(\fieldI{f}{}{\infty},$\epsilon$)\} & $\phi$ & \{(\fieldI{f}{}{\infty},$\epsilon$)\} & \{(\fieldI{f}{}{\infty},$\epsilon$)\} & \{(\fieldI{f}{}{\infty},$\epsilon$)\} & \{(\fieldI{f}{}{\infty},$\epsilon$)\}\\ \hline

\{(\fieldI{f}{}{\infty},\fieldD{g}{})\} & \{(\fieldI{f}{}{\infty},\fieldD{g}{})\} & $\phi$ & \{(\fieldI{f}{}{\infty},\fieldD{g}{})\} & \{(\fieldI{f}{}{\infty},\fieldD{g}{})\} & \{(\fieldI{f}{}{\infty},\fieldD{g}{})\} & $\phi$ & \{(\fieldI{f}{}{\infty},\fieldD{g}{})\} & \{(\fieldI{f}{}{\infty},\fieldD{g}{})\} & \{(\fieldI{f}{}{\infty},\fieldD{g}{})\}\\ \hline
 
\{(\fieldI{f}{}{\infty},\fieldI{g}{}{y})\} & \{(\fieldI{f}{}{\infty},\fieldI{g}{}{y})\} & $\phi$ & \{(\fieldI{f}{}{\infty},\fieldI{g}{}{y\#n})\} & \{(\fieldI{f}{}{\infty},\fieldI{g}{}{y})\} & \{(\fieldI{f}{}{\infty},\fieldI{g}{}{y})\} & $\phi$ & \{(\fieldI{f}{}{\infty},\fieldI{g}{}{y\#n})\} & \{(\fieldI{f}{}{\infty},\fieldI{g}{}{y})\} & \{(\fieldI{f}{}{\infty},\fieldI{g}{}{y})\}\\ \hline

\{(\fieldI{f}{}{\infty},\fieldI{g}{}{\infty})\} & \{(\fieldI{f}{}{\infty},\fieldI{g}{}{\infty})\} & $\phi$ & \{(\fieldI{f}{}{\infty},\fieldI{g}{}{\infty})\} & \{(\fieldI{f}{}{\infty},\fieldI{g}{}{\infty})\} & \{(\fieldI{f}{}{\infty},\fieldI{g}{}{\infty})\} & $\phi$ & \{(\fieldI{f}{}{\infty},\fieldI{g}{}{\infty})\} & \{(\fieldI{f}{}{\infty},\fieldI{g}{}{\infty})\} & \{(\fieldI{f}{}{\infty},\fieldI{g}{}{\infty})\}\\ \hline


\end{tabular}
}\\
\end{tabular}
\end{figure*}

}

\item Multiplication by a scalar($\star$): Let $i, j
  \in \nat, i \leq k, j \leq k$. Then, for a path $\alpha$,
  the multiplication by a scalar $i$, $i\star\alpha$ is
  defined in Table~\ref{tab:path-mult}. The operation is
  extended to set of paths as:
\begin{eqnarray*}
  i \star {S} &=&   \left\{ \begin{array}{lcl}
    \emptyset &\quad& i = 0 \\
    \{ i\star \alpha \ \vert\ \alpha \in S\} &\quad& i\in \nat \cup \{\infty\} 
  \end{array} \right.
\end{eqnarray*}

\begin{table*}[t]
\caption{Multiplication by a scalar\label{tab:path-mult}}
\begin{center}
\scalebox{0.8}{
\renewcommand{\arraystretch}{1.1}
\begin{tabular*}{0.75\textwidth}{@{\extracolsep{\fill}} l@{\ }|cccc }
  \hline
  $\star\qquad \alpha$ & $\epsilon$ & $f^{\drct}$      & ${f^{\indrct
      j}}$ & ${f^{\indrct \infty}}$ \\ 
  $i$ &&&& \\ \hline 
  $i$      & $\epsilon$ & ${f^{\indrct i}}$ & $f^{\indrct
    m}$, $m = \left\{\begin{array}{lcl}
  i*j    && \mbox { if } i*j \leq k \\
  \infty && \mbox{ Otherwise} 
  \end{array}\right.$
  & ${f^{\indrct \infty}}$ \\ 
  $\infty$ & $\epsilon$ &${f^{\indrct \infty}}$ &
  ${f^{\indrct \infty}}$& ${f^{\indrct \infty}}$\\ \hline 
\end{tabular*}
}
\end{center}
\end{table*}
	

\end{itemize}
