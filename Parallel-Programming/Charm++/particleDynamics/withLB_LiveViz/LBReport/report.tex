\documentclass[10pt,times]{report}
\usepackage{url}
\usepackage{graphicx}
\setkeys{Gin}{width=\textwidth,height=\textheight,keepaspectratio}
\usepackage{caption}
\usepackage{subcaption}
\usepackage{listings}
\usepackage{xcolor}
\usepackage{framed}
\usepackage{pgfplotstable}
\usepackage{pgfplots}
\lstset{language=Python, keywordstyle=\color{blue}\bfseries, }
\usepackage{amsmath}

\setlength{\parskip}{0pt}
\setlength{\parsep}{0pt}
\setlength{\headsep}{0pt}
\setlength{\topskip}{0pt}
\setlength{\topmargin}{0pt}
\setlength{\topsep}{0pt}
\setlength{\partopsep}{0pt}

\newcommand{\lbp}{LB\_Period}

\pagestyle{myheadings}

\hyphenation{in-de-pen-dent}


\begin{document}

\begin{flushleft}
\textbf{\large{RefineLB}}
\end{flushleft}

  \begin{figure}[htbp]
    \begin{center}
       \includegraphics{../100/20/RLB/24/TL.png} 
    \end{center}
    \caption{RefineLB: Cores: 24, LB\_PERIOD: 20, Iteration Count:100}
      \label{fig:1} 
  \end{figure}

  \begin{flushleft}
    \textbf{Are   they   beneficial?   Why   or   why   not?   How   much   is   the   overhead?}  
  \end{flushleft}
  Yes, they are beneficial if done less frequently. With \lbp = 10,
  RefineLB total execution time (6683 ms) is more that that of without any LB
case (6070 ms). But with \lbp = 20 we are getting better performance (RefineLB time = 5492 ms and NoLB time = 5955 ms). The reason is
that with frequent load balancing the overhead of load balancing (decision time
    and actual migration time) become dominant than the actual gain obtained by load
balancing. For example, with \lbp = 10, the load balancing overhead is $1.441\%$ as compared to $.860\%$ at \lbp = 20.    

  \textbf{\small{Other Overheads}}:  When we are doing RefineLB, the neighboring chares \textbf{MAY} get separated if they where situated  at the same PE during initial placement. This observation is true for any LBs discussed
  in this report ( except the GreedyComm which takes into account the associated communication). This also add on to communication load on CPU.

  Also the application is peculiar in two ways: \
      (1) The load is frequently changing across the chares.
      (2) With too big iteration count the application may load balance itself naturally.
    Now if we keep the iteration count be very large (e.g. with 500)
  then the RefineLB is not giving better performance (even with infrequent load balancing),  because the application is naturally balanced 
  and the load balancing effort comes as an extra overhead. 

  \begin{flushleft}
    \textbf{\small{Which  strategy  is  the  best  for  this  Particle  application?}}
  \end{flushleft}
  This strategy seems to be the best among the three discussed, with the constrains like LB period should not be too frequent
  and the iteration count should not be too big. The reason are:
  (1) The overhead of RefineLB is lower than others.
  (2) As the load is frequently changes in the application it does not worth using a high overhead load balancing strategy. Rather a small refinement 
  will be sufficient.

\pagebreak
\begin{flushleft}
\textbf{\Large{GreedyLB}}
\end{flushleft}

  \begin{figure}[htbp]
    \begin{center}
       \includegraphics{../100/20/GLB/24/TL.png} 
    \end{center}
    \caption{GreedyLB: Cores: 24, LB\_PERIOD: 20, Iteration Count:100}
      \label{fig:2} 
  \end{figure}

  \begin{flushleft}
    \textbf{Are   they   beneficial?   Why   or   why   not?   How   much   is   the   overhead?}  
  \end{flushleft}
  NOT beneficial. The following table with different \lbp\ values shows that GreedyLB's total execution times are  more than those with NoLB and RefineLB.

  \begin{table}[h]
  \begin{tabular}{|c|c|c|c|c|c|}
  \hline
  \multicolumn{1}{|l|}{LB\_Period} & \multicolumn{1}{l|}{$t_{nolb} (ms)$} & \multicolumn{1}{l|}{$t_{rlb} (ms) ( \% p_{rlb})$} & \multicolumn{1}{l|}{$t_{glb} (ms) (\%p_{glb})$} \\ \hline
  10                           &  6070                     & 6683   (1.441)                   & 9830  (11.5)                        \\ \hline
  20                           &  5955                     & 5492   (.860)                    & 8503  (7.112)                           \\ \hline
  30                           &  5876                     & 5277   (.60)                     & 5989  (5.51)                        \\ \hline
  \end{tabular}
  \caption {$t_{nolb},\ t_{rlb},\ t_{glb}$ are the executions times of NoLB, RefineLB and GreedyLB respectively and $p_{rlb},\ p_{glb}$ are the \% CPU utilization for the migrations.}
  \end{table}
 
The reason is that the overhead of GreedyLB load balancing  (decision time
    and actual migration time) is very high. Though the quality of load balancing provided by
GreedyLB is much better than RefineLB, but that does not help in this scenario because 
      (1) The load is frequently changing across the chares. So the effort put in by this expensive load balancing is kind of nullified after few iterations. 
      (2) In case of GreedyLB, the number of chares migrated in more as compared to RefineLB and that migration happens without 
      taking communication into account, so this ends up with decoupling the neighboring chares which in turn causes huge communication load on CPU.

Another interesting observation is as the \lbp\ decreases, the overhead of load
balancing diminishes and so the total execution time.      Also with higher
iteration count, the performance is bad as compared to both NoLB and RefineLB
because the application get naturally load balanced in that scenario and the
effort of this expensive load balancing becomes an overhead.

\pagebreak

\begin{flushleft}
\textbf{\Large{GreedyCommPB}}
\end{flushleft}

  \begin{figure}[htbp]
    \begin{center}
       \includegraphics{../100/20/GCLB/24/TL.png} 
    \end{center}
    \caption{GreedyLB: Cores: 24, LB\_PERIOD: 20, Iteration Count:100}
      \label{fig:3} 
  \end{figure}

  \begin{flushleft}
    \textbf{Are   they   beneficial?   Why   or   why   not?   How   much   is   the   overhead?}  
  \end{flushleft}
  NOT beneficial. The following table with different \lbp\ values shows that GreedyCommLB's total execution times are  more than those with NoLB and RefineLB.
  The reason for this is same as the reasons for GreedyLB that the overhead of migration is huge which is not beneficial in
  the scenario when the load is dynamically changing. Also similar to GreedyLB they end up decoupling the neighboring
  chares adding more to the communication load of CPU.
  
  \text{BUT} there performance is better than
  GreedyLB because they are communication aware in the sense that if two chares are going to communicate much then they will be kept on the same PE.

  \begin{table}[h]
  \begin{tabular}{|c|c|c|c|c|c|}
  \hline
  \multicolumn{1}{|l|}{LB\_Period} & \multicolumn{1}{l|}{$t_{nolb} (ms)$} & \multicolumn{1}{l|}{$t_{rlb} (ms) ( \% p_{rlb})$} & \multicolumn{1}{l|}{$t_{glb} (ms) (\%p_{glb})$} & \multicolumn{1}{l|}{$t_{gclb} (ms) (\%p_{gclb})$}\\ \hline
  10                           &  6070                     & 6683   (1.441)                   & 9830  (11.5)                &    9817 (12.56)     \\ \hline
  20                           &  5955                     & 5492   (.860)                    & 8503  (7.112)               &    7117 (8.04)         \\ \hline
  30                           &  5876                     & 5277   (.60)                     & 5989  (5.51)                &    5914 (4.54)         \\ \hline
  \end{tabular}
  \caption {The application is run on 24 cores with iteration Count 100. $t_{nolb},\ t_{rlb},\ t_{glb}\ and\ t_{gclb}$ are the executions times of NoLB, RefineLB, GreedyLB and GreedyCommLB respectively and $p_{rlb},\ p_{glb}\ 
    and\ p_{gclb}$ are the \% CPU utilization for the migrations of RefineLB, GreedyLB and GreedyCommLB respectively.}
  \end{table}

Also like GreedyLB, as the \lbp\ decreases, the overhead of load balancing diminishes and so the total execution time.
Also with higher
iteration count, the performance is bad as compared to both NoLB and RefineLB, \textbf{BUT} better that GreedyLB.

  
\end{document}

% \usepackage{multirow}
%\begin{table}[h]
%\begin{tabular}{|l|c|c|c|c|c|c|}
%\hline
%Total Iteration      & \multicolumn{1}{l|}{Period} & \multicolumn{1}{l|}{Cores} & \multicolumn{1}{l|}{WLB} & \multicolumn{1}{l|}{RLB (Overhead)} & \multicolumn{1}{l|}{GLB} & \multicolumn{1}{l|}{GCLB} \\ \hline
%\multirow{6}{*}{100} & 10                          & 12                         & 9563                     & 10005 (1.083)                       & 12688(15.68)             & 12124 (15.07))            \\ \cline{2-7} 
%                     &                             & 24                         & 6070                     & 6683(1.441)                         & 9830 (11.5)              & 9877 (12.56)              \\ \cline{2-7} 
%                     & 20                          & 12                         & 9094                     & 9063 (.486)                         & 10690 (9.4)              & 10580 (9.331)             \\ \cline{2-7} 
%                     &                             & 24                         & 5955                     & 5492 (.860)                         & 8503 (7.112)             & 7117 (8.04)               \\ \cline{2-7} 
%                     & 50                          & 12                         & 8988                     & 9151 (.44)                          & 9550 (4.24))             & 9413 ( 4.91)              \\ \cline{2-7} 
%                     &                             & 24                         & 5876                     & 5277 (.60)                          & 5989 (5.51)              & 5914 (4.54)               \\ \hline
%\multirow{6}{*}{500} & 10                          & 12                         & 9527                     & 10220                               & 12671                    & 12435                     \\ \cline{2-7} 
%                     &                             & 24                         & 6510                     & 11194                               & 12801                    & 9730                      \\ \cline{2-7} 
%                     & 20                          & 12                         & 9026                     & 9468                                & 10311                    & 10242                     \\ \cline{2-7} 
%                     &                             & 24                         & 6090                     & 8634                                & 9439                     & 7393                      \\ \cline{2-7} 
%                     & 50                          & 12                         & 8957                     & 9221                                & 9975                     & 9548                      \\ \cline{2-7} 
%                     &                             & 24                         & 5762                     & 7784                                & 6230                     & 7485                      \\ \hline
%\end{tabular}
%\end{table}
