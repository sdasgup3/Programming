\bb{
\begin{figure}
%\hrule
\begin{framed}
{\tt
  \begin{program}{0}
  \FL fun tagHeapStmt( f )  \{
  \UNL{0} \FOR (each HeapStmt $S_i$ of f) \{
  \UNL{0} if $S_i\equiv$ (p = q) \COMMENT{p, q are pointers to heap object}
  \UNL{1} Tag[$S_i$] = AliasStmt;
  \UNL{0} else if $S_i\equiv$ (p = q$\rightarrow$next) \COMMENT{next is pointer field}
  \UNL{1} Tag[$S_i$] = LinkTraverseStmt;
  \UNL{0} else if $S_i\equiv$ ($\cdots$ = q$\rightarrow$num) \COMMENT{num is data field}
  \UNL{1} Tag[$S_i$] = ReadStmt;
  \UNL{0} else if $S_i\equiv$ (q$\rightarrow$num = $\cdots$)
  \UNL{1} Tag[$S_i$] = WriteStmt;
  \UNL{0} else if $S_i\equiv$ f(p, q)
  \UNL{1} Tag[$S_i$] = FunCallStmt;
  \UNL{0} else Tag[$S_i$] = OtherStmt;
  \UNL{0} \}
  \UNL{-1} \}
  \end{program}
}
\end{framed}
%\hrule
  \caption{Algorithm of function {\tt tagHeapStmt} for
    annotation of heap related statements. \label{fig:AlgoTag}}
%\hrule  
\end{figure}
}



However, for function call statement {\tt computeHeapStates} may result 
in a set of states where a variable can be potentially bound to multiple symbolic locations, 
our algorithm chooses the most conservative location for the variable such that it approximates other locations accessed by 
the same variable.