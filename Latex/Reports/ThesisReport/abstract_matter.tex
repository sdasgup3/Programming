\vspace{2in}
\begin{abstract}
Programs in high level languages make intensive use of heap
to support dynamic data structures.  Analyzing these programs
requires precise reasoning about the heap structures. Shape
analysis refers to the class of techniques that statically
approximate the run-time structures created on the heap.  In
this work, we present a novel field sensitive shape analysis
technique to identify the shapes of the heap structures.  The
novelty of our approach lies in the way we use field
information to remember the paths that result in a particular
shape (Tree, DAG, Cycle).  We associate the field information
with a shape in two ways: (a) through boolean functions that
capture the shape transition due to change in a particular
field, and (b) through matrices that store the field
sensitive path information among two pointer variables.  This
allows us to easily identify transitions from Cycle to DAG,
from Cycle to Tree and from DAG to Tree, thus making the
shape more precise.
\end{abstract} 
