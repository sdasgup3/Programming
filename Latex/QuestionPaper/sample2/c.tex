\newcommand{\Testing}{
\question \QuestionTitle{Testing \& Debugging Techniques}

\begin{parts}

% YES
\part[1] What is the difference between black-box and white-box testing?

        \begin{solution}[1.25in] White-box tests internal structures
	or workings of a program, as opposed to the functionality
	exposed to the end-user.  Black-box examining functionality
	without any knowledge of internal
	implementation.  \end{solution}

% YES - MULTIPLE
\part[2] According to \Ch{29}, which statements are true about
\textbf{smoke testing}. \CircleAll{}
\begin{enumerate}
\item{Smoke tests should only test GUI}
\item{Smoke tests should be automated}
\item{Smoke tests are a kind of regression tests}
\item{Smoke tests should run fast}
%\item{Smoke tests are only needed for hardware not software}
\end{enumerate}
\begin {solution}
2,3,4

Smoke testing is mentioned not only in book chapter 29 but also in
SCM patterns and on the slides
\end {solution}

\Maybe{
 \part[2] In the lectures, we discussed the benefits of writing regression tests.
\begin{enumerate}
\item{What is regression testing? When will you need this kind of testing?}

\begin{solution}[1.5 in]
 - Regression testing is any type of software testing that seeks to uncover new software bugs, or regressions, in existing functional and non functional areas of a system after changes such as enhancements, patches or configuration changes, have been made to them.
 - When we change the code and want to make sure the new code does not introduced new faults.
\end{solution}

\item{Typically regression tests are automated. Do you know why? Briefly explain your answer.}

\begin{solution}[1.5 in]
 - Regression tests are rerun every time the software is changed to makes sure that things that are fixed stay fixed. When the code is very big, you may actually have lots of regressions tests to run after each change. Doing this in automated way will save a lot of time for testing. 

\end{solution}
\end{enumerate}
}

% YES
   \part[2] What is the class invariant for the example class below?
(Hint: Think of boundary values.)

\lstinputlisting[language=Java]{code/ValueRange.java}

    \begin{solution}[0 in]
      \CodeIn{lo <= hi}
    \end{solution}

\Comment{
\part[1]
Why is it important to make an intermittent bug occur predictably?
\CircleAll{}
\begin{enumerate}
\item If a bug does not occur predictably, it is much harder to debug.
\item A predictable bug at the client site can always be reproduced at the
developer site.
\item Predictable bugs are free of initialization error and dangling-pointer
problem.
\item Predictable bugs have smaller fixes than the non-predictable ones.
\end{enumerate} 
\begin{solution} 
1
\end{solution}
}

% YES - MULTIPLE
\part[2]
Why is simplifying test cases important in debugging? \CircleAll{}
\begin{enumerate}
\item A simplified test case is easier to communicate.
\item A simplified test case \emph{usually} means smaller fixes.
\item A simplified test case \emph{usually} means smaller program states.
\item A simplified test case \emph{usually} means fewer program steps.
\end{enumerate}
      \begin{solution} 
      1,3,4
      \end{solution} 
\end{parts}
} % End \Testing

\newcommand{\OOMetrics}{
\question \QuestionTitle{Metrics}
\begin{parts}

\Noo{
\part[1] What is the difference between technical and non-technical software
metrics?

        \begin{solution}[2 in]
        technical metric: Are the measures of the actual production code
        non-technical metric: Does not measure exactly the code but some proxy
        about the code. Like number of people, time taken, money spent,  bugs
        reported that the business cares about.  
        \end{solution}
}

% YES
\part[1]
    Does a large number of tests directly translate to a better quality
    software?  Why or why not?  \textbf{Give one reason} to support your answer.
        \begin{solution}[1.25in]
                \begin{enumerate}
                \item coverage
                \item quality of tests
                \item tests do not prove the absence of bugs/correctness of the system
                \end{enumerate}
        \end{solution}
%  \end{subparts} 

% YES
\part[1] When people talk about testing code, sometimes they refer 
to the static code size, e.g., ``I tested a million-line software''.
What would be one better metric to use in this context?
\begin{solution}[1.25in]
Use code coverage to measure how many lines the tests
\emph{dynamically} execute, not how many lines \emph{statically} exist
in the code.
\end{solution}
    
\part \Choose{Cyclomatic Complexity}{Code Coverage}
  \begin{subparts}

% YES
  \subpart[1] What does this code metric measure?
  \begin{solution} [1.25in]
    Cyclomatic Complexity: Measures the number of independent paths through the
    procedure.  
    Code Coverage: It is a measure used to describe the degree to
    which the source code of a program is tested by a particular test suite.  
  \end{solution}  
    
% YES
  \subpart[1] How can this metric be used to improve software testing?
  \begin{solution}[1.25in]
    Cyclomatic Complexity :Gives an upper bound on the number of tests necessary
    to execute every edge of a control graph.
    Code Coverage: A program with high code coverage has been more thoroughly
    tested and has a lower chance of containing software bugs than a program
    with low code coverage.
  \end{solution}

  \end{subparts}

% YES
\part[2] Recall the reading and lecture on Chidamber \& Kemerer
object-oriented metrics.  \Choose{Coupling
Between Object Classes (CBO)}{Weighted Method per Class (WMC)}
Why does a higher value of that metric lower the quality
of a software system?

  \begin{solution}[1.25in]
  Why does a higher value of \textbf{CBO} (Coupling
  Between Object Classes) lower the quality of a software system? CBO
  is the number of classes to which a class is coupled.

   high CBO: Excessive coupling between object classes is detrimental
   to modular design and prevents reuse.  The more independent a class
   is, the easier it is to reuse it in another application.  In order
   to improve modularity and promote encapsulation, inter-object class
   couples should be kept to a minimum.  The larger the number of
   couples, the higher the sensitivity to changes in other parts of
   the design, and therefore maintenance is more difficult. A high
   coupling has been found to indicate fault-proneness.  Rigorous
   testing is thus needed.

  Why does a higher value of \textbf{WMC} (Weighted
  Method per Class) lower the quality of a software system? WMC can be the
  number of methods defined in the class.

  high WMC: A high WMC has been found to lead to more faults. Classes
  with many methods are likely to be more application specific,
  limiting the possibility of reuse. WMC is a predictor of how much
  time and effort is required to develop and maintain the class. A
  large number of methods also means a greater potential impact on
  derived classes, since the derived classes inherit (some of) the
  methods of the base class.
  \end{solution}

\Maybe{
\part[1] Describe \textbf{one way} in which ``Lines of Code'' metric 
can be used in a software development process.
\begin{solution}[1.25in]
Estimate complexity of code, identify long methods or large classes
for refactorings, arguably judge programmer's productivity...
\end{solution}
}
\end{parts}
} % End \OOMetrics

\newcommand{\Patterns}{
% YES
\question[2] \QuestionTitle{Design Patterns}

In the lectures we discussed the following \textbf{eight design
  patterns}: Observer, Composite, Interpreter, Visitor, Iterator,
Template Method, Command, and Strategy.  The book \emph{Design
  Patterns} has 15 more patterns: Abstract Factory, Adaptor, Bridge,
Builder, Chain of Responsibility, Decorator, Fa\c{c}ade, Factory
Method, Flyweight, Mediator, Memento, Prototype, Proxy, Singleton, and
State.

Choose \textbf{one} of the 15 design patterns that we \textbf{did NOT
  discuss} in the lecture.  Briefly describe this pattern, including
its ``Intent'' (i.e., the reason for using this pattern) and one real
example.

        \begin{solution}[1.5in]
        \end{solution}
\Comment{
    \subpart[5] Did you come across any of the design patterns in your
    project that we \textbf{did NOT cover} in the lectures?
    \\ \textbf{If so}, name the pattern and describe
        \begin{enumerate}
  %        \item The Intent of the pattern. 
          \item The place in the code where this pattern is used.
          \item Why is this pattern a good choice to be used in that
            place?
        \end{enumerate}
  \textbf{If not}, choose any one of the 15 patterns that we
  \textbf{did NOT cover} in the lecture and answer the following
        \begin{enumerate}
          \item Describe the Intent of the pattern.
          \item Give one concrete example of that pattern.
          \item How that pattern could be used in your project.
        \end{enumerate}
  \begin{solution}[4.5 in]
    The answer should based on the pattern they choose. They should
    have clear understanding of the intent of the pattern and we shall
    judge if they can reason out the artfulness of the pattern in the
    mentioned place.
   \end{solution}
}


\Maybe{
\part 
Suppose you are making a spreadsheet program.  It supports large
spreadsheets\Comment{, up to 10,000 by 10,000 cells}.
\begin{subparts}
\subpart Suppose that one cell can update its content based on the
changes in other cell(s) that it references.  For example, if the cell
\CodeIn{B10} has the rule \CodeIn{=SUM(A9, A10)}, then \CodeIn{B10}
must be updated to the sum of the values in cells \CodeIn{A9} and
\CodeIn{A10} whenever there is a change in any one of those two cells.  In this
scenario, answer the following questions.
  \begin{subparts}
  % MAYBE
  \subpart [2] \textbf{Which design pattern} should you use to monitor
  the changes and \textbf{why}? (Hint: we are \textbf{not} asking about the design
  pattern to evaluate the expression\Comment{, but about the pattern to monitor
  for the changes}.)
        \begin{solution}[2 in]
        observer
        \end{solution}

  % MAYBE
  \subpart[1]
    \textbf{Which participants of this pattern} are the cells \CodeIn{B10},
    \CodeIn{A9} and \CodeIn{A10} mapped to?
        \begin{solution}[1.5 in]
        \CodeIn{B10} is the \textbf{observer} of the \textbf{subjects} \CodeIn{A9} and
        \CodeIn{A10}. 
        \end{solution}

   \end{subparts}     

% FROM DARKO: Can we turn this into one or more multiple choice questions?
% Maybe we describe a few scenario (e.g., like above about monitoring
% changes, then like below about evaluation expressions, then maybe
% something that hints that there'll be change in the algorithm so that
% they should use a visitor, etc.)  Then we just give them a list of
% patterns to choose from for each of the scenarios.  It makes the
% grading very easy (pick one pattern) as long as the questions are
% rather precise.  If we canNOT be precise, then we should ask them
% not only to choose one pattern but also describe \textbf{why},
% because then even a seemingly incorrect choice may have a very good
% justification for why.
% NOO
\subpart[2] Each cell can have a rule.  Rules support all arithmetic
operations plus lots of functions, such as trig
functions. \textbf{What design pattern} could you use to implement the
rules and \textbf{why}?
\begin{solution}[2.5 in] 
interpreter 
\end{solution}
\end{subparts}
}

\Maybe{
\part
In the code of MP1, we have Collection and Book classes with Element
as their superclass. Also we have the Library object as a container
for all the books and collection.  Answer the following questions.
\begin{subparts}
  \subpart [3] If we want to use the Composite design pattern to model
  them, what should be the type of component, composite and leaf
  separately?
    \begin{solution} [2 in] 
      For the first question, component == Element; composite ==
      library, collection; leaf == book. For question two, component
      is the same, while composite is the library, collection, book;
      and leaf should be the paragraph.
    \end{solution}

  \subpart [3] Suppose another class Paragraph is added such that
  there should be many paragraphs inside one book? Now if we want to
  use the Composite design pattern to model them, what should be the
  type of component, composite and leaf separately?
    \begin{solution} [2 in] 
      component is the same, while composite is the
      library, collection, book; and leaf should be the paragraph.
    \end{solution}
\end{subparts}    
}
} % End \Patterns

\newcommand{\MergeWithPrevious}{
\Maybe{
% MAYBE
\question \QuestionTitle{Observer Pattern in Real Life}

\begin{parts}
% MAYBE
\part[4] One example of the Observer pattern from outside of software can be
the university alert system: As an Illinois student, you
subscribe to the university's alert notification system to receive
important messages.
\begin{subparts}
% MAYBE
\subpart[2]
For this example, what is the observer and what is the subject?

\begin{solution}[2 in]
For the first example, the observer is the person who subscribe, and subject is the mailing system.
For the second example, the observer is the Illinois student, and the subject is the alert system.
\end{solution}

% NOO
\subpart[1]
Can you give another example from real life that can be modeled using
the Observer pattern?

\begin{solution}[2 in]
 People subscribe their favorite Newspaper for reading.
\end{solution}

% MAYBE (if merged with the other MAYBE?)
\subpart [1]
Would you model a radio station and its listeners using the Observer
pattern?  Describe why or why not.

\begin{solution}[2 in]
I don't think that a radio station is an ideal example, because it
does not keep a set of its listeners to inform them of changes.
\end{solution}
\end{subparts}
\end{parts}
} % Maybe
}

\newcommand{\Iterators}{
\question[3] \QuestionTitle{Iterator Design Pattern}

\Comment{
  % CONSIDER ONLY IF NEEDED FOR LENGTH
  Describe \textbf{one difference} between internal and external iterators?
  \begin{solution} [2 in]
  Internal Iterator: Instead of getting an iterator from the aggregate
  and then being in control of the iteration yourself, you can ask the
  aggregate to do the iterate for you and you give it the code that
  you want to have executed each time on each iteration.
  \end{solution}
} 

In the code shown below, the class \CodeIn{SongCollection}
stores a list of \CodeIn{SongInfo} objects.

Your aim is to execute some operation, e.g.,
\CodeIn{findOnYouTube(SongInfo s)}, on each song info in the song
collection.  One way to do that is using an \textbf{internal} iterator
on \CodeIn{SongCollection}.  We are providing you some classes that
implement the operation and call the internal iterator method
\ClassMethod{SongCollection}{apply(Command cmd)}.  Write the body of
that method. 

\lstinputlisting[language=Java]{code/Internal_Iter.java}

\lstinputlisting[language=Java]{code/SongCollection.java}

  \begin{solution}
\lstinputlisting[language=Java]{code/SongCollectionApplySolution.java}
  \end{solution}
} % End \Iterators

\newcommand{\Visitor}{
\question[5] \QuestionTitle{Visitor Design Pattern}

Suppose you start from this hierarchy for \CodeIn{Car} parts:

\lstinputlisting[language=Java]{code/CarHierarchy.java}

If you want to add operations to this hierarchy, e.g., to print car
parts, you would need to change all the classes for each new operation
if you use straight polymorphism\Comment{, e.g., as shown here:
\lstinputlisting[language=Java]{code/CarHierarchy_Impl.java}}.

Using the Visitor design pattern, we need to change the hierarchy once
and can then add new operations without modifying the car classes.
Our goal is to add methods that print all parts of a given car using a
visitor. We are providing you some changes to the original classes and
a working visitor class \CodeIn{CarElementPrintVisitor} that prints
the parts.

\lstinputlisting[language=Java]{code/VisitorDemo.java}

\textbf{You should write the bodies of the \CodeIn{accept} methods.}

\newpage

\lstinputlisting[language=Java]{code/Wheel_Body_Car.java}

\begin{solution}[0 in]
\lstinputlisting[language=Java]{code/CarVisitorSolution.java}
\end{solution}

% FROM DARKO: This would need to be related to the previous; it should
% not be on Tractors and Appraisers, but on Cars mentioned above!  Or
% if it can't be on those exact cars, maybe we need (1) to add more hierarchy
% about car parts or (2) to remove this question.
% Sandeep:  Right now I am commenting this question. I will uncomment it in case
% i come up with some better alternative by tonight
\Comment{
\part[2]
Why the ``Visit'' method should have in their name the type of the
parameter?  For example, why the call to visit should be like
VisitN(TypeN n) instead of Visit(TypeN n).

\begin{solution}[2in]
  Languages like Java, C++ have dynamic dispatch based on a single
  object.  Java-style method overloading isn't multiple dispatch. In
  multiple dispatch, the 'switch' is done on the dynamic (run-time)
  type tag of the object, but Java's (and C++'s) overloading only
  considers the static (compile-time) type of the object reference.

  This become a problem as the call to visit (without using the type
  name) require dynamically figuring out the actual method based on
  the dynamic type.  To solve this we use VisitN(TypeN n) instead of
  Visit(TypeN n).

  For example,
   class TractorHead extends Appraiser {
    public void appraise(Vehicle v) { println("Boring...");           }
    public void appraise(Tractor t) { println("I LOVE TRACTORS!!!!"); }
  }
  //Consider Tractor extends Vehicle.
  Appraiser myAppraiser = new TractorHead();
  Vehicle   myVehicle   = new Tractor();

  myAppraiser.appraise(myVehicle);        // outputs: "Boring..."

  Clearly we have DynamicDispatch because only the TractorHead has the behavior
  of printing 'Boring...'. But this is single dynamic dispatch because it was
  'dynamic' only on a single argument (myAppraiser). If we had true
  MultipleDispatch (even just DoubleDispatch) then "I LOVE TRACTORS" would have
  been printed; Languages like Java, C++ does not support this.  
\end{solution}
}

% FROM DARKO: If/since we have the previous question, we may not need
% this one below, but let's keep it as a backup for now.  Basically
% they ask about the same; write a body of the visitor for the class
% (``whole'') that has some parts.  You already have above:
%      for (ICarElement elem: elements) {
%        elem.accept(visitor);
%      }
% So you may not need this from below:
%    public Number visitPlus(PlusExpression p) {
%      return p.operand1.accept(this) + p.operand2.accept(this);
%    }
% ALSO, DID YOU LOOK INTO THE LECTURE SLIDES THAT DISCUSS HOW THIS
% ITERATION OVER ``PARTS'' CAN BE DONE EITHER IN THE COMPOSITE ITSELF
% OR IN THE VISITOR, OR OUTSIDE OF THE BOTH OF THEM?

% Sandeep :I am commenting this question 
% Sandeep: yes Darko I looked at that where you had shown the three cases where
% the component handles the traversal, or the visitor handles the traversal or
% the traversal is done by client using an iterator. In this example below the
% visitor calls the accept (i.e. the second case).

\Comment{
\part[5]
Suppose we are implementing a spreadsheet program where each cell can have a
rule.  Also suppose that the only rule supported is binary addition operation.
Our aim is to compute the value of a cell in the spreadsheet.  To implement the
rules, we have the abstract class \CodeIn{Expression} and the subclasses
\CodeIn{PlusExpression} and \CodeIn{CellExpression}.
 
The class \CodeIn{PlusExpression} is responsible for implementing the binary addition
rule and the class \CodeIn{CellExpression} is responsible for representing the address of
a particular cell (e.g. D3).  Now to compute the value of a cell we define
method \CodeIn{value(Spreadsheet)} for each subclass Expression as follows:

  \begin{lstlisting}


  abstract class Expression {
    public abstract Number value(Spreadsheet s);
  }

  class PlusExpression extends Expression {
    Expression operand1, operand2;
    ...
    public Number value(Spreadsheet s) {
     return operand1.value(s) + operand2.value(s);
    }
    ...
  }

  class CellExpression extends Expression {
    int row, column;
    ...
    public Number value(Spreadsheet s) {
      //Returns the actual value of the cell with address (row,column)
      return s.cellvalue(row, column); 
    }
    ...
  }
  \end{lstlisting}
  In the lecture, we discussed how Visitor design pattern helps  to centralize
  this algorithm of computing the value.  An implementation with Visitor design
  pattern is given as follows. Fill in your code just below the comment ``WRITE
  THE CODE HERE''

  \begin{lstlisting}
  interface ExpressionVisitor {
        public Number visitPlus(PlusExpression p);
        public Number visitCell(CellExpression c);
  }

  class ValueVisitor implements ExpressionVisitor {

    Spreadsheet s;
    public Number visitPlus(PlusExpression p) {
      //WRITE THE CODE HERE



    }
    public Number visitCell(CellExpression c) {
      return s.cellvalue(c.row, c.column);
    }
  
  }

  abstract class Expression {
    public Number accept(ExpressionVisitor visitor);
  }

  class PlusExpression extends Expression {
    Expression operand1, operand2;
    ...
    public Number accept(ExpressionVisitor visitor) {
      visitor.visitPlus(this);
    }  
  }

  class CellExpression extends Expression {
    int row, column;
    ...
    public Number accept(ExpressionVisitor visitor) {
      visitor.visitCell(this);
    }  
  }
  \end{lstlisting}

  \begin{solution} [0 in]
  \begin{lstlisting}
    public Number visitPlus(PlusExpression p) {
      return p.operand1.accept(this) + p.operand2.accept(this);
    }
  \end{lstlisting}
  \end{solution} 
}
} % End \Visitor

\newcommand{\Observer}{
\question[2] \QuestionTitle{Observer Design Pattern}

% FROM DARKO: Look promising, edit for final.
An apartment complex has a shared washer that all tenants use.
This is modeled by the classes \CodeIn{Tenant} and \CodeIn{Washer}.  The
tenants want to get notified when the washer has finished to perform
some action.\Comment{
\lstinputlisting[language=Java]{code/Tenant_Washer.java}

} Later on, we add a class \CodeIn{Manager} that would like to get
notified as well.  This seems like a good candidate to apply the
Observer pattern.  The following is an implementation of the pattern.
\CodeIn{Listener} is the abstract Observer class that is extended by
the classes \CodeIn{Tenant} and \CodeIn{Manager}.  Write the
\CodeIn{notifyFinish} method to inform all the listeners that the
washer has finished.

\lstinputlisting[language=Java]{code/Listener.java}

\begin{solution} [0 in]
\lstinputlisting[language=Java]{code/ObserverSolution.java}
\end{solution}
} % End \Observer

% LocalWords: SCM Cyclomatic Chidamber Kemerer CBO WMC Adaptor ade
% LocalWords: refactorings superclass SongCollection SongInfo cmd
% LocalWords: findOnYouTube CarElementPrintVisitor VisitN TypeN
% LocalWords: TractorHead println myAppraiser myVehicle cellvalue
% LocalWords: DynamicDispatch MultipleDispatch DoubleDispatch
% LocalWords: PlusExpression CellExpression ExpressionVisitor
% LocalWords: visitPlus visitCell ValueVisitor notifyFinish
