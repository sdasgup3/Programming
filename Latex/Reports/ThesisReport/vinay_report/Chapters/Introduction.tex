
\section{Brief Introduction}
% \label{intro:graphs}

Shape Analysis is  a static analysis technique which works on the heap to find the possible shape of the heap allocated objects.
This is useful in  many areas like garbage collection, parallelization, compile time optimization, instruction scheduling etc. 

In this report we discuss in detail the effectiveness of the work of Sandeep ~\cite{Sandeep} about field sensitive shape analysis.
~\cite{Sandeep} shows its preciseness using few typical examples, but lacks deep evaluation of the approach in order to check 
its scalability and preciseness on large practical benchmark programs. The present work suggests several enhancements over ~\cite{Sandeep},
involving modifications of the data flow values and intelligent ways of storing the same, which makes the analysis more precise and memory 
efficient. This work also involves implementing the interprocedural version of the analysis as a dynamic plugin on GCC and deep evaluation of the
same using large benchmarks.

This work also proposes a refined version of analysis namely subset-based field sensitive analysis (refer Chapter ~\ref{Subsetbased}) which 
infers more precise shapes depending upon which field pointers are actually accessed in a function. 

The interprocedural analysis is provided with two flavors: context sensitive and context insensitive; a trade-off analysis
on which of these techniques would be good for this shape analysis technique is described.
% and a detailed trade-off analysis between
% complexity and effectiveness/preciseness between the two is discussed with supporting results.
Also a new method of merging contexts at function calls is proposed whose complexity lies middle way between the above two, namely
shape sensitive interprocedural analysis, which uses the shape information of
the arguments of function to decide which context to merge with.

\section{Organization Of Thesis}
We discuss some of the prior works on shape analysis in
Chapter~\ref{Background}. Chapter~\ref{Enhancements} discusses all the enhancements suggested to~\cite{Sandeep}.
The Subset-based analysis is detailed in Chapter~\ref{Subsetbased}. 
Chapter~\ref{Interprocedural} elaborates on the Interprocedural analysis along with its proposed flavours.
Implementation details along with detailed evaluation using benchmark programs are given in Chapter~\ref{Implementation}.
We conclude the presentation in Chapter~\ref{Conclusion} and give directions for future work.
