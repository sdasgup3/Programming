\section{Related Work}

???????????? ---about history of it ????required\\
??????????-------should I write about the definitions and notations of Dir,Int matrices???-------\\

This work actually has its roots from Ghiya's \cite{Ghiya96} analysis.In this data flow analysis the data flow
values used are direction matrices,interference matrices which has entries for every heap pointer
pair.This analysis doesn't take into consideration any field information so its fast but results are
not precise. \\
    In Sandeep's \cite{Sandeep} analysis limited field sensitivity is used to infer the shape of the heap i.e a
TREE,DAG or a CYCLE. For identifying the shape they capture field sensitivity information in
two ways;first by using Boolean variables to remember the direct connection between two pointer
variables ,second by the use of field sensitive matrices that store the approximate path information.
And at every statement Boolean equations are formed which will be able to infer the shape.
The information coming out of this field sensitive analysis is a set of data flow values named Direction Matrix,
Interference Matrix , Boolean Equations and field based Boolean variables for every
statement(those which access the heap pointer variables) in the program across functions.It's pretty clear
that the data flow values of \cite{Ghiya96} and \cite{Sandeep} are somewhat similar. This
information only is refined and used as per requirement in  subset based analysis.Very little details are given
about the inter procedural analysis of the same,but our report discusses it in detail.

