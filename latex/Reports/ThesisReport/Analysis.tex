\chapter{Analysis}\label{ch:Analysis}

For $\{\p, \q\} \subseteq \heap,\ f \in\fields,\ n \in \nat$
and $\mbox{\tt op} \in \{+, -\}$, 
we have the following eight basic statements that can access or modify the heap structures. 
\begin{enumerate}
  \item Allocations
  	\begin{enumerate}
    	\item {\tt p = malloc();}
  	\end{enumerate}
  \item Pointer Assignments
  	\begin{enumerate}
    	\item {\tt p = NULL;}
    	\item {\tt p = q;}
    	\item {\tt p = q $\rightarrow$ f;}
    	\item {\tt p = \&(q $\rightarrow$ f);}
    	\item {\tt p = q op n;}
  	\end{enumerate}
  \item Structure Updates
  	\begin{enumerate}
    	\item {\tt p $\rightarrow$ f = q;}
    	\item {\tt p $\rightarrow$ f = NULL;}
  	\end{enumerate}
\end{enumerate}
%%
Our intend is to determine, at each program point, the field
sensitive matrices $D_F$ and $I_F$, and the
boolean variables capturing field connectivity. We formulate
the problem as an instance of forward data flow analysis,
where the data flow values are the matrices and the boolean
variables as mentioned above.  For simplicity, we construct
basic blocks containing a single statement each. Before
presenting the equations for data flow analysis, we define
the confluence operator (\merge) for various data flow values
as used by our analysis. Using the superscripts $x$ and $y$
to denote the values coming along two paths,
\begin{eqnarray*}
  \merge( f_{\p\q}^x, f_{\p\q}^y ) &=& f_{\p\q}^x \vee f_{\p\q}^y , f \in \fields,
  \p,\q \in \heap \\
   \merge(\p_{\subC}^x, \p_{\subC}^y) &=& \p_{\subC}^x \vee
   \p_{\subC}^y, \p \in \heap \\
   \merge(\p_{\subD}^x, \p_{\subD}^y) &=& \p_{\subD}^x \vee
   \p_{\subD}^y, \p \in \heap \\
   \merge(D_F^{x}, D_F^{y}) &=& D_F \mbox{ where }
   D_F[{\p}, {\q}]  = D_F^{x}[{\p}, {\q}] \cup
   D_F^{y}[{\p}, {\q}],   \forall {\p}, {\q} \in \heap\\ 
   \merge(I_F^{x}, I_F^{y}) &=& I_F \mbox{ where }
   I_F[{\p}, {\q}]  = I_F^{x}[{\p}, {\q}] \cup
   I_F^{y}[{\p}, {\q}],   \forall {\p}, {\q} \in \heap
\end{eqnarray*}
The transformation of data flow values due to a statement
$st$ is captured by the following set of equations:
\begin{eqnarray*}
  D_F^{\dout}[\p,\q] &=& (\remOne{D_F^{\din}[\p,\q]}{D_F^{\dkill}[\p,\q]}) \cup
  D_F^{\dgen}[\p,\q] \\
  I_F^{\dout}[\p,\q] &=& (\remOne{I_F^{\din}[\p,\q]}{I_F^{\dkill}[\p,\q]}) \cup
  I_F^{\dgen}[\p,\q] \\
  \p_{\subC}^{\dout} &=& (\p_{\subC}^{\din} \wedge
  \neg\p_{\subC}^{\dkill} ) \vee \p_{\subC}^{\dgen}  \\
  \p_{\subD}^{\dout} &=& (\p_{\subD}^{\din} \wedge
  \neg\p_{\subD}^{\dkill} ) \vee \p_{\subD}^{\dgen} 
\end{eqnarray*}
Field connectivity information is updated directly by the
statement.

%%%%%%%%%%%%%%%%%%%%%%%%%%%%%%%%%%%%%%%%%%%%%%%%%%%%%
\section{Analysis of Basic Statements} \label{Analysis_of_Basic_Statements}
%%%%%%%%%%%%%%%%%%%%%%%%%%%%%%%%%%%%%%%%%%%%%%%%%%%%%

We now present the basic statements that can access or modify
the heap structures, and our analysis of each kind of
statements.
\begin{enumerate}
%@@@@@@@@@@@@@@@@@@@@@@@@@@@@@@
\item {{\tt p = malloc()}}:
%@@@@@@@@@@@@@@@@@@@@@@@@@@@@@@
After this statement all the existing relationships of $\p$ get killed and it will point to a
newly allocated object. We will consider that $\p$ can have an empty path to itself and it can interfere
with itself using empty paths (or $\epsilon$ paths).

\begin{eqnarray*}
  \KillC{p}  = \InC{p} &\quad& \KillD{p} = \InD{p} \\
  \GenC{p} = \false 	 &\quad& \GenD{p} = \false 
\end{eqnarray*}
$\forall \s \in \heap, \s \not= \p:$
\begin{eqnarray*}
  D_F^{\dkill}[\p,\s]  =  D_F^{\din}[\p,\s] &\quad& D_F^{\dgen}[\p,\s]    =  \emptyset \\ 
  D_F^{\dkill}[\s,\p]  =  D_F^{\din}[\s,\p] &\quad& D_F^{\dgen}[\s,\p]    =  \emptyset \\
  D_F^{\dkill}[\p,\p]  =  D_F^{\din}[\p,\p] &\quad& D_F^{\dgen}[\p,\p]    =  \epsilonset\\
  I_F^{\dkill}[\p,\s]  =  I_F^{\din}[\p,\s] &\quad& I_F^{\dgen}[\p,\s]    =  \emptyset \\
  I_F^{\dkill}[\p,\p]  =  I_F^{\din}[\p,\p] &\quad& I_F^{\dgen}[\p,\p]    =  \epsilonpairset 
\end{eqnarray*}

%@@@@@@@@@@@@@@@@@@@@@@@@@@@@@@
\item{\tt p = NULL}:
%@@@@@@@@@@@@@@@@@@@@@@@@@@@@@@
This statement only kills the existing relations of \p.
\begin{eqnarray*}
  \KillC{p}  = \InC{p} &\quad& \KillD{p} = \InD{p} \\
  \GenC{p} = \false 	 &\quad& \GenD{p} = \false \\
\end{eqnarray*}
$\forall \s \in \heap:$
\begin{eqnarray*}
  D_F^{\dkill}[\p,\s]  =  D_F^{\din}[\p,\s] &\quad& D_F^{\dgen}[\p,\s]    =  \emptyset \\ 
  D_F^{\dkill}[\s,\p]  =  D_F^{\din}[\s,\p] &\quad& D_F^{\dgen}[\s,\p]    =  \emptyset \\
  I_F^{\dkill}[\p,\s]  =  I_F^{\din}[\p,\s] &\quad& I_F^{\dgen}[\p,\s]    =  \emptyset 
\end{eqnarray*}

%@@@@@@@@@@@@@@@@@@@@@@@@@@@@@@
\item{\tt p = q, p = \&(q$\rightarrow$f), p = q op n}:
%@@@@@@@@@@@@@@@@@@@@@@@@@@@@@@
In our analysis  we consider these three pointer assignment statements as
  equivalent. 
After this statement all the existing relationships of $\p$ gets
killed and it will point to same heap object as pointed to
by $\q$. In case $\q$ currently points to null, $\p$ will also points
to null after the statement. So $\p$ will have the same field
sensitive Direction and Interference relationships as $\q$. 
The kill effect of this statement is same as that of the previous statement. 
The generated boolean functions for heap object $\p$ corresponding 
to DAG or Cycle attribute will be same as that of $\q$, with all occurrences of $\q$ 
replaced by $\p$.

\begin{eqnarray*}
  \KillC{p} = \InC{p} &\quad& \KillD{p} = \InD{p} \\ 
  \GenC{p}   = \InC{q}[q/p]
  &\quad& \GenD{p} = \InD{q}[q/p] 
\end{eqnarray*}
{where $X[\q/\p]$ creates a copy of $X$ with all occurrences
  of \q\ replaced by \p.}

$\forall \s \in \heap, \s \not= \p, \forall f \in \fields$  
\begin{eqnarray*}
  f_{\p\s} = f_{\q\s},  \quad  f_{\s\p} = f_{\s\q} \\
  D_F^{\dkill}[\p,\s]  =  D_F^{\din}[\p,\s] &\quad&
  D_F^{\dgen}[\p,\s]    =  D_F^{\din}[\q,\s]   \\ 
  D_F^{\dkill}[\s,\p]  =  D_F^{\din}[\s,\p] &\quad&
  D_F^{\dgen}[\s,\p]    =  D_F^{\din}[\s,\q]   \\
  D_F^{\dkill}[\p,\p]  =  D_F^{\din}[\p,\p] &\quad&
  D_F^{\dgen}[\p,\p]    =  D_F^{\din}[\q,\q]  \rule{0mm}{0pt}\\
  I_F^{\dkill}[\p,\s]  =  I_F^{\din}[\p,\s] &\quad&
  I_F^{\dgen}[\p,\s]    =   I_F^{\din}[\q,\s] \\
  I_F^{\dkill}[\p,\p]  =  I_F^{\din}[\p,\p] &\quad&
  I_F^{\dgen}[\p,\p]    = I_F^{\din}[\q,\q] 
\end{eqnarray*}

%@@@@@@@@@@@@@@@@@@@@@@@@@@@@@@
\item {\tt p$\rightarrow$f = null}:
%@@@@@@@@@@@@@@@@@@@@@@@@@@@@@@
This statement breaks the existing link $f$ emanating from
$\p$, thus killing relations of $\p$, that are due to the link
$f$. The statement does not generate any new relations. The killed
relationships involve breaking of the all the direct links
of $\p$, which is manifested by setting $f_{\p\q}$ to 0, $\forall \q \in \heap$ 
as well as the indirect links of $\p$, which is done by updating the $D_F$ and $I_F$
matrices. Other links between pointers which come up
due to field $f$ of $\p$ are conservatively approximated to exist even after the statement. 
The relation \IFM{\p}{\p} is not much relevant to our analysis, so the corresponding
kill information is ignored. 

\begin{eqnarray*}
  \KillC{p}  = \false, &\quad& \KillD{p} = \false \\
  \GenC{p} = \false, &\quad& \GenD{p} = \false  \\
\end{eqnarray*}
$\forall \q,\s \in \heap, \s \not= \p:$
\begin{eqnarray*}
  f_{\p\q} &=& \false \\
  D_F^{\dkill}[\p,\q] &=& \project{D_F^{\din}[\p,\q]}{f}
  \qquad
  D_F^{\dkill}[\s,\q]  = \emptyset \\
  I_F^{\dkill}[\p,\s] &=& \{(\alpha, \beta) \ \vert\ (\alpha,
  \beta) \in I_F^{\din}[\p, \q] \mbox,\ \alpha \equiv f^\anysup \} \\ 
  I_F^{\dkill}[\q,\s] &=& \emptyset \mbox{ if } \q \not= \p \qquad\qquad
  I_F^{\dkill}[\p,\p] = \emptyset 
\end{eqnarray*}

%@@@@@@@@@@@@@@@@@@@@@@@@@@@@@@
\item {\tt p$\rightarrow$f = q}: 
%@@@@@@@@@@@@@@@@@@@@@@@@@@@@@@
This statement first breaks
  the existing link $f$ and then re-links the the heap object
  pointed to by $\p$ to the heap object pointed to by
  $\q$. The kill effects are exactly same as described in the
  case of  {\tt p$\rightarrow$f = null}. We only describe the
  generated relationships here. 
  
 The fact that the shape of the variable $\p$
becomes DAG after the statement is captured by the boolean functions $\GenD{p}$ and
$f_{\p\q}$. The functions simply say that variable $\p$ reaches a DAG
because there are more than one paths ($\num{I_F[p,q]} > 1$)
from $\p$ to $\q$. It also keeps track of the path $f_{pq}$ in this case.
The function $\GenC{q}$ (or $\GenC{p}$)  captures the fact that cycle on $\q$ (or $\p$)
 consists of field $f$ from $\p$ to $\q$ ($f_{\p\q}$)
and some path from $\q$ to $\p$ ($\num{\DFM{q}{p}} \geq
1$). The function $\GenC{p}$ also captures the fact that cycle on $\p$
can be due to the link $f_{\p\q}$ reaching an already existing cycle on $\q$.
These are summarized as follows:

  \begin{eqnarray*}
    \GenC{p}   &= & (f_{\p\q} \wedge \InC{q}) \vee (f_{\p\q}
    \wedge (\num{\DFM{\q}{\p}} \geq 1)) \qquad 
    \GenD{p}   =  f_{\p\q} \wedge (\num{\IFM{\p}{\q}} > 1)	\\
    \GenC{q}   &= & f_{\p\q} \wedge (\num{\DFM{\q}{\p}}
    \geq 1) \qquad\qquad\qquad\qquad\quad \GenD{q} = \false\\ 
    f_{\p\q} &=& \true \\
  \end{eqnarray*}

For nodes $\s \in \heap$ other than $\p$ or $\q$, the function $\GenC{\s}$ 
captures the fact that cycle on $\s$ consists of some path from $\s$ to $\p$ (or $\q$) i.e.\ 
$\num{\DFM{\s}{\p}} \geq 1$ (or $\num{\DFM{\s}{\q}} \geq 1$)
and the fact that a Cycle on $\p$ (or $\q$) has just created due to the
statement. Again the function $\GenD{\s}$  simply say that variable $\s$ reaches a DAG
because there are more than one way of interference between $\s$ and $\q$ i.e.\ $\num{\IFM{\s}{\q}} > 1$. 
It also keeps track of the paths $f_{pq}$ and $\DFM{\s}{\p}$ in this case.

  \begin{eqnarray*}
    \GenC{\s} &=& \ \ \ \ ((\num{\DFM{\s}{\p}} \geq 1) \wedge f_{\p\q}
    \wedge \InC{q}) \\
    && \vee\ ((\num{\DFM{\s}{\p}} \geq 1) \wedge f_{\p\q} \wedge
    (\num{\DFM{\q}{\p}} \geq 1)) \\ 
    && \vee\ ((\num{\DFM{\s}{\q}} \geq 1) \wedge f_{\p\q}
    \wedge (\num{\DFM{\q}{\p}} \geq 1))  \quad 
    \forall \s \in \heap, \s \not= \p, \s \not=\q \\
    \GenD{\s}   &=& (\num{\DFM{\s}{\p}} \geq 1) \wedge
    f_{\p\q} \wedge (\num{\IFM{\s}{\q}} > 1) \quad 
    \forall \s \in \heap, \s \not= \p, \s \not=\q 
  \end{eqnarray*}

After the statement, all the nodes which have paths towards $\p$ (including $\p$) will have path towards 
all the nodes reachable from $\q$ (including $\q$). Again all nodes having paths to $\p$ can potentially interfere
with all the node $\q$ interferes with. Thus the relations generated for $D_F$ and $I_F$ are as follows. 

For $\myr, \s \in \heap$:
\begin{eqnarray*}
  D_F^{\dgen}[\myr, \s] &=& \num{D_F^{\din}[\q, \s]} \star
  D_F^{\din}[\myr, \p],\ \s \not= \p,\ \myr \not\in
  \{\p, \q\}\\ 
  D_F^{\dgen}[\myr, \p] &=& \num{D_F^{\din}[\q,\p]} \star
  D_F^{\din}[\myr, \p], \ \myr \not= \p\\
 D_F^{\dgen}[\p, \myr] &=& \num{D_F^{\din}[\q, \myr]} \star
 (\remOne{D_F^{\din}[\p, \p]}{\epsilonset} \cup \{f^{\indrct
   1}\}),\ \myr \not= \q \\ 
  D_F^{\dgen}[\p, \q] &=&
  \{\fieldD{f}{}\}\ \cup\ (\num{D_F^{\din}[\q, \q] -
    \epsilonset} \star
  \{\fieldI{f}{}{1}\})\ \cup\  (\num{D_F^{\din}[\q, \q]}
  \star ( \remOne{D_F^{\din}[\p, \p]}{\epsilonset})) \\ 
  D_F^{\dgen}[\q, \q] &=& 1 \star D_F^{\din}[\q, \p] \\
  D_F^{\dgen}[\q, \myr] &=& \num{D_F^{\din}[\q, \myr]} \star
  D_F^{\din}[\q, \p],\ \myr \not\in \{\p,
  \q\} \\ 
%\end{eqnarray*}
%\begin{eqnarray*}
  I_F^{\dgen}[\p, \q] &=& \{(\fieldD{f}{}, \epsilon)\}
  \cup ((1 \star (\remOne{D_F^{\din}[\p,
      \p]}{\epsilonset})) \times \epsilonset) \\	 
  I_F^{\dgen}[\p, \myr] &=& 
  (1 \star (\remOne{D_F^{\din}[\p, \p]}{\epsilonset})) \times
  \{\beta\ \vert\ (\alpha, \beta) \in I_F^{\din}[\q, \myr] \}
  \\
  && \cup\ \{f^\drct\} \times 
  \{\beta\ \vert\ (\epsilon, \beta) \in I_F^{\din}[\q, \myr] \}
  \\
  && \cup\ \{f^{\indrct 1}\} \times 
  \{\beta\ \vert\ (\alpha, \beta) \in I_F^{\din}[\q, \myr],
  \alpha \not= \epsilon \},\ \myr \not\in \{\p, \q\} \\
  I_F^{\dgen}[\s, \q] &=& (1 \star D_F^{\din}[\s, \p]) \times
  \epsilonset,\ \s \not\in \{\p, \q\} \\ 
  I_F^{\dgen}[\s, \myr] &=& (1 \star D_F^{\din}[\s, \p])
  \times \{\beta\ \vert\ (\alpha, \beta) \in I_F^{\din}[\q,
    \myr]\},\ \s \not\in \{\p, \q\},\ \myr \not\in \{\p,
  \q\},\ \s \not= \myr 
\end{eqnarray*}

\item {\tt p = q$\rightarrow$f}: The relations killed by the
  statement are same as that in case of {\tt p = NULL}. The
  relations created by this statement are heavily
  approximated by our analysis.  After this statement $\p$
  points to the heap object which is accessible from pointer
  $\q$ through $f$ link. The only inference we can draw is
  that \p\ is reachable from any pointer \myr\ such that
  \myr\ reaches $\q\rightarrow f$ before the assignment. This
  information is available because $I_F^{\din}[\q, \myr]$
  will have an entry of the form $(f^\drct, \alpha)$ for some
  $\alpha$. 

  As \p\ could potentially point to a cycle(DAG) reachable
  from \q, we set:
  $$\GenC{p} = \InC{q} \qquad \GenD{p} = \InD{q}$$ We record
  the fact that \q\ reaches \p\ through the path
  $f$. Also, any object reachable from \q\ using field $f$
  is marked as reachable from \p\ through any
  possible field.
\begin{eqnarray*}
  f_{\q\p} &=& \true  \qquad
  {h_{\p\myr}} =
  \num{\project{D_F^{\din}[\q,\myr]}{f}} \geq 1\quad
  \forall h \in \fields, \forall \myr \in \heap 
\end{eqnarray*}

The equations to compute the generated relations for  $D_F$ and $I_F$
can be divided into three components. We explain each of the
component, and give the equations.

As a side-effect of the statement, any node \s\ that is
reachable from \q\ through field $f$ before the statement,
becomes reachable from \p. However, the information available
is not sufficient to determine the path from \p\ to
\s. Therefore, we conservatively assume that any path
starting from \p\ can potentially reach \s. This is achieved
in the analysis by using a universal path set \upath\ for
$D_F[\p,\s]$. The set \upath\ is defined as:
\[ \upath \quad=\quad \epsilonset \cup \bigcup_{f\in\fields} \{f^{\drct},
f^{\indrct\infty}\} \] Because it is also not possible to determine
if there exist a path from $\p$ to itself,  we safely
conclude a self loop on $p$ in case a cycle is reachable
from \q\ (i.e., \q.\shape\ evaluates to \Cycle). 
%%The interference from \p\ to itself is computed
%%conservatively in all cases. 
These
observations result in the following equations:
\begin{eqnarray*}
  D_1[\p, \s] &=&  \upath \quad
  \forall \s \in \heap, \s \not=\p \wedge
  \project{D_F^{\din}[\q, \s]}{f} \not= \emptyset
  \\
  D_1[\p, \p] &=& \left\{\begin{array}{@{}ll}
    \upath& \q.\shape \mbox{ evaluates to } \Cycle \\
    \epsilonset & \mbox{Otherwise}
  \end{array}\right.  \\
  I_1[\p, \p] &=& \upath \times \upath  
\end{eqnarray*}

Any node \s\ (including \q), that has paths to \q\ before the
statement, will have paths to \p\ after the
statement. However, we can not know the exact number of paths
\s\ to \p, and therefore use upper limit ($\infty$) as an
approximation:
\begin{eqnarray*}
  D_2[\s, \p] &=& \infty \star D_F^{\din}[\s, \q] \quad
  \forall \s \in \heap, \s \not=\q \\
  {D_2[\q,\p]} &=& \{ f^{\drct}\} \cup (\infty \star (\remOne{D_F^{\din}[\q, \q]}{\epsilonset})) \cup \upath  \\ 
  %I_2[\s, \p] &=& (\infty \star D_F^{\din}[\s, \q]) \times
  %\epsilonset \quad 
  %\forall \s \in \heap, \s \not=\q \\
  %I_2[\q,\p] &=& \{(f^{\drct}, \epsilon\}
  I_2[\s, \p] &=& D_2[\s, \p] \times
  \epsilonset \quad \forall \s \in \heap 
\end{eqnarray*}

The third category of nodes to consider are those that
interfere with the node reachable from \q\ using direct path
$f$. Such a node \s\ will have paths to \p\ after the
statement. Also the nodes that interfere with the node reachable from \q\ using direct or indirect path
$f$ will interfere with \p\ after the statement. Thus, we have:
\begin{eqnarray*}
  D_3[\s, \p] &=& \{ \alpha \ \vert \ (f^{\drct}, \alpha) \in I_F^{\din}[\q, \s] \}  \\
  I_3[\s, \p] &=& \{\alpha \ \vert\ (f^\anysup, \alpha) \in I_F^{\din}[\q, \s]\} \times \upath 
\end{eqnarray*}
Finally, we compute the $I_F$ and $D_F$ relations as:
\begin{eqnarray*}
D_F^{\dgen}[\myr, \s] &= & D_1[\myr, \s] \cup D_2[\myr, \s] \cup D_3[\myr, \s] \quad \forall\myr,\s \in \heap  \\
I_F^{\dgen}[\myr, \s] &= & I_1[\myr, \s] \cup I_2[\myr, \s] \cup I_3[\myr, \s] \quad \forall\myr,\s \in \heap 
\end{eqnarray*}
\end{enumerate}

%%%%%%%%%%%%%%%%%%%%%%%%%%%%%%%%%%%%%%%%%%%%%%%%%%%%%
\section{Inter-procedural Analysis} \label{Interprocedural_Analysis}
%%%%%%%%%%%%%%%%%%%%%%%%%%%%%%%%%%%%%%%%%%%%%%%%%%%%%
To handle procedure calls we use simple inter-procedural
analysis that works by creating an invocation graph of the
program. To handle recursive calls, for which the invocation
structure is statically unknown, we use approximate summary
for one of the nodes involved in the recursion chain, and use
it to break the cycle. At each call site, the two matrices
($D_F$ and $I_F$) along with the boolean functions are fed as
input to the called procedure, after proper mapping between
formal and actual arguments. The called procedure is then
analyzed to create the corresponding output matrices and the
boolean functions. This approach is similar to the work by
Ghiya et.~al.~\cite{Ghiya96}.
