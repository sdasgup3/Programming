\section{Related Work}
It was for functional languages that first shape analysis was looked at. Jones and Muchnick~\cite{Muchnick79} 
suggested a method for finding shape of unbounded data objects in LISP like
languages using regular tree grammars. They associate with each program point a set of 
shape graphs and to handle termination of analysis they use k-limiting approach. So they treat all nodes
whose distance is more than k from root as a single summarized node. Due to its large consumption of space 
and time the analysis is not practical.

Chase et al.~\cite{Chase90}  has used the concept of heap reference counts. They associate each node with a reference 
count and then try to find that part of the heap where all nodes
have reference count as one, such portions are said to be a tree or list. 
They have tried to tackle the problems with k-limiting to some extent, however their work fails obtaining 
accurate results for recursive data structures.

Sagiv et al.~\cite{Sagiv99},~\cite{Sagiv02} have presented a family of abstract interpretation algorithms based on three-valued logic. 
They use abstraction, a method for summarizing node and to handle destructive updates they have 
come up with re-materialization which refers to the process of splitting summary nodes. An exponential number of shape nodes may arise because of
abstract interpretation, so its not suitable for practical purposes. Sagiv and Noam~\cite{SagivInter02} have also looked into inter procedural shape 
analysis for recursive programs but they work only on linked lists. 

The main idea of Brian et al.~\cite{hackett05region} is to  decompose heap abstractions and independently analyze different parts of the heap.
They also decompose memory abstraction horizontally and vertically. Now for the local work to propagate globally they proposed 
and used a context sensitive inter procedural shape analysis algorithm.

A dynamic shape analysis technique was proposed by Jump et al.~\cite{maria09dynamic}. They compute a class field 
summary graph that summarizes the dynamic object graph. This summary graph also records the in-degree and out-degree of each 
object which are the recursive degree metrics. In their analysis they keep track of those node which are of fixed degree and
also those whose degree is in a particular range. Since running the analysis after each pointer statement is very costly they 
do it by piggybacking with garbage collection. 

Susan et al.~\cite{sagivDemand95} have a polynomial worst case method for inter procedural analysis provided its a demand 
data flow analysis. It will determine whether a single given data flow value holds at some give point. But the class of problems 
it can handle was limited. Alexy~\cite{interAbstr06} represent heap portions independently by using the notions of abstraction and separation logic.
Their representation helps to easily separate the portion which is reachable and which is not from a procedure. Its limitation is that it 
supports only linked lists, doubly linked lists and trees.

Ghiya et al.~\cite{Ghiya96} estimates the shapes of heap structures pointed to by pointers as a \emph{Tree}, \emph{Dag} or \emph{Cycle}. 
They use Direction, Interference matrices and shape attributes as their data flow values which gets generated and killed after each pointer 
statements. As this is very closely related to our work, we have given a detailed view of it in the Appendix section.

Marron et al.~\cite{marron06static} uses a graph based heap model with objects being vertices and pointers being the labeled edges. 
A node is considered as a set of cells and each node is associated with a layout saying \emph{Singleton}, \emph{List}, \emph{Tree}, 
\emph{Multipath} or \emph{Cycle}. These layouts honors the order \emph{Singleton} $<$ \emph{List} $<$ \emph{Tree} $<$ \emph{Multipath} $<$ \emph{Cycle}. 
This means that suppose a node is of layout \emph{Tree} then it may have properties of \emph{Singleton}, \emph{List} or \emph{Tree}.
They have methods by which summarized nodes are split to concrete nodes (and edges) but its only for for the most common cases encountered, 
this enables them to handle strong updates. Then they have proposed a context sensitive analysis~\cite{marron08context} for the same graph based
heap models. For this they have come up with operations \emph{project}$/$\emph{extend}. \emph{project} removes that part of heap which is 
unaffected by a called procedure and \emph{extend} rejoins the unreachable portion back after return. 

The work of Sandeep et al.~\cite{Sandeep11thesis}, which is extended in the present paper, is explained in detail in the
following section for a better understanding of the basis of this work.

\section[Precise Shape Analysis using Field Sensitivity]{Analysis of Sandeep et. al.~\cite{Sandeep11thesis}\footnote{The contents of this section are borrowed from~\cite{Sandeep11thesis}}}
Most of the definitions and technical terms used in this section are borrowed from the aforementioned paper.
As we will be using these details so throughout the report we have mentioned this as a separate section in our report.
They have presented a shape analysis technique that uses limited field sensitivity to infer the shape. 
As this technique is able to handle destructive updates, so precise shape information can be obtained.
They generated data flow values in the form of field sensitive matrices and boolean equations at each program point to obtain the shape information.

\subsection{Definitions And Notations}
At a particular program point, the heap structure is viewed as a directed
graph, the nodes of which represent the allocated objects and
the edges represent the connectivity through pointer fields.
Pictorially, inside a node all the relevant pointer
variables are shown that can point to the heap object corresponding to
that node. The edges are labeled by the name of the
corresponding pointer field. 

Let $\heap$ denotes the set of all heap directed
pointers at a particular program point and $\fields$
denotes the set of all pointer fields at that program point.
Given two heap-directed pointers $\p$, $\q$ $\in$
$\heap$, a path from $\p$ to $\q$ is the sequence
of pointer fields that need to be traversed in the heap to
reach from $\p$ to $\q$.  The length of a path is
defined as the number of pointer fields in the path.  As the
path length between two heap objects may be unbounded, only the first field of a path is stored. 
To distinguish between a path of length one
(direct path) from a path of length greater than one
(indirect path) that start at the same field, the
superscript $\drct$ for a direct path and $\indrct$ for an
indirect path are used. In pictures, solid edges are used for direct
paths, and dotted edges for indirect paths.

It is also possible to have multiple paths between two
pointers starting at a given field $f$, with at most one
direct path $f^\drct$. However, the number of indirect paths
$f^\indrct$ may be unbounded. As there can only be a finite
number of first fields, first fields of paths are stored,
including the count for the indirect paths, between two
pointer variables in a set. To bound the size of the set, a limit $k$ is put 
on number of repetitions of a particular field. If the number goes beyond $k$, the number of
paths with that field is treated as $\infty$. 

\begin{example} {\rm
\begin{figure}
\centering
\newcommand{\smf}{$f$}%%\scriptstyle f$}
\newcommand{\smh}{$h$}%%\scriptstyle h$}
\newcommand{\smg}{$g$}%%\scriptstyle g$}
\begin{tabular}{@{}cc@{}}
{\small \tt
    \begin{tabular}[b]{l}
      S1. q = p; \\
      S2. {\bf while}(...) \{ \\
      S3. \ \ \ \ q$\rightarrow$g = s; \\
      S4. \ \ \ \ q = q$\rightarrow$f; \\
      S5. \} \\
    \end{tabular}
  } &
 \scalebox{0.8}{ \psset{unit=1mm}
  \begin{pspicture}(0,0)(50,30)
    %%\psframe(0,0)(50,30)
    \putnode{q0}{origin}{3}{3}{}
    \putnode{q1}{q0}{0}{0}{\pscirclebox{\mbox{\p}}}
    \putnode{q2}{q1}{10}{0}{\pscirclebox[framesep=2.5]{\mbox{}}}
    \putnode{q3}{q2}{10}{0}{\mbox{\ldots}}
    \putnode{q4}{q3}{10}{0}{\pscirclebox{\mbox{\q}}}
    \putnode{q5}{q4}{10}{0}{\pscirclebox[framesep=2.5]{\mbox{}}}
    
    \ncline{->}{q1}{q2}
    \Bput[0.2]{\smf}
    \ncline{->}{q2}{q3}
    \Bput[0.2]{\smf}
    \ncline{->}{q3}{q4}
    \Bput[0.2]{\smf}
    \ncline{->}{q4}{q5}
    \Bput[0.2]{\smf}
    
    \putnode{s0}{q4}{15}{15}{\pscirclebox{\mbox{\s}}}
    \nccurve[angleA=90,angleB=145,ncurv=1]{->}{q1}{s0}
    \Aput[.1]{\smg}
    
    \putnode{t0}{q2}{0}{15}{\pscirclebox[framesep=2.5]{\mbox{}}}
    \nccurve[linestyle=dotted, dotsep=.5, angleA=-55,angleB=-165,
      ncurv=.2,nodesepA=-.3]{->}{t0}{s0}
    
    \ncline[linestyle=dotted, dotsep=.5]{->}{t0}{s0}
    \nccurve[linestyle=dotted, dotsep=.5,angleA=55,angleB=165,
      ncurv=.3,nodesepA=-.3]{->}{t0}{s0}
    

    \ncline[nodesepA=-.5,nodesepB=-.5]{->}{q1}{t0}
    \Aput[.1]{\smh}
    
    \ncline{->}{q2}{s0}
    \Bput[0.2]{\smg}
    \ncline[nodesepA=-.75,nodesepB=-.75]{->}{q4}{s0}
    \Bput[0.2]{\smg}
\end{pspicture}} \\
\scalebox{0.80}{ (a) A code fragment} & \scalebox{0.80}{ (b) 
  \begin{tabular}[t]{@{}p{95mm}}
    A possible heap graph for code in (a). Solid edges are the
    direct paths, dotted edges are the indirect paths.
  \end{tabular}}
  \end{tabular}
\caption{Paths in a heap graph}
\label{fig:pathsAndSets}
\end{figure}


Figure~\ref{fig:pathsAndSets}(a) shows a code fragment and
Fig.~\ref{fig:pathsAndSets}(b) shows a possible heap graph
at a program point after line {\tt S5}. In any execution, there is one path
between $\p$ and $\q$, starting with field $f$, whose
length is statically unknown. This information is stored by
as the set $\{\fieldI{f}{}{1}\}$. Further, there are
unbounded number of paths between $\p$ and $\s$, all
starting with field $f$. There is also a direct path from
$\p$ to $\s$ using field $g$, and 3 paths starting with
field $h$ between $\p$ and $\s$. Assuming the limit $k
\geq 3$, this information can be represented by the set $\{
g^\drct, f^{\indrct\infty}, h^{\indrct 3} \}$. On the other hand, if $k < 3$,
then the set would be $\{ g^\drct, f^{\indrct\infty}, h^{\indrct\infty} \}$.
  } \hfill\psframebox{}
\end{example}

For brevity, 
$f^{\anysup}$ is used for the cases when it is irrelevant to
distinguish between direct or indirect path starting at the
first field $f$. Next field sensitive matrices are defined.

%FIELD SENSITIVE DIRECTION MATRIX 
\begin{definition}
\label{DFM_matrix}
Field sensitive Direction matrix
$D_F$ is a matrix that stores information 
about paths between two pointer variables.
Given $\p, \q \in
\heap, f \in \fields$:
\begin{eqnarray*}
  \epsilon & \in \DFM{p}{p}& \mbox{ where $\epsilon$
    denotes the empty path.} \\
  f^\drct  &\in  \DFM{p}{q} & \mbox{ if there is a direct
    path $f$ from $\p$ to $\q$.}\\
  f^{\indrct m} & \in  \DFM{p}{q} & 
  \mbox{\begin{tabular}[t]{p{105mm}}if there are $m$ indirect
      paths starting with field $f$ from $\p$ to $\q$ and $m
      \leq k.$
    \end{tabular}
  } \\
  f^{\indrct\infty} & \in  \DFM{p}{q} &
  \mbox{\begin{tabular}[t]{p{105mm}}if there are $m$ indirect
      paths starting with field $f$ from $\p$ to $\q$ and $m >
      k.$
  \end{tabular}}  
\end{eqnarray*}
\end{definition}

Let \nat\ denote the set of natural numbers. The
following partial order are defined for approximate paths used by the
analysis. For $ f \in \fields,\ m,n \in \nat,\ n \leq m$:
$$
\epsilon \sqsubseteq \epsilon, \quad 
f^\drct \sqsubseteq  f^\drct,  \quad
f^{\indrct\infty}  \sqsubseteq  f^{\indrct\infty}, \quad
f^{\indrct m} \sqsubseteq f^{\indrct\infty}, \quad
f^{\indrct n} \sqsubseteq f^{\indrct m} \enspace .
$$
The partial order is extended to set of paths $S_{P_1},
S_{P_2}$ as\footnote{Note that for the analysis, for a given
  field $f$, these sets contain at most one entry of type
  $f^\drct$ and at most one entry of type $f^\indrct$}:
\begin{eqnarray*}
  S_{P_1} \sqsubseteq S_{P_2} &\Leftrightarrow& \forall \alpha \in
  S_{P_1}, \exists \beta \in S_{P_2}\ s.t. \alpha \sqsubseteq \beta \enspace .
\end{eqnarray*}
For pair of paths:
\begin{eqnarray*}
  (\alpha, \beta) \sqsubseteq (\alpha', \beta') 
  \Leftrightarrow 
   (\alpha \sqsubseteq \alpha')  \wedge
  (\beta \sqsubseteq  \beta')
\end{eqnarray*}
For set of pairs of paths $R_{P_1}, R_{P_2}$:
\begin{eqnarray*}
  R_{P_1} \sqsubseteq R_{P_2} \Leftrightarrow \forall
  (\alpha, \beta) \in
  R_{P_1}, \exists (\alpha', \beta') \in
  R_{P_2}\ s.t. (\alpha, \beta) \sqsubseteq (\alpha', \beta')
\end{eqnarray*}


Two pointers $\p,\q \in \heap$ are said to
interfere if there exists $\s \in \heap$ such that both
$\p$ and $\q$ have paths reaching $\s$. Note that $\s$ could
be $\p$ (or $\q$) itself, in which case the path from $\p$
(from $\q$) is $\epsilon$.

%FIELD SENSITIVE DIRECTION MATRIX 
\begin{definition}\label{IFM_matrix}
Field sensitive Interference matrix $I_F$ between
two pointers captures the ways in which these pointers are
interfering.  For $\p, \q, \s \in \heap, \p \not= \q$,
the following relation holds for $D_F$ and $I_F$: 
\begin{eqnarray*}
  \DFM{p}{s} \times \DFM{q}{s} &\sqsubseteq&
  \IFM{p}{q} \enspace . \label {eq:rel-df-if}
\end{eqnarray*}
\end{definition}

The analysis computes over-approximations for the matrices
$D_F$ and $I_F$ at each program point. While it is possible
to compute only $D_F$ and use above equation to
compute $I_F$, computing both explicitly results in better
approximations for $I_F$. Note that interference relation is
symmetric, i.e.,
\begin{eqnarray*}
  (\alpha, \beta) \in I_F[\p, \q] \Leftrightarrow
   (\beta, \alpha) \in I_F[\q, \p] \enspace .
\end{eqnarray*}
While describing the analysis, the
above relation is used to show the computation of only one of the two
entries. 

\begin{figure}
\centering
\begin{tabular}{c@{$\qquad\qquad$}c}
\psset{unit=1.2mm}
\begin{pspicture}(0,0)(15,20)
	\psset{linecolor=black}
  \putnode{p0}{origin}{3}{9}{\pscirclebox{\p}}
  \putnode{q0}{p0}{5}{7}{\pscirclebox{\q}}
  %\putnode{r0}{p0}{0}{-14}{\pscirclebox{\myr}}
  \putnode{r0}{p0}{10}{0}{\pscirclebox{\myr}}
	\psset{linecolor=black}
  \ncline[nodesep=-.3]{<-}{p0}{q0}
  \aput[0.2](0.4){$\scriptstyle f$}
  \ncline{<-}{p0}{r0}
  \aput[0.2](0.4){$\scriptstyle g$}
%   %\ncline[nodesep=-.5]{->}{s0}{r0}
%   %\aput[0.2](0.2){$\scriptstyle f_5$}
%   \nccurve[angleA=45,angleB=0,linestyle=dotted,dotsep=0.2,ncurv=1,nodesepA=-.8]{->}{p0}{s0}
%   \aput[0.2](0.2){$\scriptstyle f_2$}
%   %\nccurve[angleA=180,angleB=225,linestyle=dotted,dotsep=0.2,ncurv=1,nodesepB=-.5]{->}{q0}{r0}
%   %\bput[0.2](0.2){$\scriptstyle f_4$}
\end{pspicture} &
\scalebox{0.8}{
\renewcommand{\arraystretch}{1.2}
\begin{tabular}[b]{|c|c|c|c|}
\hline
$D_F$          & \p                          & \q                             &  \myr  \\ \hline \hline 
\p 	           & $\{\epsilon\}$              & $\emptyset$            & $\emptyset$     \\ \hline 
\q             & $\{\fieldD{f}{}\}$                  & $\{\epsilon\}$                 & $\emptyset$   \\ \hline
\myr             & $\{\fieldD{g}{}\}$                 & $\emptyset$                    & $\{\epsilon\}$       \\ \hline
\end{tabular}} \\
\scalebox{0.80}{ (a) Heap graph} & \scalebox{0.80}{ (b) Direction Matrix}  \\ \\
% \multicolumn{2}{c}{
& \multirow{6}{*}{
\begin{tabular}{c}
$f_{\q\p} = \true$ \\
$f_{\p\q} = \false$ \\
$g_{\myr\p} = \true$ \\ 
\end{tabular}
} \\

\scalebox{0.80}{
\renewcommand{\arraystretch}{1.2}
\newcommand{\iwd}{0.23\columnwidth}
\begin{tabular}[b]{|c|c|c|c|}
\hline 
$\ I_F$     & $\p$	               & $\q$ &  $\myr$             \\ \hline \hline 
%%
$\p$ & $\{\epsilon, \epsilon\}$    & $\{(\epsilon, \fieldD{f}{}) \}$               & $\{(\epsilon, \fieldD{g}{})\}$      \\      \hline 
%%
$\q$       & $\{(\fieldD{f}{}, \epsilon)\}$                 & $\{\epsilon, \epsilon\}$          & $\{(\fieldD{f}{}, \fieldD{g}{})\}$         \\  \hline              
%%
$\myr$       & $\{(\fieldD{g}{}, \epsilon)\}$             & $\{(\fieldD{g}{}, \fieldD{f}{})\}$             & $\{\epsilon, \epsilon\}$         \\ \hline              
%%
\end{tabular}} &  \\
% } \\
% \multicolumn{2}{c}{\scalebox{0.80}{ (c) Interference Matrix} }
\scalebox{0.80}{(c) Interference Matrix}  & \scalebox{0.80}{(d) Bolean variables}
\end{tabular}
%\caption{A heap graph and its field sensitive path matrices\label{fig:DFM_IFM}}
\end{figure}


\begin{example}{\rm
Figure~\ref{fig:DFM_IFM} shows a heap graph and the
corresponding field sensitive matrices as computed by the
analysis.  } \hfill\psframebox{}
\end{example}

As mentioned earlier, for each variable $\p \in \heap$, the
analysis uses attributes $\p_{\subD}$ and $\p_{\subC}$ to
store boolean functions telling whether $\p$ can reach a DAG or
cycle respectively in the heap. The boolean functions
consist of the values from matrices $D_F$, $I_F$, and the field
connectivity information. For $f \in \fields, \p, \q \in
\heap$, field connectivity is captured by boolean variables
of the form $f_{pq}$, which is true when $f$ field of $\p$ points
directly to $\q$. 
The shape of \p, \p.\shape, can be obtained by evaluating
the functions for the attributes $\p_{\subC}$ and
$\p_{\subD}$, and using Table~\ref{tbl:det_shape}.
\begin{table}
\caption{Determining shape from boolean
  attributes\label{tbl:det_shape}}
\begin{center}
\begin{tabular*}{0.75\textwidth}{@{\extracolsep{\fill}} cccc|c }
\hline
$\p_{\subC}$ &$\quad$ & $\p_{\subD}$ &$\quad$& $\p.\shape$ \\ 
\hline
\true  && Don't Care  && Cycle        \\ 
\false  && \true          && DAG    \\ 
\false  && \false          && Tree   \\ 
\hline
\end{tabular*}
\end{center}
\end{table}

The following operations are used in the analysis. Let $S$
denote the set of approximate paths between two nodes, $P$
denote a set of pair of paths, and $k \in \nat$ denotes the
limit on maximum indirect paths stored for a given
field. Then,
\begin{itemize}
\item Projection: For $f \in \fields$,
  \project{S}{f}\ extracts the paths starting at field $f$.
  $$\project{S}{f} \equiv S \cap \{f^{\drct}, f^{\indrct 1}, \ldots,
  f^{\indrct k}, f^{\indrct \infty}\} \enspace .$$

\item Counting: The count on the number of paths is defined
  as :
  \begin{eqnarray*}
   \num{\epsilon} =     1, \qquad
   \num{f^{\drct}} =     1, &\qquad&
   \num{f^{\indrct\infty}} =     \infty, \qquad
   \num{f^{\indrct j}} =  j \mbox{ for } j \in \nat \\
   \num{S} &=&   \sum_{\alpha \in S}\num{\alpha} 
  \end{eqnarray*}

Also,  
  \begin{eqnarray*}
   \num{(\alpha, f^{\indrct m})} &=&  \left\{ \begin{array}{@{}ll}  
   			m        &  \mbox{if } \alpha \in \{f^{\drct}, \epsilon\} \\
			m*n      &  \mbox{if } \alpha = f^{\indrct n} \\
			\infty   &  \mbox{if } \alpha = f^{\indrct\infty} \\
			\end{array}\right. \\
   \num{(f^{\indrct m}, \beta)} &=&  \left\{ \begin{array}{@{}ll}  
   			m        &  \mbox{if } \beta \in \{f^{\drct}, \epsilon\} \\
			m*n      &  \mbox{if } \beta = f^{\indrct n} \\
			\infty   &  \mbox{if } \beta = f^{\indrct\infty} \\
			\end{array}\right. \\
   \num{(\alpha, \beta)} &=&  \begin{array}{@{}ll}
   			 1       &  \mbox{if } \alpha, \beta \in \{f^{\drct}, \epsilon\} \\
			\end{array} \\
   \num{(f^{\indrct\infty}, \beta)} &=&  \begin{array}{@{}ll}
   			 \infty       &  \mbox{where }  \beta \in \{f^{\drct}, \epsilon, f^{\indrct\infty}\} \\
			\end{array} \\
   \num{(\alpha, f^{\indrct\infty})} &=&  \begin{array}{@{}ll}
   			 \infty       &  \mbox{where } \alpha \in \{f^{\drct}, \epsilon, f^{\indrct\infty}\} \\
			\end{array} \\
	\num{P} &=&   \sum_{(\alpha, \beta) \in P}\num{(\alpha, \beta)}		
  \end{eqnarray*}

\item Path removal, intersection and union over set of
  approximate paths : For singleton sets of paths
    $\{\alpha\}$ and $\{\beta\}$, path removal
    (\remOne{\{\alpha\}}{\{\beta\}}), intersection
    ($\{\alpha\} \cap \{\beta\}$) and union($\{\alpha\} \cup
    \{\beta\}$) operations are defined as given in
    Table~\ref{tab:path-ops}. These definitions can be extended
    to set of paths in a natural way. For example,
	for general sets of paths, $S_1$ and $S_2$, 
	the definition of removal can be extended as:
\begin{eqnarray*}
  \remOne{S_1}{S_2}  = \bigcap_{\beta \in
    {S_2}}(\bigcup_{\alpha \in {S_1}}  \remOne{\{\alpha\}}{\{\beta\}}  )
\end{eqnarray*}

\begin{table}[t]
\caption{Path removal, intersection and union
  operations, where $\gamma$ denotes any other path.\label{tab:path-ops}}
\begin{center}
\scalebox{0.8}{
\renewcommand{\arraystretch}{1.1}
\begin{tabular}{c@{$\qquad\qquad$}c}
(a) Path removal & (b) Intersection \\ 
%\begin{tabular*}{0.55\textwidth}{@{\extracolsep{\fill}} cc|ccccc}
\begin{tabular*}{0.55\textwidth}{@{}c@{}c@{}|ccccc}
  \hline
  $\remOne{}{}$ & $\{\beta\}$ & $\{\epsilon\}$ & $\{f^{\drct}\}$ & $\{f^{\indrct j}\}$& $\{f^{\indrct\infty}\}$ &  $\{\gamma\}$ \\
  $\{\alpha\}$ && & & & & 
  \\ \hline %\hline
  $\{\epsilon\}$ && $\emptyset$  & $\{\epsilon\}$ & $\{\epsilon\}$ & $\{\epsilon\}$ & $\{\epsilon\}$ \\ %\hline
  $\{f^{\drct}\}$ && $\{f^{\drct}\}$ & $\emptyset$ & $\{f^{\drct}\}$ & $\{f^{\drct}\}$ & $\{f^{\drct}\}$ \\ %\hline
  $\{f^{\indrct i}\}$ && $\{f^{\indrct i}\}$ & $\emptyset$ & $\{f^{\indrct m}\}$& $\emptyset$ & $\{f^{\indrct i}\}$ \\ %\hline
  $\{f^{\indrct\infty}\}$ && $\{f^{\indrct\infty}\}$ &$\emptyset$ & $\{f^{\indrct \infty}\}$& $\{f^{\indrct
    \infty}\}$ & $\{f^{\indrct\infty}\}$ \\ \hline
\end{tabular*} &
%\begin{tabular*}{0.50\textwidth}{@{\extracolsep{\fill}} cc|ccccc}
\begin{tabular*}{0.50\textwidth}{@{}c@{}c@{}|ccccc}
  \hline
  $\cap$&$\{\beta\} $& $\{\epsilon\}$ & $\{f^{\drct}\}$ &
  $\{f^{\indrct j}\}$& $\{f^{\indrct\infty}\}$ & $\{\gamma\}$ \\
  $\{\alpha\}$ && & & & & 
  \\ \hline %\hline
  $\{\epsilon\}$ && $\epsilon$  & $\emptyset$ & $\emptyset$ & $\emptyset$ & $\emptyset$ \\ %\hline
  $\{f^{\drct}\}$ && $\emptyset$ & $\{f^{\drct}\}$ & $\emptyset$ & $\emptyset$ & $\emptyset$ \\ %\hline
  $\{f^{\indrct i}\}$ && $\emptyset$ & $\emptyset$ & $\{f^{\indrct n}\}$& $\{f^{\indrct i}\}$ & $\emptyset$ \\ %\hline
  $\{f^{\indrct\infty}\}$ && $\emptyset$ &$\emptyset$ & $\{f^{\indrct j}\}$& $\{f^{\indrct \infty}\}$ & $\emptyset$ \\ \hline
\end{tabular*} \\
& \\
\multicolumn{2}{c}{(c) Union} \\
\multicolumn{2}{c}{
%\begin{tabular*}{0.75\textwidth}{@{\extracolsep{\fill}} cc|ccccc}
\begin{tabular*}{0.72\textwidth}{@{}c@{}c@{}|ccccc}
  \hline
  $\cup$&$\{\beta\} $& $\{\epsilon\}$ & $\{f^{\drct}\}$ &
  $\{f^{\indrct j}\}$& $\{f^{\indrct\infty}\}$ & $\{\gamma\}$ \\
  $\{\alpha\}$ && & & & &  
 \\ \hline %\hline
  $\{\epsilon\}$ && $\{\epsilon\}$  & $\{\epsilon, \fieldD{f}{}\}$ & $\{\epsilon, \fieldI{f}{}{j} \}$ & $\{\epsilon, \fieldI{f}{}{\infty}\}$ & $\{\epsilon, \gamma\}$  \\ %\hline
  $\{f^{\drct}\}$ && $\{\fieldD{f}{}, \epsilon\}$ & $\{\fieldD{f}{}\}$ & $\{\fieldD{f}{}, \fieldI{f}{}{j}\}$ & $\{\fieldD{f}{}, \fieldI{f}{}{\infty}\}$ & $\{\fieldD{f}{}, \gamma\}$  \\ %\hline
  $\{f^{\indrct i}\}$ && $\{\fieldI{f}{}{i}, \epsilon\}$ & $\{\fieldI{f}{}{i}, \fieldD{f}{}\}$ & $\{\fieldI{f}{}{t}\}$& $\{\fieldI{f}{}{\infty}\}$ & $\{\fieldI{f}{}{i}, \gamma\}$  \\ %\hline
  $\{f^{\indrct\infty}\}$ && $\{\fieldI{f}{}{\infty},
 \epsilon\}$ & $\{\fieldI{f}{}{\infty}, \fieldD{f}{}\}$ &
 $\{\fieldI{f}{}{\infty}\}$& $\{\fieldI{f}{}{\infty}\}$ &
 $\{\fieldI{f}{}{\infty}, \gamma\}$  \\ \hline
\end{tabular*}} \\& \\
\multicolumn{2}{c}{
 $i, j  \in \nat$, $m = {\tt max}(i-j, 0)$, $n = {\tt
  min}(i, j)$ and $t = \left\{\begin{array}{lcl}
  i+j    && \mbox { if } i+j \leq k \\
  \infty && \mbox{ Otherwise} 
  \end{array}\right.$.
}
\end{tabular}}
\end{center}
\end{table}


\item Path removal, intersection and union over set of pair
  of paths : For singleton sets of paths
    $\{\alpha, \beta\}$ and $\{\gamma, \delta\}$, union($\{\alpha, \beta\} \cup
    \{\gamma, \delta\}$), intersection
    ($\{\alpha, \beta\} \cap  \{\gamma, \delta\}$) and path removal
    (\remOne{\{\alpha, \beta\}}{\{\gamma, \delta\}}) operations are defined as given in
    Figure~\ref{tab:union_path_pair}, ~\ref{tab:intersection_path_pair} and ~\ref{tab:removal_path_pair} respectively. As before these 
	definitions can be extended to set of pair of paths in a natural way. For example,
	for general sets of paths, $P_1$ and $P_2$, 
	the definition of removal can be extended as:
\begin{eqnarray*}
  \remOne{P_1}{P_2}  = \bigcap_{(\gamma, \delta) \in
    {P_2}}(\bigcup_{(\alpha, \beta) \in {P_1}}  \remOne{\{\alpha, \beta\}}{\{\gamma, \delta\}}  )
\end{eqnarray*}

\begin{figure}
\caption{Union operation between two singleton sets of pair of paths, where $\{\alpha, \beta\}$ denote any other pair of paths and
$x$, $y$, $m$, $n$ can be any positive integer or $\infty$
\label{tab:union_path_pair}}
\begin{tabular}{c}
\\
\scalebox{0.65}{
\begin{tabular}{|c|c|c|c|c|c|}
\hline
$\cup$ & \{(\fieldD{f}{},\fieldD{g}{})\} & \{(\fieldI{f}{}{m},\fieldD{g}{})\} & \{(\fieldD{f}{},\fieldI{g}{}{n})\} & \{(\fieldI{f}{}{m},\fieldI{g}{}{n})\} & \{($\alpha$, $\beta$)\}\\ \hline

\{($\epsilon$,$\epsilon$)\} & \{($\epsilon$,$\epsilon$),(\fieldD{f}{},\fieldD{g}{}),  & \{($\epsilon$,$\epsilon$),(\fieldI{f}{}{m},\fieldD{g}{}),  & \{($\epsilon$,$\epsilon$),(\fieldD{f}{},\fieldI{g}{}{n}),  & \{($\epsilon$,$\epsilon$),(\fieldI{f}{}{m},\fieldI{g}{}{n}),  & \{($\epsilon$,$\epsilon$),\\
& ($\epsilon$,\fieldD{g}{}),(\fieldD{f}{},$\epsilon$ )\} & ($\epsilon$,\fieldD{g}{}),(\fieldI{f}{}{m},$\epsilon$ )\} & ($\epsilon$,\fieldI{g}{}{n}),(\fieldD{f}{},$\epsilon$ )\} & ($\epsilon$,\fieldI{g}{}{n}),(\fieldI{f}{}{m},$\epsilon$ )\} & ($\alpha$, $\beta$) \}\\ \hline

\{($\epsilon$,\fieldD{g}{})\} & \{($\epsilon$,\fieldD{g}{}),(\fieldD{f}{},\fieldD{g}{})\} & \{($\epsilon$,\fieldD{g}{}),(\fieldI{f}{}{m},\fieldD{g}{})\} & \{($\epsilon$,\fieldD{g}{}),(\fieldD{f}{},\fieldI{g}{}{n}),  & \{($\epsilon$,\fieldD{g}{}),(\fieldI{f}{}{m},\fieldI{g}{}{n}),  & \{($\epsilon$,\fieldD{g}{}), \\
& & & ($\epsilon$,\fieldI{g}{}{n}),(\fieldD{f}{},\fieldD{g}{})\} & ($\epsilon$,\fieldI{g}{}{n}),(\fieldI{f}{}{m},\fieldD{g}{})\} & ($\alpha$, $\beta$)\}\\ \hline

\{($\epsilon$,\fieldI{g}{}{y})\} & \{($\epsilon$,\fieldI{g}{}{y}),(\fieldD{f}{},\fieldD{g}{}) & \{($\epsilon$,\fieldI{g}{}{y}),(\fieldI{f}{}{m},\fieldD{g}{}) & \{($\epsilon$,\fieldI{g}{}{n+y}),(\fieldD{f}{},\fieldI{g}{}{n+y})\} & \{($\epsilon$,\fieldI{g}{}{n+y}),(\fieldI{f}{}{m},\fieldI{g}{}{n+y})\} & \{($\epsilon$,\fieldI{g}{}{y}), \\
& ($\epsilon$,\fieldD{g}{}),(\fieldD{f}{},\fieldI{g}{}{y})\} & ($\epsilon$,\fieldD{g}{}),(\fieldI{f}{}{m},\fieldI{g}{}{y})\} & & & ($\alpha$, $\beta$)\}\\ \hline

\{(\fieldD{f}{},$\epsilon$)\} & \{(\fieldD{f}{},$\epsilon$),(\fieldD{f}{},\fieldD{g}{})\} & \{(\fieldD{f}{},$\epsilon$),(\fieldI{f}{}{m},\fieldD{g}{}) & \{(\fieldD{f}{},$\epsilon$),(\fieldD{f}{},\fieldI{g}{}{n})\} & \{(\fieldD{f}{},$\epsilon$),(\fieldI{f}{}{m},\fieldI{g}{}{n}) & \{(\fieldD{f}{},$\epsilon$), \\
& & (\fieldI{f}{}{m},$\epsilon$),(\fieldD{f}{},\fieldD{g}{})\} & & (\fieldI{f}{}{m},$\epsilon$). (\fieldD{f}{},\fieldI{g}{}{n})\} & ($\alpha$, $\beta$)\}\\ \hline

\{(\fieldI{f}{}{x},$\epsilon$)\} & \{(\fieldI{f}{}{x},$\epsilon$),(\fieldD{f}{},\fieldD{g}{}) & \{(\fieldI{f}{}{x+m},$\epsilon$),(\fieldI{f}{}{x+m},\fieldD{g}{})\} & \{(\fieldI{f}{}{x},$\epsilon$),(\fieldD{f}{},\fieldI{g}{}{n}) & \{(\fieldI{f}{}{x+m},$\epsilon$),(\fieldI{f}{}{x+m},\fieldI{g}{}{n})\} & \{(\fieldI{f}{}{x},$\epsilon$), \\
& (\fieldI{f}{}{x},\fieldD{g}{}),(\fieldD{f}{},$\epsilon$)\} & & (\fieldI{f}{}{x},\fieldI{g}{}{n}),(\fieldD{f}{},$\epsilon$)\} & & ($\alpha$, $\beta$)\}\\ \hline

\{(\fieldD{f}{},\fieldD{g}{})\} & \{(\fieldD{f}{},\fieldD{g}{})\} & \{(\fieldD{f}{},\fieldD{g}{}), & \{(\fieldD{f}{},\fieldD{g}{}),  & \{(\fieldD{f}{},\fieldD{g}{}),(\fieldD{f}{},\fieldI{g}{}{n}), & \{(\fieldD{f}{},\fieldD{g}{}),\\
& & (\fieldI{f}{}{m},\fieldD{g}{})\} & (\fieldD{f}{},\fieldI{g}{}{n})\} & (\fieldI{f}{}{m},\fieldD{g}{}) , (\fieldI{f}{}{m},\fieldI{g}{}{n})\} & ($\alpha$, $\beta$)\} \\ \hline

\{(\fieldI{f}{}{x},\fieldD{g}{})\} & \{(\fieldI{f}{}{x},\fieldD{g}{}), & \{(\fieldI{f}{}{x+m},\fieldD{g}{})\} & \{(\fieldI{f}{}{x},\fieldD{g}{}) , (\fieldD{f}{},\fieldI{g}{}{n}), & \{(\fieldI{f}{}{x+m},\fieldD{g}{}), & \{(\fieldI{f}{}{x},\fieldD{g}{}), \\
& (\fieldD{f}{},\fieldD{g}{})\} & & (\fieldI{f}{}{x},\fieldI{g}{}{n}) , (\fieldD{f}{},\fieldD{g}{})\} & (\fieldI{f}{}{x+m},\fieldI{g}{}{n})\} & ($\alpha$, $\beta$)\}\\ \hline

 \{(\fieldD{f}{},\fieldI{g}{}{y})\} & \{(\fieldD{f}{},\fieldD{g}{}), & \{(\fieldD{f}{},\fieldI{g}{}{y}) , (\fieldI{f}{}{m},\fieldD{g}{}), & \{(\fieldD{f}{},\fieldI{g}{}{n+y})\} & \{(\fieldD{f}{},\fieldI{g}{}{n+y}),  (\fieldD{f}{},\fieldI{g}{}{y}), & \{(\fieldD{f}{},\fieldI{g}{}{y}) , \\
& (\fieldD{f}{},\fieldI{g}{}{y})\} & (\fieldI{f}{}{m},\fieldI{g}{}{y}) , (\fieldD{f}{},\fieldD{g}{}), & & (\fieldI{f}{}{m},\fieldI{g}{}{n+y})\} & ($\alpha$, $\beta$)\}\\ \hline

 \{(\fieldI{f}{}{x},\fieldI{g}{}{y})\} & \{(\fieldD{f}{},\fieldD{g}{}),(\fieldI{f}{}{x},\fieldI{g}{}{y}), & \{(\fieldI{f}{}{x+m},\fieldD{g}{}), & \{(\fieldD{f}{},\fieldI{g}{}{n+y}), & \{(\fieldI{f}{}{m+x},\fieldI{g}{}{n+y})\} & \{(\fieldI{f}{}{x},\fieldI{g}{}{y}),  \\
& (\fieldD{f}{},\fieldI{g}{}{x}),(\fieldI{f}{}{x},\fieldD{g}{})\} & (\fieldI{f}{}{x+m},\fieldI{g}{}{y})\} & (\fieldI{f}{}{x},\fieldI{g}{}{n+y})\} & & ($\alpha$, $\beta$)\} \\ \hline

\end{tabular}
}
\\ \\
\scalebox{0.65}{
\begin{tabular}{|c|c|c|c|c|c|c|}
\hline
$\cup$ & \{($\epsilon$,$\epsilon$)\} & \{($\epsilon$,\fieldD{g}{})\} & \{($\epsilon$,\fieldI{g}{}{n})\} & \{(\fieldD{f}{},$\epsilon$)\} & \{(\fieldI{f}{}{m},$\epsilon$)\} & \{($\alpha$, $\beta$)\}\\ \hline

\{($\epsilon$,$\epsilon$)\} & \{($\epsilon$,$\epsilon$)\} & \{($\epsilon$,$\epsilon$),($\epsilon$,\fieldD{g}{})\} & \{($\epsilon$,$\epsilon$),($\epsilon$,\fieldI{g}{}{n})\} & \{($\epsilon$,$\epsilon$),(\fieldD{f}{},$\epsilon$)\} & \{($\epsilon$,$\epsilon$),(\fieldI{f}{}{x},$\epsilon$)\} & \{($\epsilon$,$\epsilon$)\} \\
&&&&&& ($\alpha$, $\beta$)\} \\\hline

\{($\epsilon$,\fieldD{g}{})\} & \{($\epsilon$,\fieldD{g}{}),($\epsilon$,$\epsilon$)\} & \{($\epsilon$,\fieldD{g}{})\} & \{($\epsilon$,\fieldD{g}{}),($\epsilon$,\fieldI{g}{}{n})\} & \{($\epsilon$,\fieldD{g}{}),(\fieldD{f}{},$\epsilon$), & \{($\epsilon$,\fieldD{g}{}),(\fieldI{f}{}{m},$\epsilon$), & \{($\epsilon$,\fieldD{g}{}),  \\
& & & & ($\epsilon$,$\epsilon$),(\fieldD{f}{},\fieldD{g}{})\} & ($\epsilon$,$\epsilon$),(\fieldI{f}{}{m},\fieldD{g}{})\} & ($\alpha$, $\beta$)\} \\ \hline

\{($\epsilon$,\fieldI{g}{}{y})\} & \{($\epsilon$,\fieldI{g}{}{y}),($\epsilon$,$\epsilon$)\} & \{($\epsilon$,\fieldI{g}{}{y}),($\epsilon$,\fieldD{g}{})\} & \{($\epsilon$,\fieldI{g}{}{y+n})\} & \{($\epsilon$,\fieldI{g}{}{y}),(\fieldD{f}{},$\epsilon$), & \{($\epsilon$,\fieldI{g}{}{y}),(\fieldI{f}{}{m},$\epsilon$), & \{($\epsilon$,\fieldI{g}{}{y}),\\
& & & & ($\epsilon$,$\epsilon$),(\fieldD{f}{},\fieldI{g}{}{y})\} & ($\epsilon$,$\epsilon$),(\fieldI{f}{}{m},\fieldI{g}{}{y})\} & ($\alpha$, $\beta$)\}\\ \hline

\{(\fieldD{f}{},$\epsilon$)\} & \{(\fieldD{f}{},$\epsilon$),($\epsilon$,$\epsilon$)\} & \{(\fieldD{f}{},$\epsilon$),($\epsilon$,\fieldD{g}{}), & \{(\fieldD{f}{},$\epsilon$),($\epsilon$,\fieldI{g}{}{n}), & \{(\fieldD{f}{},$\epsilon$)\} & \{(\fieldD{f}{},$\epsilon$),(\fieldI{f}{}{x},$\epsilon$)\} & \{(\fieldD{f}{},$\epsilon$),\\ 
& & (\fieldD{f}{},\fieldD{g}{}),($\epsilon$,$\epsilon$)\} & (\fieldD{f}{},\fieldI{g}{}{n}),($\epsilon$,$\epsilon$)\} & & & ($\alpha$, $\beta$)\}\\ \hline

\{(\fieldI{f}{}{x},$\epsilon$)\} & \{(\fieldI{f}{}{x},$\epsilon$),($\epsilon$,$\epsilon$)\} & \{(\fieldI{f}{}{x},$\epsilon$),($\epsilon$,\fieldD{g}{}), & \{(\fieldI{f}{}{x},$\epsilon$),($\epsilon$,\fieldI{g}{}{n}), & \{(\fieldI{f}{}{x},$\epsilon$),(\fieldD{f}{},$\epsilon$)\} & \{(\fieldI{f}{}{x+m},$\epsilon$)\} & \{(\fieldI{f}{}{x},$\epsilon$), \\ 
& & (\fieldI{f}{}{x},\fieldD{g}{}),($\epsilon$,$\epsilon$)\} & (\fieldI{f}{}{x},\fieldI{g}{}{n}),($\epsilon$,$\epsilon$)\} & & & ($\alpha$, $\beta$)\}\\ \hline 

\{(\fieldD{f}{},\fieldD{g}{})\} & \{(\fieldD{f}{},\fieldD{g}{}),($\epsilon$,$\epsilon$), & \{(\fieldD{f}{},\fieldD{g}{}),($\epsilon$,\fieldD{g}{})\} & \{(\fieldD{f}{},\fieldD{g}{}),($\epsilon$,\fieldI{g}{}{n}), & \{(\fieldD{f}{},\fieldD{g}{}),(\fieldD{f}{},$\epsilon$)\} & \{(\fieldD{f}{},\fieldD{g}{}),(\fieldI{f}{}{x},$\epsilon$), & \{(\fieldD{f}{},\fieldD{g}{}),  \\
& (\fieldD{f}{},$\epsilon$),($\epsilon$,\fieldD{g}{})\} & & (\fieldD{f}{},\fieldI{g}{}{n}),($\epsilon$,\fieldD{g}{})\} & & (\fieldD{f}{},$\epsilon$),(\fieldI{f}{}{m},\fieldD{g}{})\} & ($\alpha$, $\beta$)\}\\ \hline 

\{(\fieldI{f}{}{x},\fieldD{g}{})\} & \{(\fieldI{f}{}{x},\fieldD{g}{}),($\epsilon$,$\epsilon$), & \{(\fieldI{f}{}{x},\fieldD{g}{}),($\epsilon$,\fieldD{g}{})\} & \{(\fieldI{f}{}{x},\fieldD{g}{}),($\epsilon$,\fieldI{g}{}{n}), & \{(\fieldI{f}{}{x},\fieldD{g}{}),(\fieldD{f}{},$\epsilon$), & \{(\fieldI{f}{}{x+m},\fieldD{g}{}),(\fieldI{f}{}{x+m},$\epsilon$)\} & \{(\fieldI{f}{}{x},\fieldD{g}{}), \\
& (\fieldI{f}{}{x},$\epsilon$),($\epsilon$,\fieldD{g}{})\} & & (\fieldI{f}{}{x},\fieldI{g}{}{n}),($\epsilon$,\fieldD{g}{})\} & (\fieldI{f}{}{x},$\epsilon$),(\fieldD{f}{},\fieldD{g}{})\} & & ($\alpha$, $\beta$)\}\\ \hline

\{(\fieldD{f}{},\fieldI{g}{}{y})\} & \{(\fieldD{f}{},\fieldI{g}{}{y}),($\epsilon$,$\epsilon$), & \{(\fieldD{f}{},\fieldI{g}{}{y}),($\epsilon$,\fieldD{g}{}), & \{(\fieldD{f}{},\fieldI{g}{}{y+n}),($\epsilon$,\fieldI{g}{}{y+n})\} & \{(\fieldD{f}{},\fieldI{g}{}{y}),(\fieldD{f}{},$\epsilon$)\} & \{(\fieldD{f}{},\fieldI{g}{}{y}),(\fieldI{f}{}{m},$\epsilon$), & \{(\fieldD{f}{},\fieldI{g}{}{y}), \\ 
& (\fieldD{f}{},$\epsilon$),($\epsilon$,\fieldI{g}{}{y})\} & (\fieldD{f}{},\fieldD{g}{}),($\epsilon$,\fieldI{g}{}{y})\} & & & (\fieldD{f}{},$\epsilon$),(\fieldI{f}{}{m},\fieldI{g}{}{y})\} & ($\alpha$, $\beta$)\}\\ \hline

\{(\fieldI{f}{}{x},\fieldI{g}{}{y})\} & \{(\fieldI{f}{}{x},\fieldI{g}{}{y}),($\epsilon$,$\epsilon$), & \{(\fieldI{f}{}{x},\fieldI{g}{}{y}),($\epsilon$,\fieldD{g}{}), & \{(\fieldI{f}{}{x},\fieldI{g}{}{y+n}),($\epsilon$,\fieldI{g}{}{y+n})\} & \{(\fieldI{f}{}{x},\fieldI{g}{}{y}),(\fieldD{f}{},$\epsilon$), & \{(\fieldI{f}{}{x+m},\fieldI{g}{}{y}),(\fieldI{f}{}{x+m},$\epsilon$)\} & \{(\fieldI{f}{}{x},\fieldI{g}{}{y}), \\ 
& (\fieldI{f}{}{x},$\epsilon$),($\epsilon$,\fieldI{g}{}{y})\} & (\fieldI{f}{}{x},\fieldD{g}{}),($\epsilon$,\fieldI{g}{}{y})\} & & (\fieldI{f}{}{x},$\epsilon$),(\fieldD{f}{},\fieldI{g}{}{y})\} & & ($\alpha$, $\beta$)\}\\ \hline

\end{tabular}
}\\
\end{tabular}
\end{figure}

\begin{figure}
\caption{Intersection operation between two singleton sets of pair of paths, where $\{\alpha, \beta\}$ denote any other pair of paths and $a \star b$ = min(a,b).
\label{tab:intersection_path_pair}}
\begin{tabular}{c}
\\
\scalebox{0.65}{ 
\begin{tabular}{|c|c|c|c|c|c|c|c|c|c|}
\hline

$\bigcap$ & \{($\epsilon$,$\epsilon$)\} & \{($\epsilon$,\fieldD{g}{})\} & \{($\epsilon$,\fieldI{g}{}{n})\} & \{($\epsilon$,\fieldI{g}{}{\infty})\} & \{(\fieldD{f}{},$\epsilon$)\} & \{(\fieldD{f}{},\fieldD{g}{})\} & \{(\fieldD{f}{},\fieldI{g}{}{n})\} & \{(\fieldD{f}{},\fieldI{g}{}{\infty})\} & \{($\alpha$, $\beta$)\} \\ \hline


\{($\epsilon$,$\epsilon$)\} & \{($\epsilon$,$\epsilon$)\} & $\phi$ & $\phi$ & $\phi$ & $\phi$ & $\phi$ & $\phi$ & $\phi$ & $\phi$\\ \hline

\{($\epsilon$,\fieldD{g}{})\} & $\phi$ & \{($\epsilon$,\fieldD{g}{})\} & $\phi$ & $\phi$ & $\phi$ & $\phi$ & $\phi$ & $\phi$ & $\phi$\\ \hline

\{($\epsilon$,\fieldI{g}{}{y})\} & $\phi$ & $\phi$ & \{($\epsilon$,\fieldI{g}{}{y \star n})\} & \{($\epsilon$,\fieldI{g}{}{y})\} & $\phi$ & $\phi$ & $\phi$ & $\phi$ & $\phi$\\ \hline

\{($\epsilon$,\fieldI{g}{}{\infty})\} & $\phi$ & $\phi$ & \{($\epsilon$,\fieldI{g}{}{n})\} & \{($\epsilon$,\fieldI{g}{}{\infty})\} & $\phi$ & $\phi$ & $\phi$ & $\phi$ & $\phi$\\ \hline

\{(\fieldD{f}{},$\epsilon$)\} & $\phi$ & $\phi$ & $\phi$ & $\phi$ & \{(\fieldD{f}{},$\epsilon$)\} & $\phi$ & $\phi$ & $\phi$ & $\phi$\\ \hline

\{(\fieldD{f}{},\fieldD{g}{})\} & $\phi$ & $\phi$ & $\phi$ & $\phi$ & $\phi$ & \{(\fieldD{f}{},\fieldD{g}{})\} & $\phi$ & $\phi$ & $\phi$\\ \hline
 
\{(\fieldD{f}{},\fieldI{g}{}{y})\} & $\phi$ & $\phi$ & $\phi$ & $\phi$ & $\phi$ & $\phi$ & \{(\fieldD{f}{},\fieldI{g}{}{y \star n})\} & \{(\fieldD{f}{},\fieldI{g}{}{y})\} & $\phi$\\ \hline

\{(\fieldD{f}{},\fieldI{g}{}{\infty})\} & $\phi$ & $\phi$ & $\phi$ & $\phi$ & $\phi$ & $\phi$ & \{(\fieldD{f}{},\fieldI{g}{}{n})\} & \{(\fieldD{f}{},\fieldI{g}{}{\infty})\} & $\phi$\\ \hline

\{(\fieldI{f}{}{x},$\epsilon$)\} & $\phi$ & $\phi$ & $\phi$ & $\phi$ & $\phi$ & $\phi$ & $\phi$ & $\phi$ & $\phi$\\ \hline

\{(\fieldI{f}{}{x},\fieldD{g}{})\} & $\phi$ & $\phi$ & $\phi$ & $\phi$ & $\phi$ & $\phi$ & $\phi$ & $\phi$ & $\phi$\\ \hline

\{(\fieldI{f}{}{x},\fieldI{g}{}{y})\} & $\phi$ & $\phi$ & $\phi$ & $\phi$ & $\phi$ & $\phi$ & $\phi$ & $\phi$ & $\phi$\\ \hline

\{(\fieldI{f}{}{x},\fieldI{g}{}{\infty})\} & $\phi$ & $\phi$ & $\phi$ & $\phi$ & $\phi$ & $\phi$ & $\phi$ & $\phi$ & $\phi$\\ \hline

\{(\fieldI{f}{}{\infty},$\epsilon$)\} & $\phi$ & $\phi$ & $\phi$ & $\phi$ & $\phi$ & $\phi$ & $\phi$ & $\phi$ & $\phi$\\ \hline

\{(\fieldI{f}{}{\infty},\fieldD{g}{})\} & $\phi$ & $\phi$ & $\phi$ & $\phi$ & $\phi$ & $\phi$ & $\phi$ & $\phi$ & $\phi$\\ \hline

\{(\fieldI{f}{}{\infty},\fieldI{g}{}{y})\} & $\phi$ & $\phi$ & $\phi$ & $\phi$ & $\phi$ & $\phi$ & $\phi$ & $\phi$ & $\phi$\\ \hline

\{(\fieldI{f}{}{\infty},\fieldI{g}{}{\infty})\} & $\phi$ & $\phi$ & $\phi$ & $\phi$ & $\phi$ & $\phi$ & $\phi$ & $\phi$ & $\phi$\\ \hline


\end{tabular}
}

\\ \\ \\

\scalebox{0.65}{
\begin{tabular}{|c|c|c|c|c|c|c|c|c|c|}
\hline

$\bigcap$ & \{(\fieldI{f}{}{m},$\epsilon$)\} & \{(\fieldI{f}{}{m},\fieldD{g}{})\} & \{(\fieldI{f}{}{m},\fieldI{g}{}{n})\} & \{(\fieldI{f}{}{m},\fieldI{g}{}{\infty})\} & \{(\fieldI{f}{}{\infty},$\epsilon$)\} & \{(\fieldI{f}{}{\infty},\fieldD{g}{})\} & \{(\fieldI{f}{}{\infty},\fieldI{g}{}{n})\} & \{(\fieldI{f}{}{\infty},\fieldI{g}{}{\infty})\} & \{($\alpha$, $\beta$)\}\\ \hline


\{($\epsilon$,$\epsilon$)\} & $\phi$ & $\phi$ & $\phi$ & $\phi$ & $\phi$ & $\phi$ & $\phi$ & $\phi$ & $\phi$\\ \hline

\{($\epsilon$,\fieldD{g}{})\} & $\phi$ & $\phi$ & $\phi$ & $\phi$ & $\phi$ & $\phi$ & $\phi$ & $\phi$ & $\phi$\\ \hline

\{($\epsilon$,\fieldI{g}{}{y})\} & $\phi$ & $\phi$ & $\phi$ & $\phi$ & $\phi$ & $\phi$ & $\phi$ & $\phi$ & $\phi$\\ \hline

\{($\epsilon$,\fieldI{g}{}{\infty})\} & $\phi$ & $\phi$ & $\phi$ & $\phi$ & $\phi$ & $\phi$ & $\phi$ & $\phi$ & $\phi$\\ \hline

\{(\fieldD{f}{},$\epsilon$)\} & $\phi$ & $\phi$ & $\phi$ & $\phi$ & $\phi$ & $\phi$ & $\phi$ & $\phi$ & $\phi$\\ \hline

\{(\fieldD{f}{},\fieldD{g}{})\} & $\phi$ & $\phi$ & $\phi$ & $\phi$ & $\phi$ & $\phi$ & $\phi$ & $\phi$ & $\phi$\\ \hline

\{(\fieldD{f}{},\fieldI{g}{}{y})\} & $\phi$ & $\phi$ & $\phi$ & $\phi$ & $\phi$ & $\phi$ & $\phi$ & $\phi$ & $\phi$ \\ \hline

\{(\fieldD{f}{},\fieldI{g}{}{\infty})\} & $\phi$ & $\phi$ & $\phi$ & $\phi$ & $\phi$ & $\phi$ & $\phi$ & $\phi$ & $\phi$\\ \hline

\{(\fieldI{f}{}{x},$\epsilon$)\} & \{(\fieldI{f}{}{x \star m},$\epsilon$)\} & $\phi$ & $\phi$ & $\phi$ & \{(\fieldI{f}{}{x},$\epsilon$)\} & $\phi$ & $\phi$ & $\phi$ & $\phi$\\ \hline

\{(\fieldI{f}{}{x},\fieldD{g}{})\} & $\phi$ & \{(\fieldI{f}{}{x \star m},\fieldD{g}{})\} & $\phi$ & $\phi$ & $\phi$ & \{(\fieldI{f}{}{x},\fieldD{g}{})\} & $\phi$ & $\phi$ & $\phi$\\ \hline

\{(\fieldI{f}{}{x},\fieldI{g}{}{y})\} & $\phi$ & $\phi$ & \{(\fieldI{f}{}{x \star m},\fieldI{g}{}{y \star n})\} & \{(\fieldI{f}{}{x \star m},\fieldI{g}{}{y})\} & $\phi$ & $\phi$ & \{(\fieldI{f}{}{x},\fieldI{g}{}{y \star n})\} & \{(\fieldI{f}{}{x},\fieldI{g}{}{y})\} & $\phi$\\ \hline

\{(\fieldI{f}{}{x},\fieldI{g}{}{\infty})\} & $\phi$ & $\phi$ & \{(\fieldI{f}{}{x \star m},\fieldI{g}{}{n})\} & \{(\fieldI{f}{}{x \star m},\fieldI{g}{}{\infty})\} & $\phi$ & $\phi$ & \{(\fieldI{f}{}{x},\fieldI{g}{}{n})\} & \{(\fieldI{f}{}{x},\fieldI{g}{}{n})\} & $\phi$\\ \hline

\{(\fieldI{f}{}{\infty},$\epsilon$)\} & \{(\fieldI{f}{}{m},$\epsilon$)\} & $\phi$ & $\phi$ & $\phi$ & \{(\fieldI{f}{}{\infty},$\epsilon$)\} & $\phi$ & $\phi$ & $\phi$ & $\phi$\\ \hline

\{(\fieldI{f}{}{\infty},\fieldD{g}{})\} & $\phi$ & \{(\fieldI{f}{}{m},\fieldD{g}{})\} & $\phi$ & $\phi$ & $\phi$ & \{(\fieldI{f}{}{\infty},\fieldD{g}{})\} & $\phi$ & $\phi$  & $\phi$\\ \hline

\{(\fieldI{f}{}{\infty},\fieldI{g}{}{y})\} & $\phi$ & $\phi$ & \{(\fieldI{f}{}{m},\fieldI{g}{}{y \star n})\} & \{(\fieldI{f}{}{m},\fieldI{g}{}{y})\} & $\phi$ & $\phi$ & \{(\fieldI{f}{}{\infty},\fieldI{g}{}{y \star n})\} & \{(\fieldI{f}{}{\infty},\fieldI{g}{}{y})\} & $\phi$\\ \hline

\{(\fieldI{f}{}{\infty},\fieldI{g}{}{\infty})\} & $\phi$ & $\phi$ & \{(\fieldI{f}{}{m},\fieldI{g}{}{n})\} & \{(\fieldI{f}{}{m},\fieldI{g}{}{\infty})\} &  $\phi$ &  $\phi$ & \{(\fieldI{f}{}{\infty},\fieldI{g}{}{n})\} & \{(\fieldI{f}{}{\infty},\fieldI{g}{}{\infty})\} & $\phi$\\ \hline

\end{tabular}
}\\
\end{tabular}
\end{figure}

\begin{figure}
\caption{Removal operation between two singleton sets of pair of paths, where $\{\alpha, \beta\}$ denote any other pair of paths and $a\#b$ = max(a-b,0).
\label{tab:removal_path_pair}}
\begin{tabular}{c}
\\
\scalebox{0.65}{
\begin{tabular}{|c|c|c|c|c|c|c|c|c|c|}
\hline

\remOne{}{} & \{($\epsilon$,$\epsilon$)\} & \{($\epsilon$,\fieldD{g}{})\} & \{($\epsilon$,\fieldI{g}{}{n})\} & \{($\epsilon$,\fieldI{g}{}{\infty})\} & \{(\fieldD{f}{},$\epsilon$)\} & \{(\fieldD{f}{},\fieldD{g}{})\} & \{(\fieldD{f}{},\fieldI{g}{}{n})\} & \{(\fieldD{f}{},\fieldI{g}{}{\infty})\} & \{($\alpha$, $\beta$)\}\\ \hline

\{($\epsilon$,$\epsilon$)\} & $\phi$ & \{($\epsilon$,$\epsilon$)\} & \{($\epsilon$,$\epsilon$)\} & \{($\epsilon$,$\epsilon$)\} & \{($\epsilon$,$\epsilon$)\} & \{($\epsilon$,$\epsilon$)\} & \{($\epsilon$,$\epsilon$)\} & \{($\epsilon$,$\epsilon$)\} & \{($\epsilon$,$\epsilon$)\}\\ \hline

\{($\epsilon$,\fieldD{g}{})\} & $\phi$ & $\phi$ & \{($\epsilon$,\fieldD{g}{})\} & \{($\epsilon$,\fieldD{g}{})\} & \{($\epsilon$,\fieldD{g}{})\} & \{($\epsilon$,\fieldD{g}{})\} & \{($\epsilon$,\fieldD{g}{})\} & \{($\epsilon$,\fieldD{g}{})\} & \{($\epsilon$,\fieldD{g}{})\}\\ \hline

\{($\epsilon$,\fieldI{g}{}{y})\} &  $\phi$ & $\phi$ & \{($\epsilon$,\fieldI{g}{}{y})\} & \{($\epsilon$,\fieldI{g}{}{y})\} & \{($\epsilon$,\fieldI{g}{}{y})\} & $\phi$ & \{($\epsilon$,\fieldI{g}{}{y})\} & \{($\epsilon$,\fieldI{g}{}{y\#n})\} & \{($\epsilon$,\fieldI{g}{}{y})\}\\ \hline

\{($\epsilon$,\fieldI{g}{}{\infty})\} & $\phi$ & $\phi$ & \{($\epsilon$,\fieldI{g}{}{\infty})\} & \{($\epsilon$,\fieldI{g}{}{\infty})\} & \{($\epsilon$,\fieldI{g}{}{\infty})\} & $\phi$ & \{($\epsilon$,\fieldI{g}{}{\infty})\} & \{($\epsilon$,\fieldI{g}{}{\infty})\} & \{($\epsilon$,\fieldI{g}{}{\infty})\}\\ \hline

\{(\fieldD{f}{},$\epsilon$)\} & $\phi$ & \{(\fieldD{f}{},$\epsilon$)\} & \{(\fieldD{f}{},$\epsilon$)\} & \{(\fieldD{f}{},$\epsilon$)\} & $\phi$ & $\phi$ & $\phi$ & $\phi$& \{(\fieldD{f}{},$\epsilon$)\} \\ \hline

\{(\fieldD{f}{},\fieldD{g}{})\} & \{(\fieldD{f}{},\fieldD{g}{})\} & $\phi$ & \{(\fieldD{f}{},\fieldD{g}{})\} & \{(\fieldD{f}{},\fieldD{g}{})\} & $\phi$ & $\phi$ & $\phi$ & $\phi$ & \{(\fieldD{f}{},\fieldD{g}{})\} \\ \hline

\{(\fieldD{f}{},\fieldI{g}{}{y})\} & \{(\fieldD{f}{},\fieldI{g}{}{y})\} & $\phi$ & \{(\fieldD{f}{},\fieldI{g}{}{y\#n})\} & $\phi$ & $\phi$ & $\phi$ & $\phi$ & $\phi$ & \{(\fieldD{f}{},\fieldI{g}{}{y})\}  \\ \hline

\{(\fieldD{f}{},\fieldI{g}{}{\infty})\} & \{(\fieldD{f}{},\fieldI{g}{}{\infty})\} & $\phi$ & \{(\fieldD{f}{},\fieldI{g}{}{\infty})\} & \{(\fieldD{f}{},\fieldI{g}{}{\infty})\} & $\phi$ & $\phi$ & $\phi$ & $\phi$ & \{(\fieldD{f}{},\fieldI{g}{}{\infty})\} \\ \hline

\{(\fieldI{f}{}{x},$\epsilon$)\} & $\phi$ & \{(\fieldI{f}{}{x},$\epsilon$)\} & \{(\fieldI{f}{}{x},$\epsilon$)\} & \{(\fieldI{f}{}{x},$\epsilon$)\} & $\phi$ & $\phi$ & $\phi$ & $\phi$ & \{(\fieldI{f}{}{x},$\epsilon$)\} \\ \hline

\{(\fieldI{f}{}{x},\fieldD{g}{})\} & \{(\fieldI{f}{}{x},\fieldD{g}{})\} & $\phi$ & \{(\fieldI{f}{}{x},\fieldD{g}{})\} & \{(\fieldI{f}{}{x},\fieldD{g}{})\} & $\phi$ & $\phi$ & $\phi$ & $\phi$ & \{(\fieldI{f}{}{x},\fieldD{g}{})\} \\ \hline

\{(\fieldI{f}{}{x},\fieldI{g}{}{y})\} & \{(\fieldI{f}{}{x},\fieldI{g}{}{y})\} & $\phi$ & \{(\fieldI{f}{}{x},\fieldI{g}{}{y\#n})\} & $\phi$ & $\phi$ & $\phi$ & $\phi$ & $\phi$ & \{(\fieldI{f}{}{x},\fieldI{g}{}{y}) \} \\ \hline

\{(\fieldI{f}{}{x},\fieldI{g}{}{\infty})\} & \{(\fieldI{f}{}{x},\fieldI{g}{}{\infty})\} & $\phi$ & \{(\fieldI{f}{}{x},\fieldI{g}{}{\infty})\} & \{(\fieldI{f}{}{x},\fieldI{g}{}{\infty})\} & $\phi$ & $\phi$ & $\phi$ & $\phi$ & \{(\fieldI{f}{}{x},\fieldI{g}{}{\infty})  \\ \hline

\{(\fieldI{f}{}{\infty},$\epsilon$)\} & $\phi$ & \{(\fieldI{f}{}{\infty},$\epsilon$)\} & \{(\fieldI{f}{}{\infty},$\epsilon$)\} & \{(\fieldI{f}{}{\infty},$\epsilon$)\} & $\phi$ & $\phi$ & $\phi$ & $\phi$ & \{(\fieldI{f}{}{\infty},$\epsilon$)\}\\ \hline

\{(\fieldI{f}{}{\infty},\fieldD{g}{})\} & \{(\fieldI{f}{}{\infty},\fieldD{g}{})\} & $\phi$ & \{(\fieldI{f}{}{\infty},\fieldD{g}{})\} & \{(\fieldI{f}{}{\infty},\fieldD{g}{})\} & $\phi$ & $\phi$ & $\phi$ & $\phi$ & \{(\fieldI{f}{}{\infty},\fieldD{g}{})\}  \\ \hline

\{(\fieldI{f}{}{\infty},\fieldI{g}{}{y})\} & \{(\fieldI{f}{}{\infty},\fieldI{g}{}{y})\} & $\phi$ & \{(\fieldI{f}{}{\infty},\fieldI{g}{}{y\#n})\} & $\phi$ & $\phi$ & $\phi$ & $\phi$ & $\phi$ & \{(\fieldI{f}{}{\infty},\fieldI{g}{}{y})\}  \\ \hline

\{(\fieldI{f}{}{\infty},\fieldI{g}{}{\infty})\} & \{(\fieldI{f}{}{\infty},\fieldI{g}{}{\infty})\} & $\phi$ & \{(\fieldI{f}{}{\infty},\fieldI{g}{}{\infty})\} & \{(\fieldI{f}{}{\infty},\fieldI{g}{}{\infty})\} & $\phi$ & $\phi$ & $\phi$ & $\phi$ & \{(\fieldI{f}{}{\infty},\fieldI{g}{}{\infty})\}  \\ \hline

\end{tabular}
}\\ \\
\scalebox{0.65}{
\begin{tabular}{|c|c|c|c|c|c|c|c|c|c|}
\hline

\remOne{}{} & \{(\fieldI{f}{}{m},$\epsilon$)\} & \{(\fieldI{f}{}{m},\fieldD{g}{})\} & \{(\fieldI{f}{}{m},\fieldI{g}{}{n})\} & \{(\fieldI{f}{}{m},\fieldI{g}{}{\infty})\} & \{(\fieldI{f}{}{\infty},$\epsilon$)\} & \{(\fieldI{f}{}{\infty},\fieldD{g}{})\} & \{(\fieldI{f}{}{\infty},\fieldI{g}{}{n})\} & \{(\fieldI{f}{}{\infty},\fieldI{g}{}{\infty})\} & \{($\alpha$,$\beta$)\}\\ \hline


\{($\epsilon$,$\epsilon$)\} & $\phi$ & \{($\epsilon$,$\epsilon$)\} & \{($\epsilon$,$\epsilon$)\} & \{($\epsilon$,$\epsilon$)\} & \{($\epsilon$,$\epsilon$)\} & \{($\epsilon$,$\epsilon$)\} & \{($\epsilon$,$\epsilon$)\} & \{($\epsilon$,$\epsilon$)\} & \{($\epsilon$,$\epsilon$)\}\\ \hline

\{($\epsilon$,\fieldD{g}{})\} & \{($\epsilon$,\fieldD{g}{})\} & $\phi$ & \{($\epsilon$,\fieldD{g}{})\} & \{($\epsilon$,\fieldD{g}{})\} & \{($\epsilon$,\fieldD{g}{})\} & $\phi$ & \{($\epsilon$,\fieldD{g}{})\} & \{($\epsilon$,\fieldD{g}{})\} & \{($\epsilon$,\fieldD{g}{})\}\\ \hline
 
\{($\epsilon$,\fieldI{g}{}{y})\} & \{($\epsilon$,\fieldI{g}{}{y})\} & $\phi$ & \{($\epsilon$,\fieldI{g}{}{y\#n})\} & $\phi$ & \{($\epsilon$,\fieldI{g}{}{y})\} & $\phi$ & \{($\epsilon$,\fieldI{g}{}{y})\} & $\phi$  & \{($\epsilon$,\fieldI{g}{}{y})\}\\ \hline
 
\{($\epsilon$,\fieldI{g}{}{\infty})\} & \{($\epsilon$,\fieldI{g}{}{\infty})\} & $\phi$ & \{($\epsilon$,\fieldI{g}{}{\infty})\} & \{($\epsilon$,\fieldI{g}{}{\infty})\} & \{($\epsilon$,\fieldI{g}{}{\infty})\} & $\phi$ & \{($\epsilon$,\fieldI{g}{}{\infty})\} & \{($\epsilon$,\fieldI{g}{}{\infty})\} & \{($\epsilon$,\fieldI{g}{}{\infty})\}\\ \hline
 
\{(\fieldD{f}{},$\epsilon$)\} & $\phi$ & \{(\fieldD{f}{},$\epsilon$)\} & \{(\fieldD{f}{},$\epsilon$)\} & \{(\fieldD{f}{},$\epsilon$)\} & $\phi$ & \{(\fieldD{f}{},$\epsilon$)\} & \{(\fieldD{f}{},$\epsilon$)\} & \{(\fieldD{f}{},$\epsilon$)\} & \{(\fieldD{f}{},$\epsilon$)\}\\ \hline

\{(\fieldD{f}{},\fieldD{g}{})\} & \{(\fieldD{f}{},\fieldD{g}{})\} & $\phi$ & \{(\fieldD{f}{},\fieldD{g}{})\} & \{(\fieldD{f}{},\fieldD{g}{})\} & \{(\fieldD{f}{},\fieldD{g}{})\} & $\phi$ & \{(\fieldD{f}{},\fieldD{g}{})\} & \{(\fieldD{f}{},\fieldD{g}{})\} & \{(\fieldD{f}{},\fieldD{g}{})\}\\ \hline

\{(\fieldD{f}{},\fieldI{g}{}{y})\} & \{(\fieldD{f}{},\fieldI{g}{}{y})\} & $\phi$ & \{(\fieldD{f}{},\fieldI{g}{}{y\#n})\} &  $\phi$ & \{(\fieldD{f}{},\fieldI{g}{}{y})\} & $\phi$ & \{(\fieldD{f}{},\fieldI{g}{}{y\#n})\} & $\phi$ & \{(\fieldD{f}{},\fieldI{g}{}{y})\}\\ \hline

\{(\fieldD{f}{},\fieldI{g}{}{\infty})\} & \{(\fieldD{f}{},\fieldI{g}{}{\infty})\} & $\phi$ & \{(\fieldD{f}{},\fieldI{g}{}{\infty})\} & \{(\fieldD{f}{},\fieldI{g}{}{\infty})\} & \{(\fieldD{f}{},\fieldI{g}{}{\infty})\} & $\phi$ & \{(\fieldD{f}{},\fieldI{g}{}{\infty})\} & \{(\fieldD{f}{},\fieldI{g}{}{\infty})\} & \{(\fieldD{f}{},\fieldI{g}{}{\infty})\}\\ \hline

\{(\fieldI{f}{}{x},$\epsilon$)\} & \{(\fieldI{f}{}{x},$\epsilon$)\} & \{(\fieldI{f}{}{x\#m},$\epsilon$)\} & \{(\fieldI{f}{}{x\#m},$\epsilon$)\} & \{(\fieldI{f}{}{x\#m},$\epsilon$)\} & \{(\fieldI{f}{}{x},$\epsilon$)\} & $\phi$ & $\phi$ & $\phi$ & \{(\fieldI{f}{}{x},$\epsilon$)\}\\ \hline

\{(\fieldI{f}{}{x},\fieldD{g}{})\} & \{(\fieldI{f}{}{x\#m},\fieldD{g}{})\} & $\phi$ & \{(\fieldI{f}{}{x\#m},\fieldD{g}{})\} & \{(\fieldI{f}{}{x\#m},\fieldD{g}{})\} & $\phi$ & $\phi$ & $\phi$ & $\phi$ & \{(\fieldI{f}{}{x},\fieldD{g}{})\}\\ \hline
 
\{(\fieldI{f}{}{x},\fieldI{g}{}{y})\} & \{(\fieldI{f}{}{x\#m},\fieldI{g}{}{y})\} & $\phi$ & \{(\fieldI{f}{}{x\#m},\fieldI{g}{}{y\#n})\} & $\phi$ & $\phi$ & $\phi$ & $\phi$ & $\phi$ & \{(\fieldI{f}{}{x},\fieldI{g}{}{y})\}\\ \hline

\{(\fieldI{f}{}{x},\fieldI{g}{}{\infty})\} & \{(\fieldI{f}{}{x\#m},\fieldI{g}{}{\infty})\} & $\phi$ & \{(\fieldI{f}{}{x\#m},\fieldI{g}{}{\infty})\} & \{(\fieldI{f}{}{x\#m},\fieldI{g}{}{\infty})\} & $\phi$ & $\phi$ & $\phi$ & $\phi$ & \{(\fieldI{f}{}{x},\fieldI{g}{}{\infty})\}\\ \hline

\{(\fieldI{f}{}{\infty},$\epsilon$)\} & $\phi$ & \{(\fieldI{f}{}{\infty},$\epsilon$)\} & \{(\fieldI{f}{}{\infty},$\epsilon$)\} & \{(\fieldI{f}{}{\infty},$\epsilon$)\} & $\phi$ & \{(\fieldI{f}{}{\infty},$\epsilon$)\} & \{(\fieldI{f}{}{\infty},$\epsilon$)\} & \{(\fieldI{f}{}{\infty},$\epsilon$)\} & \{(\fieldI{f}{}{\infty},$\epsilon$)\}\\ \hline

\{(\fieldI{f}{}{\infty},\fieldD{g}{})\} & \{(\fieldI{f}{}{\infty},\fieldD{g}{})\} & $\phi$ & \{(\fieldI{f}{}{\infty},\fieldD{g}{})\} & \{(\fieldI{f}{}{\infty},\fieldD{g}{})\} & \{(\fieldI{f}{}{\infty},\fieldD{g}{})\} & $\phi$ & \{(\fieldI{f}{}{\infty},\fieldD{g}{})\} & \{(\fieldI{f}{}{\infty},\fieldD{g}{})\} & \{(\fieldI{f}{}{\infty},\fieldD{g}{})\}\\ \hline
 
\{(\fieldI{f}{}{\infty},\fieldI{g}{}{y})\} & \{(\fieldI{f}{}{\infty},\fieldI{g}{}{y})\} & $\phi$ & \{(\fieldI{f}{}{\infty},\fieldI{g}{}{y\#n})\} & \{(\fieldI{f}{}{\infty},\fieldI{g}{}{y})\} & \{(\fieldI{f}{}{\infty},\fieldI{g}{}{y})\} & $\phi$ & \{(\fieldI{f}{}{\infty},\fieldI{g}{}{y\#n})\} & \{(\fieldI{f}{}{\infty},\fieldI{g}{}{y})\} & \{(\fieldI{f}{}{\infty},\fieldI{g}{}{y})\}\\ \hline

\{(\fieldI{f}{}{\infty},\fieldI{g}{}{\infty})\} & \{(\fieldI{f}{}{\infty},\fieldI{g}{}{\infty})\} & $\phi$ & \{(\fieldI{f}{}{\infty},\fieldI{g}{}{\infty})\} & \{(\fieldI{f}{}{\infty},\fieldI{g}{}{\infty})\} & \{(\fieldI{f}{}{\infty},\fieldI{g}{}{\infty})\} & $\phi$ & \{(\fieldI{f}{}{\infty},\fieldI{g}{}{\infty})\} & \{(\fieldI{f}{}{\infty},\fieldI{g}{}{\infty})\} & \{(\fieldI{f}{}{\infty},\fieldI{g}{}{\infty})\}\\ \hline


\end{tabular}
}\\
\end{tabular}
\end{figure}


\begin{table}[t]
\caption{Multiplication by a scalar\label{tab:path-mult}}
\begin{center}
\renewcommand{\arraystretch}{1.1}
\begin{tabular*}{0.75\textwidth}{@{\extracolsep{\fill}} l@{\ }|cccc }
  \hline
  $\star\qquad \alpha$ & $\epsilon$ & $f^{\drct}$      & ${f^{\indrct
      j}}$ & ${f^{\indrct \infty}}$ \\ 
  $i$ &&&& \\ \hline 
  $i$      & $\epsilon$ & ${f^{\indrct i}}$ & $f^{\indrct
    m}$, $m = \left\{\begin{array}{lcl}
  i*j    && \mbox { if } i*j \leq k \\
  \infty && \mbox{ Otherwise} 
  \end{array}\right.$
  & ${f^{\indrct \infty}}$ \\ 
  $\infty$ & $\epsilon$ &${f^{\indrct \infty}}$ &
  ${f^{\indrct \infty}}$& ${f^{\indrct \infty}}$\\ \hline 
\end{tabular*}
\end{center}
\end{table}

\item Multiplication by a scalar($\star$): Let $i, j
  \in \nat, i \leq k, j \leq k$. Then, for a path $\alpha$,
  the multiplication by a scalar $i$,\ $i\star\alpha$ is
  defined in Table~\ref{tab:path-mult}. The operation is
  extended to set of paths as:
\begin{eqnarray*}
  i \star {S} &=&   \left\{ \begin{array}{lcl}
    \emptyset &\quad& i = 0 \\
    \{ i\star \alpha \ \vert\ \alpha \in S\} &\quad& i\in \nat \cup \{\infty\} 
  \end{array} \right.
\end{eqnarray*}

\end{itemize}

\subsection{Analysis}
For $\{\p, \q\} \subseteq \heap,\ f \in\fields,\ n \in \nat$
and $\mbox{\tt op} \in \{+, -\}$, 
the following eight basic statements are identified that can access or modify the heap structures. 
\begin{enumerate}
  \item Allocations
  	\begin{enumerate}
    	\item {\tt p = malloc();}
  	\end{enumerate}
  \item Pointer Assignments
  	\begin{enumerate}
    	\item {\tt p = NULL;}
    	\item {\tt p = q;}
    	\item {\tt p = q $\rightarrow$ f;}
    	\item {\tt p = \&(q $\rightarrow$ f);}
    	\item {\tt p = q op n;}
  	\end{enumerate}
  \item Structure Updates
  	\begin{enumerate}
    	\item {\tt p $\rightarrow$ f = q;}
    	\item {\tt p $\rightarrow$ f = NULL;}
  	\end{enumerate}
\end{enumerate}
%%
They intend to determine, at each program point, the field
sensitive matrices $D_F$ and $I_F$, and the
boolean variables capturing field connectivity. The problem is formulated
 as an instance of forward data flow analysis,
where the data flow values are the matrices and the boolean
variables as mentioned above.  For simplicity sake the basic blocks
are constructed with single statements each
The definition of 
the confluence operator (\merge) for various data flow values
as used by the analysis is given below. The superscripts $x$ and $y$
are used to denote the values coming along two paths,
\begin{eqnarray*}
  \merge( f_{\p\q}^x, f_{\p\q}^y ) &=& f_{\p\q}^x \vee f_{\p\q}^y , f \in \fields,
  \p,\q \in \heap \\
   \merge(\p_{\subC}^x, \p_{\subC}^y) &=& \p_{\subC}^x \vee
   \p_{\subC}^y, \p \in \heap \\
   \merge(\p_{\subD}^x, \p_{\subD}^y) &=& \p_{\subD}^x \vee
   \p_{\subD}^y, \p \in \heap \\
   \merge(D_F^{x}, D_F^{y}) &=& D_F \mbox{ where }
   D_F[{\p}, {\q}]  = D_F^{x}[{\p}, {\q}] \cup
   D_F^{y}[{\p}, {\q}],   \forall {\p}, {\q} \in \heap\\ 
   \merge(I_F^{x}, I_F^{y}) &=& I_F \mbox{ where }
   I_F[{\p}, {\q}]  = I_F^{x}[{\p}, {\q}] \cup
   I_F^{y}[{\p}, {\q}],   \forall {\p}, {\q} \in \heap
\end{eqnarray*}
The transformation of data flow values due to a statement
$st$ is captured by the following set of equations:
\begin{eqnarray*}
  D_F^{\dout}[\p,\q] &=& (\remOne{D_F^{\din}[\p,\q]}{D_F^{\dkill}[\p,\q]}) \cup
  D_F^{\dgen}[\p,\q] \\
  I_F^{\dout}[\p,\q] &=& (\remOne{I_F^{\din}[\p,\q]}{I_F^{\dkill}[\p,\q]}) \cup
  I_F^{\dgen}[\p,\q] \\
  \p_{\subC}^{\dout} &=& (\p_{\subC}^{\din} \wedge
  \neg\p_{\subC}^{\dkill} ) \vee \p_{\subC}^{\dgen}  \\
  \p_{\subD}^{\dout} &=& (\p_{\subD}^{\din} \wedge
  \neg\p_{\subD}^{\dkill} ) \vee \p_{\subD}^{\dgen} 
\end{eqnarray*}
Field connectivity information is updated directly by the
statement.


Few details about each of this basic statements are given below. The data flow values for each of these statements 
are shown in Figure~\ref{fig:Data Flow Equations}.

\begin{itemize}
\item {\tt p = malloc}:
    After this statement all the existing relationships of $\p$ get killed and it will point to a
newly allocated object.It is considered that  $\p$ can have an empty path to itself and it can interfere
with itself using empty paths (or $\epsilon$ paths).

\item {\tt p = NULL}:
   This statement only kills the existing relations of \p.
	
\item{\tt p=q,  p=\&(q$\rightarrow$f),  p=q op n}:
All these three pointer assignment statements are considered equivalent. 
After this statement all the existing relationships of $\p$ gets
killed and it will point to same heap object as pointed to
by $\q$. In case $\q$ currently points to null, $\p$ will also points
to null after the statement. So $\p$ will have the same field
sensitive Direction and Interference relationships as $\q$. 
The kill effect of this statement is same as that of the previous statement. 
The generated boolean functions for heap object $\p$ corresponding 
to DAG or Cycle attribute will be same as that of $\q$, with all occurrences of $\q$ 
replaced by $\p$.
	
\item{\tt p $\rightarrow$ f = NULL}: 
This statement breaks the existing link $f$ emanating from
$\p$, thus killing relations of $\p$, that are due to the link
$f$. The statement does not generate any new relations. 
           
\item{\tt p $\rightarrow$ f = q}:
This statement first breaks
  the existing link $f$ and then re-links the the heap object
  pointed to by $\p$ to the heap object pointed to by
  $\q$. The kill effects are exactly same as described in the
  case of  {\tt p$\rightarrow$f = null}.  Only the
  generated relationships are described in Figure~\ref{fig:Data Flow Equations}. 

\item{\tt p = q $\rightarrow$ f}:
The relations killed by the
  statement are same as that in case of {\tt p = NULL}. The
  relations created by this statement are heavily
  approximated as not much information is available about the heap node pointed by {\tt q $\rightarrow$ f} before the statement.
  After this statement $\p$ points to the heap object which is accessible from pointer
  $\q$ through $f$ link. 

\end{itemize}

\begin{figure}[p] 
\begin{spacing}{2.0}
%\begin{tabular}{|c|l|}
\begin{tabular}{|p{2.6cm}|p{12cm}|}
\hline
 {\tt p = malloc()} & 

\scalebox{0.70}{
\begin{tabular}{lll}
	  \KillC{p}  = \InC{p} & \KillD{p} = \InD{p} &\\
	  \GenC{p} = \false  & \GenD{p} = \false &\\

	  $\forall \s \in \heap,\ \s \not= \p,$ && \\
	  $D_F^{\dkill}[\p,\s]$  =  $D_F^{\din}[\p,\s]$ & $D_F^{\dkill}[\s,\p]$  =  $D_F^{\din}[\s,\p]$ & $D_F^{\dkill}[\p,\p]$  =  $D_F^{\din}[\p,\p]$ \\
	  $D_F^{\dgen}[\p,\s]$   =  $\emptyset$         & $D_F^{\dgen}[\s,\p]$   =  $\emptyset$          &$D_F^{\dgen}[\p,\p]$   =  $\epsilonset$ \\
	  $I_F^{\dkill}[\p,\s]$  =  $I_F^{\din}[\p,\s]$ & $I_F^{\dkill}[\p,\p]$  =  $I_F^{\din}[\p,\p]$   &\\
	  $I_F^{\dgen}[\p,\s]$   =  $\emptyset$         & $I_F^{\dgen}[\p,\p]$   =  $\epsilonpairset$     &
\end{tabular}
} \\
\hline
 {\tt p = NULL} & 
\scalebox{0.70}{
\begin{tabular}{ll}
	  \KillC{p}  = \InC{p} & \KillD{p} = \InD{p} \\
	  \GenC{p} = \false    & \GenD{p} = \false \\

	   $\forall \s \in \heap,$ & \\
	  $D_F^{\dkill}[\p,\s]$  =  $D_F^{\din}[\p,\s]$ & $D_F^{\dkill}[\s,\p]$  =  $D_F^{\din}[\s,\p]$ \\
	  $D_F^{\dgen}[\p,\s]$    =  $\emptyset$  & $D_F^{\dgen}[\s,\p]$    =  $\emptyset$ \\
	  $I_F^{\dkill}[\p,\s]$  =  $I_F^{\din}[\p,\s]$ & $I_F^{\dgen}[\p,\s]$    =  $\emptyset$ 
  \end{tabular}
} \\
\hline
\begin{tabular}{l}
{\tt p = q} \\
{\tt p = \&(q$\rightarrow$f)} \\
{\tt p = q op n}
\end{tabular}
 &
\scalebox{0.70}{
  \begin{tabular}{l}

	\KillC{p} = \InC{p} \quad\qquad \KillD{p} = \InD{p} \\ 
	\GenC{p}   = \InC{q}[q/p] \quad \GenD{p} = \InD{q}[q/p] \\ 
	where $X[\q / \p]$ creates a copy of $X$ with all occurrences
	of $\q$ replaced by $\p$.\\

      $\forall \s \in \heap,  \s \not= \p,  \forall f \in \fields,$ \\

	$f_{\p\s} = f_{\q\s}$  \quad  $f_{\s\p} = f_{\s\q}$ \\
	$D_F^{\dkill}[\p, \s]$  =  $D_F^{\din}[\p, \s]$ \quad $D_F^{\dkill}[\s, \p]$  =  $D_F^{\din}[\s, \p]$ \quad $D_F^{\dkill}[\p, \p]$  =  $D_F^{\din}[\p, \p]$ \\
	$D_F^{\dgen}[\p, \s]$    =  $D_F^{\din}[\q, \s]$ \quad $D_F^{\dgen}[\s, \p]$    =  $D_F^{\din}[\s, \q]$ \quad $D_F^{\dgen}[\p, \p]$    =  $D_F^{\din}[\q, \q]$  \\
	$I_F^{\dkill}[\p, \s]$  =  $I_F^{\din}[\p, \s]$ \qquad $I_F^{\dgen}[\p, \s]$    =   $I_F^{\din}[\q, \s]$ \\
	$I_F^{\dkill}[\p, \p]$  =  $I_F^{\din}[\p, \p]$ \qquad $I_F^{\dgen}[\p, \p]$    = $I_F^{\din}[\q, \q]$ 
  \end{tabular}
} \\
\hline

{\tt p$\rightarrow$f = null}
&
	\scalebox{0.70}{
	  \begin{tabular}{l}
	  \KillC{p}  = \false,  \quad \KillD{p} = \false \\
	  \GenC{p} = \false,  \quad \GenD{p} = \false  \\
	$\forall \q, \s \in \heap,  \s \not= \p, $ \\
	  $f_{\p\q}$ = \false \\
	  $D_F^{\dkill}[\p, \q]$  = $\project{D_F^{\din}[\p, \q]}{f}$ \qquad \; \;  {\blue ${D_{F}^{{\dkill}}[\s, \q]}  = \emptyset  $}   %-------Modified
	  \\
 	  $I_F^{\dkill}[\p, \s] = \{(\alpha,  \beta) \ \vert\ (\alpha, \beta) \in I_F^{\din}[\p,  \q] \mbox, \ \alpha \equiv f^\anysup \}$ \\ 

 	  {\blue ${I_F^{\dkill}[\q, \s]} = \emptyset \mbox{ if } \q \not= \p$} \qquad\qquad    %----------------Modified 
 	  {\blue ${I_F^{\dkill}[\p, \p]} = \emptyset$}   %----------------Modified 

  \end{tabular}
} \\

\hline
\end{tabular}
\end{spacing}
\end{figure}

\begin{figure}[p]
\begin{spacing}{2.0}
\begin{tabular}{|p{2.6cm}|p{12cm}|}
\hline
{\tt p$\rightarrow$f = q}
&
    \scalebox{0.70}{
    \begin{tabular}{l}

	The KILL relations are same as that of {\tt p$\rightarrow$f = null} \\
	  $\GenC{p}   =  (f_{\p\q} \wedge \InC{q}) \vee (f_{\p\q} \wedge (\num{\DFM{\q}{\p}} \geq 1))$ \quad {\blue ${\GenD{p}}   =  f_{\p\q} \wedge (\num{\IFM{\p}{\q}} > 1)$}	\\
 	  $\GenC{q}   = f_{\p\q} \wedge (\num{\DFM{\q}{\p}} \geq 1)$ \qquad\qquad\qquad\qquad $\GenD{q} = \false$\\ 
	  $f_{\p\q} = \true$ \\
	  $\GenC{\s} =  ((\num{\DFM{\s}{\p}} \geq 1) \wedge f_{\p\q} \wedge \InC{q})$ $\vee\ ((\num{\DFM{\s}{\p}} \geq 1) \wedge f_{\p\q} \wedge
 	  (\num{\DFM{\q}{\p}} \geq 1))$ \\ 
	  \quad \quad\quad   $\vee\ ((\num{\DFM{\s}{\q}} \geq 1) \wedge f_{\p\q}
	  \wedge (\num{\DFM{\q}{\p}} \geq 1))$,  \ \   $\forall \s \in \heap,  \s \not= \p,  \s \not=\q$ \\
	  $\GenD{\s}   = (\num{\DFM{\s}{\p}} \geq 1) \wedge
	  f_{\p\q} \wedge (\num{\IFM{\s}{\q}} > 1)$,  \ $\forall \s \in \heap,  \s \not= \p,  \s \not=\q $ \\
	 {\blue $ {D_F^{\dgen}[\myr,  \s]} = \num{D_F^{\din}[\q,  \s]} \star         %-----------------modified
	  D_F^{\din}[\myr,  \p], \ \s \not= \p, \ \myr \not\in
	  \{\p,  \q\} $}\\ 
	  $D_F^{\dgen}[\myr,  \p] = \num{D_F^{\din}[\q, \p]} \star
	  D_F^{\din}[\myr,  \p],  \ \myr \not= \p$\\
	  $D_F^{\dgen}[\p,  \myr] = \num{D_F^{\din}[\q,  \myr]} \star
	(\remOne{D_F^{\din}[\p,  \p]}{\epsilonset} \cup \{f^{\indrct
	  1}\})$,  \ \  $\myr \not= \q$ \\ 
	  $D_F^{\dgen}[\p,  \q] =
	  \{\fieldD{f}{}\}\ \cup\ (\num{D_F^{\din}[\q,  \q] -
	    \epsilonset} \star
	  \{\fieldI{f}{}{1}\})\ \cup\ 
	   (\num{D_F^{\din}[\q,  \q]} \star ( \remOne{D_F^{\din}[\p,  \p]}{\epsilonset}))$ \\ 
	  $D_F^{\dgen}[\q,  \q] = 1 \star D_F^{\din}[\q,  \p]$ \\
	  $D_F^{\dgen}[\q,  \myr] = \num{D_F^{\din}[\q,  \myr]} \star
	  D_F^{\din}[\q,  \p], \ \myr \not\in \{\p, 
	  \q\}$ \\ 
	  {\blue ${I_F^{\dgen}[\p,  \q]} = \{(\fieldD{f}{},  \epsilon)\}    %-----------------modified
	  \cup ((1 \star (\remOne{D_F^{\din}[\p, 
	      \p]}{\epsilonset})) \times \epsilonset)$} \\	 
	  $I_F^{\dgen}[\p,  \myr]$ = 
 	  $(1 \star (\remOne{D_F^{\din}[\p,  \p]}{\epsilonset})) \times 
	  \{\beta\ \vert\ (\alpha,  \beta) \in I_F^{\din}[\q,  \myr] \}$  
	  $\cup\ \{f^\drct\} \times 
	  \{\beta\ \vert\ (\epsilon,  \beta) \in I_F^{\din}[\q,  \myr] \}$ \\
	  \quad \quad\quad  $\cup\ \{f^{\indrct 1}\} \times 
	   \{\beta\ \vert\ (\alpha,  \beta) \in I_F^{\din}[\q,  \myr], 
	  \alpha \not= \epsilon \}$,  
	  \qquad\qquad  $\myr \not\in \{\p,  \q\}$ \\
	  {\blue ${I_F^{\dgen}[\s,  \q]} = (1 \star D_F^{\din}[\s,  \p]) \times    %-----------------modified
	  \epsilonset, \ \s \not\in \{\p,  \q\}$} \\ 
	  {\blue ${I_F^{\dgen}[\s,  \myr]} = (1 \star D_F^{\din}[\s,  \p])         %-----------------modified
	  \times \{\beta\ \vert\ (\alpha,  \beta) \in I_F^{\din}[\q, 
	    \myr]\}$}, 
	     \qquad $\s \not\in \{\p,  \q\}, \ \myr \not\in \{\p, 
	  \q\}, \ \s \not= \myr$

      \end{tabular}
  } \\
\hline
{\tt p = q$\rightarrow$f} &
    \scalebox{0.70}{
    \begin{tabular}{l}
	The KILL relations are same as that of {\tt p = NULL} \\
	$\GenC{p} = \InC{q} \qquad \GenD{p} = \InD{q}$ \\

	$f_{\q\p} = \true$ \quad
	${h_{\p\myr}} =
	\num{\project{D_F^{\din}[\q, \myr]}{f}} \geq 1\quad
	\forall h \in \fields,  \forall \myr \in \heap$  \\

	$ \upath \quad=\quad \epsilonset \cup \ \bigcup_{f\in\fields} \{f^{\drct}, 
	f^{\indrct\infty}\}$  \\

	{\blue ${D_1[\p,  \s]} =  \upath \quad
	\forall \s \in \heap,  \s \not=\p \wedge
	\project{D_F^{\din}[\q,  \s]}{f} \not= \emptyset$} \\

	$D_1[\p,  \p] = \left \{ \begin{array}{@{}ll}
	  \upath& \q.\shape \mbox{ evaluates to } \Cycle \\
	  \epsilonset & \mbox{Otherwise}
	\end{array}. \right. $  \quad \quad
	$I_1[\p,  \p] = \upath \times \upath $ \\

	{\blue ${D_2[\s,  \p]} = \infty \star D_F^{\din}[\s,  \q]$ \quad
	$\forall \s \in \heap,  \s \not=\q$} \\
	{\blue ${D_2[\q, \p]} = \{ f^{\drct}\} \cup (\infty \star (\remOne{D_F^{\din}[\q,  \q]}{\epsilonset})) \cup \upath$}  \\ 
	$I_2[\s,  \p] = D_2[\s,  \p] \times
	\epsilonset \quad \forall \s \in \heap $ \\	

	$D_3[\s,  \p] = \{ \alpha \ \vert \ (f^{\drct},  \alpha) \in I_F^{\din}[\q,  \s] \}$  \\
	$I_3[\s,  \p] = \{\alpha \ \vert\ (f^\anysup,  \alpha) \in I_F^{\din}[\q,  \s]\} \times \upath$ \\ \\

	Finally $I_F$ and $D_F$ relations are: \\
	$D_F^{\dgen}[\myr,  \s] =  D_1[\myr,  \s] \cup D_2[\myr,  \s] \cup D_3[\myr,  \s] \quad \forall\myr, \s \in \heap$  \\
	$I_F^{\dgen}[\myr,  \s] =  I_1[\myr,  \s] \cup I_2[\myr,  \s] \cup I_3[\myr,  \s] \quad \forall\myr, \s \in \heap$ 
	  

    \end{tabular}
    } \\
    \hline
%     \\
%     \footnotesize Data Flow Equations
%   
\end{tabular}
\end{spacing}
\caption{Data flow values corresponding to each statement.}  \label{fig:Data Flow Equations}
\end{figure}

