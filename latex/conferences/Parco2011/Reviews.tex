\documentclass{IOS-Book-Article}

\usepackage{times}
\normalfont
\usepackage[T1]{fontenc}
\usepackage{graphicx}
\usepackage{url}
\usepackage{pstricks}
\usepackage{pst-node}
\usepackage{pst-rel-points}
\usepackage{flexiprogram}

\newcommand{\isInterfering}{\mbox{isInterfering}}

\begin{document}
\begin{frontmatter}                           % The preamble begins here.

\title{ParCo 2011 : Actions Taken on Reviewers's Comments} 
\subtitle{ID 197 : Heap Dependence Analysis for Sequential Programs}
%
\end{frontmatter}

\section{Review 1}
\subsection{Shape analysis algorithm not specified in
  detail.}  Section 2 extended the abstract overview of the
shape analysis algorithm.  The details of this analysis can
be found in Sandeep's Thesis~\cite{Sandeep11thesis}.

\subsection{Differentiation against closely related work not
  crisp.} 
{\red TO DO}

\subsection{No comparative evaluation against previously
  published work.} 
{\red TO DO}

\subsection{What if the data structure of Figure 1 is cycle ?
  In that case there could be a dependence between {\tt S3}
  and {\tt S5}.}  Thanks for pointing this out. The
observation is absolutely correct.  The data structure in
Figure 1 is assumed to be a singly linked list. It is now
mentioned in the motivational example.

\subsection{Why shape analysis ``framework'' in section 2 ?}
Using ``framework'' we want to emphasise that any shape
analysis algorith which can give an interface function
function {\tt \isInterfering($p$, $\alpha$, $q$, $\beta$)}
can be plugged in our dependence detection analysis.


\subsection{Define ``abstract'' path (referred in section 2)
  ? Illustrate the concept on the example in Figure 1.} 
As our shape analysis is field-sensitive, so the anaysis must
remember the paths between the pointer variables. As the path
length may be unbounded so we abstracted the path as a fixed
length prefix of the actual path. Also it may happen that the
number of paths starting from the same fixed length prefix is
unbounded. But we will consider only k (a fixed constant)
such paths i.e.\ k-limiting on the number of paths starting
with the same fixed length prefixes. The details of this
abstraction can be found in Sandeep's
Thesis~\cite{Sandeep11thesis}.

In Section 2, we illustrated the concept of abstract path on
Figure 1.

\subsection{In section 2, last paragraph, the prime should be
  on the first mention of  alpha and beta in this paragraph
  not the second.} 
Thanks for pointing this out. This has been taken care of.

\subsection{In Table 1, use an emptyset symbol instead of a phi.}
Thanks for pointing this out. This has been taken care of.

\subsection{Mention up front that your analysis is intra-procedural.}
Thanks for pointing this out. It is mentioned up front that
the analysis is intra procedural.  

\subsection{Classify your shape analysis according to path- /
  flow-sensitivity.} 
{\red TO DO}


\section{Review 2}
\subsection{How to convert RE, which is used to represent
  access paths, to some canonical representation before
  comparing exponents.} 
{\red TO DO}

\subsection{How relevant is the inclusion of Lamport's work.} 
{\red TO DO}

\section{Review 3}
\subsection{To present some advanced examples or results with
  prototype implementation at the time of presentation.} 
{\red TO DO}

\bibliographystyle{plain}
\bibliography{parrefs}

\end{document}
