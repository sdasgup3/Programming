\chapter{Motivation} \label{ch:Motiv}
%
For each pointer variable, our analysis computes the shape
attribute of the data structure pointed to by the variable.
Following the existing
literature~\cite{Ghiya96,Sagiv96,Ghiya96practicaltechniques,marron06static}, we define the shape attribute $\p.\shape$
for a pointer $\p$ as follows:
%%
\begin{eqnarray*}
  \p.\shape = \left\{ \begin{array}{@{}rl@{}}
    \Cycle & \mbox{If a cycle can be reached from $\p$} \\ 
    \Dag & \mbox{Else if a DAG can be reached from $\p$} \\
    \Tree & \mbox{Otherwise} \\
  \end{array} \right.
\end{eqnarray*}
%%
where the heap is visualized as a directed graph, and cycle
and DAG have there natural graph-theoretic meanings.

We use the code fragment in Fig.~\ref{fig:motivA}(a) to
motivate the need for a field sensitive shape analysis.

\begin{example}{\rm
\begin{figure}
\centering
\scalebox{0.8}{
  \begin{tabular}{@{}cc@{}}
{\tt
      \begin{tabular}[b]{l}
        S1. p = malloc(); \\
        S2. q = malloc(); \\
        S3. p$\rightarrow$f = q; \\
        S4. p$\rightarrow$h = q; \\
        S5. q$\rightarrow$g = p; \\
        S6. q$\rightarrow$g = NULL;   \\
        S7. p$\rightarrow$h = NULL; 
      \end{tabular}
    }  &
    %\scalebox{0.80}{\begin{tabular}[b]{|c|@{}c@{}|@{}c@{}|}
    \begin{tabular}[b]{|c|@{}c@{}|@{}c@{}|}
      \hline
          {\bf After Stmt} & $D$ & $I$ \\ \hline
          {\tt S1}
          & \begin{tabular}{|c|cc|}
              {\tt } & $\p$  &  $\q$ \\ \hline
              $\p$ & 1  & 0 \\
              $\q$ & 0  & 0\\
			\hline
            \end{tabular} &
          \begin{tabular}{|c|cc|}
            {\tt } & $\p$  &  $\q$ \\ \hline
            $\p$ & 1  & 0 \\
            $\q$ & 0  & 0\\
			\hline
          \end{tabular}\\ \hline
         {\tt S2}
         & \begin{tabular}{|c|cc|}
             {\tt } & $\p$  &  $\q$ \\ \hline
             $\p$ & 1  & 0 \\
             $\q$ & 0  & 1\\
			\hline
           \end{tabular}&
		\begin{tabular}{|c|cc|}
		{\tt } & $\p$  &  $\q$ \\ \hline
		$\p$ & 1  & 0 \\
		$\q$ & 0  & 1\\
		\hline
		\end{tabular} \\ \hline
		{\tt S3}
	 & \begin{tabular}{|c|cc|} \hline 
	{\tt } & $\p$  &  $\q$ \\ \hline
	$\p$ & 1  & 1 \\
	$\q$ & 0  & 1\\
	\hline
	\end{tabular}&
	\begin{tabular}{|c|cc|} \hline 
	{\tt } & $\p$  &  $\q$ \\ \hline
	$\p$ & 1  & 1 \\
	$\q$ & 1  & 1\\
	\hline
	\end{tabular}\\ \hline
	{\tt S4}
	 & \begin{tabular}{|c|cc|} \hline 
	{\tt } & $\p$  &  $\q$ \\ \hline
	$\p$ & 1  & 1 \\
	$\q$ & 0  & 1\\
	\hline
	\end{tabular} &
	\begin{tabular}{|c|cc|} \hline 
	{\tt } & $\p$  &  $\q$ \\ \hline
	$\p$ & 1  & 1 \\
	$\q$ & 1  & 1\\
	\hline
	\end{tabular}\\ \hline
	{\tt S5, S6, S7}
	 & \begin{tabular}{|c|cc|} \hline 
	{\tt } & $\p$  &  $\q$ \\ \hline
	$\p$ & 1  & 1 \\
	$\q$ & 1  & 1\\
	\hline
	\end{tabular} &
	\begin{tabular}{|c|cc|} \hline 
	{\tt } & $\p$  &  $\q$ \\ \hline
	$\p$ & 1  & 1 \\
	$\q$ & 1  & 1\\
	\hline
	\end{tabular} \\ \hline
    \end{tabular} \\ 
     (a) A code fragment & 
     (b) \begin{tabular}[t]{@{}p{40mm}}
      Direction ($D$) and Interference ($I$) matrices as
      computed by Ghiya et. al.
    \end{tabular}
  \end{tabular}
	}
%\caption{A motivating example\label{fig:motivA}}
\end{figure}

\begin{figure}
\centering \scalebox{.80}{
\begin{tabular}[b]{|l@{}|@{}c@{}|@{}c@{}|} \hline
 {\bf After} & $D_F$ & $I_F$ \\ 
 {\bf Stmt} & & \\ \hline
 {\tt S1} & 
%\begin{tabular}{|p{3mm}|p{12mm}p{21mm}|} \hline 
\begin{tabular}{|p{3mm}|p{12mm}p{12mm}|} \hline 
  & $\p$  &  $\q$ \\ \hline
  $\p$ & $\left\{\epsilon\right\}$  & $\emptyset$ \\
  $\q$ & $\emptyset$  & $\emptyset$\\
  \hline
\end{tabular} &
\begin{tabular}{|p{3mm}|p{28mm}p{37mm}|} \hline 
%\begin{tabular}{|p{3mm}|p{28mm}p{28mm}|} \hline 
  & $\p$  &  $\q$ \\ \hline
  $\p$ & $\left\{(\epsilon, \epsilon)\right\}$  & $\emptyset$ \\
  $\q$ & $\emptyset$  & $\emptyset$\\
  \hline
\end{tabular} \\ \hline

{\tt S2} & 
%\begin{tabular}{|p{3mm}|p{12mm}p{21mm}|} \hline 
\begin{tabular}{|p{3mm}|p{12mm}p{12mm}|} \hline 
  & $\p$  &  $\q$ \\ \hline
  $\p$ & $\left\{\epsilon\right\}$  &    $\emptyset$ \\
  $\q$ &    $\emptyset$       & $\left\{\epsilon\right\}$\\
  \hline
\end{tabular} &
\begin{tabular}{|p{3mm}|p{28mm}p{37mm}|} \hline 
%\begin{tabular}{|p{3mm}|p{28mm}p{28mm}|} \hline 
  & $\p$  &  $\q$ \\ \hline
  $\p$ & $\left\{(\epsilon, \epsilon)\right\}$  & $\emptyset$ \\
  $\q$ &        $\emptyset$  & $\left\{(\epsilon, \epsilon)\right\}$\\
  \hline
\end{tabular} \\ \hline

{\tt S3} &
%\begin{tabular}{|p{3mm}|p{12mm}p{21mm}|} \hline 
\begin{tabular}{|p{3mm}|p{12mm}p{12mm}|} \hline 
  & $\p$  &  $\q$ \\ \hline
  $\p$ & $\left\{\epsilon\right\}$  &    $\left\{f\right\}$ \\
  $\q$ &    $\emptyset$       & $\left\{\epsilon\right\}$\\
  \hline
\end{tabular} &
\begin{tabular}{|p{3mm}|p{28mm}p{37mm}|} \hline 
%\begin{tabular}{|p{3mm}|p{28mm}p{28mm}|} \hline 
  & $\p$  &  $\q$ \\ \hline
  $\p$ & $\left\{(\epsilon, \epsilon)\right\}$  & $\left\{(f, \epsilon)\right\}$ \\
  $\q$ &        $\left\{(\epsilon, f)\right\}$  & $\left\{(\epsilon, \epsilon)\right\}$\\
  \hline
\end{tabular} \\ \hline

{\tt S4} &
%\begin{tabular}{|p{3mm}|p{12mm}p{21mm}|} \hline 
\begin{tabular}{|p{3mm}|p{12mm}p{12mm}|} \hline 
  & $\p$  &  $\q$ \\ \hline
  $\p$ & $\left\{\epsilon\right\}$  &    $\left\{f, h\right\}$ \\
  $\q$ &    $\emptyset$       & $\left\{\epsilon\right\}$\\
  \hline
\end{tabular} &
\begin{tabular}{|p{3mm}|p{28mm}p{37mm}|} \hline 
%\begin{tabular}{|p{3mm}|p{28mm}p{28mm}|} \hline 
  & $\p$  &  $\q$ \\ \hline
  $\p$ & $\left\{(\epsilon, \epsilon)\right\}$  & $\left\{(f, \epsilon), (h, \epsilon)\right\}$ \\
  $\q$ &        $\left\{(\epsilon, f), (\epsilon, h)\right\}$  & $\left\{(\epsilon, \epsilon)\right\}$\\
  \hline 
\end{tabular} \\ \hline

{\tt S5} &
\begin{tabular}{|p{3mm}|p{12mm}p{12mm}|} \hline 
  & $\p$  &  $\q$ \\ \hline
  $\p$ & $\left\{\epsilon, f, h\right\}$ &    $\left\{f, h\right\}$ \\ 
  $\q$ &    $\left\{g\right\}$       & $\left\{\epsilon, g\right\}$\\ 
  \hline
\end{tabular} &
\begin{tabular}{|p{3mm}|p{28mm}p{37mm}|} \hline 
%\begin{tabular}{|p{3mm}|p{28mm}p{28mm}|} \hline 
  & $\p$  &  $\q$ \\ \hline
  $\p$ & $\left\{(\epsilon, \epsilon)\right\}$  & 
  		$\left\{(f, \epsilon), (h, \epsilon),  (\epsilon, g)\right\}$ \\ 
  $\q$ &        $\left\{(\epsilon, f), (\epsilon, h), (g, \epsilon)\right\}$                   
  & $\left\{(\epsilon, \epsilon)\right\}$ \\ 
  \hline
\end{tabular} \\ \hline

{\tt S6}&
%\begin{tabular}{|p{3mm}|p{12mm}p{21mm}|} \hline 
\begin{tabular}{|p{3mm}|p{12mm}p{12mm}|} \hline 
  & $\p$  &  $\q$ \\ \hline
  $\p$ & $\left\{\epsilon\right\}$  &    $\left\{f, h\right\}$ \\
  $\q$ &   $\emptyset$       & $\left\{\epsilon\right\}$\\
  \hline
\end{tabular} &
\begin{tabular}{|p{3mm}|p{28mm}p{37mm}|} \hline 
%\begin{tabular}{|p{3mm}|p{28mm}p{28mm}|} \hline 
  & $\p$  &  $\q$ \\ \hline
  $\p$ & $\left\{(\epsilon, \epsilon)\right\}$  & $\left\{(f, \epsilon), (h, \epsilon)\right\}$ \\
  $\q$ &        $\left\{(\epsilon, f), (\epsilon, h)\right\}$  & $\left\{(\epsilon, \epsilon)\right\}$\\
  \hline
\end{tabular} \\ \hline

{\tt S7} &
%\begin{tabular}{|p{3mm}|p{12mm}p{21mm}|} \hline 
\begin{tabular}{|p{3mm}|p{12mm}p{12mm}|} \hline 
  & $\p$  &  $\q$ \\ \hline
  $\p$ & $\left\{\epsilon\right\}$  &    $\left\{f\right\}$ \\
  $\q$ &    $\emptyset$       & $\left\{\epsilon\right\}$\\
  \hline
\end{tabular} &
\begin{tabular}{|p{3mm}|p{28mm}p{37mm}|} \hline 
%\begin{tabular}{|p{3mm}|p{28mm}p{28mm}|} \hline 
  & $\p$  &  $\q$ \\ \hline
  $\p$ & $\left\{(\epsilon, \epsilon)\right\}$  & $\left\{(f, \epsilon)\right\}$ \\
  $\q$ &        $\left\{(\epsilon, f)\right\}$  & $\left\{(\epsilon, \epsilon)\right\}$\\
  \hline
\end{tabular} \\ \hline
\end{tabular}} \\[2mm]

{\footnotesize (a) Direction ($D_F$) and Interference ($I_F$) Matrices.} \\ [5mm]

\scalebox{.80}{  \begin{tabular}{|l|l|l|l|l|l|l|l|} \hline
    {\bf After Stmt} & {\tt S1 } & {\tt S2 } &  {\tt S3 } &  {\tt
      S4 } &  {\tt S5 } &  {\tt S6 } &  {\tt S7 } \\ \hline
    {\bf Boolean Vars} & & & & & & & \\ 
    $f_{pq}$ & false & false & true   & true  & true & true & true \\ 
    $h_{pq}$ & false & false & false & true  & true & true & false \\ 
    $g_{qp} $& false & false & false & false & true  & false & false\\ \hline 
  \end{tabular}} \\[2mm]

{\footnotesize (b) Values for boolean variables corresponding to
  relevant pointer fields.}

  \caption{Field Sensitive information for the 
    code in Figure~\ref{fig:motivA}(a) \label{fig:motivB}}
 \end{figure}


Consider the code segment in the Fig.~\ref{fig:motivA}(a),
At {\tt S4}, a DAG is created that is reachable from $\p$. 
At {\tt S5}, a cycle is created that is reachable from
both $\p$ and $\q$. This cycle is destroyed at line
{\tt S6} and the DAG is destroyed at {\tt S7}.

Field insensitive shape analysis algorithms use conservative
kill information and hence they are, in general, unable to
infer the shape transition from cycle to DAG or from DAG to
Tree.  For example, the algorithm by Ghiya
et. al.~\cite{Ghiya96} can correctly report the shape
transition from DAG to cycle\ (at {\tt S5}), but fails to
infer the shape transition from cycle to DAG (at {\tt S6})
and from DAG\ to \Tree\ (at {\tt S7}). This is evident from
Fig.~\ref{fig:motivA}(b) that shows the Direction ($D$)
and Interference ($I$) matrices computed using their
algorithm.  We get conservative shape information at {\tt S6}
and {\tt S7} because the kill-effect of statements {\tt S6}
and {\tt S7} are not taken into account for computing $D$
and $I$.  } \hfill\psframebox{}
\end{example}

We now show how we have incorporated limited field
sensitivity at each program point  in our shape
analysis. The details of our analysis will be presented later
(Chapter~\ref{ch:Analysis}).

\begin{example}{\rm 
The statement at {\tt S4} creates a new DAG structure
reachable from $\p$, because there are two paths ($\p
\rightarrow f$ and $\p \rightarrow h$) reaching $\q$. Any
field sensitive shape analysis algorithm must remember all
paths from $\p$ to $\q$.  Our analysis approximates any
path between two variables by the first field that is
dereferenced on the path.  Further, as there may be an
unbounded number of paths between two variables, we use
$k$-limiting to approximate the number of paths starting at a
given field.

Our analysis remembers the path information using the
following: (a) $D_F$: Modified direction matrix that stores
the first fields of the paths between two pointers; (b)
$I_F$: Modified interference matrix that stores the pairs of
first fields corresponding to the pairs of interfering paths,
and (c) Boolean Variables that remember the fields directly
connecting two pointer variables.

Figures~\ref{fig:motivB}(a) and~\ref{fig:motivB}(b) show the
values computed by our analysis for the example program.  In
this case, the fact that the shape of the variable $\p$
becomes DAG after {\tt S4} is captured by the following
boolean functions\footnote{The functions and values shown in
  this example and in Fig.~\ref{fig:motivB} are simplified
  to avoid references to concepts not defined yet.}
:
\begin{eqnarray*}
  p_{\subD} &=& (h_{pq} \wedge (\num{\IFM{p}{q}} > 1)) \vee
  (f_{pq} \wedge (\num{\IFM{p}{q}} > 1)), \qquad h_{pq} =
  \true \enspace .
\end{eqnarray*}
Where $h_{pq}$ is a boolean variable that is true if $h$
field of $\p$ points to $\q$, $f_{pq}$ is a boolean
variable that is true if $f$ field of $\p$ points to
$\q$, $I_F$ is field sensitive interference matrix,
$\num{I_F[p,q]}$ is the count of number of interfering paths
between $\p$ and $\q$. 

The functions simply say that variable $\p$ reaches a DAG
because there are more than one paths ($\num{I_F[p,q]} > 1$)
from $\p$ to $\q$. It also keeps track of the paths ($f_{pq}$
and $h_{pq}$ in this case).  Later, at statement {\tt S7},
the path due to $h_{pq}$ is broken, causing $\num{I_F[p,q]} =
1$. This causes $\p_{\subD}$ to become false.  Note that we
{\em do not} evaluate the boolean functions immediately, but
associate the unevaluated functions with the statements. When
we want to find out the shape at a given statement, only then
we evaluate the function using the $D_F$ and $I_F$ matrices,
and the values of boolean variables at that statement.

Our analysis uses  another attribute ${\Cycle}$ to capture the
cycles reachable  from a  variable. For our  example program,
assuming  the   absence  of  cycles  before   {\tt  S5},  the
simplified  functions for  detecting cycle  on $\p$ after
{\tt S5} are:
\begin{eqnarray*}
p_{\subC} &=& g_{qp} \wedge (\num{\DFM{p}{q}} \geq 1), \qquad
g_{qp} = \true \enspace .
\end{eqnarray*}

Here, the functions captures the fact that cycle on $\p$
 consists of field $g$ from $\q$ to $\p$ ($g_{qp}$)
and some path from $\p$ to $\q$ ($\num{\DFM{p}{q}} \geq
1$). This cycle is broken either when the path from $\p$
to $\q$ is broken ($\num{\DFM{p}{q}} = 0$) or when the link $g$
changes ($g_{qp} = \false$). The latter occurs after
{\tt S6} in Fig.~\ref{fig:motivB}(a).  
}
\hfill\psframebox{}  
\end{example}



In the rest of the report, we formalize the intuitions
presented above and describe our analysis in details.
%
