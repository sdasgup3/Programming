\newcommand{\SQA}{
\question \QuestionTitle{Software Quality Assurance and XP}

\begin{parts}
% YES
\part[2] In the lecture on Software Quality Assurance (SQA), we
mentioned that one of the goals of measuring metrics is to be able to
answer the question: \textit{``How accurate are our estimates?''}  In
this context, \textbf{what does the term ``velocity'' mean} in an XP
process?

\begin{solution}[1.25in]
   How many units or user stories can you implement in each unit of time? 
\end{solution}

\part A crucial issue that arises in software development
is how to know whether the code that the developers are writing will
result in software that the customer actually needs.

\begin{subparts}

% YES
\subpart[2] In this context, briefly describe the \textbf{two} terms,
\textbf{Verification} and \textbf{Validation}.

\begin{solution}[1.25in]
   Verification - Did we build the product right? Code is correct with
   respect to some requirements/spec. Building code the right way.

   Validation - Did we build the right product? Code results in
   software that customers want. Does the spec for the software match
   what the customer really wants?
\end{solution}

% YES
\subpart[2] \textbf{Describe one} XP practice that can be used to improve the
chances that the software being developed will meet customer needs.

\begin{solution}[1.25in]
   XP solution - Customer on site, Short iterations, Customer tests
\end{solution}
\end{subparts}
\end{parts}
}

\newcommand{\XP}{
\question \QuestionTitle{XP Project Management}

\begin{parts}
\part During Iteration 1 of the course project, your team could
submit a ``spike'' as a deliverable.

\begin{subparts}

% YES - MULTIPLE
\subpart[2] Which of these correctly describe an XP spike? \CircleAll{}
\begin{enumerate}
\item A spike may be used to try out a tough design.
\item A spike is another name for a major milestone.
%\item A spike helps to reduce uncertainty in an estimate.
%\item A spike results in a \textit{complete} analysis of a new
%  technology to present to the client.
\item A spike is a way to quickly try out an unfamiliar technology.
\item A spike must be done at the beginning of every XP project.
\item A first attempt at a solution which may be thrown away.
\end{enumerate}

\begin{solution}[0.1in]
1,3,5 (Any 2 of these 3 will get full points)
\end{solution}

% YES - PROJECT
\subpart[2] In terms of testing\Comment{ and refactoring}, how is working
on a spike different from regular XP iterations?  \Comment{Describe how your
later iterations differed from the initial spike.}

\begin{solution}[1.25in]
Spike is likely to involve no testing and no refactoring.
\end{solution}

\end{subparts}
\part \Choose{Planning Game}{Refactoring}

\begin{subparts}
% YES
\subpart[2] Why is that practice advocated in XP?
\begin{solution}[1.25in]
\end{solution}

% YES - PROJECT
\subpart[2] How did your team follow the practice during the course project?
\begin{solution}[1.25in]
\end{solution}
\end{subparts}

% YES
\part[2] In MP4,\Comment{ we gave some suggestions for writing a good user story.}
you learned that some development groups write their user stories on physical
3inch~X~5inch index cards (called ``bits of plain card'' on the web site given
for MP4).  Why are user stories written on such small cards?

\begin{solution}[1.5 in]
This serves as  a physical constraint which limits the possible length of the story. That's because user story should be concise, with short but clear goals.
\end{solution}
\end{parts}
}


\newcommand{\Mistakes}{
\question \QuestionTitle{Testing Mistakes}

\begin{parts} 
\part  In ``Classic Testing Mistakes'', Brian Marick writes about
 some common pitfalls in the way many organizations approach software
testing.  For each of the following, \textbf{describe one reason} why
it is considered a ``Testing Mistake'', according to the reading and
the lecture on this topic.

\begin{subparts}
% YES
\subpart[1] ``Sticking stubbornly to the test plan''.  (Hint: We saw
a related video during a lecture.)

\begin{solution}[1.5 in]
Selective Attention could make the testers miss important things
outside of the scope of the test plan.
\end{solution}

% YES
\subpart[1] ``The testing team is the only team responsible for
assuring software quality''.

\begin{solution}[1.6 in]
Reason:

 The testers usually have no authority to prevent poor quality
   software from being shipped.  

 The testers sometimes can't really improve quality, they can
   usually only measure it, especially when testing starts too late.

 Telling developers that software quality is someone Else job
   tends to make them sloppy about the code that they write.

\end{solution}

% YES
\subpart[1] ``Over-reliance on beta testing''.

\begin{solution}[1.6 in]
\begin{enumerate}
\item  The customers probably aren't that representative. 

\item  Even of those beta users who actually use the product, most will 
not use it seriously.

\item  Beta users - just like customers in general - don't report usability
problems unless prompted. They simply silently decide they won't buy
the final version.

\item  Beta users - just like customers in general - often won't report a
bug, especially if they're not sure what they did to cause it, or if
they think it is obvious enough that someone else must have already
reported it.

\item  When beta users report a bug, the bug report is often unusable. It
costs much more time and effort to handle a user bug report than one
generated internally.

\end{enumerate}
\end{solution}

\end{subparts}

% YES
\part[2] One of the ``Classic Testing Mistakes'' is \textbf{Attempting to 
automate every test}. Describe \textbf{two} cases when manual testing
is better than automated testing.  Similarly, describe \textbf{two}
cases when automated testing is better than manual testing.

\begin{solution}[1.6 in]

Manual testing is appropriate when: 

\begin{itemize} 

\item tests need to be run once 
\item tester doesn't know how to program 
\item the test is expensive to automate 
\item When writing some kinds of GUI tests, for example 
\end{itemize}

Automated testing is appropriate when:
\begin{itemize}

\item the tests need to be run over and over as the software evolves 
or is changed
\item when we want to use them as documentation.
\item The Main Advantage with Automated Testing is that you can do
regression testing quickly.  Developer can verify previous
functionality, if it is working fine or not after adding new
functionality into the System.
\end{itemize}

\end{solution}

\Comment{
\begin{subparts}

\subpart Provide a cost-based justification of why some tests should
be manual.

\begin{solution}[1.5 in]
If a manual test costs x hours each time it is run, changing this
to an automated test is justified if the much higher one-time cost is
less than x times the number of times the manual test is likely to be
run.
\end{solution}
}

\end{parts}
}% End of \Mistakes

\Comment{add a question about MOCK testing; it's in the book (look for
SCAFFOLDING/STUB) and was mentioned on the final review slides}

\Comment{add a question about GUI testing, because they all worked with GUI
in their projects}

\newcommand{\Advocacy}{

\question \QuestionTitle{Bug Advocacy}

\begin{parts}

% YES
\part[2] The lecture and slides on Bug Advocacy talked about a process
called \textbf{Bug Triage}. Why is this process needed during software
development?

\begin{solution}[2.1 in]
Bug Triage helps developers to decide which bugs to fix, and which
ones to leave in the software product.
\end{solution}

\Noo{
\part[3] This question concerns the analogy between bug advocacy and sales. 
In bug advocacy, (i) \textbf{what are you trying to sell?}, (ii) \textbf{why do
you have to sell it?}, and (iii) \textbf{to whom do you have to sell it?}

\begin{solution}[1.5 in]
\begin {enumerate}
\item Trying to sell: bugs 
\item Selling to: developers who are in a position to fix them 
\item Reason for selling: need to give the developers a reason to
choose to fix this bug among all the bug reports that they get.
\end {enumerate}
\end{solution}
}

\Maybe{
\part[2] Describe \textbf{two reasons} that you might use to motivate the
developers to fix a bug that you found in their software. (Hint:
Remember the slide on ``Motivating the bug fixer.'')

\begin{solution}[1.25in]
-- It looks really bad.
-- It will affect lots of people. % FROM DARKO: how can it affect a lot of people if defined as ``extremely unlikely''?
-- Getting to it is trivially easy.
-- It is a breach of contract.
-- A bug like it has embarrassed the company, or a competitor.
-- It looks like an interesting puzzle and piques the programmer’s
   curiosity.
-- Management (that is, someone with influence) has said that they
   really want it fixed
-- You said you want this particular bug fixed, and the programmer
   likes you, trusts your judgment is susceptible to flattery from you
   or owes you a favor.
\end{solution}
}
\end{parts}
}% End of \Advocacy

\newcommand{\Reuse}{
\question[3] \QuestionTitle{Components and Reuse}

% YES - PROJECT
For your final projects, we highly recommended that you extend an
existing Jenkins plug-in instead of creating one from scratch.  The
lecture on ``Components and Reuse'' mentioned that such reuse
should only be done when the benefits of reuse outweigh the costs of
reuse.  Describe \textbf{two benefits} and \textbf{one drawback} of
reusing an existing Jenkins plug-in for your course project instead of
building one from scratch.

\begin{solution}[1.5in]
Benefits include saving time when the existing plug-in provides a lot
of functionality needed, existing plug-ins usually have some tests,
existing plug-ins help avoid duplicate work, existing plug-ins have
support...

Drawbacks include a potentially high learning curve for functionality
that could have been implemented easily
\end{solution}

\Maybe{
\part[2] Libraries provide one way to achieve software reuse.  \textbf{Choose one} of the following alternative approaches
to software reuse and \textbf{describe it}: \textit{Components},
\textit{Framework}

\begin{solution}[1.5 in]
\end{solution}   
Frameworks: Encapsulate both design and code that can be reused
in different scenarios. Developers have to understand the implementation in order to reuse them.

Components: Enable the reuse of code only, and developers do not have to understand the internals to reuse them.
}
} % End \Reuse


%%%%% SAM, SUE AND ALEX %%%%%%%%%%%%%%%%%%%%%%
\Comment{
\question \textbf{Components and Reuse.} 

\begin{parts}

% NEWER VERSION
\part[3] Sue and Alex are software architects who are discussing how 
they enable software reuse in their different companies. Sue
says, \textbf{``We try to enable the reuse of not just code, but
design as well.  Generic designs are implemented as abstract and
concrete classes so that developers can reuse the design in different
scenarios.''}  Alex explains, \textbf{``Each team writes code that can
be reused by other teams by just calling an API, without needing to
understand how the code works.''}  Which of these architects is
referring to the ``Frameworks'' that you learned about in the lecture
on ``Components and reuse''?


% OLDER VERSION 
\part[3] Sam, Sue and Alex are software architects. They meet at a 
conference and each talks about how they enable software reuse in
their different companies. Sam says, \textbf{``We simply provide a set of math
functions like \texttt{min}, \texttt{sin} that developers can simply
call in their own programs''}. Sue responds, \textbf{``We try to enable the
reuse of not just code, but design as well.  We try to have generic
designs, implemented in the form of abstract and concrete classes so
that developers can easily adapt the design to different scenarios.''}
Finally, Alex explains.  \textbf{``Each team writes code such that they can be
reused by other teams by just calling an API, without needing to
undestand how the code works.''}  These architects are talking
about \textbf{components}, \textbf{libraries} and \textbf{frameworks}.
Based on the slides on ``Components and Reuse'' and the reading on
``Frameworks'', match the architects to the more technical term that
they are talking about.

}



% This question is to be removed. % FROM DARKO: I DON'T UNDERSTAND WHO
% SAID THAT THIS SHOULD BE REMOVED; IF OWOLABI, THEN PLEASE MARK YOUR
% BACKUP QUESTIONS AS SUCH AND LIST THEM LOWER.
\Comment{explain ``this''. Alternatively ask how to test thick vs. thin GUIs.  
Lecture after break on testing GUI for Jenkins
\question GUI Testing

\begin{parts}

\part[2] In class and during the course project, we often suggested 
that a good way to test your GUI is to put as little code in the GUI
layer itself, so that you can write automated tests for the business
logic and then do manual testing for the GUI. Describe \textbf{one reason}
why this does not work well in practice.

\begin{solution}
Most GUIs end up as ``thick GUIs'' which do more than handling the
presentation of the software, but rather also contain some business
logic. Manual testing may not uncover some of the problems with
business logic written as part of the GUI.
\end{solution}

\end{parts}
}
\Comment{
\subpart In \textbf{one sentence}, explain what a ``Spike'' is.

\begin{solution}[1.5 in]
A first attempt at a solution to a problem which may be thrown away.
\end{solution}

\subpart \textbf{Describe one reason} why projects which follow XP process 
might want to use a Spike \textit{as a way to help in making better
estimates of future work}.  In other words, describe one circumstance
which may necessitate the use of a Spike during an XP project. (Hint:
Think about why you used a Spike for your course project.)

\begin{solution} [1.5 in]
 - The developers may be unfamiliar with a technology and may need a
   Spike to learn it.
 - You may need a Spike to try out some tough design 
\end{solution}
}

% LocalWords:  XP SQA refactoring Marick GUIs
