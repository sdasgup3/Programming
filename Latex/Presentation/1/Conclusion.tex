\chapter{Conclusion and Future Work}
\label{ch:conclusion}
In this report we have presented our work on heap induced 
dependence analysis that can be utilized by a parallelizing 
compiler to extract both fine-grained and coarse-grained parallelism 
from sequential programs. Our method gives an easy to implement 
technique for the same. It is divided into two phases: 
the intra-procedural dependence analysis phase and loop 
based analysis phase, with carefully chosen interfaces between phases to 
combine work done by individual phases. The first phase is helpful 
to find dependences in both statement level and function 
call level, whereas, the second phase refines the analysis in the 
context of loop. This modularity gives us flexibility to work 
on testing and improving each phase independently. 

Our intra-procedural analysis abstracts each actual heap location 
by symbolic location, which is defined by set of access 
paths leading to same heap node. It successfully computes 
the read and write sets of heap access paths at each program point 
and identifies dependences based on the aliasing information produced 
by the specific shape analysis framework~\cite{sandeep}. 
Our loop dependence analysis abstracts the dependence 
information in forms of linear equations, that can be solved using traditional 
dependence analysis tests like GCD, Lamport tests that already 
exist for finding array dependences. 
Our intra-procedural analysis use conservative approximation of function calls 
assuming worst case scenario. We give a direction to extend the 
intra-procedural analysis to inter-procedural one, which is able to precise 
function calls more precisely. 


Our analysis is too conservative for complex cyclic structures and can 
not extract any parallelism between any two statements or 
function calls or different iterations of loop body . We have 
to further develop our shape analysis 
technique to handle more frequently occurring complex and cyclic 
structures and programming patterns to find precise dependences. 
In this work the analysis only keeps information about the first 
link field of the access paths and blindly summarizes the rest of the path. Hence 
it losses good amount of information for interference analysis.  
We want to improve the summarization technique for better abstraction of access paths. 
In case of loop sensitive dependence analysis, our technique assumes the loops without 
any irregular control flow constructs. We can further extend our technique to automatically 
detect good loops~\cite{ghiya98detecting} which do not contain any irregular control flow constructs. 

We have implemented a basic prototype model for intra-procedural 
and loop based dependence analysis. This prototype model with manual intervention 
detects dependences from heap intensive sequential programs. We have tested 
the model with a very few number of heap intensive C benchmark programs. 
It can be further developed to implement inter-procedural dependence 
analysis and to show its effectiveness on large benchmarks. 


