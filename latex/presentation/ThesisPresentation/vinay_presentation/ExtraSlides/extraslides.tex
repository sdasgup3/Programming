%\documentclass{beamer}
\documentclass[mathserif,10pt]{beamer}

\usepackage{beamerthemesplit}
\usepackage{graphics}
\usepackage{epsfig}
\usepackage{algorithm}
\usepackage{verbatim}
\usepackage{listings}
\usepackage{framed}
\usepackage{pstricks}
\usepackage{pst-node,pst-tree}
\usepackage{pst-rel-points}
\usepackage{flexiprogram}
\usepackage[UKenglish]{babel}
\usepackage{hyperref}
\usepackage{pst-coil}
\usepackage{color}
\usepackage{epsfig}
\usepackage{tikz}
\usepackage{multirow}
\usepackage{graphviz}
\usefonttheme{serif}



\newcommand{\cmt}[1]{}
\newcommand{\epsilonset}{\ensuremath{\{\epsilon\}}}
\newcommand{\epsilonpairset}{\ensuremath{\{\epsilon,\epsilon\}}}
\newcommand{\num}[1]{\ensuremath{|#1|}}
\newcommand{\upath}{\ensuremath{\mathcal{U}}}
\newcommand{\mb}[1]{\mbox{{\tt #1}}}
\newcommand{\ttf}[1]{{\tt #1}}
\newcommand{\rtarrow}{$\rightarrow$}
\newcommand{\Tree}{{\tt Tree}}
\newcommand{\Dag}{{\tt Dag}}
\newcommand{\Cycle}{{\tt Cycle}}
\newcommand{\p}{\ensuremath{p}}
\newcommand{\q}{\ensuremath{q}}
\newcommand{\s}{\ensuremath{s}}
\newcommand{\f}{\ensuremath{f}}
\newcommand{\myr}{\ensuremath{r}}
\newcommand{\shape}{\mbox{shape}}
\newcommand{\drct}{\ensuremath{D}}
\newcommand{\indrct}{\ensuremath{I}}
\newcommand{\heap}{\ensuremath{\mathcal{H}}}
\newcommand{\fields}{\ensuremath{\mathcal{F}}}
\newcommand{\DFM}[2]{\ensuremath{D_F[#1,#2]}}
\newcommand{\IFM}[2]{\ensuremath{I_F[#1,#2]}}
\newcommand{\nat}{\ensuremath{\mathcal{N}}}
\newcommand{\fieldD}[2]{\ensuremath{{#1}_{#2}^\drct}}
\newcommand{\fieldI}[3]{\ensuremath{{#1}_{#2}^{\indrct#3}}}
\newcommand{\subC}{\mbox{\scalebox{0.6}{\Cycle}}}
\newcommand{\subD}{\mbox{\scalebox{0.6}{\Dag}}}
\newcommand{\false}{\textbf{False}}
\newcommand{\true}{\textbf{True}}

\newcommand{\dout}{\mbox{\footnotesize\em out}}
\newcommand{\din}{\mbox{\footnotesize\em in}}
\newcommand{\dkill}{\mbox{\footnotesize\em kill}}
\newcommand{\dgen}{\mbox{\footnotesize\em gen}}


\newcommand{\GenC}[1]{\ensuremath{{#1}_{\subC}^{\dgen}}}
\newcommand{\GenD}[1]{\ensuremath{{#1}_{\subD}^{\dgen}}}
\newcommand{\KillC}[1]{\ensuremath{{#1}_{\subC}^{\dkill}}}
\newcommand{\KillD}[1]{\ensuremath{{#1}_{\subD}^{\dkill}}}
\newcommand{\InC}[1]{\ensuremath{{#1}_{\subC}^{\din}}}
\newcommand{\InD}[1]{\ensuremath{{#1}_{\subD}^{\din}}}
\newcommand{\OutC}[1]{\ensuremath{{#1}_{\subC}^{\dout}}}
\newcommand{\OutD}[1]{\ensuremath{{#1}_{\subD}^{\dout}}}

\newcommand{\Left}{\ensuremath{Left}}
\newcommand{\Right}{\ensuremath{Right}}
\newcommand{\Parent}{\ensuremath{Parent}}  

\newcommand{\project}[2]{\ensuremath{#1\triangleright\!\!#2}}
%\noindent

\newcommand{\labelitemi}{$\bullet$}

\setcounter{tocdepth}{1}

\lstset{
basicstyle=\footnotesize,
breakatwhitespace=true,
language=[ANSI]C,
columns=fullflexible,
keepspaces=true,
breaklines=true,
tabsize=3, 
showstringspaces=false,
extendedchars=true
}


% \lstset{language=[ANSI]C}
% \lstset{% general command to set parameter(s)
% basicstyle=\footnotesize\tt, % print whole listing small
% identifierstyle=, % nothing happens
% commentstyle=\color{red}, % white comments
% showstringspaces=false, % no special string spaces
% lineskip=1pt,
% captionpos=b,
% frame=single,
% breaklines=true
% %\insertauthor[width={3cm},center,respectlinebreaks]
% }
% \lstset{classoffset=0,
% morekeywords={},keywordstyle=\color{black},
% classoffset=1,
% classoffset=0}% restore default

%\usetheme{Warsaw}
\usetheme{CambridgeUS}
%\usetheme{Antibes}
%\usecolortheme{lily}
%\useinnertheme{rectangles} 
%\useoutertheme{infolines} 
%\setbeamercolor{alerted text}{fg=cyan}
%\beamertemplatetransparentcovereddynamicmedium
%\definecolor{bbrown}{rgb}{.6588,.4,.1647}
%\definecolor{blueviolet}{rgb}{.098039216,.050980392,.929411765}
%\definecolor{periwinkle}{rgb}{.423529412,.458823529,.988235294}
%\mode<presentation>
%{ \usetheme{boxes} }
\usecolortheme{dolphin}



\definecolor{orange}{rgb}{1,0.5,0}

\title[Shape Analysis]{Field Sensitive Shape Analysis: Implementation and Improvements}
\author[P.Vinay Kumar Reddy]{
\large{\textbf{P.Vinay Kumar Reddy}}
\newline
\newline \small{Supervisor}
\newline \large{\textbf{Dr. Amey Karkare}}
}
\institute[CSE, IIT Kanpur]{\textbf{Department of Computer Science and Engineering}
\newline \textbf{Indian Institute of Technology, Kanpur}}
\date{June , 2012}
%\begin{comment}
\begin{document}

\section{Precise Shape Analysis using Field Sensitivity}
\frame
{
    \frametitle{Analysis}
    
    \begin{itemize}
     \item  Heap structure at a program point as a directed graph. %\pause
	    \begin{itemize}
	      \item  {\blue $D_F$}: Direction matrix. \cmt{ that stores the first fields of the paths between two pointers.}
% 	      \pause
	      \item {\blue $I_F$}: Interference matrix. \cmt{ that stores the pairs of first fields corresponding to the pairs of interfering paths.}
% 	      \pause
	      \item {\blue Boolean Variables}: For $f \in \fields, \p, \q \in \heap$, \\ 
			\begin{center}
			$f_{pq} = \left\{\begin{array}{@{}ll}
			                  \true & \f \ field \ of \  \p \ points \ directly \ to \ \q \\
					  \false & Otherwise
			                 \end{array}\right.$
			\end{center}
	    \end{itemize}

    \end{itemize}
}

\section{Subset Based Analysis}
\subsection{Example explained}
\begin{frame}[fragile]
  \frametitle{\subsecname}
\begin{tabular}{cc}
\begin{lstlisting}
Struct Node {
  Struct Node *Left,*Right;
  Struct Node *Parent;
  int key;
};
typedef Struct Node Node;

bool search(Node *root,int key){
if(root)
  return (key==root->key)||
	  search(root->Left,key)||
	  search(root->Right,key);
return 0;
}
\end{lstlisting} 
&
\begin{lstlisting}
 void insert(Node *root,int key){
    Node *s=root; 
    ..
    //New node is inserted as a child of s(s can be any node of the tree)
    Node *p;
    S1. p=(Node *)malloc(sizeof(Node));    	
    S2. p->Left=NULL;			
    S3. p->Right=NULL;			
    S4. p->key=key;				
    S5. s->Left=p;				
    S6. p->Parent = s;			
    ..
}
\end{lstlisting} \\
\end{tabular}
\pause

 \begin{eqnarray*}
  S5:  {\tt s_{cycle}} &=& \{  (\Left_{s,p}  \wedge {\blue \InC{p}} ) \vee (\Left_{s,p} \wedge   ( \num{D[p,s]} >= 1)) \} \\ \pause
  S6:  {\tt s_{cycle}} &=& \{ \Parent_{p,s} \wedge ( \num{D[s,p]} >= 1) \} \vee \{  (\Left_{s,p}  \wedge False) \vee \\
 && (\Left_{s,p} \wedge   ( \num{D[p,s]} >= 1)) \}
  \end{eqnarray*}

% Boolean function of $root$ at the entry of \emph{search} function.
% \begin{eqnarray*}
% {\tt root_{cycle}} &=& \{ \Parent_{p,root} \wedge ( \num{D[root,p]} >= 1) \} \vee \{  (\Left_{root,p}  \wedge False) \vee \\
%  && (\Left_{root,p} \wedge   ( \num{D[p,root]} >= 1)) \} \\
% \end{eqnarray*}
\end{frame}

\frame
{
  \frametitle{\subsecname}
\scalebox{0.70}{
\begin{tabular}[b]{|@{}c@{}|@{}c@{}|} \hline
\begin{tabular}{|p{6mm}|p{15mm}p{15mm}|} \hline 
   D         & $\p$  		& $root$   \\ \hline
  $\p$ 	& $\epsilon$	& $\{\fieldD{Parent}{}\}$	 \\ \hline
  $root$ 	& $\{\fieldD{Left}{}\}$	& $\emptyset$	\\ \hline
\end{tabular}
 &
\begin{tabular}{|p{6mm}|p{34mm}p{34mm}|} \hline 
   I         & $\p$  		& $root$   \\ \hline
  $\p$ 	& $\{(\epsilon,\epsilon)\}$	&  $\{( \epsilon,\fieldD{Left}{}),(\fieldD{Parent}{},\epsilon)\}$	 \\ \hline
  $root$ 	& $\{(\fieldD{Left}{}, \epsilon),(\epsilon,\fieldD{Parent}{})\}$		& $\emptyset$	\\ \hline
\end{tabular} \\ \hline
\end{tabular}  \ $\Left_{root,p} = \true$
}

\begin{eqnarray*}
{\tt root_{cycle}} &=& \{ \Parent_{p,root}^{\#} \wedge ( \num{D^{\#}[root,p]} >= 1) \} \vee \{ (\Left_{root,p}^{\#} \wedge False ) \vee \\
&& (\Left_{root,p}^{\#} \wedge   ( \num{D^{\#}[p,root]} >= 1)) \} \\
\end{eqnarray*}
\begin{eqnarray*}
 Parent_{p,root}^{\#} &=& \false  \ \ \ (Parent \not \in S_{Search} ) \\
 Left_{root,p}^{\#} &=& Left_{root,p} \  (Left \in S_{Search} ) \\
 D^{\#}[root,p] &=& D[root,p] - \project{D[root,p]}{Parent} \\
		&=& D[root,p] - \phi \\
		&=& \{\fieldD{Left}{}\} \\
 D^{\#}[p,root] &=& D[p,root] - \project{D[p,root]}{Parent} \\
		&=& D[p,root] - \{\fieldD{Parent}{}\} \\
		&=& \phi
\end{eqnarray*}

}

\section{Inter Procedural Analysis}
\subsection{Shape Sensitive}
\begin{frame}[fragile]
  \frametitle{\subsecname}
\centering    
\begin{tabular}[t]{cl}
\begin{lstlisting}
struct Node
{
  struct Node *f,*g;
};

void foo(struct Node *s)
{
  struct Node *t;
  S: t=s;
  ....
}
\end{lstlisting}
&
\begin{lstlisting}
int main()
{
  struct Node *p,*q;
  p=(struct Node *)malloc(sizeof(struct Node));
  q=(struct Node *)malloc(sizeof(struct Node));

  S1: p->f=q;
  S2: q->f=p;
  c1: foo(p);
  S3: q->f=NULL;
  c2: foo(p);
}
\end{lstlisting}
\end{tabular}

\end{frame}

\frame
{
  \frametitle{\subsecname}
  \begin{itemize}
   \item Each function will be attached with some set of dataflow value versions
   \item IN[0] $\rightarrow$ Tree , IN[1] $\rightarrow$ Dag , IN[2] $\rightarrow$ Cycle 
  \end{itemize}

\begin{tabular}{lc}
  \multirow{15}{*}{
  \scalebox{0.70}{
  \begin{tabular}{ll}
  S1: & p $\rightarrow$ f=q; \\
  S2: & q $\rightarrow$ f=p;\\
  c1: & foo(p);\\
  S3: & q $\rightarrow$ f=NULL;\\
  c2: &  foo(p);\\
  \end{tabular}
  }
  } & \\ \pause

   & \scalebox{0.6}{
\begin{tabular}{cc}
\renewcommand{\arraystretch}{1.2}
\begin{tabular}[b]{|c|c|c|c|}
\hline
$D_F$     & \p & \q &  \s  \\ \hline \hline 
\p 	& $\{\epsilon\}$ & $\{\fieldD{f}{} \}$ & $\{\epsilon\}$  \\ \hline 
\q             & $\{\fieldD{f}{}\}$      & $\{\epsilon\}$          & $\{\fieldD{f}{}\}$  \\ \hline
\s             & $\{\epsilon\}$           &$\{\fieldD{f}{}\}$      & $\{\epsilon\}$      \\ \hline
\end{tabular} 
&

\renewcommand{\arraystretch}{1.2}
% % \newcommand{\iwd}{0.23\columnwidth}
\begin{tabular}[b]{|c|c|c|c|}
\hline 
$\ I_F$     & $\p$	               & $\q$
&  $\s$              \\ \hline \hline 
%%
$\p$ & $\{ (\epsilon, \epsilon)\}$    & $\{(\fieldD{f}{},\epsilon),(\epsilon,\fieldD{f}{}) \} $   & $\{ (\epsilon, \epsilon)\}$ \\ \hline
$\q$ & $\{(\fieldD{f}{},\epsilon),(\epsilon,\fieldD{f}{}) \} $   & $\{ (\epsilon, \epsilon)\}$ & $\{(\fieldD{f}{},\epsilon),(\epsilon,\fieldD{f}{}) \} $ \\ \hline
$\s$ & $\{ (\epsilon, \epsilon)\}$    & $\{(\fieldD{f}{},\epsilon),(\epsilon,\fieldD{f}{}) \} $   & $\{ (\epsilon, \epsilon)\}$ \\ \hline
\end{tabular} \\
% \caption{C1:Context Insensitive}
\end{tabular}
} \\
& \scriptsize (a) c1:IN[2] \\ \\ \pause

& \scalebox{0.6}{
\begin{tabular}{cc}
\renewcommand{\arraystretch}{1.2}
\begin{tabular}[b]{|c|c|c|c|}
\hline
$D_F$     & \p & \q &  \s  \\ \hline \hline 
\p 	& $\{\epsilon\}$ &  & $\{\epsilon\}$  \\ \hline 
\q 	&  & $\{\epsilon\}$ &   \\ \hline 
\s 	& $\{\epsilon\}$ &  & $\{\epsilon\}$  \\ \hline 
\end{tabular} 
&

\renewcommand{\arraystretch}{1.2}
% % \newcommand{\iwd}{0.23\columnwidth}
\begin{tabular}[b]{|c|c|c|c|}
\hline 
$\ I_F$     & $\p$	               & $\q$
&  $\s$              \\ \hline \hline 
%%
$\p$ & $\{ (\epsilon, \epsilon)\}$    &   & $\{ (\epsilon, \epsilon)\}$ \\ \hline
$\q$ &     & $\{ (\epsilon, \epsilon)\}$   &  \\ \hline
$\s$ & $\{ (\epsilon, \epsilon)\}$    &    & $\{ (\epsilon, \epsilon)\}$ \\ \hline
\end{tabular} \\
% \caption{C1:Context Insensitive}
\end{tabular}
} \\

& \scriptsize (b) c2: IN[0] \\ \\ \pause

   & \scalebox{0.6}{
\begin{tabular}{cc}
\renewcommand{\arraystretch}{1.2}
\begin{tabular}[b]{|c|c|c|c|}
\hline
$D_F$     & \p & \q &  \s  \\ \hline \hline 
\p 	& $\{\epsilon\}$ & $\{\fieldD{f}{} \}$ & $\{\epsilon\}$  \\ \hline 
\q             & $\{\fieldD{f}{}\}$      & $\{\epsilon\}$          & $\{\fieldD{f}{}\}$  \\ \hline
\s             & $\{\epsilon\}$           &$\{\fieldD{f}{}\}$      & $\{\epsilon\}$      \\ \hline
\end{tabular} 
&

\renewcommand{\arraystretch}{1.2}
% % \newcommand{\iwd}{0.23\columnwidth}
\begin{tabular}[b]{|c|c|c|c|}
\hline 
$\ I_F$     & $\p$	               & $\q$
&  $\s$              \\ \hline \hline 
%%
$\p$ & $\{ (\epsilon, \epsilon)\}$    & $\{(\fieldD{f}{},\epsilon),(\epsilon,\fieldD{f}{}) \} $   & $\{ (\epsilon, \epsilon)\}$ \\ \hline
$\q$ & $\{(\fieldD{f}{},\epsilon),(\epsilon,\fieldD{f}{}) \} $   & $\{ (\epsilon, \epsilon)\}$ & $\{(\fieldD{f}{},\epsilon),(\epsilon,\fieldD{f}{}) \} $ \\ \hline
$\s$ & $\{ (\epsilon, \epsilon)\}$    & $\{(\fieldD{f}{},\epsilon),(\epsilon,\fieldD{f}{}) \} $   & $\{ (\epsilon, \epsilon)\}$ \\ \hline
\end{tabular} \\
% \caption{C1:Context Insensitive}
\end{tabular}
} \\
& \scriptsize (a) Context Insensitive c2:IN
\end{tabular}
}


\section{Implementation,Testing and Results}
\subsection{Implementation}

\begin{frame}[fragile]
  \frametitle{\subsecname}
  \begin{itemize}
   \item Plugins
    \begin{itemize}
     \item Plugin’s make the developer add new features to the compiler without modifying the compiler itself. 
     \item Adding a pass to the compiler
     \item No need of compiling source code every time 
    \end{itemize}
   \item Tree
      \begin{itemize}
       \item The central data structure used by the internal representation of GCC is the Tree. 
       \item A tree is a pointer type, but the object to which it points may be of a variety of types.
       \item TREE\_CODE , TREE\_TYPE, TREE\_FIELD 
      \end{itemize}
    \item {\small GIMPLE}
      \begin{itemize}
       \item 3-address representation with at max one load/store per statement. 
       \item GCC front end converts source code to GIMPLE
       \item is\_gimple\_assign(stmt)		
      \end{itemize} 
  \end{itemize}
\end{frame}


% \subsection{Results}
% \frame
% {
%   \frametitle{New Results}
%   \begin{table}[htbp]
\scalebox{0.70}{
\begin{tabular}{|l|l|c|l|c|c|}
\hline
\multicolumn{ 1}{|c|}{Benchmark} & \multicolumn{ 2}{c|}{Ghiya's Analysis} & \multicolumn{ 2}{c|}{Field Sensitive} & \multicolumn{ 1}{l|}{Result} \\ \cline{ 2- 5}
\multicolumn{ 1}{|c|}{} & Shape & \multicolumn{1}{l|}{Time(Sec)} & Shape & \multicolumn{1}{l|}{Time(Sec)} & \multicolumn{ 1}{l|}{} \\ \hline
100\_create\_iter.cpp & (30,0,0) & 0.004 & (30,0,0) & 0.051 &  \\ \hline
100\_create\_recur.cpp & (36,0,0) & 0.003 & (36,0,0) & 0.043 &  \\ \hline
200\_delall\_iter\_create\_fixed.cpp & (168,0,0) & 0.012 & (168,0,0) & 0.112 &  \\ \hline
200\_delall\_iter\_create\_iter.cpp & (63,0,0) & 0.008 & (63,0,0) & 0.094 &  \\ \hline
200\_delall\_recur\_create\_fixed.cpp & (12,0,0) & 0.003 & (12,0,0) & 0.046 &  \\ \hline
200\_delall\_recur\_create\_iter.cpp & (56,0,0) & 0.008 & (56,0,0) & 0.12 &  \\ \hline
\red{300\_insert\_iter\_create\_fixed.cpp} & (341,19,0) & 0.042 & (354,0,6) & 0.49 & \# \\ \hline
\red{300\_insert\_iter\_create\_iter.cpp} & (161,19,0) & 0.053 & (174,0,6) & 0.489 & \# \\ \hline
300\_insert\_recur\_create\_fixed.cpp & (225,0,27) & 0.02 & (225,0,27) & 0.176 &  \\ \hline
300\_insert\_recur\_create\_iter.cpp & (113,0,27) & 0.018 & (113,0,27) & 0.239 &  \\ \hline
\blue{400\_remove\_iter\_create\_fixed.cpp} & (314,0,26) & 0.015 & (337,3,0) & 0.242 & \$ \\ \hline
\blue{400\_remove\_iter\_create\_iter.cpp} & (154,0,26) & 0.028 & (177,3,0) & 0.289 & \$ \\ \hline
400\_remove\_recur\_create\_fixed.cpp & (243,0,18) & 0.015 & (243,0,18) & 0.179 &  \\ \hline
400\_remove\_recur\_create\_iter.cpp & (108,0,18) & 0.015 & (108,0,18) & 0.164 &  \\ \hline
\end{tabular}
}

\centering{ \footnotesize \$- Field sensitive analysis is more precise  \quad  \footnotesize \#- Ghiya's analysis is more precise}
\label{LatestRes1}
\end{table}


	
% }
% 
% \frame
% {
%   \frametitle{New Results}
%   \begin{table}[htbp]
\scalebox{0.70}{
\begin{tabular}{|l|l|c|l|c|c|}
\hline
\multicolumn{ 1}{|c|}{Benchmark} & \multicolumn{ 2}{c|}{Ghiya's Analysis} & \multicolumn{ 2}{c|}{Field Sensitive} & \multicolumn{ 1}{l|}{Result} \\ \cline{ 2- 5}
\multicolumn{ 1}{|c|}{} & Shape & \multicolumn{1}{l|}{Time(Sec)} & Shape & \multicolumn{1}{l|}{Time(Sec)} & \multicolumn{ 1}{l|}{} \\ \hline
500\_search\_iter\_create\_fixed.cpp & (168,0,0) & 0.011 & (168,0,0) & 0.099 &  \\ \hline
500\_search\_iter\_create\_iter.cpp & (63,0,0) & 0.007 & (63,0,0) & 0.105 &  \\ \hline
500\_search\_recur\_create\_fixed.cpp & (208,0,0) & 0.014 & (208,0,0) & 0.12 &  \\ \hline
500\_search\_recur\_create\_iter.cpp & (88,0,0) & 0.009 & (88,0,0) & 0.151 &  \\ \hline
\red{600\_append\_iter\_create\_fixed.cpp} & (358,6,0) & 0.033 & (343,0,21) & 0.417 & \# \\ \hline
\red{600\_append\_iter\_create\_iter.cpp} & (163,6,0) & 0.022 & (148,0,21) & 0.439 & \# \\ \hline
\red{600\_append\_recur\_create\_fixed.cpp} & (354,0,38) & 0.032 & (342,0,50) & 0.533 & \# \\ \hline
\blue{600\_append\_recur\_create\_iter.cpp} & (144,0,38) & 0.039 & (153,0,29) & 0.494 & \$ \\ \hline
700\_merge\_iter\_create\_fixed.cpp & (311,0,109) & 0.082 & (311,0,109) & 1.657 &  \\ \hline
700\_merge\_iter\_create\_iter.cpp & (242,0,142) & 0.107 & (242,0,142) & 1.937 &  \\ \hline
700\_merge\_recur\_create\_fixed.cpp & (498,0,114) & 0.077 & (498,0,114) & 1.396 &  \\ \hline
700\_merge\_recur\_create\_iter.cpp & (228,0,114) & 0.072 & (228,0,114) & 1.362 &  \\ \hline
\blue{800\_reverse\_iter\_create\_fixed.cpp} & (233,0,47) & 0.016 & (241,0,39) & 0.29 & \$ \\ \hline
\blue{800\_reverse\_iter\_create\_iter.cpp} & (83,0,47) & 0.036 & (91,0,39) & 0.272 & \$ \\ \hline
\blue{800\_reverse\_recur\_create\_fixed.cpp} & (499,0,62) & 0.042 & (507,0,54) & 0.764 & \$ \\ \hline
\blue{800\_reverse\_recur\_create\_iter.cpp} & (244,0,62) & 0.062 & (252,0,54) & 0.758 & \$ \\ \hline
\end{tabular}
}

\centering{ \footnotesize \$-Field sensitive analysis is more precise  \quad  \footnotesize \#- Ghiya's analysis is more precise}
\label{LatestRes2}
\end{table}

% }


\subsection{Results}
\frame
{
  \frametitle{New Results}
  
\begin{table}[htbp]
\scalebox{0.70}{
\begin{tabular}{|l|l|c|l|c|c|}
\hline
\multicolumn{ 1}{|c|}{Benchmark} & \multicolumn{ 2}{c|}{Ghiya} & \multicolumn{ 2}{c|}{Field} & Result \\ \cline{ 2- 6}
\multicolumn{ 1}{|l|}{} & Shape & \multicolumn{1}{l|}{Time(Sec)} & Shape & \multicolumn{1}{l|}{Time(Sec)} &  \\ \hline
100\_create\_iter.cpp & (30,0,0) & 0 & (30,0,0) & 0.033 &  \\ \hline
100\_create\_recur.cpp & (36,0,0) & 0 & (36,0,0) & 0.03 &  \\ \hline
200\_delall\_iter\_create\_fixed.cpp & (168,0,0) & 0.001 & (168,0,0) & 0.036 &  \\ \hline
200\_delall\_iter\_create\_iter.cpp & (63,0,0) & 0.001 & (63,0,0) & 0.033 &  \\ \hline
200\_delall\_recur\_create\_fixed.cpp & (12,0,0) & 0 & (12,0,0) & 0.031 &  \\ \hline
200\_delall\_recur\_create\_iter.cpp & (56,0,0) & 0.001 & (56,0,0) & 0.034 &  \\ \hline
\red{300\_insert\_iter\_create\_fixed.cpp} & (341,19,0) & 0.003 & (354,0,6) & 0.08 & \# \\ \hline
\red{300\_insert\_iter\_create\_iter.cpp} & (161,19,0) & 0.003 & (174,0,6) & 0.075 &  \# \\ \hline
300\_insert\_recur\_create\_fixed.cpp & (225,0,27) & 0.002 & (225,0,27) & 0.044 &  \\ \hline
300\_insert\_recur\_create\_iter.cpp & (113,0,27) & 0.002 & (113,0,27) & 0.049 &  \\ \hline
\blue{400\_remove\_iter\_create\_fixed.cpp} & (314,0,26) & 0.002 & (340,0,0) & 0.05 &  \$ \\ \hline
\blue{400\_remove\_iter\_create\_iter.cpp} & (154,0,26) & 0.002 & (180,0,0) & 0.058 &  \$ \\ \hline
400\_remove\_recur\_create\_fixed.cpp & (243,0,18) & 0.002 & (243,0,18) & 0.043 &  \\ \hline
400\_remove\_recur\_create\_iter.cpp & (108,0,18) & 0.002 & (108,0,18) & 0.047 &  \\ \hline
\end{tabular}
}

\centering{ \footnotesize \$-Field sensitive analysis is more precise  \quad  \footnotesize \#- Ghiya's analysis is more precise}
\label{LatestRes_noI1}
\end{table}
	
}

\frame
{
  \frametitle{New Results}
  \begin{table}[htbp]
\scalebox{0.70}{
\begin{tabular}{|l|l|c|l|c|c|}
\hline
\multicolumn{ 1}{|c|}{Benchmark} & \multicolumn{ 2}{c|}{Ghiya} & \multicolumn{ 2}{c|}{Field} & Result \\ \cline{ 2- 6}
\multicolumn{ 1}{|l|}{} & Shape & \multicolumn{1}{l|}{Time(Sec)} & Shape & \multicolumn{1}{l|}{Time(Sec)} &  \\ \hline
500\_search\_iter\_create\_fixed.cpp & (168,0,0) & 0.001 & (168,0,0) & 0.034 &  \\ \hline
500\_search\_iter\_create\_iter.cpp & (63,0,0) & 0.001 & (63,0,0) & 0.033 &  \\ \hline
500\_search\_recur\_create\_fixed.cpp & (208,0,0) & 0.001 & (208,0,0) & 0.043 &  \\ \hline
500\_search\_recur\_create\_iter.cpp & (88,0,0) & 0.001 & (88,0,0) & 0.036 &  \\ \hline
\red{600\_append\_iter\_create\_fixed.cpp} & (358,6,0) & 0.003 & (343,0,21) & 0.069 & \# \\ \hline
\red{600\_append\_iter\_create\_iter.cpp} & (163,6,0) & 0.002 & (148,0,21) & 0.06 &  \#\\ \hline
\blue{600\_append\_recur\_create\_fixed.cpp} & (354,0,38) & 0.003 & (363,0,29) & 0.084 & \$ \\ \hline
\blue{600\_append\_recur\_create\_iter.cpp }& (144,0,38) & 0.003 & (153,0,29) & 0.066 &  \$\\ \hline
700\_merge\_iter\_create\_fixed.cpp & (311,0,109) & 0.006 & (311,0,109) & 0.187 &  \\ \hline
700\_merge\_iter\_create\_iter.cpp & (242,0,142) & 0.009 & (242,0,142) & 0.195 &  \\ \hline
700\_merge\_recur\_create\_fixed.cpp & (498,0,114) & 0.008 & (498,0,114) & 0.17 &  \\ \hline
700\_merge\_recur\_create\_iter.cpp & (228,0,114) & 0.008 & (228,0,114) & 0.158 &  \\ \hline
\blue{800\_reverse\_iter\_create\_fixed.cpp }& (233,0,47) & 0.002 & (241,0,39) & 0.046 &  \$ \\ \hline
\blue{800\_reverse\_iter\_create\_iter.cpp} & (83,0,47) & 0.002 & (91,0,39) & 0.061 &  \$ \\ \hline
\blue{800\_reverse\_recur\_create\_fixed.cpp} & (499,0,62) & 0.006 & (507,0,54) & 0.111 &  \$ \\ \hline
\blue{800\_reverse\_recur\_create\_iter.cpp} & (244,0,62) & 0.006 & (252,0,54) & 0.109 &  \$ \\ \hline
\end{tabular}
}

\centering{ \footnotesize \$-Field sensitive analysis is more precise  \quad  \footnotesize \#- Ghiya's analysis is more precise}
\label{LatestRes_noI2}
\end{table}
}


\end{document}
