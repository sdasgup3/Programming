
\documentclass[11pt]{article}
\usepackage{url}
\begin{document}
 
\section{Introduction}
\subsection{Brief Introduction}

Shape analysis,the aim of this static analysis technique is to infer some useful properties about the
the programs changing heap structure  which can be used in many areas like garbage collection,parallelization,
compiler time optimization,instruction scheduling etc. \\
  
  In this report we discuss the effectiveness of Sandeep's \cite{Sandeep} work which is about field sensitive shape analysis
in comparison to that of Ghiya's \cite{Ghiya96}.Some of the  implementation details about Sandeep's \cite{Sandeep} analysis
 is also provided .When considering the inter procedural anlysis should we go  for context sensitive or a context
insensitive analysis; feasibility of each of it is discussed along with the supporting arguments and data.
In the context insensitive case a new method of merging contexts at function calls is looked at.\\

  Assume a particular function in a program which uses not all field pointers ,and we were asked
to find the shape of a heap pointer in that function.If we go by Sandeep's\cite{Sandeep} ,it doesn't take into consideration
the field pointers absent in that function because of which the shape may have changed giving less
precise shape i.e instead of considering just a subset of all the fields,everything was considered .We create a subset
for each function and  use the field sensitive information obtained by Sandeep's \cite{Sandeep} work,
to find the shape of the heap pointer inside that function.This subset-based field sensitive
analysis can give even more precise results.
  
\subsection{Organization Of Thesis}
  Some of the earlier related work is discussed in Chapter 2. Chapter 3
tells about the details of implementation of field sensitive analysis, optimization's suggested and results
run on benchmarks.Details about the subset-based analysis is discussed in chapter 4 with the help of examples.The advantages and disadvantages of
some inter procedural analysis methods are given in Chapter 5 along with supporting numbers.Finally details about 
future work in chapter 6.



\section{Subset Based Analysis}
  \subsection{Motivating Example}




\bibliographystyle{unsrt}   % this means that the order of references
			    % is dtermined by the order in which the
			    % \cite and \nocite commands appear
\bibliography{references}  % list here all the bibliographies that


\end{document}
