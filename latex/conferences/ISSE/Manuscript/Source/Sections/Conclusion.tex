In this  paper we proposed  a field sensitive  shape analysis
technique to  infer shapes of heap  structures.  Our approach
uses field based boolean variables along with field sensitive
path  matrices to infer  the shapes  of pointer  variables in
terms  of  boolean  equations.   The path  matrices  help  in
remembering connectivity  information of pointers,  while the
field based  boolean variables help in  remembering the exact
updates affecting  the shapes.   This allows our  analysis to
generate precise kill  information for field updates, thereby
capturing the shape transitions from Cycle to DAG, from Cycle
to Tree  and from  DAG to Tree. 

We  have  shown  some  scenarios  that can  be  handled  more
precisely  by  our analysis  as  compared  to existing  field
insensitive  analyses.   To  show  the effectiveness  of  our
analysis,  we implemented our  analysis as  a plug-in  for GCC
version  4.5.0.    The  implementation  is   an  instance  of
call-string    based   interprocedural    data-flow   analysis
framework.  We evaluated  our analysis on standard benchmarks
and showed that the results are more precise than an existing
field insensitive analysis.   We have shown some enhancements
that can  be easily incorporated in our  analysis to increase
its effectiveness in some cases.

Our   shape   analysis  can   be   used   by  compilers   for
optimizations, parallelization and  verification.  There is a
lot of scope  to improve the memory and  the time required by
the  analysis. In a  large program,  typically a  pointer has
path to only  a few other pointers.  Therefore,  in future we
plan   to  use   sparse  matrices   to  represent   the  path
information.  We also plan  to implement with a demand-driven
variation of  the analysis that can  switch between precision
(field sensitivity) and speed (field insensitivity) depending
on  the needs  of the  target  application. We  also plan  to
implement  and  evaluate the  enhancements  to our  analysis,
namely Shape-based context sensitive interprocedural analysis   
and Field-subset based analysis, in near future.
