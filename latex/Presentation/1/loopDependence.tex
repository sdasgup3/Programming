\chapter{Loop Sensitive Dependence Analysis}
\label{ch:loopdep}
This chapter focuses on detecting the presence of dependences on loops which 
traverse recursive heap data structure. Two statements \ttf{S} and \ttf{T} may induce (a) \ttf{Loop Independent Dependence} (LID), where statements \ttf{S} and \ttf{T}
access same memory location in a single iteration of the loop, (b) \ttf{Loop Carried Dependence} (LCD), if the 
memory location accessed by statement \ttf{S} in a given iteration, is accessed 
by statement \ttf{T} in other iteration. In either case at least one of the accesses must be write access. 

Our approach for dependence analysis, as explained earlier, does not work well 
for loops. This is because it combines the paths accessed in different iterations 
of a loop.  To get better result in
presence of loops we need to keep the accesses made by
different iterations of a loop separate. To do so, we have
devised another novel approach, which works as follows: Given
a loop, we first identify the navigators for the loop,
then by a single symbolic traversal over the loop, we compute
the read and write accesses made by each statement in terms
of the values of the navigators. The read and write sets thus
obtained are generalized to represent arbitrary iteration of
the loop, using the iteration number as a parameter. These generalized 
sets, in terms of equations, are tested by \emph{GCD} or \emph{Lamport} test 
to find out any integer solution of those equations. Presence of loop dependences
indicates that the iterations are not independent, hence 
can not be executed in parallel. The top level algorithm of loop analysis is 
outlined in Figure~\ref{fig:algoLoopAnalyse}.

We assume that the loop under analysis is heap intensive i,e., 
reads/writes heap and the execution of the loop does not stop 
prematurely using irregular control flow constructs 
such as \emph{return}, \emph{continue}, \emph{break} 
statements or function calls like \emph{exit}, \emph{abort}. 
Hence testing loop condition is the only way to exit 
control from the loop. 

The rest of the 
chapter is organised as follows: Section~\ref{sec:nav} gives a
brief description of finding navigator of the loop. Section~\ref{sec:DepDetect} 
explains about how to compute read and wriets sets of access 
paths and how our approach identifies both loop independent and
loop carried dependences.
\begin{figure}
%\hrule
\begin{framed}
{\tt
  \begin{program}{0}
  \FL fun loopAnalyse(Loop)  \{
%  \UNL{0} GoodLoop = detectGoodLoop(f);
  \UNL{0} <NavigatorVar, NavigatorExpr>  = identifyNavigator(Loop);%\COMMENT{returns both navigator variable and navigator expr}
  \UNL{0} ReadWriteSet = generateReadWrite(NavigatorVar);
  \UNL{0} identifyDep(ReadWriteSet);
  \UNL{-1} \}
  \end{program}
}
\end{framed}
%\hrule
  \caption{Dep. detection for loop. \label{fig:algoLoopAnalyse}}
%\hrule  
\end{figure}
%%%%%%%%%%%%%%%%%%%%%%%%%%%%%%%%%%%%%%%%%%%%%%%%%%%%%%%%%%
\section{Identifying Navigator}
\label{sec:nav}
Dependence analysis for loops relies on the computation of read and write sets, 
in terms of access path, for each statement in a single symbolic iteration of the loop. 
Access paths are computed with respect to \emph{navigator} as mentioned 
in~\cite{ghiya98detecting}. A navigator consists of (a)\emph{navigator variable} \ttf{NavigatorVar}, pointer variable 
used to traverse the loop and (b)\emph{navigator expression} \ttf{NavigatorExpr}, ordered set of pointer field references. 
Navigator variable in association with the navigator expression, iterates the loop traversing the data structure.

Navigator variable is closely related to the variable {\tt TestVar} used to test the stopping criteria
for the loop in the program. The algorithm generates the definition chain \ttf{DefChain} of 
{\tt TestVar} using statements inside the loop. If the definition chain of \ttf{TestVar} encounters a 
loop resident statement twice, recurrence is reported. Otherwise the creation of definition chain returns 
null if it fails to find a loop resident statement for \ttf{DefChain}. 
Hence {\tt DefChain}, thus generated for the 
later case returns an access path consisting of a pointer variable followed by an ordered sequence of pointer field references. 
The base pointer variable obtained from the access path is potential candidate to be navigator variable and the  
sequence of field references results in navigator expression. The details for identifying navigator can be found 
in~\cite{ghiya98detecting}. 

\begin{example}{\rm
Consider the code shown in Figure~\ref{ch:Intro}.2(b). We identify 
\ttf{p} as loop condition test variable. Definition chain for \ttf{p} comes from the sequence of following loop-resident statements, 
{\tt S7: p = r}, {\tt S4: r = q\rtarrow next}, {\tt S2: q = p\rtarrow next} and {\tt S7: p = r}. Note that statement {\tt S7} is
encountered twice, leading to recurrence. Hence the statements {\tt S7}, {\tt S4}, {\tt S2} are added to {\tt DefChain} which returns 
access path as {\tt p\rtarrow next\rtarrow next}. Hence the resulting navigator consists of navigator variable {\tt p} and navigator expression 
{\tt next\rtarrow next}
}
\hfill\psframebox{}  \end{example}
%%%%%%%%%%%%%%%%%%%%%%%%%%%%%%%%%%%%%%%%%%%%%%%%%%%%%%%%%%%%%%%%%%%%%%%%%%%%%%%%%%%%
\section{Computing Read/Write Sets}
As mentioned earlier, our analysis computes read and 
write sets for each statement residing in the loop in 
a single symbolic execution of the loop. Read and write sets 
consist of paths that access heap locations for reading or writing. Unlike previous analysis, 
full length access paths are used by loop dependence analysis. For each 
loop-residing statement full length access paths, referred as \ttf{AccPath}, 
are computed in terms of navigator variable. Access paths \ttf{AccPath} 
are computed from definition chains, that are evaluated by recursively 
traversing all the reaching definitions of the pointer variable 
used by the statement until 
the navigator variable is encountered. 

\begin{figure}
%\hrule
\begin{framed}
{\tt
  \begin{program}{0}
  \FL fun generateReadWrite(NavigatorVar)  \{
%  \UNL{0} CFG[f] = (N, E, Entry, Exit); \COMMENT {Control flow graph of function f}
 % \UNL{0} InitSet = initialize(); \COMMENT{Initialize all parameters and
%  globals}
  \UNL{0} for each statement $S_i$ in Loop \{
  \UNL{1} UseVar = UseVarSet[$S_i$];
  \UNL{1} DefChain[$S_i$] = findDefChain(UseVar, NavigatorVar);
  \UNL{1} AccPath[$S_i$] = findAccPath(DefChain[$S_i$]);
  \UNL{1} if (Tag[$S_i$] == ReadStmt) \{
  \UNL{2} ReadSet[$S_i$] = AccPath[$S_i$];
  \UNL{2} WriteSet[$S_i$] = $\phi$;
  \UNL{1} \}
   \UNL{1} else if (Tag[$S_i$] == WriteStmt) \{
  \UNL{2} WriteSet[$S_i$] = AccPath[$S_i$];
  \UNL{2} ReadSet[$S_i$] = $\phi$;
  \UNL{1} \}
   \UNL{1} else \{
  \UNL{2} ReadSet[$S_i$] = $\phi$;
  \UNL{2} WriteSet[$S_i$] = $\phi$;
  \UNL{1} \}
  \UNL{0} \}
  \UNL{-1} \}
  \end{program}
}
\end{framed}
%\hrule
  \caption{Generating Read and Write sets. \label{fig:algoReadWrite}}
%\hrule  
\end{figure}

These access paths, thus constructed, return 
read/write sets based on statements reading or writing heap data. 
Function \ttf{generateReadWrite} showed in Figure~\ref{fig:algoReadWrite} 
computes such sets of access paths with respect to navigator variable \ttf{NavigaotrVar}.
Definition chain, \ttf{DefChain}, produced by function \ttf{findDefChain}, 
is processed by function \ttf{findAccPath} to compute access path. 
\ttf{ReadSet}/\ttf{WriteSet} sets, for each statement, are then computed from \ttf{AccPath}. 
The access paths, thus obtained, 
are generalized by arbitrary iteration of the loop, using iteration number as parameter, for further processing. 
\begin{example}{\rm
Let us again consider the example given in Figure~\ref{ch:Intro}.2(b). 
Navigator variable and navigator expression for the loop are \ttf{p} and \ttf{next\rtarrow{next}} 
respectively. Table~\ref{fig:tableLoopReadWrite} shows the read and write sets of full access paths constructed for each loop residing statement. Note that, 
the access paths are not abstracted and are constructed in terms of \ttf{p}. 
}
\hfill\psframebox{}  \end{example}
\begin{table}
\centering
\begin{tabular}{| l | c | c |}
\hline 
Statement & Read Set & Write Set \tn
\hline \hline
{\tt S2} & $\phi$ & $\phi$ \tn
{\tt S3} & \ttf{p\rtarrow{next}} & $\phi$ \tn
{\tt S4} & $\phi$ & $\phi$ \tn
{\tt S5} & $\phi$ & \ttf{p\rtarrow{next}\rtarrow{next}} \tn
{\tt S6} & $\phi$ & $\phi$ \tn
\hline
\end{tabular}
\caption{Read and write sets for each loop residing statement} 
\label{fig:tableLoopReadWrite}
%\hrule
\end{table}
%%%%%%%%%%%%%%%%%%%%%%%%%%%%%%%%%%%%%%%%%%%%%%%%%%%%%%%%%%%%%%%%%%%%%%%%%%%%%%%%%
%%%%%%%%%%%%%%%%%%%%%%%%%%%%%%%%%%%%%%%%%%%%%%%%%%%%%%%%%%%%%%%%%%%%%%%%%%%%%%%%%%%
\section{Loop Dependence Detection}
\label{sec:DepDetect}
Let \ttf{S} and \ttf{T} be two statements inside a loop. Further, 
let \ttf{write(S,i)} denote the set of access paths written by statement 
\ttf{S} in the iteration number \ttf{i}, and let \ttf{read(T,j)} 
denote the set of access paths read by statement \ttf{T} in the iteration
number \ttf{j}. Predicate \ttf{sharing}($\ttf{Set}_\ttf{1}$, $\ttf{Set}_\ttf{2}$) returns 
true if two access paths $\ttf{AccPath}_\ttf{1}\in\ttf{Set}_\ttf{1}$ and $\ttf{AccPath}_\ttf{2}\in\ttf{Set}_\ttf{2}$ 
share a common heap node. Then
\begin{itemize}
\item \ttf{T} is loop independent flow dependent on \ttf{S} if there is an 
execution path from \ttf{S} to \ttf{T} that does not cross the loop boundary and there exist \ttf{i} 
within loop bounds such that \ttf{sharing(write(S,i), read(T,i))} is true.
\item \ttf{T} is loop carried flow dependent on \ttf{S} if there exist 
\ttf{i} and \ttf{j} within loop bounds such that \ttf{j}>\ttf{i}, and \ttf{sharing(write(S,i), read(T,j))} is true.
\end{itemize}
Note that, in case loop bounds can not be computed at compile time,
we can assume them to be (-$\infty$,$\infty$). We can similarly define 
loop independent and loop carried anti-dependence and output-dependence.
Read and write sets of access paths, thus obtained for 
each statement inside a loop are tested for both loop independent 
and loop carried dependences.
%%%%%%%%%%%%%%%%%%%%%%%%%%%%%%%%%%%%%%%%%%%%%%%%%%%%%%%%%%%%%%%%%%%%%%%%%%%%%%%%%%%%%%
%%%%%%%%%%%%%%%%%%%%%%%%%%%%%%%%%%%%%%%%%%%%%%%%%%%%%%%%%%%%%%%%%%%%%%%%%%%%%%%%%%%%%% 
\subsection{Identifying Loop Independent Dependence}
Loop independent dependences can be detected for 
any two statements by checking for any sharing of node by their 
respective read/write sets. Sharing of a node, in this level, occurs due to 
the shape of the underlying data structure. Shape analysis gives the 
probable shape attribute of the navigator variable traversing dynamic data structure. Based 
on the shape we can figure out whether 
there exists any dependence due to sharing within the underlying data structure. 
\begin{description}
\item[Observation 1:]If the shape attribute of navigator variable is Tree, then there 
exist no sharing of nodes by different access paths rooted at the navigator variable. Two access paths can only visit 
a common node if the paths are equivalent. Let \ttf{lp} be the navigator variable. Hence \ttf{lp\rtarrow{f}} and \ttf{lp\rtarrow{f}} 
are equivalent paths leading to a common node, whereas \ttf{lp\rtarrow{f}} and \ttf{lp\rtarrow{g}} lead to different nodes.

\item[Observation 2:] If the shape attribute is DAG, the navigator expression will lead navigator 
variable to a distinct node in each iteration of the loop. If an access path is a proper subpath 
of another access path then they surely visit distinct nodes. However, paths being either equivalent or distinct, having different 
pointer field references may access a common node. For example, \ttf{lp\rtarrow{f}} is a proper subpath of 
\ttf{lp\rtarrow{f}\rtarrow{g}} whereas, \ttf{lp\rtarrow{f}} and 
\ttf{lp\rtarrow{g}} are not. Hence in the former case they do not share a common node, whereas  
in later case they might result in sharing of node.

\item[Observation 3:] If shape attribute of navigator variable is Cycle, we make conservative decision such that the loop can 
not be executed in parallel.
\end{description}
To detect various types of LIDs, read and write sets of different statements are tested 
accordingly for detecting sharing of node. Let two statements \ttf{S} and \ttf{T} access paths 
\ttf{PathS} and \ttf{PathT} respectively and \ttf{read(S) = \{PathS\}} and \ttf{write(T) = \{PathT\}}. 
Hence there is loop independent flow dependence from \ttf{S} to \ttf{T} if \ttf{sharing(PathS, PathT)} returns true.
We check for the shape of the underlying data structure 
and test the paths as follows:
\begin{enumerate}
\item If the paths \ttf{PathS} and \ttf{PathT} are equivalent and the data structure is either Tree or DAG, 
the paths will access same node. Hence,
\[ \ttf{sharing (PathS, PathT)} = True\] for both Tree and DAG structure.
\item If \ttf{PathS} is subpath of \ttf{PathT} then these paths do not lead to any common node for both Tree 
and DAG data structure. Hence, 
\[ \ttf{sharing (PathS, PathT)} = False\] for both Tree and DAG structure.
\item If \ttf{PathS} and \ttf{PathT} are not equivalent and one is not subpath of other, then these two paths share a common node only if 
shape of the underlying structure is DAG. For Tree structure they lead to disjoint nodes. 
\[ \ttf{sharing (PathS, PathT)} = False\] if shape attribute is TREE.
\[ \ttf{sharing (PathS, PathT)} = True\] if shape attribute is DAG.
\end{enumerate}
\begin{example}{\rm
Consider the loop shown in Figure~\ref{ch:Intro}.2(b) and the corresponding 
read and write sets for each statement in Table~\ref{fig:tableLoopReadWrite}. Read set of statement
\ttf{S3} and write set of statement \ttf{S5} are checked for sharing of any  node. As the shape attribute 
of the navigator variable \ttf{p} is Tree, and the paths \ttf{p\rtarrow{next}} and 
\ttf{p\rtarrow{next}\rtarrow{next}} are not equivalent the following will result.
\[\ttf{sharing(p\rightarrow{next}, p\rightarrow{next}\rightarrow{next})} = False\]
Hence no loop independent dependence is detected. 
}
\hfill\psframebox{}  \end{example}
%%%%%%%%%%%%%%%%%%%%%%%%%%%%%%%%%%%%%%%%%%%%%%%%%%%%%%%%%%%%%%%%%%%%%%%%%%%%%%%%%%%%
%%%%%%%%%%%%%%%%%%%%%%%%%%%%%%%%%%%%%%%%%%%%%%%%%%%%%%%%%%%%%%%%%%%%%%%%%%%%%%%%%%%%
\subsection{Identifying Loop Carried Dependence}
Loop carried dependence is incurred in the loop when two statements from different 
iterations access same memory location. LCDs can be introduced 
when the statements in a single iteration of loop access both current node and neighbour heap nodes.  
Current node means the node which is being currently accessed by the navigator 
variable, whereas, neighbour nodes mean nodes other than the one being currently 
accessed. 
\begin{example}{\rm
Let the shape attribute of the navigator variable \ttf{lp} be Tree and 
a loop is traversing a list using navigator variable \ttf{lp} and navigator expression \ttf{next}. 
Statement \ttf{lp\rtarrow{num}++} will not incur any loop carried dependence as 
the location pointed to by \ttf{lp} can't be accessed in consecutive iterations. 
However, statement like \ttf{lp\rtarrow{num} = lp\rtarrow{next}\rtarrow{num}} 
will still introduce an LCD because both current and neighbour nodes are accessed 
in the same iteration and neighbour node is visited using pointer field \ttf{next} 
which is also a navigator expression. 
}
\hfill\psframebox{}  \end{example}
As mentioned earlier, LCDs are only introduced by different iterations of loop, provided there is 
no sharing of nodes hidden in the data structure. However, DAG has sharing within the 
structure, it can be traversed by a loop in a manner such that shared nodes 
are not accessed by loop and access pattern of the structure is Tree. Hence, loops having this type 
of access pattern can also be tested for loop carried dependencies. 

The read-write sets computed by each loop resident statements are generalized 
for arbitrary iteration number, resulting in a set of equations. Generalization is done 
using random times of navigator expression, that is used by the navigator variable to traverse 
over the underlying data structure. These equations are then tested 
by \ttf{GCD}~\cite{Kennedy01Optimizing} or \ttf{Lamport}~\cite{Lamport1974parallel} test, as explained before. If the equations have any integer solution, 
dependence is reported. 
Here we demonstrate two examples to show the novelty lying in our approach.
\begin{figure}[t]
  \begin{center}
    \scalebox{.85}{\begin{tabular}{@{}c@{}|@{}c@{}}
      {\tt
\begin{program}{10}
  \FL\ \ldots
  \UNL{0} p = list;
  \UNL{0} \WHILE (p\rtarrow{next} != NULL) \{
  \NL{1}     \ldots = p\rtarrow{num};
  \NL{1}     p\rtarrow{next}\rtarrow{num} = \ldots;
  \NL{1}     p = p\rtarrow{next}\rtarrow{next};
  \UNL{0} \}
  \UNL{0} \ldots
\end{program}
}
      &
      {\tt
\begin{program}{20}
  \FL\ \ldots
  \UNL{0} p = list;
  \UNL{0} \WHILE (p\rtarrow{next} != NULL) \{
  \NL{1}     \ldots = p\rtarrow{num};
  \NL{1}     p\rtarrow{next}\rtarrow{num} = \ldots;
  \NL{1}     p = p\rtarrow{next};
  \UNL{0} \}
  \UNL{0} \ldots
\end{program}
} \\
      {(a) \emph{List-1}: Loop without Dependence } &
      {(b) \emph{List-2}: Loop with Dependence }
    \end{tabular}}
  \end{center}
  \hrule
  \caption{\label{fig:loopdep} Identifying Loop Dependences}

\end{figure}
\begin{example}{\rm
Let us consider Figure~\ref{fig:loopdep}. The code snippet in Figure~\ref{fig:loopdep}(a) is the 
reformulation of code given in Figure~\ref{ch:Intro}.2(b). The navigator variable for
 both the loops is \ttf{p}. For the code in Figure~\ref{fig:loopdep}(a), the navigator expression is
  \ttf{next\rtarrow{next}}. Using \ttf{i} to represent the iteration number, the generalized access path read by
    S11 is \ttf{p}\rtarrow$\ttf{{next}^{2i}}$ and the generalized access path written by S12 is 
    \ttf{p}\rtarrow$\ttf{{next}^{2j+1}}$. Clearly there is no loop independent dependence as the shape of 
    the data structure is Tree and statements do not generate equivalent access paths. 
    To find out loop
    carried dependence, we have to find out whether for
    iterations \ttf{i} and \ttf{j} and \ttf{i}$\neq$\ttf{j}, the two paths point to the same
    heap location. This reduces to finding out if there is a
    possible solution to the following equation:
      \[ \ttf{p\rightarrow{next}^{2i} =
      p\rightarrow{next}^{2j+1}}\] 
      In other words, we have to find out if integer solutions
      to the following equation are possible :
      \[\ttf{2*i = 2*j + 1}\]
      GCD or Lamport test tell us\footnote{In this example, visual inspection 
    tells us that for any \ttf{i} and \ttf{j} LHS is even number while RHS is an odd number. Hence the equation can not have a 
    solution. In general, the equations are more complicated and we need to use standard tests as mentioned.} that
      this equation can not have integer solutions. Thus,
      there is no dependence among the statements.
      
        For the code in Figure~\ref{fig:loopdep}(b), the navigator expression 
      is \ttf{next}. In this case the equation to find 
      out the loop carried dependences among statements S21
      and S22 reduces to : 
      \[ \ttf{i = j + 1}\]
      which has integer solutions. So we have to
      conservatively report dependence between the two statements
}
\hfill\psframebox{}  \end{example}







